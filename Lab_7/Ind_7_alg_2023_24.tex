 \documentclass[11pt]{report}

\usepackage[T2A]{fontenc}

\usepackage[utf8]{inputenc}

\usepackage[russian]{babel}

\usepackage{amsmath,amssymb}

\usepackage{graphicx}

\graphicspath{ {d:/HSE/OR/CW/CW5pict/} }

\begin{document}

\pagestyle{empty}

{\bf Индивидуальное задание.}

Вариант N 1

Даны точки: G (-4, 0, -2), K (-4, -4, 2), N (2, -4, 1), S (-22, 12, -11), T (5, 4, -3).
Составить словарь с ключами - точками (например, Point3D(3, 2, 1)) и значениями - именами точек (G, N и т.д.).


Найти и вывести на экран


расстояние от точек K и S до плоскости GNT,


уравнение плоскости GNT,


параметрические уравнения прямой KS.


Определить и вывести на экран, какие 4 из заданных точек лежат в одной плоскости (пользуясь словарем точек, вывести имена точек).


Определить и вывести на экран, какие 3 из заданных точек лежат на одной прямой (пользуясь словарем точек, вывести имена точек).

\newpage
Вариант N 2

Даны точки: F (-3, 5, -4), G (2, 1, -2), J (4, 5, -1), L (14, 25, 4), T (4, -3, 0).
Составить словарь с ключами - точками (например, Point3D(3, 2, 1)) и значениями - именами точек (F, G и т.д.).


Найти и вывести на экран


расстояние от точек T и L до плоскости FGJ,


уравнение плоскости FGJ,


параметрические уравнения прямой TL.


Определить и вывести на экран, какие 4 из заданных точек лежат в одной плоскости (пользуясь словарем точек, вывести имена точек).


Определить и вывести на экран, какие 3 из заданных точек лежат на одной прямой (пользуясь словарем точек, вывести имена точек).

\newpage
Вариант N 3

Даны точки: F (2, 1, -4), H (2, 0, -1), M (4, 1, -1), P (3, 5, 2), R (7, -11, -10).
Составить словарь с ключами - точками (например, Point3D(3, 2, 1)) и значениями - именами точек (F, M и т.д.).


Найти и вывести на экран


расстояние от точек H и R до плоскости FMP,


уравнение плоскости FMP,


параметрические уравнения прямой HR.


Определить и вывести на экран, какие 4 из заданных точек лежат в одной плоскости (пользуясь словарем точек, вывести имена точек).


Определить и вывести на экран, какие 3 из заданных точек лежат на одной прямой (пользуясь словарем точек, вывести имена точек).

\newpage
Вариант N 4

Даны точки: H (-3, 5, 4), K (-4, 1, -4), L (5, -15, -10), M (1, 5, -2), R (2, 0, -4).
Составить словарь с ключами - точками (например, Point3D(3, 2, 1)) и значениями - именами точек (K, M и т.д.).


Найти и вывести на экран


расстояние от точек H и L до плоскости KMR,


уравнение плоскости KMR,


параметрические уравнения прямой HL.


Определить и вывести на экран, какие 4 из заданных точек лежат в одной плоскости (пользуясь словарем точек, вывести имена точек).


Определить и вывести на экран, какие 3 из заданных точек лежат на одной прямой (пользуясь словарем точек, вывести имена точек).

\newpage
Вариант N 5

Даны точки: H (4, 5, -4), K (5, 3, 4), L (2, 2, 1), R (-10, -2, -11), T (5, -1, -2).
Составить словарь с ключами - точками (например, Point3D(3, 2, 1)) и значениями - именами точек (K, L и т.д.).


Найти и вывести на экран


расстояние от точек H и R до плоскости KLT,


уравнение плоскости KLT,


параметрические уравнения прямой HR.


Определить и вывести на экран, какие 4 из заданных точек лежат в одной плоскости (пользуясь словарем точек, вывести имена точек).


Определить и вывести на экран, какие 3 из заданных точек лежат на одной прямой (пользуясь словарем точек, вывести имена точек).

\newpage
Вариант N 6

Даны точки: F (4, 4, 3), H (1, 2, 3), L (25, 19, 1), M (4, 4, -2), N (-3, -1, -3).
Составить словарь с ключами - точками (например, Point3D(3, 2, 1)) и значениями - именами точек (F, M и т.д.).


Найти и вывести на экран


расстояние от точек H и L до плоскости FMN,


уравнение плоскости FMN,


параметрические уравнения прямой HL.


Определить и вывести на экран, какие 4 из заданных точек лежат в одной плоскости (пользуясь словарем точек, вывести имена точек).


Определить и вывести на экран, какие 3 из заданных точек лежат на одной прямой (пользуясь словарем точек, вывести имена точек).

\newpage
Вариант N 7

Даны точки: F (-4, 0, -3), H (-2, 3, -2), K (1, -3, -1), P (-3, -4, 0), R (-5, 4, -6).
Составить словарь с ключами - точками (например, Point3D(3, 2, 1)) и значениями - именами точек (F, K и т.д.).


Найти и вывести на экран


расстояние от точек H и R до плоскости FKP,


уравнение плоскости FKP,


параметрические уравнения прямой HR.


Определить и вывести на экран, какие 4 из заданных точек лежат в одной плоскости (пользуясь словарем точек, вывести имена точек).


Определить и вывести на экран, какие 3 из заданных точек лежат на одной прямой (пользуясь словарем точек, вывести имена точек).

\newpage
Вариант N 8

Даны точки: G (5, 5, 5), L (3, -4, 3), Q (1, -13, 1), R (-2, 2, -2), T (-1, -1, 3).
Составить словарь с ключами - точками (например, Point3D(3, 2, 1)) и значениями - именами точек (G, L и т.д.).


Найти и вывести на экран


расстояние от точек R и Q до плоскости GLT,


уравнение плоскости GLT,


параметрические уравнения прямой RQ.


Определить и вывести на экран, какие 4 из заданных точек лежат в одной плоскости (пользуясь словарем точек, вывести имена точек).


Определить и вывести на экран, какие 3 из заданных точек лежат на одной прямой (пользуясь словарем точек, вывести имена точек).

\newpage
Вариант N 9

Даны точки: G (2, -2, 0), H (4, -4, -1), Q (-1, -5, 3), R (-3, 4, -3), S (3, -1, -1).
Составить словарь с ключами - точками (например, Point3D(3, 2, 1)) и значениями - именами точек (G, H и т.д.).


Найти и вывести на экран


расстояние от точек R и Q до плоскости GHS,


уравнение плоскости GHS,


параметрические уравнения прямой RQ.


Определить и вывести на экран, какие 4 из заданных точек лежат в одной плоскости (пользуясь словарем точек, вывести имена точек).


Определить и вывести на экран, какие 3 из заданных точек лежат на одной прямой (пользуясь словарем точек, вывести имена точек).

\newpage
Вариант N 10

Даны точки: G (-4, 3, 4), H (5, 1, -4), K (3, 5, -4), M (5, -1, 4), Q (-40, 19, 4).
Составить словарь с ключами - точками (например, Point3D(3, 2, 1)) и значениями - именами точек (G, K и т.д.).


Найти и вывести на экран


расстояние от точек H и Q до плоскости GKM,


уравнение плоскости GKM,


параметрические уравнения прямой HQ.


Определить и вывести на экран, какие 4 из заданных точек лежат в одной плоскости (пользуясь словарем точек, вывести имена точек).


Определить и вывести на экран, какие 3 из заданных точек лежат на одной прямой (пользуясь словарем точек, вывести имена точек).

\newpage
Вариант N 11

Даны точки: F (-3, -3, 5), L (5, -4, -3), M (5, 1, -3), Q (-19, -11, 21), R (3, 4, 5).
Составить словарь с ключами - точками (например, Point3D(3, 2, 1)) и значениями - именами точек (F, L и т.д.).


Найти и вывести на экран


расстояние от точек R и Q до плоскости FLM,


уравнение плоскости FLM,


параметрические уравнения прямой RQ.


Определить и вывести на экран, какие 4 из заданных точек лежат в одной плоскости (пользуясь словарем точек, вывести имена точек).


Определить и вывести на экран, какие 3 из заданных точек лежат на одной прямой (пользуясь словарем точек, вывести имена точек).

\newpage
Вариант N 12

Даны точки: G (2, 5, 1), H (-1, 5, 3), L (3, -2, 4), Q (17, 35, 16), S (-1, -1, -2).
Составить словарь с ключами - точками (например, Point3D(3, 2, 1)) и значениями - именами точек (G, H и т.д.).


Найти и вывести на экран


расстояние от точек L и Q до плоскости GHS,


уравнение плоскости GHS,


параметрические уравнения прямой LQ.


Определить и вывести на экран, какие 4 из заданных точек лежат в одной плоскости (пользуясь словарем точек, вывести имена точек).


Определить и вывести на экран, какие 3 из заданных точек лежат на одной прямой (пользуясь словарем точек, вывести имена точек).

\newpage
Вариант N 13

Даны точки: F (-2, -1, 3), L (5, 3, 1), M (-4, -4, 1), Q (6, 11, 11), T (0, 0, 1).
Составить словарь с ключами - точками (например, Point3D(3, 2, 1)) и значениями - именами точек (F, M и т.д.).


Найти и вывести на экран


расстояние от точек L и Q до плоскости FMT,


уравнение плоскости FMT,


параметрические уравнения прямой LQ.


Определить и вывести на экран, какие 4 из заданных точек лежат в одной плоскости (пользуясь словарем точек, вывести имена точек).


Определить и вывести на экран, какие 3 из заданных точек лежат на одной прямой (пользуясь словарем точек, вывести имена точек).

\newpage
Вариант N 14

Даны точки: G (-2, -2, 4), K (1, 2, 0), L (4, 3, 0), N (-3, -1, -2), Q (0, -4, 16).
Составить словарь с ключами - точками (например, Point3D(3, 2, 1)) и значениями - именами точек (G, K и т.д.).


Найти и вывести на экран


расстояние от точек L и Q до плоскости GKN,


уравнение плоскости GKN,


параметрические уравнения прямой LQ.


Определить и вывести на экран, какие 4 из заданных точек лежат в одной плоскости (пользуясь словарем точек, вывести имена точек).


Определить и вывести на экран, какие 3 из заданных точек лежат на одной прямой (пользуясь словарем точек, вывести имена точек).

\newpage
Вариант N 15

Даны точки: K (2, 3, 2), M (0, -1, 0), Q (-6, -13, -6), S (-3, -3, 2), T (0, 0, 3).
Составить словарь с ключами - точками (например, Point3D(3, 2, 1)) и значениями - именами точек (K, M и т.д.).


Найти и вывести на экран


расстояние от точек T и Q до плоскости KMS,


уравнение плоскости KMS,


параметрические уравнения прямой TQ.


Определить и вывести на экран, какие 4 из заданных точек лежат в одной плоскости (пользуясь словарем точек, вывести имена точек).


Определить и вывести на экран, какие 3 из заданных точек лежат на одной прямой (пользуясь словарем точек, вывести имена точек).

\newpage
Вариант N 16

Даны точки: F (-1, -1, 1), L (4, -2, 2), M (1, 4, 2), N (0, -3, 4), R (-3, -24, 10).
Составить словарь с ключами - точками (например, Point3D(3, 2, 1)) и значениями - именами точек (F, M и т.д.).


Найти и вывести на экран


расстояние от точек L и R до плоскости FMN,


уравнение плоскости FMN,


параметрические уравнения прямой LR.


Определить и вывести на экран, какие 4 из заданных точек лежат в одной плоскости (пользуясь словарем точек, вывести имена точек).


Определить и вывести на экран, какие 3 из заданных точек лежат на одной прямой (пользуясь словарем точек, вывести имена точек).

\newpage
Вариант N 17

Даны точки: F (4, 3, 2), H (3, 0, -4), J (5, -1, 5), L (-1, 14, 5), N (3, 4, 5).
Составить словарь с ключами - точками (например, Point3D(3, 2, 1)) и значениями - именами точек (F, J и т.д.).


Найти и вывести на экран


расстояние от точек H и L до плоскости FJN,


уравнение плоскости FJN,


параметрические уравнения прямой HL.


Определить и вывести на экран, какие 4 из заданных точек лежат в одной плоскости (пользуясь словарем точек, вывести имена точек).


Определить и вывести на экран, какие 3 из заданных точек лежат на одной прямой (пользуясь словарем точек, вывести имена точек).

\newpage
Вариант N 18

Даны точки: F (3, 1, 3), G (2, 0, 2), L (5, -2, 3), N (5, -3, 2), Q (-4, 6, 2).
Составить словарь с ключами - точками (например, Point3D(3, 2, 1)) и значениями - именами точек (F, G и т.д.).


Найти и вывести на экран


расстояние от точек L и Q до плоскости FGN,


уравнение плоскости FGN,


параметрические уравнения прямой LQ.


Определить и вывести на экран, какие 4 из заданных точек лежат в одной плоскости (пользуясь словарем точек, вывести имена точек).


Определить и вывести на экран, какие 3 из заданных точек лежат на одной прямой (пользуясь словарем точек, вывести имена точек).

\newpage
Вариант N 19

Даны точки: F (5, -1, 0), G (-3, 3, 5), J (2, 2, -3), P (1, 4, 5), T (5, -4, -27).
Составить словарь с ключами - точками (например, Point3D(3, 2, 1)) и значениями - именами точек (G, J и т.д.).


Найти и вывести на экран


расстояние от точек F и T до плоскости GJP,


уравнение плоскости GJP,


параметрические уравнения прямой FT.


Определить и вывести на экран, какие 4 из заданных точек лежат в одной плоскости (пользуясь словарем точек, вывести имена точек).


Определить и вывести на экран, какие 3 из заданных точек лежат на одной прямой (пользуясь словарем точек, вывести имена точек).

\newpage
Вариант N 20

Даны точки: F (5, -3, 4), K (0, 4, 1), L (2, 2, 5), Q (-10, 18, -5), S (0, -3, -4).
Составить словарь с ключами - точками (например, Point3D(3, 2, 1)) и значениями - именами точек (F, K и т.д.).


Найти и вывести на экран


расстояние от точек L и Q до плоскости FKS,


уравнение плоскости FKS,


параметрические уравнения прямой LQ.


Определить и вывести на экран, какие 4 из заданных точек лежат в одной плоскости (пользуясь словарем точек, вывести имена точек).


Определить и вывести на экран, какие 3 из заданных точек лежат на одной прямой (пользуясь словарем точек, вывести имена точек).

\newpage
Вариант N 21

Даны точки: F (-4, 2, -1), K (1, 5, 3), L (-4, -2, -5), P (-4, 3, 0), T (-3, 3, 2).
Составить словарь с ключами - точками (например, Point3D(3, 2, 1)) и значениями - именами точек (F, K и т.д.).


Найти и вывести на экран


расстояние от точек T и L до плоскости FKP,


уравнение плоскости FKP,


параметрические уравнения прямой TL.


Определить и вывести на экран, какие 4 из заданных точек лежат в одной плоскости (пользуясь словарем точек, вывести имена точек).


Определить и вывести на экран, какие 3 из заданных точек лежат на одной прямой (пользуясь словарем точек, вывести имена точек).

\newpage
Вариант N 22

Даны точки: H (-1, -4, -1), K (3, 5, 1), M (1, 5, -3), Q (7, 5, 9), T (-2, -3, -2).
Составить словарь с ключами - точками (например, Point3D(3, 2, 1)) и значениями - именами точек (K, M и т.д.).


Найти и вывести на экран


расстояние от точек H и Q до плоскости KMT,


уравнение плоскости KMT,


параметрические уравнения прямой HQ.


Определить и вывести на экран, какие 4 из заданных точек лежат в одной плоскости (пользуясь словарем точек, вывести имена точек).


Определить и вывести на экран, какие 3 из заданных точек лежат на одной прямой (пользуясь словарем точек, вывести имена точек).

\newpage
Вариант N 23

Даны точки: F (-2, -3, -2), H (5, 4, -2), K (2, -4, 3), L (4, 33, 10), N (-1, 3, 0).
Составить словарь с ключами - точками (например, Point3D(3, 2, 1)) и значениями - именами точек (F, K и т.д.).


Найти и вывести на экран


расстояние от точек H и L до плоскости FKN,


уравнение плоскости FKN,


параметрические уравнения прямой HL.


Определить и вывести на экран, какие 4 из заданных точек лежат в одной плоскости (пользуясь словарем точек, вывести имена точек).


Определить и вывести на экран, какие 3 из заданных точек лежат на одной прямой (пользуясь словарем точек, вывести имена точек).

\newpage
Вариант N 24

Даны точки: F (5, -1, 0), G (-1, 4, 5), L (1, -3, 1), R (23, -16, -15), T (-4, 0, 5).
Составить словарь с ключами - точками (например, Point3D(3, 2, 1)) и значениями - именами точек (F, G и т.д.).


Найти и вывести на экран


расстояние от точек T и R до плоскости FGL,


уравнение плоскости FGL,


параметрические уравнения прямой TR.


Определить и вывести на экран, какие 4 из заданных точек лежат в одной плоскости (пользуясь словарем точек, вывести имена точек).


Определить и вывести на экран, какие 3 из заданных точек лежат на одной прямой (пользуясь словарем точек, вывести имена точек).

\newpage
Вариант N 25

Даны точки: K (1, 4, 2), Q (-7, 36, 22), R (4, 0, 4), S (4, 4, 3), T (3, -4, -3).
Составить словарь с ключами - точками (например, Point3D(3, 2, 1)) и значениями - именами точек (K, S и т.д.).


Найти и вывести на экран


расстояние от точек R и Q до плоскости KST,


уравнение плоскости KST,


параметрические уравнения прямой RQ.


Определить и вывести на экран, какие 4 из заданных точек лежат в одной плоскости (пользуясь словарем точек, вывести имена точек).


Определить и вывести на экран, какие 3 из заданных точек лежат на одной прямой (пользуясь словарем точек, вывести имена точек).

\newpage
Вариант N 26

Даны точки: H (1, 3, 2), L (-24, -10, 21), N (-4, 2, 1), P (1, 5, -4), T (-2, -4, -2).
Составить словарь с ключами - точками (например, Point3D(3, 2, 1)) и значениями - именами точек (N, P и т.д.).


Найти и вывести на экран


расстояние от точек H и L до плоскости NPT,


уравнение плоскости NPT,


параметрические уравнения прямой HL.


Определить и вывести на экран, какие 4 из заданных точек лежат в одной плоскости (пользуясь словарем точек, вывести имена точек).


Определить и вывести на экран, какие 3 из заданных точек лежат на одной прямой (пользуясь словарем точек, вывести имена точек).

\newpage
Вариант N 27

Даны точки: J (4, -4, -2), N (3, 5, 2), Q (8, -40, -18), R (-1, 1, -2), S (4, 3, 5).
Составить словарь с ключами - точками (например, Point3D(3, 2, 1)) и значениями - именами точек (J, N и т.д.).


Найти и вывести на экран


расстояние от точек R и Q до плоскости JNS,


уравнение плоскости JNS,


параметрические уравнения прямой RQ.


Определить и вывести на экран, какие 4 из заданных точек лежат в одной плоскости (пользуясь словарем точек, вывести имена точек).


Определить и вывести на экран, какие 3 из заданных точек лежат на одной прямой (пользуясь словарем точек, вывести имена точек).

\newpage
Вариант N 28

Даны точки: F (-4, 5, -3), L (-3, -4, -1), Q (1, -40, 7), R (0, 5, 4), S (-4, 4, -1).
Составить словарь с ключами - точками (например, Point3D(3, 2, 1)) и значениями - именами точек (F, L и т.д.).


Найти и вывести на экран


расстояние от точек R и Q до плоскости FLS,


уравнение плоскости FLS,


параметрические уравнения прямой RQ.


Определить и вывести на экран, какие 4 из заданных точек лежат в одной плоскости (пользуясь словарем точек, вывести имена точек).


Определить и вывести на экран, какие 3 из заданных точек лежат на одной прямой (пользуясь словарем точек, вывести имена точек).

\newpage
Вариант N 29

Даны точки: G (-3, 0, -1), H (4, -3, -3), K (1, 2, -1), L (1, 0, -21), P (-4, 0, 4).
Составить словарь с ключами - точками (например, Point3D(3, 2, 1)) и значениями - именами точек (G, K и т.д.).


Найти и вывести на экран


расстояние от точек H и L до плоскости GKP,


уравнение плоскости GKP,


параметрические уравнения прямой HL.


Определить и вывести на экран, какие 4 из заданных точек лежат в одной плоскости (пользуясь словарем точек, вывести имена точек).


Определить и вывести на экран, какие 3 из заданных точек лежат на одной прямой (пользуясь словарем точек, вывести имена точек).

\newpage
Вариант N 30

Даны точки: G (4, -1, 3), J (0, -1, 3), M (0, 3, -4), S (2, 2, -2), T (-20, -1, 3).
Составить словарь с ключами - точками (например, Point3D(3, 2, 1)) и значениями - именами точек (G, J и т.д.).


Найти и вывести на экран


расстояние от точек S и T до плоскости GJM,


уравнение плоскости GJM,


параметрические уравнения прямой ST.


Определить и вывести на экран, какие 4 из заданных точек лежат в одной плоскости (пользуясь словарем точек, вывести имена точек).


Определить и вывести на экран, какие 3 из заданных точек лежат на одной прямой (пользуясь словарем точек, вывести имена точек).

\newpage
Вариант N 31

Даны точки: F (-4, 1, -2), G (1, 1, 5), M (0, 3, -3), R (-2, 7, -19), T (3, -4, 2).
Составить словарь с ключами - точками (например, Point3D(3, 2, 1)) и значениями - именами точек (F, G и т.д.).


Найти и вывести на экран


расстояние от точек T и R до плоскости FGM,


уравнение плоскости FGM,


параметрические уравнения прямой TR.


Определить и вывести на экран, какие 4 из заданных точек лежат в одной плоскости (пользуясь словарем точек, вывести имена точек).


Определить и вывести на экран, какие 3 из заданных точек лежат на одной прямой (пользуясь словарем точек, вывести имена точек).

\newpage
Вариант N 32

Даны точки: G (2, -2, 5), K (-1, 2, -1), N (5, 0, 4), R (-10, -10, 9), T (1, -2, 3).
Составить словарь с ключами - точками (например, Point3D(3, 2, 1)) и значениями - именами точек (G, K и т.д.).


Найти и вывести на экран


расстояние от точек T и R до плоскости GKN,


уравнение плоскости GKN,


параметрические уравнения прямой TR.


Определить и вывести на экран, какие 4 из заданных точек лежат в одной плоскости (пользуясь словарем точек, вывести имена точек).


Определить и вывести на экран, какие 3 из заданных точек лежат на одной прямой (пользуясь словарем точек, вывести имена точек).

\newpage
Вариант N 33

Даны точки: F (3, -2, 1), H (0, -3, -4), Q (-3, -17, -5), S (5, 3, 3), T (-2, -3, -4).
Составить словарь с ключами - точками (например, Point3D(3, 2, 1)) и значениями - именами точек (F, S и т.д.).


Найти и вывести на экран


расстояние от точек H и Q до плоскости FST,


уравнение плоскости FST,


параметрические уравнения прямой HQ.


Определить и вывести на экран, какие 4 из заданных точек лежат в одной плоскости (пользуясь словарем точек, вывести имена точек).


Определить и вывести на экран, какие 3 из заданных точек лежат на одной прямой (пользуясь словарем точек, вывести имена точек).

\newpage
Вариант N 34

Даны точки: F (5, 5, -4), H (-1, -3, 3), Q (5, 32, -19), R (4, 1, 4), T (5, -4, 1).
Составить словарь с ключами - точками (например, Point3D(3, 2, 1)) и значениями - именами точек (F, H и т.д.).


Найти и вывести на экран


расстояние от точек R и Q до плоскости FHT,


уравнение плоскости FHT,


параметрические уравнения прямой RQ.


Определить и вывести на экран, какие 4 из заданных точек лежат в одной плоскости (пользуясь словарем точек, вывести имена точек).


Определить и вывести на экран, какие 3 из заданных точек лежат на одной прямой (пользуясь словарем точек, вывести имена точек).

\newpage
Вариант N 35

Даны точки: G (5, -2, -4), K (4, -2, 4), M (1, 4, -2), P (-1, -4, 3), T (17, 2, -18).
Составить словарь с ключами - точками (например, Point3D(3, 2, 1)) и значениями - именами точек (G, M и т.д.).


Найти и вывести на экран


расстояние от точек K и T до плоскости GMP,


уравнение плоскости GMP,


параметрические уравнения прямой KT.


Определить и вывести на экран, какие 4 из заданных точек лежат в одной плоскости (пользуясь словарем точек, вывести имена точек).


Определить и вывести на экран, какие 3 из заданных точек лежат на одной прямой (пользуясь словарем точек, вывести имена точек).

\newpage
Вариант N 36

Даны точки: F (4, -1, -4), G (3, 1, 1), J (-1, -2, 5), K (-2, 5, -2), L (-21, -17, 25).
Составить словарь с ключами - точками (например, Point3D(3, 2, 1)) и значениями - именами точек (F, G и т.д.).


Найти и вывести на экран


расстояние от точек K и L до плоскости FGJ,


уравнение плоскости FGJ,


параметрические уравнения прямой KL.


Определить и вывести на экран, какие 4 из заданных точек лежат в одной плоскости (пользуясь словарем точек, вывести имена точек).


Определить и вывести на экран, какие 3 из заданных точек лежат на одной прямой (пользуясь словарем точек, вывести имена точек).

\newpage
Вариант N 37

Даны точки: F (-3, -1, 5), M (5, -2, 3), P (-4, 0, 4), Q (32, -8, 0), S (0, 4, 2).
Составить словарь с ключами - точками (например, Point3D(3, 2, 1)) и значениями - именами точек (F, M и т.д.).


Найти и вывести на экран


расстояние от точек S и Q до плоскости FMP,


уравнение плоскости FMP,


параметрические уравнения прямой SQ.


Определить и вывести на экран, какие 4 из заданных точек лежат в одной плоскости (пользуясь словарем точек, вывести имена точек).


Определить и вывести на экран, какие 3 из заданных точек лежат на одной прямой (пользуясь словарем точек, вывести имена точек).

\newpage
Вариант N 38

Даны точки: H (0, -4, 5), J (-4, -3, 3), L (3, -1, 5), N (-1, 3, 1), Q (-7, -9, 5).
Составить словарь с ключами - точками (например, Point3D(3, 2, 1)) и значениями - именами точек (H, J и т.д.).


Найти и вывести на экран


расстояние от точек L и Q до плоскости HJN,


уравнение плоскости HJN,


параметрические уравнения прямой LQ.


Определить и вывести на экран, какие 4 из заданных точек лежат в одной плоскости (пользуясь словарем точек, вывести имена точек).


Определить и вывести на экран, какие 3 из заданных точек лежат на одной прямой (пользуясь словарем точек, вывести имена точек).

\newpage
Вариант N 39

Даны точки: F (-3, 5, 3), J (5, 2, -2), K (5, 4, 2), Q (5, -4, -14), T (4, -3, -1).
Составить словарь с ключами - точками (например, Point3D(3, 2, 1)) и значениями - именами точек (F, J и т.д.).


Найти и вывести на экран


расстояние от точек T и Q до плоскости FJK,


уравнение плоскости FJK,


параметрические уравнения прямой TQ.


Определить и вывести на экран, какие 4 из заданных точек лежат в одной плоскости (пользуясь словарем точек, вывести имена точек).


Определить и вывести на экран, какие 3 из заданных точек лежат на одной прямой (пользуясь словарем точек, вывести имена точек).

\newpage
Вариант N 40

Даны точки: H (-3, 3, 3), L (-1, 3, 1), P (5, 5, 3), Q (-27, -27, -21), S (-3, -3, -3).
Составить словарь с ключами - точками (например, Point3D(3, 2, 1)) и значениями - именами точек (H, P и т.д.).


Найти и вывести на экран


расстояние от точек L и Q до плоскости HPS,


уравнение плоскости HPS,


параметрические уравнения прямой LQ.


Определить и вывести на экран, какие 4 из заданных точек лежат в одной плоскости (пользуясь словарем точек, вывести имена точек).


Определить и вывести на экран, какие 3 из заданных точек лежат на одной прямой (пользуясь словарем точек, вывести имена точек).

\newpage
Вариант N 41

Даны точки: F (2, 5, -1), G (4, 2, 1), P (4, -1, 4), R (4, -13, 16), S (3, 5, 3).
Составить словарь с ключами - точками (например, Point3D(3, 2, 1)) и значениями - именами точек (F, G и т.д.).


Найти и вывести на экран


расстояние от точек S и R до плоскости FGP,


уравнение плоскости FGP,


параметрические уравнения прямой SR.


Определить и вывести на экран, какие 4 из заданных точек лежат в одной плоскости (пользуясь словарем точек, вывести имена точек).


Определить и вывести на экран, какие 3 из заданных точек лежат на одной прямой (пользуясь словарем точек, вывести имена точек).

\newpage
Вариант N 42

Даны точки: G (5, 4, 5), L (13, 11, 9), M (0, -3, -1), N (-3, -3, 1), S (3, -1, -3).
Составить словарь с ключами - точками (например, Point3D(3, 2, 1)) и значениями - именами точек (G, M и т.д.).


Найти и вывести на экран


расстояние от точек S и L до плоскости GMN,


уравнение плоскости GMN,


параметрические уравнения прямой SL.


Определить и вывести на экран, какие 4 из заданных точек лежат в одной плоскости (пользуясь словарем точек, вывести имена точек).


Определить и вывести на экран, какие 3 из заданных точек лежат на одной прямой (пользуясь словарем точек, вывести имена точек).

\newpage
Вариант N 43

Даны точки: F (4, 3, -3), K (3, -2, 1), L (7, 18, -15), Q (5, -3, -4), R (4, -4, 2).
Составить словарь с ключами - точками (например, Point3D(3, 2, 1)) и значениями - именами точек (F, K и т.д.).


Найти и вывести на экран


расстояние от точек R и L до плоскости FKQ,


уравнение плоскости FKQ,


параметрические уравнения прямой RL.


Определить и вывести на экран, какие 4 из заданных точек лежат в одной плоскости (пользуясь словарем точек, вывести имена точек).


Определить и вывести на экран, какие 3 из заданных точек лежат на одной прямой (пользуясь словарем точек, вывести имена точек).

\newpage
Вариант N 44

Даны точки: F (1, -1, -4), J (-3, -1, 1), Q (9, -1, -14), R (-2, 2, 0), S (-1, -1, 1).
Составить словарь с ключами - точками (например, Point3D(3, 2, 1)) и значениями - именами точек (F, J и т.д.).


Найти и вывести на экран


расстояние от точек R и Q до плоскости FJS,


уравнение плоскости FJS,


параметрические уравнения прямой RQ.


Определить и вывести на экран, какие 4 из заданных точек лежат в одной плоскости (пользуясь словарем точек, вывести имена точек).


Определить и вывести на экран, какие 3 из заданных точек лежат на одной прямой (пользуясь словарем точек, вывести имена точек).

\newpage
Вариант N 45

Даны точки: H (2, -2, -2), K (-3, 0, 5), L (-3, 5, 1), Q (12, -20, -15), S (0, -4, 1).
Составить словарь с ключами - точками (например, Point3D(3, 2, 1)) и значениями - именами точек (H, K и т.д.).


Найти и вывести на экран


расстояние от точек L и Q до плоскости HKS,


уравнение плоскости HKS,


параметрические уравнения прямой LQ.


Определить и вывести на экран, какие 4 из заданных точек лежат в одной плоскости (пользуясь словарем точек, вывести имена точек).


Определить и вывести на экран, какие 3 из заданных точек лежат на одной прямой (пользуясь словарем точек, вывести имена точек).

\newpage
Вариант N 46

Даны точки: F (0, 0, 3), K (-3, 4, 3), N (5, -1, 0), P (1, 2, 5), S (-3, 5, 10).
Составить словарь с ключами - точками (например, Point3D(3, 2, 1)) и значениями - именами точек (F, N и т.д.).


Найти и вывести на экран


расстояние от точек K и S до плоскости FNP,


уравнение плоскости FNP,


параметрические уравнения прямой KS.


Определить и вывести на экран, какие 4 из заданных точек лежат в одной плоскости (пользуясь словарем точек, вывести имена точек).


Определить и вывести на экран, какие 3 из заданных точек лежат на одной прямой (пользуясь словарем точек, вывести имена точек).

\newpage
Вариант N 47

Даны точки: G (-2, 3, 0), J (4, -4, 3), L (-3, 4, 3), M (-4, 3, -2), R (-26, 31, -12).
Составить словарь с ключами - точками (например, Point3D(3, 2, 1)) и значениями - именами точек (G, J и т.д.).


Найти и вывести на экран


расстояние от точек L и R до плоскости GJM,


уравнение плоскости GJM,


параметрические уравнения прямой LR.


Определить и вывести на экран, какие 4 из заданных точек лежат в одной плоскости (пользуясь словарем точек, вывести имена точек).


Определить и вывести на экран, какие 3 из заданных точек лежат на одной прямой (пользуясь словарем точек, вывести имена точек).

\newpage
Вариант N 48

Даны точки: G (5, 4, -4), J (4, -1, -4), K (8, 19, -4), N (0, 0, -4), P (1, 3, 1).
Составить словарь с ключами - точками (например, Point3D(3, 2, 1)) и значениями - именами точек (G, J и т.д.).


Найти и вывести на экран


расстояние от точек P и K до плоскости GJN,


уравнение плоскости GJN,


параметрические уравнения прямой PK.


Определить и вывести на экран, какие 4 из заданных точек лежат в одной плоскости (пользуясь словарем точек, вывести имена точек).


Определить и вывести на экран, какие 3 из заданных точек лежат на одной прямой (пользуясь словарем точек, вывести имена точек).

\newpage
Вариант N 49

Даны точки: H (-2, -1, 2), J (-4, -2, -3), L (3, -1, -3), R (-39, -7, -3), T (-3, -4, 0).
Составить словарь с ключами - точками (например, Point3D(3, 2, 1)) и значениями - именами точек (J, L и т.д.).


Найти и вывести на экран


расстояние от точек H и R до плоскости JLT,


уравнение плоскости JLT,


параметрические уравнения прямой HR.


Определить и вывести на экран, какие 4 из заданных точек лежат в одной плоскости (пользуясь словарем точек, вывести имена точек).


Определить и вывести на экран, какие 3 из заданных точек лежат на одной прямой (пользуясь словарем точек, вывести имена точек).

\newpage
Вариант N 50

Даны точки: H (3, 4, -1), K (2, 3, 3), L (2, 27, 3), R (2, -3, 3), S (3, -3, -4).
Составить словарь с ключами - точками (например, Point3D(3, 2, 1)) и значениями - именами точек (K, R и т.д.).


Найти и вывести на экран


расстояние от точек H и L до плоскости KRS,


уравнение плоскости KRS,


параметрические уравнения прямой HL.


Определить и вывести на экран, какие 4 из заданных точек лежат в одной плоскости (пользуясь словарем точек, вывести имена точек).


Определить и вывести на экран, какие 3 из заданных точек лежат на одной прямой (пользуясь словарем точек, вывести имена точек).

\newpage
Вариант N 51

Даны точки: F (1, -2, 2), H (5, -4, -1), J (0, -2, 1), L (4, -2, 5), S (5, 4, -2).
Составить словарь с ключами - точками (например, Point3D(3, 2, 1)) и значениями - именами точек (F, J и т.д.).


Найти и вывести на экран


расстояние от точек H и L до плоскости FJS,


уравнение плоскости FJS,


параметрические уравнения прямой HL.


Определить и вывести на экран, какие 4 из заданных точек лежат в одной плоскости (пользуясь словарем точек, вывести имена точек).


Определить и вывести на экран, какие 3 из заданных точек лежат на одной прямой (пользуясь словарем точек, вывести имена точек).

\newpage
Вариант N 52

Даны точки: H (-1, 3, -2), N (3, -1, 5), P (0, 0, 2), Q (-6, 2, -4), S (-1, -4, -1).
Составить словарь с ключами - точками (например, Point3D(3, 2, 1)) и значениями - именами точек (N, P и т.д.).


Найти и вывести на экран


расстояние от точек H и Q до плоскости NPS,


уравнение плоскости NPS,


параметрические уравнения прямой HQ.


Определить и вывести на экран, какие 4 из заданных точек лежат в одной плоскости (пользуясь словарем точек, вывести имена точек).


Определить и вывести на экран, какие 3 из заданных точек лежат на одной прямой (пользуясь словарем точек, вывести имена точек).

\newpage
Вариант N 53

Даны точки: F (-2, 0, -2), J (5, 4, 4), M (1, 4, 2), Q (13, 4, 8), R (-3, 5, 3).
Составить словарь с ключами - точками (например, Point3D(3, 2, 1)) и значениями - именами точек (F, J и т.д.).


Найти и вывести на экран


расстояние от точек R и Q до плоскости FJM,


уравнение плоскости FJM,


параметрические уравнения прямой RQ.


Определить и вывести на экран, какие 4 из заданных точек лежат в одной плоскости (пользуясь словарем точек, вывести имена точек).


Определить и вывести на экран, какие 3 из заданных точек лежат на одной прямой (пользуясь словарем точек, вывести имена точек).

\newpage
Вариант N 54

Даны точки: H (-1, 1, 1), J (4, -1, 5), L (-1, -2, 1), P (0, -4, -2), Q (-4, -7, -9).
Составить словарь с ключами - точками (например, Point3D(3, 2, 1)) и значениями - именами точек (H, J и т.д.).


Найти и вывести на экран


расстояние от точек L и Q до плоскости HJP,


уравнение плоскости HJP,


параметрические уравнения прямой LQ.


Определить и вывести на экран, какие 4 из заданных точек лежат в одной плоскости (пользуясь словарем точек, вывести имена точек).


Определить и вывести на экран, какие 3 из заданных точек лежат на одной прямой (пользуясь словарем точек, вывести имена точек).

\newpage
Вариант N 55

Даны точки: H (5, -4, 0), K (2, 2, 5), N (5, -1, 4), P (4, 1, 4), R (7, -5, 4).
Составить словарь с ключами - точками (например, Point3D(3, 2, 1)) и значениями - именами точек (K, N и т.д.).


Найти и вывести на экран


расстояние от точек H и R до плоскости KNP,


уравнение плоскости KNP,


параметрические уравнения прямой HR.


Определить и вывести на экран, какие 4 из заданных точек лежат в одной плоскости (пользуясь словарем точек, вывести имена точек).


Определить и вывести на экран, какие 3 из заданных точек лежат на одной прямой (пользуясь словарем точек, вывести имена точек).

\newpage
Вариант N 56

Даны точки: G (1, 4, 3), H (5, 0, 5), N (0, 0, 0), R (5, 20, 15), T (3, 3, 5).
Составить словарь с ключами - точками (например, Point3D(3, 2, 1)) и значениями - именами точек (G, N и т.д.).


Найти и вывести на экран


расстояние от точек H и R до плоскости GNT,


уравнение плоскости GNT,


параметрические уравнения прямой HR.


Определить и вывести на экран, какие 4 из заданных точек лежат в одной плоскости (пользуясь словарем точек, вывести имена точек).


Определить и вывести на экран, какие 3 из заданных точек лежат на одной прямой (пользуясь словарем точек, вывести имена точек).

\newpage
Вариант N 57

Даны точки: F (-1, 3, 2), G (4, -3, -1), P (2, 3, 5), S (12, -27, -25), T (3, 0, -1).
Составить словарь с ключами - точками (например, Point3D(3, 2, 1)) и значениями - именами точек (G, P и т.д.).


Найти и вывести на экран


расстояние от точек F и S до плоскости GPT,


уравнение плоскости GPT,


параметрические уравнения прямой FS.


Определить и вывести на экран, какие 4 из заданных точек лежат в одной плоскости (пользуясь словарем точек, вывести имена точек).


Определить и вывести на экран, какие 3 из заданных точек лежат на одной прямой (пользуясь словарем точек, вывести имена точек).

\newpage
Вариант N 58

Даны точки: H (5, -2, -3), K (-4, 1, -3), L (5, 17, -12), P (1, 3, -2), T (-1, -4, 3).
Составить словарь с ключами - точками (например, Point3D(3, 2, 1)) и значениями - именами точек (K, P и т.д.).


Найти и вывести на экран


расстояние от точек H и L до плоскости KPT,


уравнение плоскости KPT,


параметрические уравнения прямой HL.


Определить и вывести на экран, какие 4 из заданных точек лежат в одной плоскости (пользуясь словарем точек, вывести имена точек).


Определить и вывести на экран, какие 3 из заданных точек лежат на одной прямой (пользуясь словарем точек, вывести имена точек).

\newpage
Вариант N 59

Даны точки: F (-4, -3, 3), H (-12, -9, 6), J (4, 3, 0), K (-1, 5, -1), S (4, -4, -3).
Составить словарь с ключами - точками (например, Point3D(3, 2, 1)) и значениями - именами точек (F, J и т.д.).


Найти и вывести на экран


расстояние от точек S и H до плоскости FJK,


уравнение плоскости FJK,


параметрические уравнения прямой SH.


Определить и вывести на экран, какие 4 из заданных точек лежат в одной плоскости (пользуясь словарем точек, вывести имена точек).


Определить и вывести на экран, какие 3 из заданных точек лежат на одной прямой (пользуясь словарем точек, вывести имена точек).

\newpage
Вариант N 60

Даны точки: F (3, -2, 1), G (2, 4, -2), K (1, -4, 4), M (-3, -4, -2), R (12, 20, -2).
Составить словарь с ключами - точками (например, Point3D(3, 2, 1)) и значениями - именами точек (F, G и т.д.).


Найти и вывести на экран


расстояние от точек K и R до плоскости FGM,


уравнение плоскости FGM,


параметрические уравнения прямой KR.


Определить и вывести на экран, какие 4 из заданных точек лежат в одной плоскости (пользуясь словарем точек, вывести имена точек).


Определить и вывести на экран, какие 3 из заданных точек лежат на одной прямой (пользуясь словарем точек, вывести имена точек).

\newpage
Вариант N 61

Даны точки: F (-2, 5, 1), H (3, -2, 5), P (-4, 0, 0), Q (28, 12, -16), S (4, 3, -4).
Составить словарь с ключами - точками (например, Point3D(3, 2, 1)) и значениями - именами точек (F, P и т.д.).


Найти и вывести на экран


расстояние от точек H и Q до плоскости FPS,


уравнение плоскости FPS,


параметрические уравнения прямой HQ.


Определить и вывести на экран, какие 4 из заданных точек лежат в одной плоскости (пользуясь словарем точек, вывести имена точек).


Определить и вывести на экран, какие 3 из заданных точек лежат на одной прямой (пользуясь словарем точек, вывести имена точек).

\newpage
Вариант N 62

Даны точки: F (-4, 2, -3), K (-4, -1, 0), M (-3, 3, 5), Q (-1, 11, 15), S (-2, -3, 4).
Составить словарь с ключами - точками (например, Point3D(3, 2, 1)) и значениями - именами точек (F, K и т.д.).


Найти и вывести на экран


расстояние от точек S и Q до плоскости FKM,


уравнение плоскости FKM,


параметрические уравнения прямой SQ.


Определить и вывести на экран, какие 4 из заданных точек лежат в одной плоскости (пользуясь словарем точек, вывести имена точек).


Определить и вывести на экран, какие 3 из заданных точек лежат на одной прямой (пользуясь словарем точек, вывести имена точек).

\newpage
Вариант N 63

Даны точки: F (3, 0, -2), H (-4, -3, -1), M (2, 1, -3), N (0, -2, -2), Q (6, 7, -5).
Составить словарь с ключами - точками (например, Point3D(3, 2, 1)) и значениями - именами точек (F, M и т.д.).


Найти и вывести на экран


расстояние от точек H и Q до плоскости FMN,


уравнение плоскости FMN,


параметрические уравнения прямой HQ.


Определить и вывести на экран, какие 4 из заданных точек лежат в одной плоскости (пользуясь словарем точек, вывести имена точек).


Определить и вывести на экран, какие 3 из заданных точек лежат на одной прямой (пользуясь словарем точек, вывести имена точек).

\newpage
Вариант N 64

Даны точки: F (2, -2, 0), M (2, 5, -2), N (0, 1, 3), R (-8, -15, 23), T (0, 0, 5).
Составить словарь с ключами - точками (например, Point3D(3, 2, 1)) и значениями - именами точек (F, M и т.д.).


Найти и вывести на экран


расстояние от точек T и R до плоскости FMN,


уравнение плоскости FMN,


параметрические уравнения прямой TR.


Определить и вывести на экран, какие 4 из заданных точек лежат в одной плоскости (пользуясь словарем точек, вывести имена точек).


Определить и вывести на экран, какие 3 из заданных точек лежат на одной прямой (пользуясь словарем точек, вывести имена точек).

\newpage
Вариант N 65

Даны точки: J (4, 3, 1), K (-4, 3, -1), M (2, -4, -1), P (1, -4, 4), R (-11, -32, 16).
Составить словарь с ключами - точками (например, Point3D(3, 2, 1)) и значениями - именами точек (J, M и т.д.).


Найти и вывести на экран


расстояние от точек K и R до плоскости JMP,


уравнение плоскости JMP,


параметрические уравнения прямой KR.


Определить и вывести на экран, какие 4 из заданных точек лежат в одной плоскости (пользуясь словарем точек, вывести имена точек).


Определить и вывести на экран, какие 3 из заданных точек лежат на одной прямой (пользуясь словарем точек, вывести имена точек).

\newpage
Вариант N 66

Даны точки: F (1, 0, -4), K (-3, 0, 5), L (9, -3, 28), M (0, 3, 4), N (-3, 5, -4).
Составить словарь с ключами - точками (например, Point3D(3, 2, 1)) и значениями - именами точек (F, M и т.д.).


Найти и вывести на экран


расстояние от точек K и L до плоскости FMN,


уравнение плоскости FMN,


параметрические уравнения прямой KL.


Определить и вывести на экран, какие 4 из заданных точек лежат в одной плоскости (пользуясь словарем точек, вывести имена точек).


Определить и вывести на экран, какие 3 из заданных точек лежат на одной прямой (пользуясь словарем точек, вывести имена точек).

\newpage
Вариант N 67

Даны точки: G (0, 1, -4), N (1, -1, 3), P (-3, 4, -1), S (-2, 5, -18), T (0, 2, -4).
Составить словарь с ключами - точками (например, Point3D(3, 2, 1)) и значениями - именами точек (G, N и т.д.).


Найти и вывести на экран


расстояние от точек T и S до плоскости GNP,


уравнение плоскости GNP,


параметрические уравнения прямой TS.


Определить и вывести на экран, какие 4 из заданных точек лежат в одной плоскости (пользуясь словарем точек, вывести имена точек).


Определить и вывести на экран, какие 3 из заданных точек лежат на одной прямой (пользуясь словарем точек, вывести имена точек).

\newpage
Вариант N 68

Даны точки: F (-3, -4, 5), H (3, -1, -2), K (2, -2, -2), L (-28, 4, 22), P (-3, -1, 2).
Составить словарь с ключами - точками (например, Point3D(3, 2, 1)) и значениями - именами точек (F, K и т.д.).


Найти и вывести на экран


расстояние от точек H и L до плоскости FKP,


уравнение плоскости FKP,


параметрические уравнения прямой HL.


Определить и вывести на экран, какие 4 из заданных точек лежат в одной плоскости (пользуясь словарем точек, вывести имена точек).


Определить и вывести на экран, какие 3 из заданных точек лежат на одной прямой (пользуясь словарем точек, вывести имена точек).

\newpage
Вариант N 69

Даны точки: H (3, -3, -4), J (5, -2, 0), L (2, 5, -4), R (25, -14, 0), S (0, 1, 0).
Составить словарь с ключами - точками (например, Point3D(3, 2, 1)) и значениями - именами точек (J, L и т.д.).


Найти и вывести на экран


расстояние от точек H и R до плоскости JLS,


уравнение плоскости JLS,


параметрические уравнения прямой HR.


Определить и вывести на экран, какие 4 из заданных точек лежат в одной плоскости (пользуясь словарем точек, вывести имена точек).


Определить и вывести на экран, какие 3 из заданных точек лежат на одной прямой (пользуясь словарем точек, вывести имена точек).

\newpage
Вариант N 70

Даны точки: F (0, -2, -1), K (-1, 0, 2), L (4, -10, -13), R (4, 10, 5), T (-3, 1, 5).
Составить словарь с ключами - точками (например, Point3D(3, 2, 1)) и значениями - именами точек (F, K и т.д.).


Найти и вывести на экран


расстояние от точек R и L до плоскости FKT,


уравнение плоскости FKT,


параметрические уравнения прямой RL.


Определить и вывести на экран, какие 4 из заданных точек лежат в одной плоскости (пользуясь словарем точек, вывести имена точек).


Определить и вывести на экран, какие 3 из заданных точек лежат на одной прямой (пользуясь словарем точек, вывести имена точек).

\newpage
Вариант N 71

Даны точки: H (-1, 5, 2), K (1, -4, -3), L (0, -10, -5), M (2, 2, -1), S (4, 5, -1).
Составить словарь с ключами - точками (например, Point3D(3, 2, 1)) и значениями - именами точек (K, M и т.д.).


Найти и вывести на экран


расстояние от точек H и L до плоскости KMS,


уравнение плоскости KMS,


параметрические уравнения прямой HL.


Определить и вывести на экран, какие 4 из заданных точек лежат в одной плоскости (пользуясь словарем точек, вывести имена точек).


Определить и вывести на экран, какие 3 из заданных точек лежат на одной прямой (пользуясь словарем точек, вывести имена точек).

\newpage
Вариант N 72

Даны точки: G (-2, 1, 5), J (-4, 1, -4), M (1, 3, 4), N (4, 1, 5), T (4, 1, 32).
Составить словарь с ключами - точками (например, Point3D(3, 2, 1)) и значениями - именами точек (G, J и т.д.).


Найти и вывести на экран


расстояние от точек M и T до плоскости GJN,


уравнение плоскости GJN,


параметрические уравнения прямой MT.


Определить и вывести на экран, какие 4 из заданных точек лежат в одной плоскости (пользуясь словарем точек, вывести имена точек).


Определить и вывести на экран, какие 3 из заданных точек лежат на одной прямой (пользуясь словарем точек, вывести имена точек).

\newpage
Вариант N 73

Даны точки: F (5, -1, 4), H (8, 24, -23), K (-2, -1, 2), M (0, 4, -3), T (4, 2, -1).
Составить словарь с ключами - точками (например, Point3D(3, 2, 1)) и значениями - именами точек (F, K и т.д.).


Найти и вывести на экран


расстояние от точек T и H до плоскости FKM,


уравнение плоскости FKM,


параметрические уравнения прямой TH.


Определить и вывести на экран, какие 4 из заданных точек лежат в одной плоскости (пользуясь словарем точек, вывести имена точек).


Определить и вывести на экран, какие 3 из заданных точек лежат на одной прямой (пользуясь словарем точек, вывести имена точек).

\newpage
Вариант N 74

Даны точки: H (-1, 1, -3), K (2, 1, -1), L (8, 7, 11), M (4, 3, 3), T (-4, 0, 2).
Составить словарь с ключами - точками (например, Point3D(3, 2, 1)) и значениями - именами точек (K, M и т.д.).


Найти и вывести на экран


расстояние от точек H и L до плоскости KMT,


уравнение плоскости KMT,


параметрические уравнения прямой HL.


Определить и вывести на экран, какие 4 из заданных точек лежат в одной плоскости (пользуясь словарем точек, вывести имена точек).


Определить и вывести на экран, какие 3 из заданных точек лежат на одной прямой (пользуясь словарем точек, вывести имена точек).

\newpage
Вариант N 75

Даны точки: G (5, -1, -1), H (5, -6, -1), J (-3, 2, 1), N (1, -2, 0), T (5, -1, 2).
Составить словарь с ключами - точками (например, Point3D(3, 2, 1)) и значениями - именами точек (G, J и т.д.).


Найти и вывести на экран


расстояние от точек T и H до плоскости GJN,


уравнение плоскости GJN,


параметрические уравнения прямой TH.


Определить и вывести на экран, какие 4 из заданных точек лежат в одной плоскости (пользуясь словарем точек, вывести имена точек).


Определить и вывести на экран, какие 3 из заданных точек лежат на одной прямой (пользуясь словарем точек, вывести имена точек).

\newpage
Вариант N 76

Даны точки: F (-4, -1, 5), H (4, -2, -1), M (-3, -4, -3), P (-4, 5, 1), R (-1, -22, -11).
Составить словарь с ключами - точками (например, Point3D(3, 2, 1)) и значениями - именами точек (F, M и т.д.).


Найти и вывести на экран


расстояние от точек H и R до плоскости FMP,


уравнение плоскости FMP,


параметрические уравнения прямой HR.


Определить и вывести на экран, какие 4 из заданных точек лежат в одной плоскости (пользуясь словарем точек, вывести имена точек).


Определить и вывести на экран, какие 3 из заданных точек лежат на одной прямой (пользуясь словарем точек, вывести имена точек).

\newpage
Вариант N 77

Даны точки: H (1, 0, 1), J (-2, 2, -3), L (-27, 12, -13), P (3, 0, -1), T (-2, -4, 1).
Составить словарь с ключами - точками (например, Point3D(3, 2, 1)) и значениями - именами точек (J, P и т.д.).


Найти и вывести на экран


расстояние от точек H и L до плоскости JPT,


уравнение плоскости JPT,


параметрические уравнения прямой HL.


Определить и вывести на экран, какие 4 из заданных точек лежат в одной плоскости (пользуясь словарем точек, вывести имена точек).


Определить и вывести на экран, какие 3 из заданных точек лежат на одной прямой (пользуясь словарем точек, вывести имена точек).

\newpage
Вариант N 78

Даны точки: F (4, -4, -3), H (-3, 0, 1), Q (4, 50, 21), S (5, -4, -4), T (4, 5, 1).
Составить словарь с ключами - точками (например, Point3D(3, 2, 1)) и значениями - именами точек (F, S и т.д.).


Найти и вывести на экран


расстояние от точек H и Q до плоскости FST,


уравнение плоскости FST,


параметрические уравнения прямой HQ.


Определить и вывести на экран, какие 4 из заданных точек лежат в одной плоскости (пользуясь словарем точек, вывести имена точек).


Определить и вывести на экран, какие 3 из заданных точек лежат на одной прямой (пользуясь словарем точек, вывести имена точек).

\newpage
Вариант N 79

Даны точки: J (-2, -2, -2), K (-2, -3, 3), P (5, -3, 1), Q (-23, 1, -11), T (-1, 2, -2).
Составить словарь с ключами - точками (например, Point3D(3, 2, 1)) и значениями - именами точек (J, K и т.д.).


Найти и вывести на экран


расстояние от точек T и Q до плоскости JKP,


уравнение плоскости JKP,


параметрические уравнения прямой TQ.


Определить и вывести на экран, какие 4 из заданных точек лежат в одной плоскости (пользуясь словарем точек, вывести имена точек).


Определить и вывести на экран, какие 3 из заданных точек лежат на одной прямой (пользуясь словарем точек, вывести имена точек).

\newpage
Вариант N 80

Даны точки: J (2, 0, 1), P (4, -4, 4), R (-2, 8, -5), S (-3, 0, 0), T (-4, 5, -4).
Составить словарь с ключами - точками (например, Point3D(3, 2, 1)) и значениями - именами точек (J, P и т.д.).


Найти и вывести на экран


расстояние от точек S и R до плоскости JPT,


уравнение плоскости JPT,


параметрические уравнения прямой SR.


Определить и вывести на экран, какие 4 из заданных точек лежат в одной плоскости (пользуясь словарем точек, вывести имена точек).


Определить и вывести на экран, какие 3 из заданных точек лежат на одной прямой (пользуясь словарем точек, вывести имена точек).

\newpage
Вариант N 81

Даны точки: H (5, 0, 0), K (5, 4, -2), L (-1, 4, 5), R (35, 4, -37), T (-3, 0, 5).
Составить словарь с ключами - точками (например, Point3D(3, 2, 1)) и значениями - именами точек (K, L и т.д.).


Найти и вывести на экран


расстояние от точек H и R до плоскости KLT,


уравнение плоскости KLT,


параметрические уравнения прямой HR.


Определить и вывести на экран, какие 4 из заданных точек лежат в одной плоскости (пользуясь словарем точек, вывести имена точек).


Определить и вывести на экран, какие 3 из заданных точек лежат на одной прямой (пользуясь словарем точек, вывести имена точек).

\newpage
Вариант N 82

Даны точки: F (-4, -3, 5), J (-3, -4, -3), N (0, -3, -1), Q (12, 1, 7), R (-4, 1, -4).
Составить словарь с ключами - точками (например, Point3D(3, 2, 1)) и значениями - именами точек (F, J и т.д.).


Найти и вывести на экран


расстояние от точек R и Q до плоскости FJN,


уравнение плоскости FJN,


параметрические уравнения прямой RQ.


Определить и вывести на экран, какие 4 из заданных точек лежат в одной плоскости (пользуясь словарем точек, вывести имена точек).


Определить и вывести на экран, какие 3 из заданных точек лежат на одной прямой (пользуясь словарем точек, вывести имена точек).

\newpage
Вариант N 83

Даны точки: F (-2, 4, -4), G (2, -3, 1), J (2, 5, 1), R (2, 45, 1), S (-4, 0, 3).
Составить словарь с ключами - точками (например, Point3D(3, 2, 1)) и значениями - именами точек (F, G и т.д.).


Найти и вывести на экран


расстояние от точек S и R до плоскости FGJ,


уравнение плоскости FGJ,


параметрические уравнения прямой SR.


Определить и вывести на экран, какие 4 из заданных точек лежат в одной плоскости (пользуясь словарем точек, вывести имена точек).


Определить и вывести на экран, какие 3 из заданных точек лежат на одной прямой (пользуясь словарем точек, вывести имена точек).

\newpage
Вариант N 84

Даны точки: H (9, -16, -12), J (0, -4, 0), P (-3, 0, 4), S (-2, -3, 5), T (1, 1, 5).
Составить словарь с ключами - точками (например, Point3D(3, 2, 1)) и значениями - именами точек (J, P и т.д.).


Найти и вывести на экран


расстояние от точек S и H до плоскости JPT,


уравнение плоскости JPT,


параметрические уравнения прямой SH.


Определить и вывести на экран, какие 4 из заданных точек лежат в одной плоскости (пользуясь словарем точек, вывести имена точек).


Определить и вывести на экран, какие 3 из заданных точек лежат на одной прямой (пользуясь словарем точек, вывести имена точек).

\newpage
Вариант N 85

Даны точки: F (1, -1, 0), G (5, 5, -2), J (-1, 4, 1), K (-16, 25, -5), P (4, -3, 3).
Составить словарь с ключами - точками (например, Point3D(3, 2, 1)) и значениями - именами точек (G, J и т.д.).


Найти и вывести на экран


расстояние от точек F и K до плоскости GJP,


уравнение плоскости GJP,


параметрические уравнения прямой FK.


Определить и вывести на экран, какие 4 из заданных точек лежат в одной плоскости (пользуясь словарем точек, вывести имена точек).


Определить и вывести на экран, какие 3 из заданных точек лежат на одной прямой (пользуясь словарем точек, вывести имена точек).

\newpage
Вариант N 86

Даны точки: J (0, 3, 0), L (0, 5, 0), P (-4, 2, -2), Q (16, 7, 8), S (5, -2, -2).
Составить словарь с ключами - точками (например, Point3D(3, 2, 1)) и значениями - именами точек (J, P и т.д.).


Найти и вывести на экран


расстояние от точек L и Q до плоскости JPS,


уравнение плоскости JPS,


параметрические уравнения прямой LQ.


Определить и вывести на экран, какие 4 из заданных точек лежат в одной плоскости (пользуясь словарем точек, вывести имена точек).


Определить и вывести на экран, какие 3 из заданных точек лежат на одной прямой (пользуясь словарем точек, вывести имена точек).

\newpage
Вариант N 87

Даны точки: F (3, -1, 0), G (-2, -1, 2), J (5, 5, 1), M (2, 2, 1), Q (-9, -7, 3).
Составить словарь с ключами - точками (например, Point3D(3, 2, 1)) и значениями - именами точек (G, J и т.д.).


Найти и вывести на экран


расстояние от точек F и Q до плоскости GJM,


уравнение плоскости GJM,


параметрические уравнения прямой FQ.


Определить и вывести на экран, какие 4 из заданных точек лежат в одной плоскости (пользуясь словарем точек, вывести имена точек).


Определить и вывести на экран, какие 3 из заданных точек лежат на одной прямой (пользуясь словарем точек, вывести имена точек).

\newpage
Вариант N 88

Даны точки: F (4, 5, -2), L (-33, -32, -19), N (-3, -2, 1), P (3, 4, 5), T (-2, 1, -4).
Составить словарь с ключами - точками (например, Point3D(3, 2, 1)) и значениями - именами точек (F, N и т.д.).


Найти и вывести на экран


расстояние от точек T и L до плоскости FNP,


уравнение плоскости FNP,


параметрические уравнения прямой TL.


Определить и вывести на экран, какие 4 из заданных точек лежат в одной плоскости (пользуясь словарем точек, вывести имена точек).


Определить и вывести на экран, какие 3 из заданных точек лежат на одной прямой (пользуясь словарем точек, вывести имена точек).

\newpage
Вариант N 89

Даны точки: H (5, -3, -7), M (1, -3, -1), P (-3, -3, 5), S (2, -2, 4), T (0, -2, -4).
Составить словарь с ключами - точками (например, Point3D(3, 2, 1)) и значениями - именами точек (M, P и т.д.).


Найти и вывести на экран


расстояние от точек T и H до плоскости MPS,


уравнение плоскости MPS,


параметрические уравнения прямой TH.


Определить и вывести на экран, какие 4 из заданных точек лежат в одной плоскости (пользуясь словарем точек, вывести имена точек).


Определить и вывести на экран, какие 3 из заданных точек лежат на одной прямой (пользуясь словарем точек, вывести имена точек).

\newpage
Вариант N 90

Даны точки: G (-3, -1, 0), L (3, -2, 0), N (-1, 1, 2), P (3, -2, 1), Q (9, 11, 12).
Составить словарь с ключами - точками (например, Point3D(3, 2, 1)) и значениями - именами точек (G, N и т.д.).


Найти и вывести на экран


расстояние от точек L и Q до плоскости GNP,


уравнение плоскости GNP,


параметрические уравнения прямой LQ.


Определить и вывести на экран, какие 4 из заданных точек лежат в одной плоскости (пользуясь словарем точек, вывести имена точек).


Определить и вывести на экран, какие 3 из заданных точек лежат на одной прямой (пользуясь словарем точек, вывести имена точек).

\newpage
Вариант N 91

Даны точки: L (0, 2, -4), N (4, -3, -4), Q (8, -1, -12), R (2, -4, 0), S (0, 5, 2).
Составить словарь с ключами - точками (например, Point3D(3, 2, 1)) и значениями - именами точек (N, R и т.д.).


Найти и вывести на экран


расстояние от точек L и Q до плоскости NRS,


уравнение плоскости NRS,


параметрические уравнения прямой LQ.


Определить и вывести на экран, какие 4 из заданных точек лежат в одной плоскости (пользуясь словарем точек, вывести имена точек).


Определить и вывести на экран, какие 3 из заданных точек лежат на одной прямой (пользуясь словарем точек, вывести имена точек).

\newpage
Вариант N 92

Даны точки: H (38, 34, -25), J (4, 5, -3), M (-2, -4, -3), N (3, 4, 0), P (-4, -2, 5).
Составить словарь с ключами - точками (например, Point3D(3, 2, 1)) и значениями - именами точек (J, N и т.д.).


Найти и вывести на экран


расстояние от точек M и H до плоскости JNP,


уравнение плоскости JNP,


параметрические уравнения прямой MH.


Определить и вывести на экран, какие 4 из заданных точек лежат в одной плоскости (пользуясь словарем точек, вывести имена точек).


Определить и вывести на экран, какие 3 из заданных точек лежат на одной прямой (пользуясь словарем точек, вывести имена точек).

\newpage
Вариант N 93

Даны точки: H (-1, 4, -1), P (-2, -4, -4), Q (-12, -12, -12), S (3, 1, 5), T (3, 0, 0).
Составить словарь с ключами - точками (например, Point3D(3, 2, 1)) и значениями - именами точек (P, S и т.д.).


Найти и вывести на экран


расстояние от точек H и Q до плоскости PST,


уравнение плоскости PST,


параметрические уравнения прямой HQ.


Определить и вывести на экран, какие 4 из заданных точек лежат в одной плоскости (пользуясь словарем точек, вывести имена точек).


Определить и вывести на экран, какие 3 из заданных точек лежат на одной прямой (пользуясь словарем точек, вывести имена точек).

\newpage
Вариант N 94

Даны точки: G (1, 0, 4), K (0, 5, 0), L (0, 4, 1), Q (-2, 15, -8), R (5, 2, 2).
Составить словарь с ключами - точками (например, Point3D(3, 2, 1)) и значениями - именами точек (G, K и т.д.).


Найти и вывести на экран


расстояние от точек R и Q до плоскости GKL,


уравнение плоскости GKL,


параметрические уравнения прямой RQ.


Определить и вывести на экран, какие 4 из заданных точек лежат в одной плоскости (пользуясь словарем точек, вывести имена точек).


Определить и вывести на экран, какие 3 из заданных точек лежат на одной прямой (пользуясь словарем точек, вывести имена точек).

\newpage
Вариант N 95

Даны точки: G (5, -4, -4), Q (-10, 8, 8), R (-1, 5, -1), S (0, 0, 0), T (4, 0, 1).
Составить словарь с ключами - точками (например, Point3D(3, 2, 1)) и значениями - именами точек (G, S и т.д.).


Найти и вывести на экран


расстояние от точек R и Q до плоскости GST,


уравнение плоскости GST,


параметрические уравнения прямой RQ.


Определить и вывести на экран, какие 4 из заданных точек лежат в одной плоскости (пользуясь словарем точек, вывести имена точек).


Определить и вывести на экран, какие 3 из заданных точек лежат на одной прямой (пользуясь словарем точек, вывести имена точек).

\newpage
Вариант N 96

Даны точки: F (3, -3, -2), G (-1, 0, 2), H (3, 0, -18), M (0, 0, -3), S (1, -3, 3).
Составить словарь с ключами - точками (например, Point3D(3, 2, 1)) и значениями - именами точек (F, G и т.д.).


Найти и вывести на экран


расстояние от точек S и H до плоскости FGM,


уравнение плоскости FGM,


параметрические уравнения прямой SH.


Определить и вывести на экран, какие 4 из заданных точек лежат в одной плоскости (пользуясь словарем точек, вывести имена точек).


Определить и вывести на экран, какие 3 из заданных точек лежат на одной прямой (пользуясь словарем точек, вывести имена точек).

\newpage
Вариант N 97

Даны точки: F (-7, 14, 28), G (-3, 2, 4), J (-2, -1, -2), N (-3, -1, 4), P (1, 3, 5).
Составить словарь с ключами - точками (например, Point3D(3, 2, 1)) и значениями - именами точек (G, J и т.д.).


Найти и вывести на экран


расстояние от точек N и F до плоскости GJP,


уравнение плоскости GJP,


параметрические уравнения прямой NF.


Определить и вывести на экран, какие 4 из заданных точек лежат в одной плоскости (пользуясь словарем точек, вывести имена точек).


Определить и вывести на экран, какие 3 из заданных точек лежат на одной прямой (пользуясь словарем точек, вывести имена точек).

\newpage
Вариант N 98

Даны точки: F (3, -4, -1), H (-2, -3, -3), J (5, 3, 3), Q (13, 11, 15), T (1, -1, -3).
Составить словарь с ключами - точками (например, Point3D(3, 2, 1)) и значениями - именами точек (F, J и т.д.).


Найти и вывести на экран


расстояние от точек H и Q до плоскости FJT,


уравнение плоскости FJT,


параметрические уравнения прямой HQ.


Определить и вывести на экран, какие 4 из заданных точек лежат в одной плоскости (пользуясь словарем точек, вывести имена точек).


Определить и вывести на экран, какие 3 из заданных точек лежат на одной прямой (пользуясь словарем точек, вывести имена точек).

\newpage
Вариант N 99

Даны точки: F (-4, 2, -4), H (5, -2, -1), K (-2, -2, -2), Q (-12, 18, -12), R (-3, -1, 2).
Составить словарь с ключами - точками (например, Point3D(3, 2, 1)) и значениями - именами точек (F, H и т.д.).


Найти и вывести на экран


расстояние от точек R и Q до плоскости FHK,


уравнение плоскости FHK,


параметрические уравнения прямой RQ.


Определить и вывести на экран, какие 4 из заданных точек лежат в одной плоскости (пользуясь словарем точек, вывести имена точек).


Определить и вывести на экран, какие 3 из заданных точек лежат на одной прямой (пользуясь словарем точек, вывести имена точек).

\newpage
Вариант N 100

Даны точки: H (-3, 4, -2), J (2, 3, 2), M (2, -1, -2), N (1, -3, -3), R (5, 21, 17).
Составить словарь с ключами - точками (например, Point3D(3, 2, 1)) и значениями - именами точек (J, M и т.д.).


Найти и вывести на экран


расстояние от точек H и R до плоскости JMN,


уравнение плоскости JMN,


параметрические уравнения прямой HR.


Определить и вывести на экран, какие 4 из заданных точек лежат в одной плоскости (пользуясь словарем точек, вывести имена точек).


Определить и вывести на экран, какие 3 из заданных точек лежат на одной прямой (пользуясь словарем точек, вывести имена точек).

\newpage
Вариант N 101

Даны точки: H (-5, 6, 10), K (-1, 0, 1), M (0, 3, 3), P (5, 0, -4), S (-3, 0, 4).
Составить словарь с ключами - точками (например, Point3D(3, 2, 1)) и значениями - именами точек (K, M и т.д.).


Найти и вывести на экран


расстояние от точек S и H до плоскости KMP,


уравнение плоскости KMP,


параметрические уравнения прямой SH.


Определить и вывести на экран, какие 4 из заданных точек лежат в одной плоскости (пользуясь словарем точек, вывести имена точек).


Определить и вывести на экран, какие 3 из заданных точек лежат на одной прямой (пользуясь словарем точек, вывести имена точек).

\newpage
Вариант N 102

Даны точки: H (4, 4, 5), N (-4, 3, 5), P (-4, 2, 5), Q (-4, -1, 5), S (-3, 0, 0).
Составить словарь с ключами - точками (например, Point3D(3, 2, 1)) и значениями - именами точек (N, P и т.д.).


Найти и вывести на экран


расстояние от точек H и Q до плоскости NPS,


уравнение плоскости NPS,


параметрические уравнения прямой HQ.


Определить и вывести на экран, какие 4 из заданных точек лежат в одной плоскости (пользуясь словарем точек, вывести имена точек).


Определить и вывести на экран, какие 3 из заданных точек лежат на одной прямой (пользуясь словарем точек, вывести имена точек).

\newpage
Вариант N 103

Даны точки: H (-4, 1, -3), K (1, -4, -4), L (-17, 20, 5), S (4, -4, -4), T (-3, 4, -1).
Составить словарь с ключами - точками (например, Point3D(3, 2, 1)) и значениями - именами точек (K, S и т.д.).


Найти и вывести на экран


расстояние от точек H и L до плоскости KST,


уравнение плоскости KST,


параметрические уравнения прямой HL.


Определить и вывести на экран, какие 4 из заданных точек лежат в одной плоскости (пользуясь словарем точек, вывести имена точек).


Определить и вывести на экран, какие 3 из заданных точек лежат на одной прямой (пользуясь словарем точек, вывести имена точек).

\newpage
Вариант N 104

Даны точки: F (1, -1, 3), L (-4, 3, 1), P (0, 4, 2), R (-2, 14, 0), T (5, 5, 2).
Составить словарь с ключами - точками (например, Point3D(3, 2, 1)) и значениями - именами точек (F, L и т.д.).


Найти и вывести на экран


расстояние от точек T и R до плоскости FLP,


уравнение плоскости FLP,


параметрические уравнения прямой TR.


Определить и вывести на экран, какие 4 из заданных точек лежат в одной плоскости (пользуясь словарем точек, вывести имена точек).


Определить и вывести на экран, какие 3 из заданных точек лежат на одной прямой (пользуясь словарем точек, вывести имена точек).

\newpage
Вариант N 105

Даны точки: K (-3, -2, -3), L (-4, 3, 1), N (5, 3, -1), Q (-19, -12, -7), R (-1, 5, -2).
Составить словарь с ключами - точками (например, Point3D(3, 2, 1)) и значениями - именами точек (K, L и т.д.).


Найти и вывести на экран


расстояние от точек R и Q до плоскости KLN,


уравнение плоскости KLN,


параметрические уравнения прямой RQ.


Определить и вывести на экран, какие 4 из заданных точек лежат в одной плоскости (пользуясь словарем точек, вывести имена точек).


Определить и вывести на экран, какие 3 из заданных точек лежат на одной прямой (пользуясь словарем точек, вывести имена точек).

\newpage
Вариант N 106

Даны точки: F (18, -9, 19), G (5, 1, -2), M (0, 4, 1), N (-2, 3, -1), P (3, 0, 4).
Составить словарь с ключами - точками (например, Point3D(3, 2, 1)) и значениями - именами точек (G, N и т.д.).


Найти и вывести на экран


расстояние от точек M и F до плоскости GNP,


уравнение плоскости GNP,


параметрические уравнения прямой MF.


Определить и вывести на экран, какие 4 из заданных точек лежат в одной плоскости (пользуясь словарем точек, вывести имена точек).


Определить и вывести на экран, какие 3 из заданных точек лежат на одной прямой (пользуясь словарем точек, вывести имена точек).

\newpage
Вариант N 107

Даны точки: G (3, 2, -2), L (1, -4, -1), Q (9, 20, -5), R (4, 2, 5), T (4, 0, -4).
Составить словарь с ключами - точками (например, Point3D(3, 2, 1)) и значениями - именами точек (G, L и т.д.).


Найти и вывести на экран


расстояние от точек R и Q до плоскости GLT,


уравнение плоскости GLT,


параметрические уравнения прямой RQ.


Определить и вывести на экран, какие 4 из заданных точек лежат в одной плоскости (пользуясь словарем точек, вывести имена точек).


Определить и вывести на экран, какие 3 из заданных точек лежат на одной прямой (пользуясь словарем точек, вывести имена точек).

\newpage
Вариант N 108

Даны точки: G (-4, -2, -1), K (4, -2, -4), P (1, 2, 4), Q (-24, -18, -21), T (3, -3, -1).
Составить словарь с ключами - точками (например, Point3D(3, 2, 1)) и значениями - именами точек (G, K и т.д.).


Найти и вывести на экран


расстояние от точек T и Q до плоскости GKP,


уравнение плоскости GKP,


параметрические уравнения прямой TQ.


Определить и вывести на экран, какие 4 из заданных точек лежат в одной плоскости (пользуясь словарем точек, вывести имена точек).


Определить и вывести на экран, какие 3 из заданных точек лежат на одной прямой (пользуясь словарем точек, вывести имена точек).

\newpage
Вариант N 109

Даны точки: G (-1, -4, 5), J (-1, 1, -1), L (1, -3, -4), P (-1, -4, -1), R (-1, -19, 23).
Составить словарь с ключами - точками (например, Point3D(3, 2, 1)) и значениями - именами точек (G, J и т.д.).


Найти и вывести на экран


расстояние от точек L и R до плоскости GJP,


уравнение плоскости GJP,


параметрические уравнения прямой LR.


Определить и вывести на экран, какие 4 из заданных точек лежат в одной плоскости (пользуясь словарем точек, вывести имена точек).


Определить и вывести на экран, какие 3 из заданных точек лежат на одной прямой (пользуясь словарем точек, вывести имена точек).

\newpage
Вариант N 110

Даны точки: L (-16, 26, -19), N (-4, 2, -3), P (-1, -4, 1), S (-4, 5, 3), T (-2, 3, 2).
Составить словарь с ключами - точками (например, Point3D(3, 2, 1)) и значениями - именами точек (N, P и т.д.).


Найти и вывести на экран


расстояние от точек T и L до плоскости NPS,


уравнение плоскости NPS,


параметрические уравнения прямой TL.


Определить и вывести на экран, какие 4 из заданных точек лежат в одной плоскости (пользуясь словарем точек, вывести имена точек).


Определить и вывести на экран, какие 3 из заданных точек лежат на одной прямой (пользуясь словарем точек, вывести имена точек).

\newpage
Вариант N 111

Даны точки: G (1, -4, -1), L (-1, -4, 2), P (5, 2, 5), Q (-3, -10, -7), S (0, 0, 3).
Составить словарь с ключами - точками (например, Point3D(3, 2, 1)) и значениями - именами точек (G, P и т.д.).


Найти и вывести на экран


расстояние от точек L и Q до плоскости GPS,


уравнение плоскости GPS,


параметрические уравнения прямой LQ.


Определить и вывести на экран, какие 4 из заданных точек лежат в одной плоскости (пользуясь словарем точек, вывести имена точек).


Определить и вывести на экран, какие 3 из заданных точек лежат на одной прямой (пользуясь словарем точек, вывести имена точек).

\newpage
Вариант N 112

Даны точки: H (-2, -4, -1), J (1, 5, 2), K (1, 0, -3), N (0, 2, 1), S (-1, -4, 0).
Составить словарь с ключами - точками (например, Point3D(3, 2, 1)) и значениями - именами точек (J, K и т.д.).


Найти и вывести на экран


расстояние от точек S и H до плоскости JKN,


уравнение плоскости JKN,


параметрические уравнения прямой SH.


Определить и вывести на экран, какие 4 из заданных точек лежат в одной плоскости (пользуясь словарем точек, вывести имена точек).


Определить и вывести на экран, какие 3 из заданных точек лежат на одной прямой (пользуясь словарем точек, вывести имена точек).

\newpage
Вариант N 113

Даны точки: J (-3, 1, 2), K (2, 5, -3), L (4, 3, -2), Q (-13, -7, 12), S (-1, 0, 4).
Составить словарь с ключами - точками (например, Point3D(3, 2, 1)) и значениями - именами точек (J, K и т.д.).


Найти и вывести на экран


расстояние от точек L и Q до плоскости JKS,


уравнение плоскости JKS,


параметрические уравнения прямой LQ.


Определить и вывести на экран, какие 4 из заданных точек лежат в одной плоскости (пользуясь словарем точек, вывести имена точек).


Определить и вывести на экран, какие 3 из заданных точек лежат на одной прямой (пользуясь словарем точек, вывести имена точек).

\newpage
Вариант N 114

Даны точки: F (0, -2, 3), G (-3, 1, -2), H (2, -1, -4), N (-4, 5, 2), R (-2, -3, -6).
Составить словарь с ключами - точками (например, Point3D(3, 2, 1)) и значениями - именами точек (F, G и т.д.).


Найти и вывести на экран


расстояние от точек H и R до плоскости FGN,


уравнение плоскости FGN,


параметрические уравнения прямой HR.


Определить и вывести на экран, какие 4 из заданных точек лежат в одной плоскости (пользуясь словарем точек, вывести имена точек).


Определить и вывести на экран, какие 3 из заданных точек лежат на одной прямой (пользуясь словарем точек, вывести имена точек).

\newpage
Вариант N 115

Даны точки: J (-3, 3, 5), L (21, 9, -19), M (3, 3, 1), N (5, 5, -3), S (-2, 5, 0).
Составить словарь с ключами - точками (например, Point3D(3, 2, 1)) и значениями - именами точек (J, M и т.д.).


Найти и вывести на экран


расстояние от точек S и L до плоскости JMN,


уравнение плоскости JMN,


параметрические уравнения прямой SL.


Определить и вывести на экран, какие 4 из заданных точек лежат в одной плоскости (пользуясь словарем точек, вывести имена точек).


Определить и вывести на экран, какие 3 из заданных точек лежат на одной прямой (пользуясь словарем точек, вывести имена точек).

\newpage
Вариант N 116

Даны точки: G (-2, 4, 1), K (1, -1, 2), P (3, -3, 5), S (13, -17, 13), T (-2, -4, 3).
Составить словарь с ключами - точками (например, Point3D(3, 2, 1)) и значениями - именами точек (G, P и т.д.).


Найти и вывести на экран


расстояние от точек K и S до плоскости GPT,


уравнение плоскости GPT,


параметрические уравнения прямой KS.


Определить и вывести на экран, какие 4 из заданных точек лежат в одной плоскости (пользуясь словарем точек, вывести имена точек).


Определить и вывести на экран, какие 3 из заданных точек лежат на одной прямой (пользуясь словарем точек, вывести имена точек).

\newpage
Вариант N 117

Даны точки: F (-4, -2, 5), G (3, -2, 5), N (3, 2, 1), Q (3, -14, 17), R (4, -1, -3).
Составить словарь с ключами - точками (например, Point3D(3, 2, 1)) и значениями - именами точек (F, G и т.д.).


Найти и вывести на экран


расстояние от точек R и Q до плоскости FGN,


уравнение плоскости FGN,


параметрические уравнения прямой RQ.


Определить и вывести на экран, какие 4 из заданных точек лежат в одной плоскости (пользуясь словарем точек, вывести имена точек).


Определить и вывести на экран, какие 3 из заданных точек лежат на одной прямой (пользуясь словарем точек, вывести имена точек).

\newpage
Вариант N 118

Даны точки: F (2, -4, 5), H (3, -4, -4), L (11, -13, 11), M (-1, -1, 3), T (-4, 0, -4).
Составить словарь с ключами - точками (например, Point3D(3, 2, 1)) и значениями - именами точек (F, M и т.д.).


Найти и вывести на экран


расстояние от точек H и L до плоскости FMT,


уравнение плоскости FMT,


параметрические уравнения прямой HL.


Определить и вывести на экран, какие 4 из заданных точек лежат в одной плоскости (пользуясь словарем точек, вывести имена точек).


Определить и вывести на экран, какие 3 из заданных точек лежат на одной прямой (пользуясь словарем точек, вывести имена точек).

\newpage
Вариант N 119

Даны точки: H (4, 4, -1), M (0, 4, 5), P (-3, 1, -4), R (-9, -5, -22), S (3, 5, 4).
Составить словарь с ключами - точками (например, Point3D(3, 2, 1)) и значениями - именами точек (M, P и т.д.).


Найти и вывести на экран


расстояние от точек H и R до плоскости MPS,


уравнение плоскости MPS,


параметрические уравнения прямой HR.


Определить и вывести на экран, какие 4 из заданных точек лежат в одной плоскости (пользуясь словарем точек, вывести имена точек).


Определить и вывести на экран, какие 3 из заданных точек лежат на одной прямой (пользуясь словарем точек, вывести имена точек).

\newpage
Вариант N 120

Даны точки: G (1, -3, 1), P (-2, 3, -1), Q (-8, 15, -5), S (-2, -3, -3), T (-3, 4, 0).
Составить словарь с ключами - точками (например, Point3D(3, 2, 1)) и значениями - именами точек (G, P и т.д.).


Найти и вывести на экран


расстояние от точек T и Q до плоскости GPS,


уравнение плоскости GPS,


параметрические уравнения прямой TQ.


Определить и вывести на экран, какие 4 из заданных точек лежат в одной плоскости (пользуясь словарем точек, вывести имена точек).


Определить и вывести на экран, какие 3 из заданных точек лежат на одной прямой (пользуясь словарем точек, вывести имена точек).

\newpage
Вариант N 121

Даны точки: G (-3, 4, 5), M (-3, 4, -4), Q (-3, 4, -22), R (-2, 1, -1), S (-4, 1, -1).
Составить словарь с ключами - точками (например, Point3D(3, 2, 1)) и значениями - именами точек (G, M и т.д.).


Найти и вывести на экран


расстояние от точек R и Q до плоскости GMS,


уравнение плоскости GMS,


параметрические уравнения прямой RQ.


Определить и вывести на экран, какие 4 из заданных точек лежат в одной плоскости (пользуясь словарем точек, вывести имена точек).


Определить и вывести на экран, какие 3 из заданных точек лежат на одной прямой (пользуясь словарем точек, вывести имена точек).

\newpage
Вариант N 122

Даны точки: G (4, 0, -1), H (-4, -1, -1), L (-4, -4, 5), Q (20, -8, 7), T (0, 2, -3).
Составить словарь с ключами - точками (например, Point3D(3, 2, 1)) и значениями - именами точек (G, H и т.д.).


Найти и вывести на экран


расстояние от точек L и Q до плоскости GHT,


уравнение плоскости GHT,


параметрические уравнения прямой LQ.


Определить и вывести на экран, какие 4 из заданных точек лежат в одной плоскости (пользуясь словарем точек, вывести имена точек).


Определить и вывести на экран, какие 3 из заданных точек лежат на одной прямой (пользуясь словарем точек, вывести имена точек).

\newpage
Вариант N 123

Даны точки: F (-1, 3, -2), J (1, -2, 1), K (-2, -1, -4), R (-5, 13, -8), S (5, 1, 1).
Составить словарь с ключами - точками (например, Point3D(3, 2, 1)) и значениями - именами точек (F, J и т.д.).


Найти и вывести на экран


расстояние от точек S и R до плоскости FJK,


уравнение плоскости FJK,


параметрические уравнения прямой SR.


Определить и вывести на экран, какие 4 из заданных точек лежат в одной плоскости (пользуясь словарем точек, вывести имена точек).


Определить и вывести на экран, какие 3 из заданных точек лежат на одной прямой (пользуясь словарем точек, вывести имена точек).

\newpage
Вариант N 124

Даны точки: H (5, 0, -3), N (-2, -4, -3), Q (-27, -9, -38), R (3, -4, -1), T (3, -3, 4).
Составить словарь с ключами - точками (например, Point3D(3, 2, 1)) и значениями - именами точек (H, N и т.д.).


Найти и вывести на экран


расстояние от точек R и Q до плоскости HNT,


уравнение плоскости HNT,


параметрические уравнения прямой RQ.


Определить и вывести на экран, какие 4 из заданных точек лежат в одной плоскости (пользуясь словарем точек, вывести имена точек).


Определить и вывести на экран, какие 3 из заданных точек лежат на одной прямой (пользуясь словарем точек, вывести имена точек).

\newpage
Вариант N 125

Даны точки: H (14, 1, 24), M (5, 2, -1), N (5, 1, 3), P (2, 1, -4), S (2, 3, -1).
Составить словарь с ключами - точками (например, Point3D(3, 2, 1)) и значениями - именами точек (M, N и т.д.).


Найти и вывести на экран


расстояние от точек S и H до плоскости MNP,


уравнение плоскости MNP,


параметрические уравнения прямой SH.


Определить и вывести на экран, какие 4 из заданных точек лежат в одной плоскости (пользуясь словарем точек, вывести имена точек).


Определить и вывести на экран, какие 3 из заданных точек лежат на одной прямой (пользуясь словарем точек, вывести имена точек).

\newpage
Вариант N 126

Даны точки: F (-1, 2, 1), G (0, 1, -3), J (3, -3, 2), S (-3, 5, 0), T (-12, 17, -23).
Составить словарь с ключами - точками (например, Point3D(3, 2, 1)) и значениями - именами точек (G, J и т.д.).


Найти и вывести на экран


расстояние от точек F и T до плоскости GJS,


уравнение плоскости GJS,


параметрические уравнения прямой FT.


Определить и вывести на экран, какие 4 из заданных точек лежат в одной плоскости (пользуясь словарем точек, вывести имена точек).


Определить и вывести на экран, какие 3 из заданных точек лежат на одной прямой (пользуясь словарем точек, вывести имена точек).

\newpage
Вариант N 127

Даны точки: H (1, 4, 0), J (-2, 5, -2), N (5, 4, -2), R (-16, 7, -2), T (0, -3, 4).
Составить словарь с ключами - точками (например, Point3D(3, 2, 1)) и значениями - именами точек (J, N и т.д.).


Найти и вывести на экран


расстояние от точек H и R до плоскости JNT,


уравнение плоскости JNT,


параметрические уравнения прямой HR.


Определить и вывести на экран, какие 4 из заданных точек лежат в одной плоскости (пользуясь словарем точек, вывести имена точек).


Определить и вывести на экран, какие 3 из заданных точек лежат на одной прямой (пользуясь словарем точек, вывести имена точек).

\newpage
Вариант N 128

Даны точки: L (4, -4, -2), M (4, -3, 4), N (2, 2, -3), P (-4, -3, 4), Q (14, 12, -17).
Составить словарь с ключами - точками (например, Point3D(3, 2, 1)) и значениями - именами точек (M, N и т.д.).


Найти и вывести на экран


расстояние от точек L и Q до плоскости MNP,


уравнение плоскости MNP,


параметрические уравнения прямой LQ.


Определить и вывести на экран, какие 4 из заданных точек лежат в одной плоскости (пользуясь словарем точек, вывести имена точек).


Определить и вывести на экран, какие 3 из заданных точек лежат на одной прямой (пользуясь словарем точек, вывести имена точек).

\newpage
Вариант N 129

Даны точки: F (-2, 5, 3), G (2, -3, -4), H (0, 2, 5), L (-10, 21, 17), T (-2, -1, -4).
Составить словарь с ключами - точками (например, Point3D(3, 2, 1)) и значениями - именами точек (F, G и т.д.).


Найти и вывести на экран


расстояние от точек H и L до плоскости FGT,


уравнение плоскости FGT,


параметрические уравнения прямой HL.


Определить и вывести на экран, какие 4 из заданных точек лежат в одной плоскости (пользуясь словарем точек, вывести имена точек).


Определить и вывести на экран, какие 3 из заданных точек лежат на одной прямой (пользуясь словарем точек, вывести имена точек).

\newpage
Вариант N 130

Даны точки: H (5, 4, 4), K (2, 3, 0), L (4, 6, 3), P (0, 0, -3), S (-3, -3, 3).
Составить словарь с ключами - точками (например, Point3D(3, 2, 1)) и значениями - именами точек (K, P и т.д.).


Найти и вывести на экран


расстояние от точек H и L до плоскости KPS,


уравнение плоскости KPS,


параметрические уравнения прямой HL.


Определить и вывести на экран, какие 4 из заданных точек лежат в одной плоскости (пользуясь словарем точек, вывести имена точек).


Определить и вывести на экран, какие 3 из заданных точек лежат на одной прямой (пользуясь словарем точек, вывести имена точек).

\newpage
Вариант N 131

Даны точки: F (-35, -15, 2), G (-3, -3, 2), J (-3, 5, -1), M (3, -2, 5), N (5, 0, 2).
Составить словарь с ключами - точками (например, Point3D(3, 2, 1)) и значениями - именами точек (G, J и т.д.).


Найти и вывести на экран


расстояние от точек M и F до плоскости GJN,


уравнение плоскости GJN,


параметрические уравнения прямой MF.


Определить и вывести на экран, какие 4 из заданных точек лежат в одной плоскости (пользуясь словарем точек, вывести имена точек).


Определить и вывести на экран, какие 3 из заданных точек лежат на одной прямой (пользуясь словарем точек, вывести имена точек).

\newpage
Вариант N 132

Даны точки: G (-4, 0, 2), L (-1, -2, 4), M (0, 4, 0), N (-3, -4, 1), Q (-6, 8, 4).
Составить словарь с ключами - точками (например, Point3D(3, 2, 1)) и значениями - именами точек (G, M и т.д.).


Найти и вывести на экран


расстояние от точек L и Q до плоскости GMN,


уравнение плоскости GMN,


параметрические уравнения прямой LQ.


Определить и вывести на экран, какие 4 из заданных точек лежат в одной плоскости (пользуясь словарем точек, вывести имена точек).


Определить и вывести на экран, какие 3 из заданных точек лежат на одной прямой (пользуясь словарем точек, вывести имена точек).

\newpage
Вариант N 133

Даны точки: F (4, 4, 4), H (-3, 4, 5), L (13, 11, -15), M (4, 5, 0), N (1, 3, 5).
Составить словарь с ключами - точками (например, Point3D(3, 2, 1)) и значениями - именами точек (F, M и т.д.).


Найти и вывести на экран


расстояние от точек H и L до плоскости FMN,


уравнение плоскости FMN,


параметрические уравнения прямой HL.


Определить и вывести на экран, какие 4 из заданных точек лежат в одной плоскости (пользуясь словарем точек, вывести имена точек).


Определить и вывести на экран, какие 3 из заданных точек лежат на одной прямой (пользуясь словарем точек, вывести имена точек).

\newpage
Вариант N 134

Даны точки: F (-4, -3, 0), G (5, -4, 4), K (3, -4, 0), R (1, -4, -4), T (1, 5, 0).
Составить словарь с ключами - точками (например, Point3D(3, 2, 1)) и значениями - именами точек (F, G и т.д.).


Найти и вывести на экран


расстояние от точек T и R до плоскости FGK,


уравнение плоскости FGK,


параметрические уравнения прямой TR.


Определить и вывести на экран, какие 4 из заданных точек лежат в одной плоскости (пользуясь словарем точек, вывести имена точек).


Определить и вывести на экран, какие 3 из заданных точек лежат на одной прямой (пользуясь словарем точек, вывести имена точек).

\newpage
Вариант N 135

Даны точки: F (0, -2, -3), H (5, -4, 0), J (0, -1, 0), K (-1, -2, 3), Q (0, 4, 15).
Составить словарь с ключами - точками (например, Point3D(3, 2, 1)) и значениями - именами точек (F, J и т.д.).


Найти и вывести на экран


расстояние от точек H и Q до плоскости FJK,


уравнение плоскости FJK,


параметрические уравнения прямой HQ.


Определить и вывести на экран, какие 4 из заданных точек лежат в одной плоскости (пользуясь словарем точек, вывести имена точек).


Определить и вывести на экран, какие 3 из заданных точек лежат на одной прямой (пользуясь словарем точек, вывести имена точек).

\newpage
Вариант N 136

Даны точки: F (4, 2, 4), J (3, -3, 1), K (-2, -3, 1), M (0, 3, 0), R (18, -33, 6).
Составить словарь с ключами - точками (например, Point3D(3, 2, 1)) и значениями - именами точек (F, J и т.д.).


Найти и вывести на экран


расстояние от точек K и R до плоскости FJM,


уравнение плоскости FJM,


параметрические уравнения прямой KR.


Определить и вывести на экран, какие 4 из заданных точек лежат в одной плоскости (пользуясь словарем точек, вывести имена точек).


Определить и вывести на экран, какие 3 из заданных точек лежат на одной прямой (пользуясь словарем точек, вывести имена точек).

\newpage
Вариант N 137

Даны точки: H (2, 1, 0), M (0, 2, -4), P (-4, -3, 0), Q (8, 12, -12), R (1, -1, 5).
Составить словарь с ключами - точками (например, Point3D(3, 2, 1)) и значениями - именами точек (H, M и т.д.).


Найти и вывести на экран


расстояние от точек R и Q до плоскости HMP,


уравнение плоскости HMP,


параметрические уравнения прямой RQ.


Определить и вывести на экран, какие 4 из заданных точек лежат в одной плоскости (пользуясь словарем точек, вывести имена точек).


Определить и вывести на экран, какие 3 из заданных точек лежат на одной прямой (пользуясь словарем точек, вывести имена точек).

\newpage
Вариант N 138

Даны точки: F (-2, -1, 1), G (4, 0, 1), J (0, -3, 1), S (-12, -12, 1), T (4, 0, -3).
Составить словарь с ключами - точками (например, Point3D(3, 2, 1)) и значениями - именами точек (G, J и т.д.).


Найти и вывести на экран


расстояние от точек F и S до плоскости GJT,


уравнение плоскости GJT,


параметрические уравнения прямой FS.


Определить и вывести на экран, какие 4 из заданных точек лежат в одной плоскости (пользуясь словарем точек, вывести имена точек).


Определить и вывести на экран, какие 3 из заданных точек лежат на одной прямой (пользуясь словарем точек, вывести имена точек).

\newpage
Вариант N 139

Даны точки: H (5, 3, 1), K (-1, -4, -3), Q (-16, -29, -28), R (2, -1, 1), T (2, 1, 2).
Составить словарь с ключами - точками (например, Point3D(3, 2, 1)) и значениями - именами точек (H, K и т.д.).


Найти и вывести на экран


расстояние от точек R и Q до плоскости HKT,


уравнение плоскости HKT,


параметрические уравнения прямой RQ.


Определить и вывести на экран, какие 4 из заданных точек лежат в одной плоскости (пользуясь словарем точек, вывести имена точек).


Определить и вывести на экран, какие 3 из заданных точек лежат на одной прямой (пользуясь словарем точек, вывести имена точек).

\newpage
Вариант N 140

Даны точки: G (2, -3, 2), H (4, 4, 3), J (3, 5, 5), Q (8, 45, 20), T (4, -4, -1).
Составить словарь с ключами - точками (например, Point3D(3, 2, 1)) и значениями - именами точек (G, J и т.д.).


Найти и вывести на экран


расстояние от точек H и Q до плоскости GJT,


уравнение плоскости GJT,


параметрические уравнения прямой HQ.


Определить и вывести на экран, какие 4 из заданных точек лежат в одной плоскости (пользуясь словарем точек, вывести имена точек).


Определить и вывести на экран, какие 3 из заданных точек лежат на одной прямой (пользуясь словарем точек, вывести имена точек).

\newpage
Вариант N 141

Даны точки: H (1, 3, 5), M (-1, -4, 2), N (0, 0, -2), Q (3, 12, -14), S (-2, -4, 3).
Составить словарь с ключами - точками (например, Point3D(3, 2, 1)) и значениями - именами точек (M, N и т.д.).


Найти и вывести на экран


расстояние от точек H и Q до плоскости MNS,


уравнение плоскости MNS,


параметрические уравнения прямой HQ.


Определить и вывести на экран, какие 4 из заданных точек лежат в одной плоскости (пользуясь словарем точек, вывести имена точек).


Определить и вывести на экран, какие 3 из заданных точек лежат на одной прямой (пользуясь словарем точек, вывести имена точек).

\newpage
Вариант N 142

Даны точки: L (1, 4, -3), M (2, 3, -2), P (-4, -3, 3), Q (-28, -27, 23), S (-4, 3, 1).
Составить словарь с ключами - точками (например, Point3D(3, 2, 1)) и значениями - именами точек (M, P и т.д.).


Найти и вывести на экран


расстояние от точек L и Q до плоскости MPS,


уравнение плоскости MPS,


параметрические уравнения прямой LQ.


Определить и вывести на экран, какие 4 из заданных точек лежат в одной плоскости (пользуясь словарем точек, вывести имена точек).


Определить и вывести на экран, какие 3 из заданных точек лежат на одной прямой (пользуясь словарем точек, вывести имена точек).

\newpage
Вариант N 143

Даны точки: G (0, 3, -2), H (0, 9, 2), J (-4, 2, 2), L (4, 4, -6), T (1, -3, -3).
Составить словарь с ключами - точками (например, Point3D(3, 2, 1)) и значениями - именами точек (G, J и т.д.).


Найти и вывести на экран


расстояние от точек H и L до плоскости GJT,


уравнение плоскости GJT,


параметрические уравнения прямой HL.


Определить и вывести на экран, какие 4 из заданных точек лежат в одной плоскости (пользуясь словарем точек, вывести имена точек).


Определить и вывести на экран, какие 3 из заданных точек лежат на одной прямой (пользуясь словарем точек, вывести имена точек).

\newpage
Вариант N 144

Даны точки: G (2, 3, -1), H (2, 2, -5), N (2, 4, 3), S (-4, 0, -2), T (-1, -3, -1).
Составить словарь с ключами - точками (например, Point3D(3, 2, 1)) и значениями - именами точек (G, N и т.д.).


Найти и вывести на экран


расстояние от точек S и H до плоскости GNT,


уравнение плоскости GNT,


параметрические уравнения прямой SH.


Определить и вывести на экран, какие 4 из заданных точек лежат в одной плоскости (пользуясь словарем точек, вывести имена точек).


Определить и вывести на экран, какие 3 из заданных точек лежат на одной прямой (пользуясь словарем точек, вывести имена точек).

\newpage
Вариант N 145

Даны точки: G (1, 2, -4), H (0, -2, 0), K (-2, 5, -4), Q (-11, 14, -4), S (3, -4, 3).
Составить словарь с ключами - точками (например, Point3D(3, 2, 1)) и значениями - именами точек (G, K и т.д.).


Найти и вывести на экран


расстояние от точек H и Q до плоскости GKS,


уравнение плоскости GKS,


параметрические уравнения прямой HQ.


Определить и вывести на экран, какие 4 из заданных точек лежат в одной плоскости (пользуясь словарем точек, вывести имена точек).


Определить и вывести на экран, какие 3 из заданных точек лежат на одной прямой (пользуясь словарем точек, вывести имена точек).

\newpage
Вариант N 146

Даны точки: G (0, 1, -1), H (-1, 1, -1), K (0, -1, 5), R (0, 9, -25), S (3, -3, -2).
Составить словарь с ключами - точками (например, Point3D(3, 2, 1)) и значениями - именами точек (G, K и т.д.).


Найти и вывести на экран


расстояние от точек H и R до плоскости GKS,


уравнение плоскости GKS,


параметрические уравнения прямой HR.


Определить и вывести на экран, какие 4 из заданных точек лежат в одной плоскости (пользуясь словарем точек, вывести имена точек).


Определить и вывести на экран, какие 3 из заданных точек лежат на одной прямой (пользуясь словарем точек, вывести имена точек).

\newpage
Вариант N 147

Даны точки: F (5, 0, -1), J (1, 2, -2), N (-3, 4, -4), Q (21, -8, 8), T (1, -3, 4).
Составить словарь с ключами - точками (например, Point3D(3, 2, 1)) и значениями - именами точек (F, J и т.д.).


Найти и вывести на экран


расстояние от точек T и Q до плоскости FJN,


уравнение плоскости FJN,


параметрические уравнения прямой TQ.


Определить и вывести на экран, какие 4 из заданных точек лежат в одной плоскости (пользуясь словарем точек, вывести имена точек).


Определить и вывести на экран, какие 3 из заданных точек лежат на одной прямой (пользуясь словарем точек, вывести имена точек).

\newpage
Вариант N 148

Даны точки: G (2, 2, -4), J (-4, 3, 0), L (0, -1, 5), Q (-22, 6, 12), S (3, 4, -3).
Составить словарь с ключами - точками (например, Point3D(3, 2, 1)) и значениями - именами точек (G, J и т.д.).


Найти и вывести на экран


расстояние от точек L и Q до плоскости GJS,


уравнение плоскости GJS,


параметрические уравнения прямой LQ.


Определить и вывести на экран, какие 4 из заданных точек лежат в одной плоскости (пользуясь словарем точек, вывести имена точек).


Определить и вывести на экран, какие 3 из заданных точек лежат на одной прямой (пользуясь словарем точек, вывести имена точек).

\newpage
Вариант N 149

Даны точки: F (-3, -3, 1), G (-3, 3, -2), J (3, 5, 4), K (27, 13, 28), P (0, 2, 5).
Составить словарь с ключами - точками (например, Point3D(3, 2, 1)) и значениями - именами точек (G, J и т.д.).


Найти и вывести на экран


расстояние от точек F и K до плоскости GJP,


уравнение плоскости GJP,


параметрические уравнения прямой FK.


Определить и вывести на экран, какие 4 из заданных точек лежат в одной плоскости (пользуясь словарем точек, вывести имена точек).


Определить и вывести на экран, какие 3 из заданных точек лежат на одной прямой (пользуясь словарем точек, вывести имена точек).

\newpage
Вариант N 150

Даны точки: J (-2, -2, 2), K (-3, 2, 3), M (-4, 4, 1), P (2, 3, -3), Q (10, 13, -13).
Составить словарь с ключами - точками (например, Point3D(3, 2, 1)) и значениями - именами точек (J, M и т.д.).


Найти и вывести на экран


расстояние от точек K и Q до плоскости JMP,


уравнение плоскости JMP,


параметрические уравнения прямой KQ.


Определить и вывести на экран, какие 4 из заданных точек лежат в одной плоскости (пользуясь словарем точек, вывести имена точек).


Определить и вывести на экран, какие 3 из заданных точек лежат на одной прямой (пользуясь словарем точек, вывести имена точек).

\newpage
Вариант N 151

Даны точки: F (-4, 4, -2), H (-3, -1, 5), K (5, 4, -1), P (-3, 3, 4), Q (-27, 0, 19).
Составить словарь с ключами - точками (например, Point3D(3, 2, 1)) и значениями - именами точек (F, K и т.д.).


Найти и вывести на экран


расстояние от точек H и Q до плоскости FKP,


уравнение плоскости FKP,


параметрические уравнения прямой HQ.


Определить и вывести на экран, какие 4 из заданных точек лежат в одной плоскости (пользуясь словарем точек, вывести имена точек).


Определить и вывести на экран, какие 3 из заданных точек лежат на одной прямой (пользуясь словарем точек, вывести имена точек).

\newpage
Вариант N 152

Даны точки: G (4, -4, 4), H (0, 5, 3), K (-4, -1, -2), L (7, 14, 4), P (5, 2, 4).
Составить словарь с ключами - точками (например, Point3D(3, 2, 1)) и значениями - именами точек (G, K и т.д.).


Найти и вывести на экран


расстояние от точек H и L до плоскости GKP,


уравнение плоскости GKP,


параметрические уравнения прямой HL.


Определить и вывести на экран, какие 4 из заданных точек лежат в одной плоскости (пользуясь словарем точек, вывести имена точек).


Определить и вывести на экран, какие 3 из заданных точек лежат на одной прямой (пользуясь словарем точек, вывести имена точек).

\newpage
Вариант N 153

Даны точки: F (1, -3, -4), H (2, -2, -4), L (-2, 1, -1), R (-11, 13, 8), T (5, -1, -4).
Составить словарь с ключами - точками (например, Point3D(3, 2, 1)) и значениями - именами точек (F, L и т.д.).


Найти и вывести на экран


расстояние от точек H и R до плоскости FLT,


уравнение плоскости FLT,


параметрические уравнения прямой HR.


Определить и вывести на экран, какие 4 из заданных точек лежат в одной плоскости (пользуясь словарем точек, вывести имена точек).


Определить и вывести на экран, какие 3 из заданных точек лежат на одной прямой (пользуясь словарем точек, вывести имена точек).

\newpage
Вариант N 154

Даны точки: L (0, 2, 5), P (-2, 5, -1), Q (13, -5, 19), R (1, 3, 3), S (1, 0, 5).
Составить словарь с ключами - точками (например, Point3D(3, 2, 1)) и значениями - именами точек (P, R и т.д.).


Найти и вывести на экран


расстояние от точек L и Q до плоскости PRS,


уравнение плоскости PRS,


параметрические уравнения прямой LQ.


Определить и вывести на экран, какие 4 из заданных точек лежат в одной плоскости (пользуясь словарем точек, вывести имена точек).


Определить и вывести на экран, какие 3 из заданных точек лежат на одной прямой (пользуясь словарем точек, вывести имена точек).

\newpage
Вариант N 155

Даны точки: H (7, 17, 9), M (4, 2, 3), N (3, -3, 1), S (4, 1, -3), T (1, 2, -1).
Составить словарь с ключами - точками (например, Point3D(3, 2, 1)) и значениями - именами точек (M, N и т.д.).


Найти и вывести на экран


расстояние от точек T и H до плоскости MNS,


уравнение плоскости MNS,


параметрические уравнения прямой TH.


Определить и вывести на экран, какие 4 из заданных точек лежат в одной плоскости (пользуясь словарем точек, вывести имена точек).


Определить и вывести на экран, какие 3 из заданных точек лежат на одной прямой (пользуясь словарем точек, вывести имена точек).

\newpage
Вариант N 156

Даны точки: F (4, -2, 0), K (2, 5, 1), Q (0, 12, 2), R (2, -1, -2), T (-1, 4, 3).
Составить словарь с ключами - точками (например, Point3D(3, 2, 1)) и значениями - именами точек (F, K и т.д.).


Найти и вывести на экран


расстояние от точек R и Q до плоскости FKT,


уравнение плоскости FKT,


параметрические уравнения прямой RQ.


Определить и вывести на экран, какие 4 из заданных точек лежат в одной плоскости (пользуясь словарем точек, вывести имена точек).


Определить и вывести на экран, какие 3 из заданных точек лежат на одной прямой (пользуясь словарем точек, вывести имена точек).

\newpage
Вариант N 157

Даны точки: G (1, 1, -1), H (3, -3, -3), M (-3, 1, -3), Q (-15, 1, -9), S (0, -2, -3).
Составить словарь с ключами - точками (например, Point3D(3, 2, 1)) и значениями - именами точек (G, M и т.д.).


Найти и вывести на экран


расстояние от точек H и Q до плоскости GMS,


уравнение плоскости GMS,


параметрические уравнения прямой HQ.


Определить и вывести на экран, какие 4 из заданных точек лежат в одной плоскости (пользуясь словарем точек, вывести имена точек).


Определить и вывести на экран, какие 3 из заданных точек лежат на одной прямой (пользуясь словарем точек, вывести имена точек).

\newpage
Вариант N 158

Даны точки: H (-4, 3, 5), K (-2, -3, 1), L (-2, 3, -17), M (-2, -2, -2), T (-4, 4, 3).
Составить словарь с ключами - точками (например, Point3D(3, 2, 1)) и значениями - именами точек (K, M и т.д.).


Найти и вывести на экран


расстояние от точек H и L до плоскости KMT,


уравнение плоскости KMT,


параметрические уравнения прямой HL.


Определить и вывести на экран, какие 4 из заданных точек лежат в одной плоскости (пользуясь словарем точек, вывести имена точек).


Определить и вывести на экран, какие 3 из заданных точек лежат на одной прямой (пользуясь словарем точек, вывести имена точек).

\newpage
Вариант N 159

Даны точки: F (1, 2, 3), J (3, -1, 4), N (1, 1, -2), R (-3, 5, -14), S (4, 3, -1).
Составить словарь с ключами - точками (например, Point3D(3, 2, 1)) и значениями - именами точек (F, J и т.д.).


Найти и вывести на экран


расстояние от точек S и R до плоскости FJN,


уравнение плоскости FJN,


параметрические уравнения прямой SR.


Определить и вывести на экран, какие 4 из заданных точек лежат в одной плоскости (пользуясь словарем точек, вывести имена точек).


Определить и вывести на экран, какие 3 из заданных точек лежат на одной прямой (пользуясь словарем точек, вывести имена точек).

\newpage
Вариант N 160

Даны точки: G (1, -4, 4), J (-4, -2, 5), M (0, 5, 0), P (5, 4, 0), R (11, -8, 2).
Составить словарь с ключами - точками (например, Point3D(3, 2, 1)) и значениями - именами точек (G, J и т.д.).


Найти и вывести на экран


расстояние от точек M и R до плоскости GJP,


уравнение плоскости GJP,


параметрические уравнения прямой MR.


Определить и вывести на экран, какие 4 из заданных точек лежат в одной плоскости (пользуясь словарем точек, вывести имена точек).


Определить и вывести на экран, какие 3 из заданных точек лежат на одной прямой (пользуясь словарем точек, вывести имена точек).

\newpage
Вариант N 161

Даны точки: F (4, -1, -4), G (4, -2, 2), H (12, -2, -6), M (-2, 5, -4), N (2, -2, 4).
Составить словарь с ключами - точками (например, Point3D(3, 2, 1)) и значениями - именами точек (G, M и т.д.).


Найти и вывести на экран


расстояние от точек F и H до плоскости GMN,


уравнение плоскости GMN,


параметрические уравнения прямой FH.


Определить и вывести на экран, какие 4 из заданных точек лежат в одной плоскости (пользуясь словарем точек, вывести имена точек).


Определить и вывести на экран, какие 3 из заданных точек лежат на одной прямой (пользуясь словарем точек, вывести имена точек).

\newpage
Вариант N 162

Даны точки: M (1, -2, -1), P (-1, -3, 3), Q (7, 1, -13), R (5, 4, -2), S (1, -2, -3).
Составить словарь с ключами - точками (например, Point3D(3, 2, 1)) и значениями - именами точек (M, P и т.д.).


Найти и вывести на экран


расстояние от точек R и Q до плоскости MPS,


уравнение плоскости MPS,


параметрические уравнения прямой RQ.


Определить и вывести на экран, какие 4 из заданных точек лежат в одной плоскости (пользуясь словарем точек, вывести имена точек).


Определить и вывести на экран, какие 3 из заданных точек лежат на одной прямой (пользуясь словарем точек, вывести имена точек).

\newpage
Вариант N 163

Даны точки: F (-4, 3, 4), H (-3, -2, -1), L (3, 3, -1), Q (8, 3, -10), S (2, 3, -3).
Составить словарь с ключами - точками (например, Point3D(3, 2, 1)) и значениями - именами точек (F, H и т.д.).


Найти и вывести на экран


расстояние от точек L и Q до плоскости FHS,


уравнение плоскости FHS,


параметрические уравнения прямой LQ.


Определить и вывести на экран, какие 4 из заданных точек лежат в одной плоскости (пользуясь словарем точек, вывести имена точек).


Определить и вывести на экран, какие 3 из заданных точек лежат на одной прямой (пользуясь словарем точек, вывести имена точек).

\newpage
Вариант N 164

Даны точки: F (1, 3, -3), J (2, -3, 5), P (-1, 0, -1), Q (-4, 3, -7), R (-1, 4, 5).
Составить словарь с ключами - точками (например, Point3D(3, 2, 1)) и значениями - именами точек (F, J и т.д.).


Найти и вывести на экран


расстояние от точек R и Q до плоскости FJP,


уравнение плоскости FJP,


параметрические уравнения прямой RQ.


Определить и вывести на экран, какие 4 из заданных точек лежат в одной плоскости (пользуясь словарем точек, вывести имена точек).


Определить и вывести на экран, какие 3 из заданных точек лежат на одной прямой (пользуясь словарем точек, вывести имена точек).

\newpage
Вариант N 165

Даны точки: F (4, 3, -3), G (3, 4, -3), J (-2, 5, 5), K (28, -30, -20), P (4, -2, 0).
Составить словарь с ключами - точками (например, Point3D(3, 2, 1)) и значениями - именами точек (G, J и т.д.).


Найти и вывести на экран


расстояние от точек F и K до плоскости GJP,


уравнение плоскости GJP,


параметрические уравнения прямой FK.


Определить и вывести на экран, какие 4 из заданных точек лежат в одной плоскости (пользуясь словарем точек, вывести имена точек).


Определить и вывести на экран, какие 3 из заданных точек лежат на одной прямой (пользуясь словарем точек, вывести имена точек).

\newpage
Вариант N 166

Даны точки: G (-4, 4, 1), J (5, 4, 1), N (5, 4, -1), R (-13, 4, 1), T (3, 2, 1).
Составить словарь с ключами - точками (например, Point3D(3, 2, 1)) и значениями - именами точек (G, J и т.д.).


Найти и вывести на экран


расстояние от точек T и R до плоскости GJN,


уравнение плоскости GJN,


параметрические уравнения прямой TR.


Определить и вывести на экран, какие 4 из заданных точек лежат в одной плоскости (пользуясь словарем точек, вывести имена точек).


Определить и вывести на экран, какие 3 из заданных точек лежат на одной прямой (пользуясь словарем точек, вывести имена точек).

\newpage
Вариант N 167

Даны точки: J (-4, 0, 3), L (16, -10, -2), N (0, -2, 2), P (1, 5, -4), T (3, 1, 0).
Составить словарь с ключами - точками (например, Point3D(3, 2, 1)) и значениями - именами точек (J, N и т.д.).


Найти и вывести на экран


расстояние от точек T и L до плоскости JNP,


уравнение плоскости JNP,


параметрические уравнения прямой TL.


Определить и вывести на экран, какие 4 из заданных точек лежат в одной плоскости (пользуясь словарем точек, вывести имена точек).


Определить и вывести на экран, какие 3 из заданных точек лежат на одной прямой (пользуясь словарем точек, вывести имена точек).

\newpage
Вариант N 168

Даны точки: F (1, 1, 5), J (3, 5, 0), L (3, 29, -9), M (3, -3, 3), T (4, 0, 4).
Составить словарь с ключами - точками (например, Point3D(3, 2, 1)) и значениями - именами точек (F, J и т.д.).


Найти и вывести на экран


расстояние от точек T и L до плоскости FJM,


уравнение плоскости FJM,


параметрические уравнения прямой TL.


Определить и вывести на экран, какие 4 из заданных точек лежат в одной плоскости (пользуясь словарем точек, вывести имена точек).


Определить и вывести на экран, какие 3 из заданных точек лежат на одной прямой (пользуясь словарем точек, вывести имена точек).

\newpage
Вариант N 169

Даны точки: J (1, 0, -1), N (3, -4, 2), P (-3, -4, 1), R (9, -16, 11), T (3, -4, -3).
Составить словарь с ключами - точками (например, Point3D(3, 2, 1)) и значениями - именами точек (J, N и т.д.).


Найти и вывести на экран


расстояние от точек T и R до плоскости JNP,


уравнение плоскости JNP,


параметрические уравнения прямой TR.


Определить и вывести на экран, какие 4 из заданных точек лежат в одной плоскости (пользуясь словарем точек, вывести имена точек).


Определить и вывести на экран, какие 3 из заданных точек лежат на одной прямой (пользуясь словарем точек, вывести имена точек).

\newpage
Вариант N 170

Даны точки: J (1, -3, 1), K (-2, 0, -2), L (1, -4, 3), Q (16, -18, 16), S (-2, 4, -3).
Составить словарь с ключами - точками (например, Point3D(3, 2, 1)) и значениями - именами точек (J, K и т.д.).


Найти и вывести на экран


расстояние от точек L и Q до плоскости JKS,


уравнение плоскости JKS,


параметрические уравнения прямой LQ.


Определить и вывести на экран, какие 4 из заданных точек лежат в одной плоскости (пользуясь словарем точек, вывести имена точек).


Определить и вывести на экран, какие 3 из заданных точек лежат на одной прямой (пользуясь словарем точек, вывести имена точек).

\newpage
Вариант N 171

Даны точки: G (5, 1, 5), J (5, -1, 5), N (0, -1, 3), P (5, 5, -2), T (-25, -11, -7).
Составить словарь с ключами - точками (например, Point3D(3, 2, 1)) и значениями - именами точек (G, N и т.д.).


Найти и вывести на экран


расстояние от точек J и T до плоскости GNP,


уравнение плоскости GNP,


параметрические уравнения прямой JT.


Определить и вывести на экран, какие 4 из заданных точек лежат в одной плоскости (пользуясь словарем точек, вывести имена точек).


Определить и вывести на экран, какие 3 из заданных точек лежат на одной прямой (пользуясь словарем точек, вывести имена точек).

\newpage
Вариант N 172

Даны точки: F (1, -4, -3), G (-2, -2, 5), H (-4, -6, -13), M (-3, -4, -4), S (-4, -2, 5).
Составить словарь с ключами - точками (например, Point3D(3, 2, 1)) и значениями - именами точек (G, M и т.д.).


Найти и вывести на экран


расстояние от точек F и H до плоскости GMS,


уравнение плоскости GMS,


параметрические уравнения прямой FH.


Определить и вывести на экран, какие 4 из заданных точек лежат в одной плоскости (пользуясь словарем точек, вывести имена точек).


Определить и вывести на экран, какие 3 из заданных точек лежат на одной прямой (пользуясь словарем точек, вывести имена точек).

\newpage
Вариант N 173

Даны точки: M (2, 2, -3), N (-1, -3, -4), P (4, 1, 1), S (14, 9, 11), T (2, 1, 2).
Составить словарь с ключами - точками (например, Point3D(3, 2, 1)) и значениями - именами точек (N, P и т.д.).


Найти и вывести на экран


расстояние от точек M и S до плоскости NPT,


уравнение плоскости NPT,


параметрические уравнения прямой MS.


Определить и вывести на экран, какие 4 из заданных точек лежат в одной плоскости (пользуясь словарем точек, вывести имена точек).


Определить и вывести на экран, какие 3 из заданных точек лежат на одной прямой (пользуясь словарем точек, вывести имена точек).

\newpage
Вариант N 174

Даны точки: F (-3, -2, -4), G (1, -1, 2), J (5, -4, 4), L (25, -19, 14), N (1, 1, 1).
Составить словарь с ключами - точками (например, Point3D(3, 2, 1)) и значениями - именами точек (G, J и т.д.).


Найти и вывести на экран


расстояние от точек F и L до плоскости GJN,


уравнение плоскости GJN,


параметрические уравнения прямой FL.


Определить и вывести на экран, какие 4 из заданных точек лежат в одной плоскости (пользуясь словарем точек, вывести имена точек).


Определить и вывести на экран, какие 3 из заданных точек лежат на одной прямой (пользуясь словарем точек, вывести имена точек).

\newpage
Вариант N 175

Даны точки: G (-2, 4, -2), K (-1, 3, 0), L (-23, 25, 1), P (5, -3, -3), S (-1, -2, 4).
Составить словарь с ключами - точками (например, Point3D(3, 2, 1)) и значениями - именами точек (G, K и т.д.).


Найти и вывести на экран


расстояние от точек S и L до плоскости GKP,


уравнение плоскости GKP,


параметрические уравнения прямой SL.


Определить и вывести на экран, какие 4 из заданных точек лежат в одной плоскости (пользуясь словарем точек, вывести имена точек).


Определить и вывести на экран, какие 3 из заданных точек лежат на одной прямой (пользуясь словарем точек, вывести имена точек).

\newpage
Вариант N 176

Даны точки: G (0, -3, 5), H (-3, -3, 17), K (1, -3, -2), P (1, -3, 1), S (5, 5, -1).
Составить словарь с ключами - точками (например, Point3D(3, 2, 1)) и значениями - именами точек (G, K и т.д.).


Найти и вывести на экран


расстояние от точек S и H до плоскости GKP,


уравнение плоскости GKP,


параметрические уравнения прямой SH.


Определить и вывести на экран, какие 4 из заданных точек лежат в одной плоскости (пользуясь словарем точек, вывести имена точек).


Определить и вывести на экран, какие 3 из заданных точек лежат на одной прямой (пользуясь словарем точек, вывести имена точек).

\newpage
Вариант N 177

Даны точки: F (-4, 1, -3), G (0, 2, 1), H (-5, 2, -19), M (4, 3, 1), T (-1, 2, -3).
Составить словарь с ключами - точками (например, Point3D(3, 2, 1)) и значениями - именами точек (G, M и т.д.).


Найти и вывести на экран


расстояние от точек F и H до плоскости GMT,


уравнение плоскости GMT,


параметрические уравнения прямой FH.


Определить и вывести на экран, какие 4 из заданных точек лежат в одной плоскости (пользуясь словарем точек, вывести имена точек).


Определить и вывести на экран, какие 3 из заданных точек лежат на одной прямой (пользуясь словарем точек, вывести имена точек).

\newpage
Вариант N 178

Даны точки: F (-3, -4, -2), H (3, 1, 4), L (-3, -2, -1), Q (-3, -8, -4), R (5, 2, 5).
Составить словарь с ключами - точками (например, Point3D(3, 2, 1)) и значениями - именами точек (F, H и т.д.).


Найти и вывести на экран


расстояние от точек R и Q до плоскости FHL,


уравнение плоскости FHL,


параметрические уравнения прямой RQ.


Определить и вывести на экран, какие 4 из заданных точек лежат в одной плоскости (пользуясь словарем точек, вывести имена точек).


Определить и вывести на экран, какие 3 из заданных точек лежат на одной прямой (пользуясь словарем точек, вывести имена точек).

\newpage
Вариант N 179

Даны точки: H (2, 5, -2), L (-37, -6, 33), M (5, -2, -1), N (-2, -1, 3), P (5, 0, -3).
Составить словарь с ключами - точками (например, Point3D(3, 2, 1)) и значениями - именами точек (M, N и т.д.).


Найти и вывести на экран


расстояние от точек H и L до плоскости MNP,


уравнение плоскости MNP,


параметрические уравнения прямой HL.


Определить и вывести на экран, какие 4 из заданных точек лежат в одной плоскости (пользуясь словарем точек, вывести имена точек).


Определить и вывести на экран, какие 3 из заданных точек лежат на одной прямой (пользуясь словарем точек, вывести имена точек).

\newpage
Вариант N 180

Даны точки: G (-2, 3, 2), H (-2, -1, 6), J (-2, 4, 1), M (-2, 3, -3), N (2, -2, -3).
Составить словарь с ключами - точками (например, Point3D(3, 2, 1)) и значениями - именами точек (G, J и т.д.).


Найти и вывести на экран


расстояние от точек M и H до плоскости GJN,


уравнение плоскости GJN,


параметрические уравнения прямой MH.


Определить и вывести на экран, какие 4 из заданных точек лежат в одной плоскости (пользуясь словарем точек, вывести имена точек).


Определить и вывести на экран, какие 3 из заданных точек лежат на одной прямой (пользуясь словарем точек, вывести имена точек).

\newpage
Вариант N 181

Даны точки: H (-1, 0, -3), L (-4, -3, 2), P (-4, 5, 3), Q (-12, 19, 11), R (0, -2, -1).
Составить словарь с ключами - точками (например, Point3D(3, 2, 1)) и значениями - именами точек (H, P и т.д.).


Найти и вывести на экран


расстояние от точек L и Q до плоскости HPR,


уравнение плоскости HPR,


параметрические уравнения прямой LQ.


Определить и вывести на экран, какие 4 из заданных точек лежат в одной плоскости (пользуясь словарем точек, вывести имена точек).


Определить и вывести на экран, какие 3 из заданных точек лежат на одной прямой (пользуясь словарем точек, вывести имена точек).

\newpage
Вариант N 182

Даны точки: G (-4, 1, -2), K (4, 3, -4), M (3, -1, 3), Q (-25, 7, -17), S (2, -2, -2).
Составить словарь с ключами - точками (например, Point3D(3, 2, 1)) и значениями - именами точек (G, K и т.д.).


Найти и вывести на экран


расстояние от точек S и Q до плоскости GKM,


уравнение плоскости GKM,


параметрические уравнения прямой SQ.


Определить и вывести на экран, какие 4 из заданных точек лежат в одной плоскости (пользуясь словарем точек, вывести имена точек).


Определить и вывести на экран, какие 3 из заданных точек лежат на одной прямой (пользуясь словарем точек, вывести имена точек).

\newpage
Вариант N 183

Даны точки: J (0, -1, 3), N (0, -1, 4), P (1, -1, 3), Q (0, -1, 7), R (-4, 13, 1).
Составить словарь с ключами - точками (например, Point3D(3, 2, 1)) и значениями - именами точек (J, N и т.д.).


Найти и вывести на экран


расстояние от точек R и Q до плоскости JNP,


уравнение плоскости JNP,


параметрические уравнения прямой RQ.


Определить и вывести на экран, какие 4 из заданных точек лежат в одной плоскости (пользуясь словарем точек, вывести имена точек).


Определить и вывести на экран, какие 3 из заданных точек лежат на одной прямой (пользуясь словарем точек, вывести имена точек).

\newpage
Вариант N 184

Даны точки: H (0, -1, -3), K (3, 3, 3), L (9, 18, 9), M (1, -2, 1), T (4, 5, 4).
Составить словарь с ключами - точками (например, Point3D(3, 2, 1)) и значениями - именами точек (K, M и т.д.).


Найти и вывести на экран


расстояние от точек H и L до плоскости KMT,


уравнение плоскости KMT,


параметрические уравнения прямой HL.


Определить и вывести на экран, какие 4 из заданных точек лежат в одной плоскости (пользуясь словарем точек, вывести имена точек).


Определить и вывести на экран, какие 3 из заданных точек лежат на одной прямой (пользуясь словарем точек, вывести имена точек).

\newpage
Вариант N 185

Даны точки: H (2, 1, 1), J (1, 4, -1), N (-2, -1, 3), R (-17, -26, 23), S (1, 2, 0).
Составить словарь с ключами - точками (например, Point3D(3, 2, 1)) и значениями - именами точек (J, N и т.д.).


Найти и вывести на экран


расстояние от точек H и R до плоскости JNS,


уравнение плоскости JNS,


параметрические уравнения прямой HR.


Определить и вывести на экран, какие 4 из заданных точек лежат в одной плоскости (пользуясь словарем точек, вывести имена точек).


Определить и вывести на экран, какие 3 из заданных точек лежат на одной прямой (пользуясь словарем точек, вывести имена точек).

\newpage
Вариант N 186

Даны точки: F (-2, 5, 5), K (5, 4, 0), M (-4, -3, -1), N (5, 0, 3), Q (41, 12, 19).
Составить словарь с ключами - точками (например, Point3D(3, 2, 1)) и значениями - именами точек (K, M и т.д.).


Найти и вывести на экран


расстояние от точек F и Q до плоскости KMN,


уравнение плоскости KMN,


параметрические уравнения прямой FQ.


Определить и вывести на экран, какие 4 из заданных точек лежат в одной плоскости (пользуясь словарем точек, вывести имена точек).


Определить и вывести на экран, какие 3 из заданных точек лежат на одной прямой (пользуясь словарем точек, вывести имена точек).

\newpage
Вариант N 187

Даны точки: G (5, 2, 5), J (3, -4, -3), K (-1, 1, 4), L (-1, -16, -19), S (3, -3, -2).
Составить словарь с ключами - точками (например, Point3D(3, 2, 1)) и значениями - именами точек (G, J и т.д.).


Найти и вывести на экран


расстояние от точек S и L до плоскости GJK,


уравнение плоскости GJK,


параметрические уравнения прямой SL.


Определить и вывести на экран, какие 4 из заданных точек лежат в одной плоскости (пользуясь словарем точек, вывести имена точек).


Определить и вывести на экран, какие 3 из заданных точек лежат на одной прямой (пользуясь словарем точек, вывести имена точек).

\newpage
Вариант N 188

Даны точки: H (5, 0, 1), K (2, 2, 0), M (-1, -2, -4), N (5, 5, -4), R (-7, -9, -4).
Составить словарь с ключами - точками (например, Point3D(3, 2, 1)) и значениями - именами точек (K, M и т.д.).


Найти и вывести на экран


расстояние от точек H и R до плоскости KMN,


уравнение плоскости KMN,


параметрические уравнения прямой HR.


Определить и вывести на экран, какие 4 из заданных точек лежат в одной плоскости (пользуясь словарем точек, вывести имена точек).


Определить и вывести на экран, какие 3 из заданных точек лежат на одной прямой (пользуясь словарем точек, вывести имена точек).

\newpage
Вариант N 189

Даны точки: G (4, -2, 5), K (4, 0, 3), N (1, 0, 1), P (3, -1, -3), T (10, -6, 13).
Составить словарь с ключами - точками (например, Point3D(3, 2, 1)) и значениями - именами точек (G, N и т.д.).


Найти и вывести на экран


расстояние от точек K и T до плоскости GNP,


уравнение плоскости GNP,


параметрические уравнения прямой KT.


Определить и вывести на экран, какие 4 из заданных точек лежат в одной плоскости (пользуясь словарем точек, вывести имена точек).


Определить и вывести на экран, какие 3 из заданных точек лежат на одной прямой (пользуясь словарем точек, вывести имена точек).

\newpage
Вариант N 190

Даны точки: G (4, 4, 2), J (0, 5, 0), K (-4, 5, -3), P (-4, 3, 2), R (-12, -1, 6).
Составить словарь с ключами - точками (например, Point3D(3, 2, 1)) и значениями - именами точек (G, J и т.д.).


Найти и вывести на экран


расстояние от точек K и R до плоскости GJP,


уравнение плоскости GJP,


параметрические уравнения прямой KR.


Определить и вывести на экран, какие 4 из заданных точек лежат в одной плоскости (пользуясь словарем точек, вывести имена точек).


Определить и вывести на экран, какие 3 из заданных точек лежат на одной прямой (пользуясь словарем точек, вывести имена точек).

\newpage
Вариант N 191

Даны точки: G (-1, -4, 2), J (1, 2, -1), L (5, 14, -7), N (-2, 3, 2), T (3, 1, 0).
Составить словарь с ключами - точками (например, Point3D(3, 2, 1)) и значениями - именами точек (G, J и т.д.).


Найти и вывести на экран


расстояние от точек T и L до плоскости GJN,


уравнение плоскости GJN,


параметрические уравнения прямой TL.


Определить и вывести на экран, какие 4 из заданных точек лежат в одной плоскости (пользуясь словарем точек, вывести имена точек).


Определить и вывести на экран, какие 3 из заданных точек лежат на одной прямой (пользуясь словарем точек, вывести имена точек).

\newpage
Вариант N 192

Даны точки: K (-4, -3, 1), N (5, -3, 3), P (-2, 5, 2), R (-30, 37, -2), T (1, 5, 0).
Составить словарь с ключами - точками (например, Point3D(3, 2, 1)) и значениями - именами точек (K, N и т.д.).


Найти и вывести на экран


расстояние от точек T и R до плоскости KNP,


уравнение плоскости KNP,


параметрические уравнения прямой TR.


Определить и вывести на экран, какие 4 из заданных точек лежат в одной плоскости (пользуясь словарем точек, вывести имена точек).


Определить и вывести на экран, какие 3 из заданных точек лежат на одной прямой (пользуясь словарем точек, вывести имена точек).

\newpage
Вариант N 193

Даны точки: J (5, 0, 4), M (3, -4, -4), N (0, -2, 3), Q (20, 6, 7), T (5, 5, -1).
Составить словарь с ключами - точками (например, Point3D(3, 2, 1)) и значениями - именами точек (J, M и т.д.).


Найти и вывести на экран


расстояние от точек T и Q до плоскости JMN,


уравнение плоскости JMN,


параметрические уравнения прямой TQ.


Определить и вывести на экран, какие 4 из заданных точек лежат в одной плоскости (пользуясь словарем точек, вывести имена точек).


Определить и вывести на экран, какие 3 из заданных точек лежат на одной прямой (пользуясь словарем точек, вывести имена точек).

\newpage
Вариант N 194

Даны точки: H (-1, 1, -4), J (-2, 1, 0), L (-4, -3, -3), P (3, -3, -4), R (18, -15, -16).
Составить словарь с ключами - точками (например, Point3D(3, 2, 1)) и значениями - именами точек (J, L и т.д.).


Найти и вывести на экран


расстояние от точек H и R до плоскости JLP,


уравнение плоскости JLP,


параметрические уравнения прямой HR.


Определить и вывести на экран, какие 4 из заданных точек лежат в одной плоскости (пользуясь словарем точек, вывести имена точек).


Определить и вывести на экран, какие 3 из заданных точек лежат на одной прямой (пользуясь словарем точек, вывести имена точек).

\newpage
Вариант N 195

Даны точки: G (3, -3, -3), H (-1, -2, -3), L (19, -3, -11), R (-1, -3, -1), T (1, -1, -4).
Составить словарь с ключами - точками (например, Point3D(3, 2, 1)) и значениями - именами точек (G, R и т.д.).


Найти и вывести на экран


расстояние от точек H и L до плоскости GRT,


уравнение плоскости GRT,


параметрические уравнения прямой HL.


Определить и вывести на экран, какие 4 из заданных точек лежат в одной плоскости (пользуясь словарем точек, вывести имена точек).


Определить и вывести на экран, какие 3 из заданных точек лежат на одной прямой (пользуясь словарем точек, вывести имена точек).

\newpage
Вариант N 196

Даны точки: H (-1, -3, -3), L (-3, 3, 4), M (3, -1, -3), P (0, 5, 5), R (12, -19, -27).
Составить словарь с ключами - точками (например, Point3D(3, 2, 1)) и значениями - именами точек (L, M и т.д.).


Найти и вывести на экран


расстояние от точек H и R до плоскости LMP,


уравнение плоскости LMP,


параметрические уравнения прямой HR.


Определить и вывести на экран, какие 4 из заданных точек лежат в одной плоскости (пользуясь словарем точек, вывести имена точек).


Определить и вывести на экран, какие 3 из заданных точек лежат на одной прямой (пользуясь словарем точек, вывести имена точек).

\newpage
Вариант N 197

Даны точки: G (5, 0, -1), L (-4, 4, 4), P (-3, 0, 0), Q (45, 0, -6), S (-1, 3, 4).
Составить словарь с ключами - точками (например, Point3D(3, 2, 1)) и значениями - именами точек (G, P и т.д.).


Найти и вывести на экран


расстояние от точек L и Q до плоскости GPS,


уравнение плоскости GPS,


параметрические уравнения прямой LQ.


Определить и вывести на экран, какие 4 из заданных точек лежат в одной плоскости (пользуясь словарем точек, вывести имена точек).


Определить и вывести на экран, какие 3 из заданных точек лежат на одной прямой (пользуясь словарем точек, вывести имена точек).

\newpage
Вариант N 198

Даны точки: F (-2, 12, 5), G (-2, 5, 0), H (-8, 5, -10), J (1, 5, 5), M (-4, 5, -3).
Составить словарь с ключами - точками (например, Point3D(3, 2, 1)) и значениями - именами точек (G, J и т.д.).


Найти и вывести на экран


расстояние от точек F и H до плоскости GJM,


уравнение плоскости GJM,


параметрические уравнения прямой FH.


Определить и вывести на экран, какие 4 из заданных точек лежат в одной плоскости (пользуясь словарем точек, вывести имена точек).


Определить и вывести на экран, какие 3 из заданных точек лежат на одной прямой (пользуясь словарем точек, вывести имена точек).

\newpage
Вариант N 199

Даны точки: F (0, -4, 1), K (0, 3, 1), L (-9, -15, 13), N (-3, -3, 5), S (5, 5, 3).
Составить словарь с ключами - точками (например, Point3D(3, 2, 1)) и значениями - именами точек (F, K и т.д.).


Найти и вывести на экран


расстояние от точек S и L до плоскости FKN,


уравнение плоскости FKN,


параметрические уравнения прямой SL.


Определить и вывести на экран, какие 4 из заданных точек лежат в одной плоскости (пользуясь словарем точек, вывести имена точек).


Определить и вывести на экран, какие 3 из заданных точек лежат на одной прямой (пользуясь словарем точек, вывести имена точек).

\newpage
Вариант N 200

Даны точки: G (3, -3, -1), L (2, -4, 2), M (-3, 2, 4), N (4, -2, -4), S (-2, 1, 3).
Составить словарь с ключами - точками (например, Point3D(3, 2, 1)) и значениями - именами точек (G, M и т.д.).


Найти и вывести на экран


расстояние от точек S и L до плоскости GMN,


уравнение плоскости GMN,


параметрические уравнения прямой SL.


Определить и вывести на экран, какие 4 из заданных точек лежат в одной плоскости (пользуясь словарем точек, вывести имена точек).


Определить и вывести на экран, какие 3 из заданных точек лежат на одной прямой (пользуясь словарем точек, вывести имена точек).

\newpage
\end{document}