 \documentclass[11pt]{report}

\usepackage[T2A]{fontenc}

\usepackage[utf8]{inputenc}

\usepackage[russian]{babel}

\usepackage{amsmath,amssymb}

\oddsidemargin=-19mm

\topmargin=-30mm

\textheight 26cm 

\hsize 18cm

\textwidth 20cm

\begin{document}

\pagestyle{empty}

{\bf Индивидуальное задание 3.}


Вариант N 1


Решить СЛАУ c параметром тремя способами (расширенная матрица, список уравнений, матричная форма).

Вначале составить список уравнений и решить вторым способом,
затем список уравнений преобразовать в матричный вид и решить третьим способом.
Затем составить из матрицы левой части и столбца правой расширенную матрицу СЛАУ и решить первым способом.
После этого провести проверку подстановкой.

Затем отдельно рассмотреть значение параметра, при котором решение СЛАУ нельзя найти по общей формуле,
полученной ранее.
Найти решение СЛАУ при этом значении параметра первым или третьим способом, используя подстановку subs.
\begin{align*}
    A = \left[\begin{matrix}-1 & -1 & -6 & 1\\3 & 3 & 9 & -1\\\delta & 4 & -1 & 9\\-4 & -4 & -15 & 2\end{matrix}\right]
\qquad b = \left[\begin{matrix}-14\\31\\42\\-45\end{matrix}\right]
\end{align*}
\newpage
Вариант N 2


Решить СЛАУ c параметром тремя способами (расширенная матрица, список уравнений, матричная форма).

Вначале составить список уравнений и решить вторым способом,
затем список уравнений преобразовать в матричный вид и решить третьим способом.
Затем составить из матрицы левой части и столбца правой расширенную матрицу СЛАУ и решить первым способом.
После этого провести проверку подстановкой.

Затем отдельно рассмотреть значение параметра, при котором решение СЛАУ нельзя найти по общей формуле,
полученной ранее.
Найти решение СЛАУ при этом значении параметра первым или третьим способом, используя подстановку subs.
\begin{align*}
    A = \left[\begin{matrix}-6 & -1 & 3 & 7\\-4 & 9 & -8 & 7\\7 & t & -2 & 3\\-2 & -10 & 11 & 0\end{matrix}\right]
\qquad b = \left[\begin{matrix}48\\59\\-4\\-11\end{matrix}\right]
\end{align*}
\newpage
Вариант N 3


Решить СЛАУ c параметром тремя способами (расширенная матрица, список уравнений, матричная форма).

Вначале составить список уравнений и решить вторым способом,
затем список уравнений преобразовать в матричный вид и решить третьим способом.
Затем составить из матрицы левой части и столбца правой расширенную матрицу СЛАУ и решить первым способом.
После этого провести проверку подстановкой.

Затем отдельно рассмотреть значение параметра, при котором решение СЛАУ нельзя найти по общей формуле,
полученной ранее.
Найти решение СЛАУ при этом значении параметра первым или третьим способом, используя подстановку subs.
\begin{align*}
    A = \left[\begin{matrix}-6 & 6 & -3 & 5\\1 & 6 & -6 & -1\\\gamma & -4 & -6 & -3\\-7 & 0 & 3 & 6\end{matrix}\right]
\qquad b = \left[\begin{matrix}-1\\-18\\-34\\17\end{matrix}\right]
\end{align*}
\newpage
Вариант N 4


Решить СЛАУ c параметром тремя способами (расширенная матрица, список уравнений, матричная форма).

Вначале составить список уравнений и решить вторым способом,
затем список уравнений преобразовать в матричный вид и решить третьим способом.
Затем составить из матрицы левой части и столбца правой расширенную матрицу СЛАУ и решить первым способом.
После этого провести проверку подстановкой.

Затем отдельно рассмотреть значение параметра, при котором решение СЛАУ нельзя найти по общей формуле,
полученной ранее.
Найти решение СЛАУ при этом значении параметра первым или третьим способом, используя подстановку subs.
\begin{align*}
    A = \left[\begin{matrix}-9 & 3 & -2 & -4\\-7 & 2 & 7 & 8\\9 & b & 4 & 3\\-2 & 1 & -9 & -12\end{matrix}\right]
\qquad b = \left[\begin{matrix}-87\\-39\\75\\-48\end{matrix}\right]
\end{align*}
\newpage
Вариант N 5


Решить СЛАУ c параметром тремя способами (расширенная матрица, список уравнений, матричная форма).

Вначале составить список уравнений и решить вторым способом,
затем список уравнений преобразовать в матричный вид и решить третьим способом.
Затем составить из матрицы левой части и столбца правой расширенную матрицу СЛАУ и решить первым способом.
После этого провести проверку подстановкой.

Затем отдельно рассмотреть значение параметра, при котором решение СЛАУ нельзя найти по общей формуле,
полученной ранее.
Найти решение СЛАУ при этом значении параметра первым или третьим способом, используя подстановку subs.
\begin{align*}
    A = \left[\begin{matrix}8 & -5 & 6 & -8\\2 & -7 & 0 & 1\\-9 & \delta & 9 & 5\\6 & 2 & 6 & -9\end{matrix}\right]
\qquad b = \left[\begin{matrix}124\\52\\42\\72\end{matrix}\right]
\end{align*}
\newpage
Вариант N 6


Решить СЛАУ c параметром тремя способами (расширенная матрица, список уравнений, матричная форма).

Вначале составить список уравнений и решить вторым способом,
затем список уравнений преобразовать в матричный вид и решить третьим способом.
Затем составить из матрицы левой части и столбца правой расширенную матрицу СЛАУ и решить первым способом.
После этого провести проверку подстановкой.

Затем отдельно рассмотреть значение параметра, при котором решение СЛАУ нельзя найти по общей формуле,
полученной ранее.
Найти решение СЛАУ при этом значении параметра первым или третьим способом, используя подстановку subs.
\begin{align*}
    A = \left[\begin{matrix}-4 & 6 & 9 & -1\\1 & 0 & 0 & 9\\5 & \delta & -4 & -6\\-5 & 6 & 9 & -10\end{matrix}\right]
\qquad b = \left[\begin{matrix}34\\-80\\-32\\114\end{matrix}\right]
\end{align*}
\newpage
Вариант N 7


Решить СЛАУ c параметром тремя способами (расширенная матрица, список уравнений, матричная форма).

Вначале составить список уравнений и решить вторым способом,
затем список уравнений преобразовать в матричный вид и решить третьим способом.
Затем составить из матрицы левой части и столбца правой расширенную матрицу СЛАУ и решить первым способом.
После этого провести проверку подстановкой.

Затем отдельно рассмотреть значение параметра, при котором решение СЛАУ нельзя найти по общей формуле,
полученной ранее.
Найти решение СЛАУ при этом значении параметра первым или третьим способом, используя подстановку subs.
\begin{align*}
    A = \left[\begin{matrix}-9 & -5 & -8 & 1\\-1 & -3 & 8 & 9\\6 & \gamma & -6 & -2\\-8 & -2 & -16 & -8\end{matrix}\right]
\qquad b = \left[\begin{matrix}56\\-2\\-21\\58\end{matrix}\right]
\end{align*}
\newpage
Вариант N 8


Решить СЛАУ c параметром тремя способами (расширенная матрица, список уравнений, матричная форма).

Вначале составить список уравнений и решить вторым способом,
затем список уравнений преобразовать в матричный вид и решить третьим способом.
Затем составить из матрицы левой части и столбца правой расширенную матрицу СЛАУ и решить первым способом.
После этого провести проверку подстановкой.

Затем отдельно рассмотреть значение параметра, при котором решение СЛАУ нельзя найти по общей формуле,
полученной ранее.
Найти решение СЛАУ при этом значении параметра первым или третьим способом, используя подстановку subs.
\begin{align*}
    A = \left[\begin{matrix}-5 & 9 & -1 & 7\\-9 & -2 & 4 & -5\\1 & \beta & -2 & 4\\4 & 11 & -5 & 12\end{matrix}\right]
\qquad b = \left[\begin{matrix}11\\-38\\8\\49\end{matrix}\right]
\end{align*}
\newpage
Вариант N 9


Решить СЛАУ c параметром тремя способами (расширенная матрица, список уравнений, матричная форма).

Вначале составить список уравнений и решить вторым способом,
затем список уравнений преобразовать в матричный вид и решить третьим способом.
Затем составить из матрицы левой части и столбца правой расширенную матрицу СЛАУ и решить первым способом.
После этого провести проверку подстановкой.

Затем отдельно рассмотреть значение параметра, при котором решение СЛАУ нельзя найти по общей формуле,
полученной ранее.
Найти решение СЛАУ при этом значении параметра первым или третьим способом, используя подстановку subs.
\begin{align*}
    A = \left[\begin{matrix}4 & 2 & -1 & -9\\7 & -7 & -8 & -7\\-3 & \alpha & 2 & -4\\-3 & 9 & 7 & -2\end{matrix}\right]
\qquad b = \left[\begin{matrix}-39\\-86\\-21\\47\end{matrix}\right]
\end{align*}
\newpage
Вариант N 10


Решить СЛАУ c параметром тремя способами (расширенная матрица, список уравнений, матричная форма).

Вначале составить список уравнений и решить вторым способом,
затем список уравнений преобразовать в матричный вид и решить третьим способом.
Затем составить из матрицы левой части и столбца правой расширенную матрицу СЛАУ и решить первым способом.
После этого провести проверку подстановкой.

Затем отдельно рассмотреть значение параметра, при котором решение СЛАУ нельзя найти по общей формуле,
полученной ранее.
Найти решение СЛАУ при этом значении параметра первым или третьим способом, используя подстановку subs.
\begin{align*}
    A = \left[\begin{matrix}-3 & -7 & -3 & 6\\-6 & -7 & -9 & 4\\-7 & \mu & 0 & -9\\3 & 0 & 6 & 2\end{matrix}\right]
\qquad b = \left[\begin{matrix}-45\\-49\\79\\4\end{matrix}\right]
\end{align*}
\newpage
Вариант N 11


Решить СЛАУ c параметром тремя способами (расширенная матрица, список уравнений, матричная форма).

Вначале составить список уравнений и решить вторым способом,
затем список уравнений преобразовать в матричный вид и решить третьим способом.
Затем составить из матрицы левой части и столбца правой расширенную матрицу СЛАУ и решить первым способом.
После этого провести проверку подстановкой.

Затем отдельно рассмотреть значение параметра, при котором решение СЛАУ нельзя найти по общей формуле,
полученной ранее.
Найти решение СЛАУ при этом значении параметра первым или третьим способом, используя подстановку subs.
\begin{align*}
    A = \left[\begin{matrix}-1 & -6 & -8 & -8\\-5 & 1 & -2 & 9\\t & -5 & 1 & 6\\4 & -7 & -6 & -17\end{matrix}\right]
\qquad b = \left[\begin{matrix}85\\-37\\-10\\122\end{matrix}\right]
\end{align*}
\newpage
Вариант N 12


Решить СЛАУ c параметром тремя способами (расширенная матрица, список уравнений, матричная форма).

Вначале составить список уравнений и решить вторым способом,
затем список уравнений преобразовать в матричный вид и решить третьим способом.
Затем составить из матрицы левой части и столбца правой расширенную матрицу СЛАУ и решить первым способом.
После этого провести проверку подстановкой.

Затем отдельно рассмотреть значение параметра, при котором решение СЛАУ нельзя найти по общей формуле,
полученной ранее.
Найти решение СЛАУ при этом значении параметра первым или третьим способом, используя подстановку subs.
\begin{align*}
    A = \left[\begin{matrix}-5 & 2 & 4 & 8\\5 & -9 & 1 & 2\\-1 & c & -2 & -2\\-10 & 11 & 3 & 6\end{matrix}\right]
\qquad b = \left[\begin{matrix}46\\-23\\3\\69\end{matrix}\right]
\end{align*}
\newpage
Вариант N 13


Решить СЛАУ c параметром тремя способами (расширенная матрица, список уравнений, матричная форма).

Вначале составить список уравнений и решить вторым способом,
затем список уравнений преобразовать в матричный вид и решить третьим способом.
Затем составить из матрицы левой части и столбца правой расширенную матрицу СЛАУ и решить первым способом.
После этого провести проверку подстановкой.

Затем отдельно рассмотреть значение параметра, при котором решение СЛАУ нельзя найти по общей формуле,
полученной ранее.
Найти решение СЛАУ при этом значении параметра первым или третьим способом, используя подстановку subs.
\begin{align*}
    A = \left[\begin{matrix}4 & -9 & -5 & 4\\-4 & -9 & -1 & 6\\6 & s & 2 & 1\\8 & 0 & -4 & -2\end{matrix}\right]
\qquad b = \left[\begin{matrix}110\\42\\60\\68\end{matrix}\right]
\end{align*}
\newpage
Вариант N 14


Решить СЛАУ c параметром тремя способами (расширенная матрица, список уравнений, матричная форма).

Вначале составить список уравнений и решить вторым способом,
затем список уравнений преобразовать в матричный вид и решить третьим способом.
Затем составить из матрицы левой части и столбца правой расширенную матрицу СЛАУ и решить первым способом.
После этого провести проверку подстановкой.

Затем отдельно рассмотреть значение параметра, при котором решение СЛАУ нельзя найти по общей формуле,
полученной ранее.
Найти решение СЛАУ при этом значении параметра первым или третьим способом, используя подстановку subs.
\begin{align*}
    A = \left[\begin{matrix}-2 & 9 & 5 & -2\\6 & -7 & 3 & 4\\\mu & -8 & -5 & -7\\-8 & 16 & 2 & -6\end{matrix}\right]
\qquad b = \left[\begin{matrix}68\\-2\\11\\70\end{matrix}\right]
\end{align*}
\newpage
Вариант N 15


Решить СЛАУ c параметром тремя способами (расширенная матрица, список уравнений, матричная форма).

Вначале составить список уравнений и решить вторым способом,
затем список уравнений преобразовать в матричный вид и решить третьим способом.
Затем составить из матрицы левой части и столбца правой расширенную матрицу СЛАУ и решить первым способом.
После этого провести проверку подстановкой.

Затем отдельно рассмотреть значение параметра, при котором решение СЛАУ нельзя найти по общей формуле,
полученной ранее.
Найти решение СЛАУ при этом значении параметра первым или третьим способом, используя подстановку subs.
\begin{align*}
    A = \left[\begin{matrix}4 & -5 & 7 & -6\\6 & 8 & 5 & 8\\-2 & \beta & -5 & -6\\-2 & -13 & 2 & -14\end{matrix}\right]
\qquad b = \left[\begin{matrix}-33\\-38\\48\\5\end{matrix}\right]
\end{align*}
\newpage
Вариант N 16


Решить СЛАУ c параметром тремя способами (расширенная матрица, список уравнений, матричная форма).

Вначале составить список уравнений и решить вторым способом,
затем список уравнений преобразовать в матричный вид и решить третьим способом.
Затем составить из матрицы левой части и столбца правой расширенную матрицу СЛАУ и решить первым способом.
После этого провести проверку подстановкой.

Затем отдельно рассмотреть значение параметра, при котором решение СЛАУ нельзя найти по общей формуле,
полученной ранее.
Найти решение СЛАУ при этом значении параметра первым или третьим способом, используя подстановку subs.
\begin{align*}
    A = \left[\begin{matrix}3 & 4 & -5 & 6\\-3 & 7 & -9 & 5\\t & 8 & 4 & 0\\6 & -3 & 4 & 1\end{matrix}\right]
\qquad b = \left[\begin{matrix}-14\\-9\\20\\-5\end{matrix}\right]
\end{align*}
\newpage
Вариант N 17


Решить СЛАУ c параметром тремя способами (расширенная матрица, список уравнений, матричная форма).

Вначале составить список уравнений и решить вторым способом,
затем список уравнений преобразовать в матричный вид и решить третьим способом.
Затем составить из матрицы левой части и столбца правой расширенную матрицу СЛАУ и решить первым способом.
После этого провести проверку подстановкой.

Затем отдельно рассмотреть значение параметра, при котором решение СЛАУ нельзя найти по общей формуле,
полученной ранее.
Найти решение СЛАУ при этом значении параметра первым или третьим способом, используя подстановку subs.
\begin{align*}
    A = \left[\begin{matrix}9 & 2 & 9 & 8\\4 & 0 & -6 & 0\\t & 5 & 7 & 7\\5 & 2 & 15 & 8\end{matrix}\right]
\qquad b = \left[\begin{matrix}48\\20\\22\\28\end{matrix}\right]
\end{align*}
\newpage
Вариант N 18


Решить СЛАУ c параметром тремя способами (расширенная матрица, список уравнений, матричная форма).

Вначале составить список уравнений и решить вторым способом,
затем список уравнений преобразовать в матричный вид и решить третьим способом.
Затем составить из матрицы левой части и столбца правой расширенную матрицу СЛАУ и решить первым способом.
После этого провести проверку подстановкой.

Затем отдельно рассмотреть значение параметра, при котором решение СЛАУ нельзя найти по общей формуле,
полученной ранее.
Найти решение СЛАУ при этом значении параметра первым или третьим способом, используя подстановку subs.
\begin{align*}
    A = \left[\begin{matrix}8 & -1 & -6 & 4\\-1 & -5 & 4 & -8\\-1 & k & 5 & 4\\9 & 4 & -10 & 12\end{matrix}\right]
\qquad b = \left[\begin{matrix}-96\\139\\-61\\-235\end{matrix}\right]
\end{align*}
\newpage
Вариант N 19


Решить СЛАУ c параметром тремя способами (расширенная матрица, список уравнений, матричная форма).

Вначале составить список уравнений и решить вторым способом,
затем список уравнений преобразовать в матричный вид и решить третьим способом.
Затем составить из матрицы левой части и столбца правой расширенную матрицу СЛАУ и решить первым способом.
После этого провести проверку подстановкой.

Затем отдельно рассмотреть значение параметра, при котором решение СЛАУ нельзя найти по общей формуле,
полученной ранее.
Найти решение СЛАУ при этом значении параметра первым или третьим способом, используя подстановку subs.
\begin{align*}
    A = \left[\begin{matrix}-6 & 8 & -7 & -1\\-8 & 8 & 7 & 6\\-5 & \gamma & -2 & -2\\2 & 0 & -14 & -7\end{matrix}\right]
\qquad b = \left[\begin{matrix}-137\\-79\\-29\\-58\end{matrix}\right]
\end{align*}
\newpage
Вариант N 20


Решить СЛАУ c параметром тремя способами (расширенная матрица, список уравнений, матричная форма).

Вначале составить список уравнений и решить вторым способом,
затем список уравнений преобразовать в матричный вид и решить третьим способом.
Затем составить из матрицы левой части и столбца правой расширенную матрицу СЛАУ и решить первым способом.
После этого провести проверку подстановкой.

Затем отдельно рассмотреть значение параметра, при котором решение СЛАУ нельзя найти по общей формуле,
полученной ранее.
Найти решение СЛАУ при этом значении параметра первым или третьим способом, используя подстановку subs.
\begin{align*}
    A = \left[\begin{matrix}0 & -1 & -6 & -5\\-2 & 6 & 1 & 8\\\delta & 2 & -6 & -9\\2 & -7 & -7 & -13\end{matrix}\right]
\qquad b = \left[\begin{matrix}27\\62\\68\\-35\end{matrix}\right]
\end{align*}
\newpage
Вариант N 21


Решить СЛАУ c параметром тремя способами (расширенная матрица, список уравнений, матричная форма).

Вначале составить список уравнений и решить вторым способом,
затем список уравнений преобразовать в матричный вид и решить третьим способом.
Затем составить из матрицы левой части и столбца правой расширенную матрицу СЛАУ и решить первым способом.
После этого провести проверку подстановкой.

Затем отдельно рассмотреть значение параметра, при котором решение СЛАУ нельзя найти по общей формуле,
полученной ранее.
Найти решение СЛАУ при этом значении параметра первым или третьим способом, используя подстановку subs.
\begin{align*}
    A = \left[\begin{matrix}-8 & -6 & -1 & -3\\6 & -7 & -8 & 7\\\alpha & -3 & -6 & -4\\-14 & 1 & 7 & -10\end{matrix}\right]
\qquad b = \left[\begin{matrix}-132\\-42\\-33\\-90\end{matrix}\right]
\end{align*}
\newpage
Вариант N 22


Решить СЛАУ c параметром тремя способами (расширенная матрица, список уравнений, матричная форма).

Вначале составить список уравнений и решить вторым способом,
затем список уравнений преобразовать в матричный вид и решить третьим способом.
Затем составить из матрицы левой части и столбца правой расширенную матрицу СЛАУ и решить первым способом.
После этого провести проверку подстановкой.

Затем отдельно рассмотреть значение параметра, при котором решение СЛАУ нельзя найти по общей формуле,
полученной ранее.
Найти решение СЛАУ при этом значении параметра первым или третьим способом, используя подстановку subs.
\begin{align*}
    A = \left[\begin{matrix}5 & 0 & -8 & 9\\-2 & 3 & -2 & 4\\7 & \gamma & 7 & 9\\7 & -3 & -6 & 5\end{matrix}\right]
\qquad b = \left[\begin{matrix}-15\\32\\113\\-47\end{matrix}\right]
\end{align*}
\newpage
Вариант N 23


Решить СЛАУ c параметром тремя способами (расширенная матрица, список уравнений, матричная форма).

Вначале составить список уравнений и решить вторым способом,
затем список уравнений преобразовать в матричный вид и решить третьим способом.
Затем составить из матрицы левой части и столбца правой расширенную матрицу СЛАУ и решить первым способом.
После этого провести проверку подстановкой.

Затем отдельно рассмотреть значение параметра, при котором решение СЛАУ нельзя найти по общей формуле,
полученной ранее.
Найти решение СЛАУ при этом значении параметра первым или третьим способом, используя подстановку subs.
\begin{align*}
    A = \left[\begin{matrix}5 & 0 & -8 & 9\\2 & -8 & 9 & -9\\0 & b & -1 & 0\\3 & 8 & -17 & 18\end{matrix}\right]
\qquad b = \left[\begin{matrix}-1\\67\\-8\\-68\end{matrix}\right]
\end{align*}
\newpage
Вариант N 24


Решить СЛАУ c параметром тремя способами (расширенная матрица, список уравнений, матричная форма).

Вначале составить список уравнений и решить вторым способом,
затем список уравнений преобразовать в матричный вид и решить третьим способом.
Затем составить из матрицы левой части и столбца правой расширенную матрицу СЛАУ и решить первым способом.
После этого провести проверку подстановкой.

Затем отдельно рассмотреть значение параметра, при котором решение СЛАУ нельзя найти по общей формуле,
полученной ранее.
Найти решение СЛАУ при этом значении параметра первым или третьим способом, используя подстановку subs.
\begin{align*}
    A = \left[\begin{matrix}7 & -5 & 8 & 3\\-6 & -8 & -5 & -8\\c & 8 & 0 & 3\\13 & 3 & 13 & 11\end{matrix}\right]
\qquad b = \left[\begin{matrix}68\\-3\\-27\\71\end{matrix}\right]
\end{align*}
\newpage
Вариант N 25


Решить СЛАУ c параметром тремя способами (расширенная матрица, список уравнений, матричная форма).

Вначале составить список уравнений и решить вторым способом,
затем список уравнений преобразовать в матричный вид и решить третьим способом.
Затем составить из матрицы левой части и столбца правой расширенную матрицу СЛАУ и решить первым способом.
После этого провести проверку подстановкой.

Затем отдельно рассмотреть значение параметра, при котором решение СЛАУ нельзя найти по общей формуле,
полученной ранее.
Найти решение СЛАУ при этом значении параметра первым или третьим способом, используя подстановку subs.
\begin{align*}
    A = \left[\begin{matrix}6 & -4 & 2 & -1\\-8 & 6 & 7 & -4\\0 & \gamma & 3 & -8\\14 & -10 & -5 & 3\end{matrix}\right]
\qquad b = \left[\begin{matrix}-80\\52\\-60\\-132\end{matrix}\right]
\end{align*}
\newpage
Вариант N 26


Решить СЛАУ c параметром тремя способами (расширенная матрица, список уравнений, матричная форма).

Вначале составить список уравнений и решить вторым способом,
затем список уравнений преобразовать в матричный вид и решить третьим способом.
Затем составить из матрицы левой части и столбца правой расширенную матрицу СЛАУ и решить первым способом.
После этого провести проверку подстановкой.

Затем отдельно рассмотреть значение параметра, при котором решение СЛАУ нельзя найти по общей формуле,
полученной ранее.
Найти решение СЛАУ при этом значении параметра первым или третьим способом, используя подстановку subs.
\begin{align*}
    A = \left[\begin{matrix}9 & 1 & 8 & 8\\3 & 4 & -7 & -9\\k & -8 & 8 & 9\\6 & -3 & 15 & 17\end{matrix}\right]
\qquad b = \left[\begin{matrix}-17\\130\\-143\\-147\end{matrix}\right]
\end{align*}
\newpage
Вариант N 27


Решить СЛАУ c параметром тремя способами (расширенная матрица, список уравнений, матричная форма).

Вначале составить список уравнений и решить вторым способом,
затем список уравнений преобразовать в матричный вид и решить третьим способом.
Затем составить из матрицы левой части и столбца правой расширенную матрицу СЛАУ и решить первым способом.
После этого провести проверку подстановкой.

Затем отдельно рассмотреть значение параметра, при котором решение СЛАУ нельзя найти по общей формуле,
полученной ранее.
Найти решение СЛАУ при этом значении параметра первым или третьим способом, используя подстановку subs.
\begin{align*}
    A = \left[\begin{matrix}0 & 2 & -9 & -7\\-2 & 1 & -7 & -9\\t & 4 & 3 & 1\\2 & 1 & -2 & 2\end{matrix}\right]
\qquad b = \left[\begin{matrix}52\\16\\-20\\36\end{matrix}\right]
\end{align*}
\newpage
Вариант N 28


Решить СЛАУ c параметром тремя способами (расширенная матрица, список уравнений, матричная форма).

Вначале составить список уравнений и решить вторым способом,
затем список уравнений преобразовать в матричный вид и решить третьим способом.
Затем составить из матрицы левой части и столбца правой расширенную матрицу СЛАУ и решить первым способом.
После этого провести проверку подстановкой.

Затем отдельно рассмотреть значение параметра, при котором решение СЛАУ нельзя найти по общей формуле,
полученной ранее.
Найти решение СЛАУ при этом значении параметра первым или третьим способом, используя подстановку subs.
\begin{align*}
    A = \left[\begin{matrix}6 & 1 & 1 & -2\\5 & 2 & -5 & -1\\0 & k & -9 & 6\\1 & -1 & 6 & -1\end{matrix}\right]
\qquad b = \left[\begin{matrix}42\\6\\-78\\36\end{matrix}\right]
\end{align*}
\newpage
Вариант N 29


Решить СЛАУ c параметром тремя способами (расширенная матрица, список уравнений, матричная форма).

Вначале составить список уравнений и решить вторым способом,
затем список уравнений преобразовать в матричный вид и решить третьим способом.
Затем составить из матрицы левой части и столбца правой расширенную матрицу СЛАУ и решить первым способом.
После этого провести проверку подстановкой.

Затем отдельно рассмотреть значение параметра, при котором решение СЛАУ нельзя найти по общей формуле,
полученной ранее.
Найти решение СЛАУ при этом значении параметра первым или третьим способом, используя подстановку subs.
\begin{align*}
    A = \left[\begin{matrix}-6 & 6 & -1 & -9\\0 & 3 & -6 & -6\\\delta & 4 & -7 & -7\\-6 & 3 & 5 & -3\end{matrix}\right]
\qquad b = \left[\begin{matrix}32\\6\\38\\26\end{matrix}\right]
\end{align*}
\newpage
Вариант N 30


Решить СЛАУ c параметром тремя способами (расширенная матрица, список уравнений, матричная форма).

Вначале составить список уравнений и решить вторым способом,
затем список уравнений преобразовать в матричный вид и решить третьим способом.
Затем составить из матрицы левой части и столбца правой расширенную матрицу СЛАУ и решить первым способом.
После этого провести проверку подстановкой.

Затем отдельно рассмотреть значение параметра, при котором решение СЛАУ нельзя найти по общей формуле,
полученной ранее.
Найти решение СЛАУ при этом значении параметра первым или третьим способом, используя подстановку subs.
\begin{align*}
    A = \left[\begin{matrix}-8 & -5 & -1 & -7\\-7 & -2 & -9 & 0\\6 & t & 1 & -7\\-1 & -3 & 8 & -7\end{matrix}\right]
\qquad b = \left[\begin{matrix}-94\\-15\\-26\\-79\end{matrix}\right]
\end{align*}
\newpage
Вариант N 31


Решить СЛАУ c параметром тремя способами (расширенная матрица, список уравнений, матричная форма).

Вначале составить список уравнений и решить вторым способом,
затем список уравнений преобразовать в матричный вид и решить третьим способом.
Затем составить из матрицы левой части и столбца правой расширенную матрицу СЛАУ и решить первым способом.
После этого провести проверку подстановкой.

Затем отдельно рассмотреть значение параметра, при котором решение СЛАУ нельзя найти по общей формуле,
полученной ранее.
Найти решение СЛАУ при этом значении параметра первым или третьим способом, используя подстановку subs.
\begin{align*}
    A = \left[\begin{matrix}-7 & 5 & 2 & -4\\-9 & 7 & 9 & -4\\t & -8 & 2 & 1\\2 & -2 & -7 & 0\end{matrix}\right]
\qquad b = \left[\begin{matrix}-44\\-56\\-36\\12\end{matrix}\right]
\end{align*}
\newpage
Вариант N 32


Решить СЛАУ c параметром тремя способами (расширенная матрица, список уравнений, матричная форма).

Вначале составить список уравнений и решить вторым способом,
затем список уравнений преобразовать в матричный вид и решить третьим способом.
Затем составить из матрицы левой части и столбца правой расширенную матрицу СЛАУ и решить первым способом.
После этого провести проверку подстановкой.

Затем отдельно рассмотреть значение параметра, при котором решение СЛАУ нельзя найти по общей формуле,
полученной ранее.
Найти решение СЛАУ при этом значении параметра первым или третьим способом, используя подстановку subs.
\begin{align*}
    A = \left[\begin{matrix}8 & 7 & 0 & 9\\-1 & -7 & 3 & -8\\b & -8 & 9 & -3\\9 & 14 & -3 & 17\end{matrix}\right]
\qquad b = \left[\begin{matrix}33\\-9\\60\\42\end{matrix}\right]
\end{align*}
\newpage
Вариант N 33


Решить СЛАУ c параметром тремя способами (расширенная матрица, список уравнений, матричная форма).

Вначале составить список уравнений и решить вторым способом,
затем список уравнений преобразовать в матричный вид и решить третьим способом.
Затем составить из матрицы левой части и столбца правой расширенную матрицу СЛАУ и решить первым способом.
После этого провести проверку подстановкой.

Затем отдельно рассмотреть значение параметра, при котором решение СЛАУ нельзя найти по общей формуле,
полученной ранее.
Найти решение СЛАУ при этом значении параметра первым или третьим способом, используя подстановку subs.
\begin{align*}
    A = \left[\begin{matrix}8 & 6 & 3 & 2\\-3 & 9 & -4 & 5\\\delta & -9 & -9 & -2\\11 & -3 & 7 & -3\end{matrix}\right]
\qquad b = \left[\begin{matrix}-7\\-68\\22\\61\end{matrix}\right]
\end{align*}
\newpage
Вариант N 34


Решить СЛАУ c параметром тремя способами (расширенная матрица, список уравнений, матричная форма).

Вначале составить список уравнений и решить вторым способом,
затем список уравнений преобразовать в матричный вид и решить третьим способом.
Затем составить из матрицы левой части и столбца правой расширенную матрицу СЛАУ и решить первым способом.
После этого провести проверку подстановкой.

Затем отдельно рассмотреть значение параметра, при котором решение СЛАУ нельзя найти по общей формуле,
полученной ранее.
Найти решение СЛАУ при этом значении параметра первым или третьим способом, используя подстановку subs.
\begin{align*}
    A = \left[\begin{matrix}4 & -8 & -2 & -2\\3 & -1 & -6 & 6\\\beta & -6 & 1 & 1\\1 & -7 & 4 & -8\end{matrix}\right]
\qquad b = \left[\begin{matrix}-24\\-28\\-28\\4\end{matrix}\right]
\end{align*}
\newpage
Вариант N 35


Решить СЛАУ c параметром тремя способами (расширенная матрица, список уравнений, матричная форма).

Вначале составить список уравнений и решить вторым способом,
затем список уравнений преобразовать в матричный вид и решить третьим способом.
Затем составить из матрицы левой части и столбца правой расширенную матрицу СЛАУ и решить первым способом.
После этого провести проверку подстановкой.

Затем отдельно рассмотреть значение параметра, при котором решение СЛАУ нельзя найти по общей формуле,
полученной ранее.
Найти решение СЛАУ при этом значении параметра первым или третьим способом, используя подстановку subs.
\begin{align*}
    A = \left[\begin{matrix}5 & -5 & 2 & -1\\-8 & -9 & 8 & 8\\k & -1 & 7 & 6\\13 & 4 & -6 & -9\end{matrix}\right]
\qquad b = \left[\begin{matrix}0\\-69\\-11\\69\end{matrix}\right]
\end{align*}
\newpage
Вариант N 36


Решить СЛАУ c параметром тремя способами (расширенная матрица, список уравнений, матричная форма).

Вначале составить список уравнений и решить вторым способом,
затем список уравнений преобразовать в матричный вид и решить третьим способом.
Затем составить из матрицы левой части и столбца правой расширенную матрицу СЛАУ и решить первым способом.
После этого провести проверку подстановкой.

Затем отдельно рассмотреть значение параметра, при котором решение СЛАУ нельзя найти по общей формуле,
полученной ранее.
Найти решение СЛАУ при этом значении параметра первым или третьим способом, используя подстановку subs.
\begin{align*}
    A = \left[\begin{matrix}-7 & -9 & 1 & 4\\8 & 8 & 8 & 4\\\delta & 3 & -4 & 5\\-15 & -17 & -7 & 0\end{matrix}\right]
\qquad b = \left[\begin{matrix}-142\\12\\-21\\-154\end{matrix}\right]
\end{align*}
\newpage
Вариант N 37


Решить СЛАУ c параметром тремя способами (расширенная матрица, список уравнений, матричная форма).

Вначале составить список уравнений и решить вторым способом,
затем список уравнений преобразовать в матричный вид и решить третьим способом.
Затем составить из матрицы левой части и столбца правой расширенную матрицу СЛАУ и решить первым способом.
После этого провести проверку подстановкой.

Затем отдельно рассмотреть значение параметра, при котором решение СЛАУ нельзя найти по общей формуле,
полученной ранее.
Найти решение СЛАУ при этом значении параметра первым или третьим способом, используя подстановку subs.
\begin{align*}
    A = \left[\begin{matrix}6 & -5 & 3 & -4\\2 & 0 & -5 & -9\\-5 & \mu & -2 & 4\\4 & -5 & 8 & 5\end{matrix}\right]
\qquad b = \left[\begin{matrix}-13\\65\\12\\-78\end{matrix}\right]
\end{align*}
\newpage
Вариант N 38


Решить СЛАУ c параметром тремя способами (расширенная матрица, список уравнений, матричная форма).

Вначале составить список уравнений и решить вторым способом,
затем список уравнений преобразовать в матричный вид и решить третьим способом.
Затем составить из матрицы левой части и столбца правой расширенную матрицу СЛАУ и решить первым способом.
После этого провести проверку подстановкой.

Затем отдельно рассмотреть значение параметра, при котором решение СЛАУ нельзя найти по общей формуле,
полученной ранее.
Найти решение СЛАУ при этом значении параметра первым или третьим способом, используя подстановку subs.
\begin{align*}
    A = \left[\begin{matrix}3 & 3 & -9 & -5\\0 & 4 & -3 & -4\\t & -5 & 0 & -9\\3 & -1 & -6 & -1\end{matrix}\right]
\qquad b = \left[\begin{matrix}-76\\-7\\19\\-69\end{matrix}\right]
\end{align*}
\newpage
Вариант N 39


Решить СЛАУ c параметром тремя способами (расширенная матрица, список уравнений, матричная форма).

Вначале составить список уравнений и решить вторым способом,
затем список уравнений преобразовать в матричный вид и решить третьим способом.
Затем составить из матрицы левой части и столбца правой расширенную матрицу СЛАУ и решить первым способом.
После этого провести проверку подстановкой.

Затем отдельно рассмотреть значение параметра, при котором решение СЛАУ нельзя найти по общей формуле,
полученной ранее.
Найти решение СЛАУ при этом значении параметра первым или третьим способом, используя подстановку subs.
\begin{align*}
    A = \left[\begin{matrix}6 & 1 & -3 & -5\\9 & -2 & 7 & -1\\c & -1 & 0 & 7\\-3 & 3 & -10 & -4\end{matrix}\right]
\qquad b = \left[\begin{matrix}-12\\102\\13\\-114\end{matrix}\right]
\end{align*}
\newpage
Вариант N 40


Решить СЛАУ c параметром тремя способами (расширенная матрица, список уравнений, матричная форма).

Вначале составить список уравнений и решить вторым способом,
затем список уравнений преобразовать в матричный вид и решить третьим способом.
Затем составить из матрицы левой части и столбца правой расширенную матрицу СЛАУ и решить первым способом.
После этого провести проверку подстановкой.

Затем отдельно рассмотреть значение параметра, при котором решение СЛАУ нельзя найти по общей формуле,
полученной ранее.
Найти решение СЛАУ при этом значении параметра первым или третьим способом, используя подстановку subs.
\begin{align*}
    A = \left[\begin{matrix}7 & 2 & 8 & -2\\-3 & 7 & 0 & -8\\4 & b & -9 & -1\\10 & -5 & 8 & 6\end{matrix}\right]
\qquad b = \left[\begin{matrix}17\\-49\\47\\66\end{matrix}\right]
\end{align*}
\newpage
Вариант N 41


Решить СЛАУ c параметром тремя способами (расширенная матрица, список уравнений, матричная форма).

Вначале составить список уравнений и решить вторым способом,
затем список уравнений преобразовать в матричный вид и решить третьим способом.
Затем составить из матрицы левой части и столбца правой расширенную матрицу СЛАУ и решить первым способом.
После этого провести проверку подстановкой.

Затем отдельно рассмотреть значение параметра, при котором решение СЛАУ нельзя найти по общей формуле,
полученной ранее.
Найти решение СЛАУ при этом значении параметра первым или третьим способом, используя подстановку subs.
\begin{align*}
    A = \left[\begin{matrix}-4 & -2 & -9 & 5\\-8 & 0 & -5 & -3\\8 & k & -3 & -3\\4 & -2 & -4 & 8\end{matrix}\right]
\qquad b = \left[\begin{matrix}-12\\-24\\52\\12\end{matrix}\right]
\end{align*}
\newpage
Вариант N 42


Решить СЛАУ c параметром тремя способами (расширенная матрица, список уравнений, матричная форма).

Вначале составить список уравнений и решить вторым способом,
затем список уравнений преобразовать в матричный вид и решить третьим способом.
Затем составить из матрицы левой части и столбца правой расширенную матрицу СЛАУ и решить первым способом.
После этого провести проверку подстановкой.

Затем отдельно рассмотреть значение параметра, при котором решение СЛАУ нельзя найти по общей формуле,
полученной ранее.
Найти решение СЛАУ при этом значении параметра первым или третьим способом, используя подстановку subs.
\begin{align*}
    A = \left[\begin{matrix}-5 & -1 & -3 & 6\\6 & -6 & 6 & -2\\-6 & \delta & -3 & 8\\-11 & 5 & -9 & 8\end{matrix}\right]
\qquad b = \left[\begin{matrix}-4\\32\\-21\\-36\end{matrix}\right]
\end{align*}
\newpage
Вариант N 43


Решить СЛАУ c параметром тремя способами (расширенная матрица, список уравнений, матричная форма).

Вначале составить список уравнений и решить вторым способом,
затем список уравнений преобразовать в матричный вид и решить третьим способом.
Затем составить из матрицы левой части и столбца правой расширенную матрицу СЛАУ и решить первым способом.
После этого провести проверку подстановкой.

Затем отдельно рассмотреть значение параметра, при котором решение СЛАУ нельзя найти по общей формуле,
полученной ранее.
Найти решение СЛАУ при этом значении параметра первым или третьим способом, используя подстановку subs.
\begin{align*}
    A = \left[\begin{matrix}9 & 3 & -9 & 0\\2 & 2 & -5 & 1\\\alpha & -6 & -6 & 9\\7 & 1 & -4 & -1\end{matrix}\right]
\qquad b = \left[\begin{matrix}57\\34\\48\\23\end{matrix}\right]
\end{align*}
\newpage
Вариант N 44


Решить СЛАУ c параметром тремя способами (расширенная матрица, список уравнений, матричная форма).

Вначале составить список уравнений и решить вторым способом,
затем список уравнений преобразовать в матричный вид и решить третьим способом.
Затем составить из матрицы левой части и столбца правой расширенную матрицу СЛАУ и решить первым способом.
После этого провести проверку подстановкой.

Затем отдельно рассмотреть значение параметра, при котором решение СЛАУ нельзя найти по общей формуле,
полученной ранее.
Найти решение СЛАУ при этом значении параметра первым или третьим способом, используя подстановку subs.
\begin{align*}
    A = \left[\begin{matrix}1 & -9 & 8 & -9\\-4 & -7 & -6 & -6\\\mu & 8 & -1 & 7\\5 & -2 & 14 & -3\end{matrix}\right]
\qquad b = \left[\begin{matrix}38\\-15\\-41\\53\end{matrix}\right]
\end{align*}
\newpage
Вариант N 45


Решить СЛАУ c параметром тремя способами (расширенная матрица, список уравнений, матричная форма).

Вначале составить список уравнений и решить вторым способом,
затем список уравнений преобразовать в матричный вид и решить третьим способом.
Затем составить из матрицы левой части и столбца правой расширенную матрицу СЛАУ и решить первым способом.
После этого провести проверку подстановкой.

Затем отдельно рассмотреть значение параметра, при котором решение СЛАУ нельзя найти по общей формуле,
полученной ранее.
Найти решение СЛАУ при этом значении параметра первым или третьим способом, используя подстановку subs.
\begin{align*}
    A = \left[\begin{matrix}6 & 7 & 8 & -2\\8 & -8 & -2 & 8\\k & -2 & 6 & -8\\-2 & 15 & 10 & -10\end{matrix}\right]
\qquad b = \left[\begin{matrix}-11\\-54\\20\\43\end{matrix}\right]
\end{align*}
\newpage
Вариант N 46


Решить СЛАУ c параметром тремя способами (расширенная матрица, список уравнений, матричная форма).

Вначале составить список уравнений и решить вторым способом,
затем список уравнений преобразовать в матричный вид и решить третьим способом.
Затем составить из матрицы левой части и столбца правой расширенную матрицу СЛАУ и решить первым способом.
После этого провести проверку подстановкой.

Затем отдельно рассмотреть значение параметра, при котором решение СЛАУ нельзя найти по общей формуле,
полученной ранее.
Найти решение СЛАУ при этом значении параметра первым или третьим способом, используя подстановку subs.
\begin{align*}
    A = \left[\begin{matrix}9 & -3 & 2 & 7\\-4 & -2 & 1 & 9\\\gamma & 3 & 2 & 5\\13 & -1 & 1 & -2\end{matrix}\right]
\qquad b = \left[\begin{matrix}-45\\-24\\-7\\-21\end{matrix}\right]
\end{align*}
\newpage
Вариант N 47


Решить СЛАУ c параметром тремя способами (расширенная матрица, список уравнений, матричная форма).

Вначале составить список уравнений и решить вторым способом,
затем список уравнений преобразовать в матричный вид и решить третьим способом.
Затем составить из матрицы левой части и столбца правой расширенную матрицу СЛАУ и решить первым способом.
После этого провести проверку подстановкой.

Затем отдельно рассмотреть значение параметра, при котором решение СЛАУ нельзя найти по общей формуле,
полученной ранее.
Найти решение СЛАУ при этом значении параметра первым или третьим способом, используя подстановку subs.
\begin{align*}
    A = \left[\begin{matrix}-3 & -9 & -2 & -2\\0 & 4 & 2 & 4\\-1 & b & 0 & -8\\-3 & -13 & -4 & -6\end{matrix}\right]
\qquad b = \left[\begin{matrix}47\\-22\\17\\69\end{matrix}\right]
\end{align*}
\newpage
Вариант N 48


Решить СЛАУ c параметром тремя способами (расширенная матрица, список уравнений, матричная форма).

Вначале составить список уравнений и решить вторым способом,
затем список уравнений преобразовать в матричный вид и решить третьим способом.
Затем составить из матрицы левой части и столбца правой расширенную матрицу СЛАУ и решить первым способом.
После этого провести проверку подстановкой.

Затем отдельно рассмотреть значение параметра, при котором решение СЛАУ нельзя найти по общей формуле,
полученной ранее.
Найти решение СЛАУ при этом значении параметра первым или третьим способом, используя подстановку subs.
\begin{align*}
    A = \left[\begin{matrix}-5 & 4 & 9 & 6\\3 & 0 & -5 & -2\\7 & \mu & 1 & 8\\-8 & 4 & 14 & 8\end{matrix}\right]
\qquad b = \left[\begin{matrix}27\\-9\\17\\36\end{matrix}\right]
\end{align*}
\newpage
Вариант N 49


Решить СЛАУ c параметром тремя способами (расширенная матрица, список уравнений, матричная форма).

Вначале составить список уравнений и решить вторым способом,
затем список уравнений преобразовать в матричный вид и решить третьим способом.
Затем составить из матрицы левой части и столбца правой расширенную матрицу СЛАУ и решить первым способом.
После этого провести проверку подстановкой.

Затем отдельно рассмотреть значение параметра, при котором решение СЛАУ нельзя найти по общей формуле,
полученной ранее.
Найти решение СЛАУ при этом значении параметра первым или третьим способом, используя подстановку subs.
\begin{align*}
    A = \left[\begin{matrix}-9 & -6 & 9 & 3\\-2 & 2 & -9 & 4\\-7 & s & -9 & 8\\-7 & -8 & 18 & -1\end{matrix}\right]
\qquad b = \left[\begin{matrix}87\\-91\\-128\\178\end{matrix}\right]
\end{align*}
\newpage
Вариант N 50


Решить СЛАУ c параметром тремя способами (расширенная матрица, список уравнений, матричная форма).

Вначале составить список уравнений и решить вторым способом,
затем список уравнений преобразовать в матричный вид и решить третьим способом.
Затем составить из матрицы левой части и столбца правой расширенную матрицу СЛАУ и решить первым способом.
После этого провести проверку подстановкой.

Затем отдельно рассмотреть значение параметра, при котором решение СЛАУ нельзя найти по общей формуле,
полученной ранее.
Найти решение СЛАУ при этом значении параметра первым или третьим способом, используя подстановку subs.
\begin{align*}
    A = \left[\begin{matrix}1 & 6 & -1 & -5\\8 & -6 & 6 & 7\\-3 & t & 6 & -2\\-7 & 12 & -7 & -12\end{matrix}\right]
\qquad b = \left[\begin{matrix}58\\-58\\43\\116\end{matrix}\right]
\end{align*}
\newpage
Вариант N 51


Решить СЛАУ c параметром тремя способами (расширенная матрица, список уравнений, матричная форма).

Вначале составить список уравнений и решить вторым способом,
затем список уравнений преобразовать в матричный вид и решить третьим способом.
Затем составить из матрицы левой части и столбца правой расширенную матрицу СЛАУ и решить первым способом.
После этого провести проверку подстановкой.

Затем отдельно рассмотреть значение параметра, при котором решение СЛАУ нельзя найти по общей формуле,
полученной ранее.
Найти решение СЛАУ при этом значении параметра первым или третьим способом, используя подстановку subs.
\begin{align*}
    A = \left[\begin{matrix}6 & -5 & 8 & 7\\2 & -2 & 6 & 5\\-5 & t & 0 & -6\\4 & -3 & 2 & 2\end{matrix}\right]
\qquad b = \left[\begin{matrix}23\\0\\-79\\23\end{matrix}\right]
\end{align*}
\newpage
Вариант N 52


Решить СЛАУ c параметром тремя способами (расширенная матрица, список уравнений, матричная форма).

Вначале составить список уравнений и решить вторым способом,
затем список уравнений преобразовать в матричный вид и решить третьим способом.
Затем составить из матрицы левой части и столбца правой расширенную матрицу СЛАУ и решить первым способом.
После этого провести проверку подстановкой.

Затем отдельно рассмотреть значение параметра, при котором решение СЛАУ нельзя найти по общей формуле,
полученной ранее.
Найти решение СЛАУ при этом значении параметра первым или третьим способом, используя подстановку subs.
\begin{align*}
    A = \left[\begin{matrix}-4 & -1 & -7 & 6\\9 & -6 & -4 & 6\\\gamma & -1 & 8 & 1\\-13 & 5 & -3 & 0\end{matrix}\right]
\qquad b = \left[\begin{matrix}-77\\-104\\71\\27\end{matrix}\right]
\end{align*}
\newpage
Вариант N 53


Решить СЛАУ c параметром тремя способами (расширенная матрица, список уравнений, матричная форма).

Вначале составить список уравнений и решить вторым способом,
затем список уравнений преобразовать в матричный вид и решить третьим способом.
Затем составить из матрицы левой части и столбца правой расширенную матрицу СЛАУ и решить первым способом.
После этого провести проверку подстановкой.

Затем отдельно рассмотреть значение параметра, при котором решение СЛАУ нельзя найти по общей формуле,
полученной ранее.
Найти решение СЛАУ при этом значении параметра первым или третьим способом, используя подстановку subs.
\begin{align*}
    A = \left[\begin{matrix}3 & 1 & -6 & 4\\-2 & 3 & -4 & 6\\-9 & \delta & 9 & -9\\5 & -2 & -2 & -2\end{matrix}\right]
\qquad b = \left[\begin{matrix}1\\1\\23\\0\end{matrix}\right]
\end{align*}
\newpage
Вариант N 54


Решить СЛАУ c параметром тремя способами (расширенная матрица, список уравнений, матричная форма).

Вначале составить список уравнений и решить вторым способом,
затем список уравнений преобразовать в матричный вид и решить третьим способом.
Затем составить из матрицы левой части и столбца правой расширенную матрицу СЛАУ и решить первым способом.
После этого провести проверку подстановкой.

Затем отдельно рассмотреть значение параметра, при котором решение СЛАУ нельзя найти по общей формуле,
полученной ранее.
Найти решение СЛАУ при этом значении параметра первым или третьим способом, используя подстановку subs.
\begin{align*}
    A = \left[\begin{matrix}4 & -1 & 1 & -2\\4 & 0 & 8 & 6\\c & -3 & 5 & -4\\0 & -1 & -7 & -8\end{matrix}\right]
\qquad b = \left[\begin{matrix}-6\\58\\22\\-64\end{matrix}\right]
\end{align*}
\newpage
Вариант N 55


Решить СЛАУ c параметром тремя способами (расширенная матрица, список уравнений, матричная форма).

Вначале составить список уравнений и решить вторым способом,
затем список уравнений преобразовать в матричный вид и решить третьим способом.
Затем составить из матрицы левой части и столбца правой расширенную матрицу СЛАУ и решить первым способом.
После этого провести проверку подстановкой.

Затем отдельно рассмотреть значение параметра, при котором решение СЛАУ нельзя найти по общей формуле,
полученной ранее.
Найти решение СЛАУ при этом значении параметра первым или третьим способом, используя подстановку subs.
\begin{align*}
    A = \left[\begin{matrix}-9 & 7 & -7 & -9\\7 & -1 & -2 & 4\\\gamma & -3 & 4 & 8\\-16 & 8 & -5 & -13\end{matrix}\right]
\qquad b = \left[\begin{matrix}-28\\77\\5\\-105\end{matrix}\right]
\end{align*}
\newpage
Вариант N 56


Решить СЛАУ c параметром тремя способами (расширенная матрица, список уравнений, матричная форма).

Вначале составить список уравнений и решить вторым способом,
затем список уравнений преобразовать в матричный вид и решить третьим способом.
Затем составить из матрицы левой части и столбца правой расширенную матрицу СЛАУ и решить первым способом.
После этого провести проверку подстановкой.

Затем отдельно рассмотреть значение параметра, при котором решение СЛАУ нельзя найти по общей формуле,
полученной ранее.
Найти решение СЛАУ при этом значении параметра первым или третьим способом, используя подстановку subs.
\begin{align*}
    A = \left[\begin{matrix}4 & 1 & 9 & -3\\3 & 1 & 6 & -9\\4 & \alpha & -5 & 5\\1 & 0 & 3 & 6\end{matrix}\right]
\qquad b = \left[\begin{matrix}-28\\45\\19\\-73\end{matrix}\right]
\end{align*}
\newpage
Вариант N 57


Решить СЛАУ c параметром тремя способами (расширенная матрица, список уравнений, матричная форма).

Вначале составить список уравнений и решить вторым способом,
затем список уравнений преобразовать в матричный вид и решить третьим способом.
Затем составить из матрицы левой части и столбца правой расширенную матрицу СЛАУ и решить первым способом.
После этого провести проверку подстановкой.

Затем отдельно рассмотреть значение параметра, при котором решение СЛАУ нельзя найти по общей формуле,
полученной ранее.
Найти решение СЛАУ при этом значении параметра первым или третьим способом, используя подстановку subs.
\begin{align*}
    A = \left[\begin{matrix}-5 & 4 & -1 & -9\\7 & 9 & 6 & -2\\-3 & s & -7 & -6\\-12 & -5 & -7 & -7\end{matrix}\right]
\qquad b = \left[\begin{matrix}45\\-46\\75\\91\end{matrix}\right]
\end{align*}
\newpage
Вариант N 58


Решить СЛАУ c параметром тремя способами (расширенная матрица, список уравнений, матричная форма).

Вначале составить список уравнений и решить вторым способом,
затем список уравнений преобразовать в матричный вид и решить третьим способом.
Затем составить из матрицы левой части и столбца правой расширенную матрицу СЛАУ и решить первым способом.
После этого провести проверку подстановкой.

Затем отдельно рассмотреть значение параметра, при котором решение СЛАУ нельзя найти по общей формуле,
полученной ранее.
Найти решение СЛАУ при этом значении параметра первым или третьим способом, используя подстановку subs.
\begin{align*}
    A = \left[\begin{matrix}2 & -6 & -6 & 8\\0 & 4 & -6 & 9\\k & -7 & -6 & 1\\2 & -10 & 0 & -1\end{matrix}\right]
\qquad b = \left[\begin{matrix}-100\\-104\\-79\\4\end{matrix}\right]
\end{align*}
\newpage
Вариант N 59


Решить СЛАУ c параметром тремя способами (расширенная матрица, список уравнений, матричная форма).

Вначале составить список уравнений и решить вторым способом,
затем список уравнений преобразовать в матричный вид и решить третьим способом.
Затем составить из матрицы левой части и столбца правой расширенную матрицу СЛАУ и решить первым способом.
После этого провести проверку подстановкой.

Затем отдельно рассмотреть значение параметра, при котором решение СЛАУ нельзя найти по общей формуле,
полученной ранее.
Найти решение СЛАУ при этом значении параметра первым или третьим способом, используя подстановку subs.
\begin{align*}
    A = \left[\begin{matrix}7 & -3 & 3 & 4\\-4 & -1 & -2 & -2\\s & 9 & -3 & -6\\11 & -2 & 5 & 6\end{matrix}\right]
\qquad b = \left[\begin{matrix}-4\\-1\\6\\-3\end{matrix}\right]
\end{align*}
\newpage
Вариант N 60


Решить СЛАУ c параметром тремя способами (расширенная матрица, список уравнений, матричная форма).

Вначале составить список уравнений и решить вторым способом,
затем список уравнений преобразовать в матричный вид и решить третьим способом.
Затем составить из матрицы левой части и столбца правой расширенную матрицу СЛАУ и решить первым способом.
После этого провести проверку подстановкой.

Затем отдельно рассмотреть значение параметра, при котором решение СЛАУ нельзя найти по общей формуле,
полученной ранее.
Найти решение СЛАУ при этом значении параметра первым или третьим способом, используя подстановку subs.
\begin{align*}
    A = \left[\begin{matrix}-5 & -6 & 8 & 6\\-4 & -9 & -4 & -4\\7 & \gamma & -4 & -6\\-1 & 3 & 12 & 10\end{matrix}\right]
\qquad b = \left[\begin{matrix}-19\\-35\\-19\\16\end{matrix}\right]
\end{align*}
\newpage
Вариант N 61


Решить СЛАУ c параметром тремя способами (расширенная матрица, список уравнений, матричная форма).

Вначале составить список уравнений и решить вторым способом,
затем список уравнений преобразовать в матричный вид и решить третьим способом.
Затем составить из матрицы левой части и столбца правой расширенную матрицу СЛАУ и решить первым способом.
После этого провести проверку подстановкой.

Затем отдельно рассмотреть значение параметра, при котором решение СЛАУ нельзя найти по общей формуле,
полученной ранее.
Найти решение СЛАУ при этом значении параметра первым или третьим способом, используя подстановку subs.
\begin{align*}
    A = \left[\begin{matrix}-9 & -1 & -1 & -6\\-4 & -2 & 8 & 1\\-6 & t & -2 & 9\\-5 & 1 & -9 & -7\end{matrix}\right]
\qquad b = \left[\begin{matrix}46\\33\\-43\\13\end{matrix}\right]
\end{align*}
\newpage
Вариант N 62


Решить СЛАУ c параметром тремя способами (расширенная матрица, список уравнений, матричная форма).

Вначале составить список уравнений и решить вторым способом,
затем список уравнений преобразовать в матричный вид и решить третьим способом.
Затем составить из матрицы левой части и столбца правой расширенную матрицу СЛАУ и решить первым способом.
После этого провести проверку подстановкой.

Затем отдельно рассмотреть значение параметра, при котором решение СЛАУ нельзя найти по общей формуле,
полученной ранее.
Найти решение СЛАУ при этом значении параметра первым или третьим способом, используя подстановку subs.
\begin{align*}
    A = \left[\begin{matrix}4 & 8 & 4 & -4\\5 & 2 & 4 & -4\\8 & b & 5 & 0\\-1 & 6 & 0 & 0\end{matrix}\right]
\qquad b = \left[\begin{matrix}80\\64\\54\\16\end{matrix}\right]
\end{align*}
\newpage
Вариант N 63


Решить СЛАУ c параметром тремя способами (расширенная матрица, список уравнений, матричная форма).

Вначале составить список уравнений и решить вторым способом,
затем список уравнений преобразовать в матричный вид и решить третьим способом.
Затем составить из матрицы левой части и столбца правой расширенную матрицу СЛАУ и решить первым способом.
После этого провести проверку подстановкой.

Затем отдельно рассмотреть значение параметра, при котором решение СЛАУ нельзя найти по общей формуле,
полученной ранее.
Найти решение СЛАУ при этом значении параметра первым или третьим способом, используя подстановку subs.
\begin{align*}
    A = \left[\begin{matrix}5 & 5 & -7 & -7\\-1 & -3 & -1 & 7\\2 & \gamma & -1 & 4\\6 & 8 & -6 & -14\end{matrix}\right]
\qquad b = \left[\begin{matrix}-35\\21\\17\\-56\end{matrix}\right]
\end{align*}
\newpage
Вариант N 64


Решить СЛАУ c параметром тремя способами (расширенная матрица, список уравнений, матричная форма).

Вначале составить список уравнений и решить вторым способом,
затем список уравнений преобразовать в матричный вид и решить третьим способом.
Затем составить из матрицы левой части и столбца правой расширенную матрицу СЛАУ и решить первым способом.
После этого провести проверку подстановкой.

Затем отдельно рассмотреть значение параметра, при котором решение СЛАУ нельзя найти по общей формуле,
полученной ранее.
Найти решение СЛАУ при этом значении параметра первым или третьим способом, используя подстановку subs.
\begin{align*}
    A = \left[\begin{matrix}-1 & -2 & -3 & 7\\5 & 5 & -2 & 5\\\delta & 6 & -1 & -8\\-6 & -7 & -1 & 2\end{matrix}\right]
\qquad b = \left[\begin{matrix}36\\87\\23\\-51\end{matrix}\right]
\end{align*}
\newpage
Вариант N 65


Решить СЛАУ c параметром тремя способами (расширенная матрица, список уравнений, матричная форма).

Вначале составить список уравнений и решить вторым способом,
затем список уравнений преобразовать в матричный вид и решить третьим способом.
Затем составить из матрицы левой части и столбца правой расширенную матрицу СЛАУ и решить первым способом.
После этого провести проверку подстановкой.

Затем отдельно рассмотреть значение параметра, при котором решение СЛАУ нельзя найти по общей формуле,
полученной ранее.
Найти решение СЛАУ при этом значении параметра первым или третьим способом, используя подстановку subs.
\begin{align*}
    A = \left[\begin{matrix}2 & -2 & -7 & 1\\3 & 3 & 6 & 2\\-8 & t & -3 & 8\\-1 & -5 & -13 & -1\end{matrix}\right]
\qquad b = \left[\begin{matrix}-35\\-10\\-117\\-25\end{matrix}\right]
\end{align*}
\newpage
Вариант N 66


Решить СЛАУ c параметром тремя способами (расширенная матрица, список уравнений, матричная форма).

Вначале составить список уравнений и решить вторым способом,
затем список уравнений преобразовать в матричный вид и решить третьим способом.
Затем составить из матрицы левой части и столбца правой расширенную матрицу СЛАУ и решить первым способом.
После этого провести проверку подстановкой.

Затем отдельно рассмотреть значение параметра, при котором решение СЛАУ нельзя найти по общей формуле,
полученной ранее.
Найти решение СЛАУ при этом значении параметра первым или третьим способом, используя подстановку subs.
\begin{align*}
    A = \left[\begin{matrix}8 & -6 & -2 & 8\\-5 & 5 & 6 & -6\\\delta & 7 & -6 & 9\\13 & -11 & -8 & 14\end{matrix}\right]
\qquad b = \left[\begin{matrix}16\\-22\\-29\\38\end{matrix}\right]
\end{align*}
\newpage
Вариант N 67


Решить СЛАУ c параметром тремя способами (расширенная матрица, список уравнений, матричная форма).

Вначале составить список уравнений и решить вторым способом,
затем список уравнений преобразовать в матричный вид и решить третьим способом.
Затем составить из матрицы левой части и столбца правой расширенную матрицу СЛАУ и решить первым способом.
После этого провести проверку подстановкой.

Затем отдельно рассмотреть значение параметра, при котором решение СЛАУ нельзя найти по общей формуле,
полученной ранее.
Найти решение СЛАУ при этом значении параметра первым или третьим способом, используя подстановку subs.
\begin{align*}
    A = \left[\begin{matrix}6 & -5 & -6 & 8\\6 & -4 & 0 & -8\\4 & s & 2 & -9\\0 & -1 & -6 & 16\end{matrix}\right]
\qquad b = \left[\begin{matrix}-9\\90\\53\\-99\end{matrix}\right]
\end{align*}
\newpage
Вариант N 68


Решить СЛАУ c параметром тремя способами (расширенная матрица, список уравнений, матричная форма).

Вначале составить список уравнений и решить вторым способом,
затем список уравнений преобразовать в матричный вид и решить третьим способом.
Затем составить из матрицы левой части и столбца правой расширенную матрицу СЛАУ и решить первым способом.
После этого провести проверку подстановкой.

Затем отдельно рассмотреть значение параметра, при котором решение СЛАУ нельзя найти по общей формуле,
полученной ранее.
Найти решение СЛАУ при этом значении параметра первым или третьим способом, используя подстановку subs.
\begin{align*}
    A = \left[\begin{matrix}2 & -7 & -6 & -3\\8 & 0 & 8 & 8\\k & 0 & 1 & -2\\-6 & -7 & -14 & -11\end{matrix}\right]
\qquad b = \left[\begin{matrix}-38\\16\\21\\-54\end{matrix}\right]
\end{align*}
\newpage
Вариант N 69


Решить СЛАУ c параметром тремя способами (расширенная матрица, список уравнений, матричная форма).

Вначале составить список уравнений и решить вторым способом,
затем список уравнений преобразовать в матричный вид и решить третьим способом.
Затем составить из матрицы левой части и столбца правой расширенную матрицу СЛАУ и решить первым способом.
После этого провести проверку подстановкой.

Затем отдельно рассмотреть значение параметра, при котором решение СЛАУ нельзя найти по общей формуле,
полученной ранее.
Найти решение СЛАУ при этом значении параметра первым или третьим способом, используя подстановку subs.
\begin{align*}
    A = \left[\begin{matrix}-3 & -1 & 5 & -6\\2 & 7 & -5 & 2\\2 & c & -8 & 2\\-5 & -8 & 10 & -8\end{matrix}\right]
\qquad b = \left[\begin{matrix}-4\\45\\6\\-49\end{matrix}\right]
\end{align*}
\newpage
Вариант N 70


Решить СЛАУ c параметром тремя способами (расширенная матрица, список уравнений, матричная форма).

Вначале составить список уравнений и решить вторым способом,
затем список уравнений преобразовать в матричный вид и решить третьим способом.
Затем составить из матрицы левой части и столбца правой расширенную матрицу СЛАУ и решить первым способом.
После этого провести проверку подстановкой.

Затем отдельно рассмотреть значение параметра, при котором решение СЛАУ нельзя найти по общей формуле,
полученной ранее.
Найти решение СЛАУ при этом значении параметра первым или третьим способом, используя подстановку subs.
\begin{align*}
    A = \left[\begin{matrix}-3 & 4 & 4 & 6\\0 & 6 & 9 & 8\\8 & \mu & 6 & 3\\-3 & -2 & -5 & -2\end{matrix}\right]
\qquad b = \left[\begin{matrix}-74\\-80\\100\\6\end{matrix}\right]
\end{align*}
\newpage
Вариант N 71


Решить СЛАУ c параметром тремя способами (расширенная матрица, список уравнений, матричная форма).

Вначале составить список уравнений и решить вторым способом,
затем список уравнений преобразовать в матричный вид и решить третьим способом.
Затем составить из матрицы левой части и столбца правой расширенную матрицу СЛАУ и решить первым способом.
После этого провести проверку подстановкой.

Затем отдельно рассмотреть значение параметра, при котором решение СЛАУ нельзя найти по общей формуле,
полученной ранее.
Найти решение СЛАУ при этом значении параметра первым или третьим способом, используя подстановку subs.
\begin{align*}
    A = \left[\begin{matrix}4 & 1 & 0 & -5\\2 & 7 & -4 & -9\\7 & \gamma & 3 & -5\\2 & -6 & 4 & 4\end{matrix}\right]
\qquad b = \left[\begin{matrix}-21\\-85\\37\\64\end{matrix}\right]
\end{align*}
\newpage
Вариант N 72


Решить СЛАУ c параметром тремя способами (расширенная матрица, список уравнений, матричная форма).

Вначале составить список уравнений и решить вторым способом,
затем список уравнений преобразовать в матричный вид и решить третьим способом.
Затем составить из матрицы левой части и столбца правой расширенную матрицу СЛАУ и решить первым способом.
После этого провести проверку подстановкой.

Затем отдельно рассмотреть значение параметра, при котором решение СЛАУ нельзя найти по общей формуле,
полученной ранее.
Найти решение СЛАУ при этом значении параметра первым или третьим способом, используя подстановку subs.
\begin{align*}
    A = \left[\begin{matrix}-6 & -3 & -5 & -2\\2 & 1 & 3 & -3\\2 & b & -3 & 9\\-8 & -4 & -8 & 1\end{matrix}\right]
\qquad b = \left[\begin{matrix}-30\\17\\13\\-47\end{matrix}\right]
\end{align*}
\newpage
Вариант N 73


Решить СЛАУ c параметром тремя способами (расширенная матрица, список уравнений, матричная форма).

Вначале составить список уравнений и решить вторым способом,
затем список уравнений преобразовать в матричный вид и решить третьим способом.
Затем составить из матрицы левой части и столбца правой расширенную матрицу СЛАУ и решить первым способом.
После этого провести проверку подстановкой.

Затем отдельно рассмотреть значение параметра, при котором решение СЛАУ нельзя найти по общей формуле,
полученной ранее.
Найти решение СЛАУ при этом значении параметра первым или третьим способом, используя подстановку subs.
\begin{align*}
    A = \left[\begin{matrix}7 & 7 & 3 & 7\\-1 & -1 & 2 & -7\\3 & \alpha & 6 & 9\\8 & 8 & 1 & 14\end{matrix}\right]
\qquad b = \left[\begin{matrix}-85\\65\\-83\\-150\end{matrix}\right]
\end{align*}
\newpage
Вариант N 74


Решить СЛАУ c параметром тремя способами (расширенная матрица, список уравнений, матричная форма).

Вначале составить список уравнений и решить вторым способом,
затем список уравнений преобразовать в матричный вид и решить третьим способом.
Затем составить из матрицы левой части и столбца правой расширенную матрицу СЛАУ и решить первым способом.
После этого провести проверку подстановкой.

Затем отдельно рассмотреть значение параметра, при котором решение СЛАУ нельзя найти по общей формуле,
полученной ранее.
Найти решение СЛАУ при этом значении параметра первым или третьим способом, используя подстановку subs.
\begin{align*}
    A = \left[\begin{matrix}-1 & 2 & 0 & -2\\2 & 4 & 2 & 2\\9 & \gamma & 7 & -6\\-3 & -2 & -2 & -4\end{matrix}\right]
\qquad b = \left[\begin{matrix}21\\38\\167\\-17\end{matrix}\right]
\end{align*}
\newpage
Вариант N 75


Решить СЛАУ c параметром тремя способами (расширенная матрица, список уравнений, матричная форма).

Вначале составить список уравнений и решить вторым способом,
затем список уравнений преобразовать в матричный вид и решить третьим способом.
Затем составить из матрицы левой части и столбца правой расширенную матрицу СЛАУ и решить первым способом.
После этого провести проверку подстановкой.

Затем отдельно рассмотреть значение параметра, при котором решение СЛАУ нельзя найти по общей формуле,
полученной ранее.
Найти решение СЛАУ при этом значении параметра первым или третьим способом, используя подстановку subs.
\begin{align*}
    A = \left[\begin{matrix}2 & -3 & 2 & 2\\9 & -7 & 5 & 8\\k & 0 & 4 & 2\\-7 & 4 & -3 & -6\end{matrix}\right]
\qquad b = \left[\begin{matrix}-6\\4\\0\\-10\end{matrix}\right]
\end{align*}
\newpage
Вариант N 76


Решить СЛАУ c параметром тремя способами (расширенная матрица, список уравнений, матричная форма).

Вначале составить список уравнений и решить вторым способом,
затем список уравнений преобразовать в матричный вид и решить третьим способом.
Затем составить из матрицы левой части и столбца правой расширенную матрицу СЛАУ и решить первым способом.
После этого провести проверку подстановкой.

Затем отдельно рассмотреть значение параметра, при котором решение СЛАУ нельзя найти по общей формуле,
полученной ранее.
Найти решение СЛАУ при этом значении параметра первым или третьим способом, используя подстановку subs.
\begin{align*}
    A = \left[\begin{matrix}8 & -2 & -6 & -7\\-9 & -7 & 3 & 9\\-4 & s & 5 & 6\\17 & 5 & -9 & -16\end{matrix}\right]
\qquad b = \left[\begin{matrix}24\\44\\27\\-20\end{matrix}\right]
\end{align*}
\newpage
Вариант N 77


Решить СЛАУ c параметром тремя способами (расширенная матрица, список уравнений, матричная форма).

Вначале составить список уравнений и решить вторым способом,
затем список уравнений преобразовать в матричный вид и решить третьим способом.
Затем составить из матрицы левой части и столбца правой расширенную матрицу СЛАУ и решить первым способом.
После этого провести проверку подстановкой.

Затем отдельно рассмотреть значение параметра, при котором решение СЛАУ нельзя найти по общей формуле,
полученной ранее.
Найти решение СЛАУ при этом значении параметра первым или третьим способом, используя подстановку subs.
\begin{align*}
    A = \left[\begin{matrix}1 & -3 & 8 & 4\\-6 & 3 & 3 & 5\\\delta & -5 & -8 & -4\\7 & -6 & 5 & -1\end{matrix}\right]
\qquad b = \left[\begin{matrix}42\\48\\-88\\-6\end{matrix}\right]
\end{align*}
\newpage
Вариант N 78


Решить СЛАУ c параметром тремя способами (расширенная матрица, список уравнений, матричная форма).

Вначале составить список уравнений и решить вторым способом,
затем список уравнений преобразовать в матричный вид и решить третьим способом.
Затем составить из матрицы левой части и столбца правой расширенную матрицу СЛАУ и решить первым способом.
После этого провести проверку подстановкой.

Затем отдельно рассмотреть значение параметра, при котором решение СЛАУ нельзя найти по общей формуле,
полученной ранее.
Найти решение СЛАУ при этом значении параметра первым или третьим способом, используя подстановку subs.
\begin{align*}
    A = \left[\begin{matrix}-1 & -6 & 7 & -1\\5 & 9 & 8 & 8\\k & -6 & 5 & 0\\-6 & -15 & -1 & -9\end{matrix}\right]
\qquad b = \left[\begin{matrix}-40\\-42\\-14\\2\end{matrix}\right]
\end{align*}
\newpage
Вариант N 79


Решить СЛАУ c параметром тремя способами (расширенная матрица, список уравнений, матричная форма).

Вначале составить список уравнений и решить вторым способом,
затем список уравнений преобразовать в матричный вид и решить третьим способом.
Затем составить из матрицы левой части и столбца правой расширенную матрицу СЛАУ и решить первым способом.
После этого провести проверку подстановкой.

Затем отдельно рассмотреть значение параметра, при котором решение СЛАУ нельзя найти по общей формуле,
полученной ранее.
Найти решение СЛАУ при этом значении параметра первым или третьим способом, используя подстановку subs.
\begin{align*}
    A = \left[\begin{matrix}-1 & -8 & -3 & 8\\8 & 3 & 6 & -6\\-4 & \mu & -5 & -4\\-9 & -11 & -9 & 14\end{matrix}\right]
\qquad b = \left[\begin{matrix}5\\-38\\42\\43\end{matrix}\right]
\end{align*}
\newpage
Вариант N 80


Решить СЛАУ c параметром тремя способами (расширенная матрица, список уравнений, матричная форма).

Вначале составить список уравнений и решить вторым способом,
затем список уравнений преобразовать в матричный вид и решить третьим способом.
Затем составить из матрицы левой части и столбца правой расширенную матрицу СЛАУ и решить первым способом.
После этого провести проверку подстановкой.

Затем отдельно рассмотреть значение параметра, при котором решение СЛАУ нельзя найти по общей формуле,
полученной ранее.
Найти решение СЛАУ при этом значении параметра первым или третьим способом, используя подстановку subs.
\begin{align*}
    A = \left[\begin{matrix}-5 & 2 & 9 & 3\\9 & 3 & -7 & 6\\\mu & 4 & -4 & 8\\-14 & -1 & 16 & -3\end{matrix}\right]
\qquad b = \left[\begin{matrix}55\\44\\66\\11\end{matrix}\right]
\end{align*}
\newpage
Вариант N 81


Решить СЛАУ c параметром тремя способами (расширенная матрица, список уравнений, матричная форма).

Вначале составить список уравнений и решить вторым способом,
затем список уравнений преобразовать в матричный вид и решить третьим способом.
Затем составить из матрицы левой части и столбца правой расширенную матрицу СЛАУ и решить первым способом.
После этого провести проверку подстановкой.

Затем отдельно рассмотреть значение параметра, при котором решение СЛАУ нельзя найти по общей формуле,
полученной ранее.
Найти решение СЛАУ при этом значении параметра первым или третьим способом, используя подстановку subs.
\begin{align*}
    A = \left[\begin{matrix}-9 & 2 & -5 & 9\\5 & 6 & -1 & -9\\4 & \beta & 0 & 3\\-14 & -4 & -4 & 18\end{matrix}\right]
\qquad b = \left[\begin{matrix}51\\-39\\15\\90\end{matrix}\right]
\end{align*}
\newpage
Вариант N 82


Решить СЛАУ c параметром тремя способами (расширенная матрица, список уравнений, матричная форма).

Вначале составить список уравнений и решить вторым способом,
затем список уравнений преобразовать в матричный вид и решить третьим способом.
Затем составить из матрицы левой части и столбца правой расширенную матрицу СЛАУ и решить первым способом.
После этого провести проверку подстановкой.

Затем отдельно рассмотреть значение параметра, при котором решение СЛАУ нельзя найти по общей формуле,
полученной ранее.
Найти решение СЛАУ при этом значении параметра первым или третьим способом, используя подстановку subs.
\begin{align*}
    A = \left[\begin{matrix}-2 & 6 & -1 & 8\\-1 & -5 & 2 & -4\\-4 & b & -5 & 0\\-1 & 11 & -3 & 12\end{matrix}\right]
\qquad b = \left[\begin{matrix}33\\-24\\24\\57\end{matrix}\right]
\end{align*}
\newpage
Вариант N 83


Решить СЛАУ c параметром тремя способами (расширенная матрица, список уравнений, матричная форма).

Вначале составить список уравнений и решить вторым способом,
затем список уравнений преобразовать в матричный вид и решить третьим способом.
Затем составить из матрицы левой части и столбца правой расширенную матрицу СЛАУ и решить первым способом.
После этого провести проверку подстановкой.

Затем отдельно рассмотреть значение параметра, при котором решение СЛАУ нельзя найти по общей формуле,
полученной ранее.
Найти решение СЛАУ при этом значении параметра первым или третьим способом, используя подстановку subs.
\begin{align*}
    A = \left[\begin{matrix}-3 & 2 & -4 & 6\\-8 & -4 & -9 & -6\\-6 & b & -1 & 9\\5 & 6 & 5 & 12\end{matrix}\right]
\qquad b = \left[\begin{matrix}77\\54\\88\\23\end{matrix}\right]
\end{align*}
\newpage
Вариант N 84


Решить СЛАУ c параметром тремя способами (расширенная матрица, список уравнений, матричная форма).

Вначале составить список уравнений и решить вторым способом,
затем список уравнений преобразовать в матричный вид и решить третьим способом.
Затем составить из матрицы левой части и столбца правой расширенную матрицу СЛАУ и решить первым способом.
После этого провести проверку подстановкой.

Затем отдельно рассмотреть значение параметра, при котором решение СЛАУ нельзя найти по общей формуле,
полученной ранее.
Найти решение СЛАУ при этом значении параметра первым или третьим способом, используя подстановку subs.
\begin{align*}
    A = \left[\begin{matrix}9 & 5 & 5 & 9\\4 & -3 & -4 & -1\\2 & b & 5 & -7\\5 & 8 & 9 & 10\end{matrix}\right]
\qquad b = \left[\begin{matrix}-111\\17\\-91\\-128\end{matrix}\right]
\end{align*}
\newpage
Вариант N 85


Решить СЛАУ c параметром тремя способами (расширенная матрица, список уравнений, матричная форма).

Вначале составить список уравнений и решить вторым способом,
затем список уравнений преобразовать в матричный вид и решить третьим способом.
Затем составить из матрицы левой части и столбца правой расширенную матрицу СЛАУ и решить первым способом.
После этого провести проверку подстановкой.

Затем отдельно рассмотреть значение параметра, при котором решение СЛАУ нельзя найти по общей формуле,
полученной ранее.
Найти решение СЛАУ при этом значении параметра первым или третьим способом, используя подстановку subs.
\begin{align*}
    A = \left[\begin{matrix}4 & -9 & 7 & 1\\-7 & -6 & -3 & 4\\3 & k & -8 & 0\\11 & -3 & 10 & -3\end{matrix}\right]
\qquad b = \left[\begin{matrix}27\\48\\63\\-21\end{matrix}\right]
\end{align*}
\newpage
Вариант N 86


Решить СЛАУ c параметром тремя способами (расширенная матрица, список уравнений, матричная форма).

Вначале составить список уравнений и решить вторым способом,
затем список уравнений преобразовать в матричный вид и решить третьим способом.
Затем составить из матрицы левой части и столбца правой расширенную матрицу СЛАУ и решить первым способом.
После этого провести проверку подстановкой.

Затем отдельно рассмотреть значение параметра, при котором решение СЛАУ нельзя найти по общей формуле,
полученной ранее.
Найти решение СЛАУ при этом значении параметра первым или третьим способом, используя подстановку subs.
\begin{align*}
    A = \left[\begin{matrix}-5 & 3 & 1 & -7\\-2 & 7 & 5 & -2\\6 & \mu & 8 & 2\\-3 & -4 & -4 & -5\end{matrix}\right]
\qquad b = \left[\begin{matrix}61\\15\\-24\\46\end{matrix}\right]
\end{align*}
\newpage
Вариант N 87


Решить СЛАУ c параметром тремя способами (расширенная матрица, список уравнений, матричная форма).

Вначале составить список уравнений и решить вторым способом,
затем список уравнений преобразовать в матричный вид и решить третьим способом.
Затем составить из матрицы левой части и столбца правой расширенную матрицу СЛАУ и решить первым способом.
После этого провести проверку подстановкой.

Затем отдельно рассмотреть значение параметра, при котором решение СЛАУ нельзя найти по общей формуле,
полученной ранее.
Найти решение СЛАУ при этом значении параметра первым или третьим способом, используя подстановку subs.
\begin{align*}
    A = \left[\begin{matrix}8 & 0 & -6 & 8\\8 & 2 & -7 & 5\\c & 8 & -4 & -2\\0 & -2 & 1 & 3\end{matrix}\right]
\qquad b = \left[\begin{matrix}40\\42\\-19\\-2\end{matrix}\right]
\end{align*}
\newpage
Вариант N 88


Решить СЛАУ c параметром тремя способами (расширенная матрица, список уравнений, матричная форма).

Вначале составить список уравнений и решить вторым способом,
затем список уравнений преобразовать в матричный вид и решить третьим способом.
Затем составить из матрицы левой части и столбца правой расширенную матрицу СЛАУ и решить первым способом.
После этого провести проверку подстановкой.

Затем отдельно рассмотреть значение параметра, при котором решение СЛАУ нельзя найти по общей формуле,
полученной ранее.
Найти решение СЛАУ при этом значении параметра первым или третьим способом, используя подстановку subs.
\begin{align*}
    A = \left[\begin{matrix}-5 & -5 & -5 & 4\\-7 & -9 & -9 & 2\\\alpha & -4 & -4 & -4\\2 & 4 & 4 & 2\end{matrix}\right]
\qquad b = \left[\begin{matrix}-22\\-12\\54\\-10\end{matrix}\right]
\end{align*}
\newpage
Вариант N 89


Решить СЛАУ c параметром тремя способами (расширенная матрица, список уравнений, матричная форма).

Вначале составить список уравнений и решить вторым способом,
затем список уравнений преобразовать в матричный вид и решить третьим способом.
Затем составить из матрицы левой части и столбца правой расширенную матрицу СЛАУ и решить первым способом.
После этого провести проверку подстановкой.

Затем отдельно рассмотреть значение параметра, при котором решение СЛАУ нельзя найти по общей формуле,
полученной ранее.
Найти решение СЛАУ при этом значении параметра первым или третьим способом, используя подстановку subs.
\begin{align*}
    A = \left[\begin{matrix}2 & -5 & 7 & 3\\4 & 1 & 9 & -2\\-4 & \beta & -1 & 6\\-2 & -6 & -2 & 5\end{matrix}\right]
\qquad b = \left[\begin{matrix}22\\-4\\14\\26\end{matrix}\right]
\end{align*}
\newpage
Вариант N 90


Решить СЛАУ c параметром тремя способами (расширенная матрица, список уравнений, матричная форма).

Вначале составить список уравнений и решить вторым способом,
затем список уравнений преобразовать в матричный вид и решить третьим способом.
Затем составить из матрицы левой части и столбца правой расширенную матрицу СЛАУ и решить первым способом.
После этого провести проверку подстановкой.

Затем отдельно рассмотреть значение параметра, при котором решение СЛАУ нельзя найти по общей формуле,
полученной ранее.
Найти решение СЛАУ при этом значении параметра первым или третьим способом, используя подстановку subs.
\begin{align*}
    A = \left[\begin{matrix}-3 & 1 & -8 & 9\\1 & -9 & 8 & 4\\\gamma & 8 & -8 & -8\\-4 & 10 & -16 & 5\end{matrix}\right]
\qquad b = \left[\begin{matrix}-43\\104\\-109\\-147\end{matrix}\right]
\end{align*}
\newpage
Вариант N 91


Решить СЛАУ c параметром тремя способами (расширенная матрица, список уравнений, матричная форма).

Вначале составить список уравнений и решить вторым способом,
затем список уравнений преобразовать в матричный вид и решить третьим способом.
Затем составить из матрицы левой части и столбца правой расширенную матрицу СЛАУ и решить первым способом.
После этого провести проверку подстановкой.

Затем отдельно рассмотреть значение параметра, при котором решение СЛАУ нельзя найти по общей формуле,
полученной ранее.
Найти решение СЛАУ при этом значении параметра первым или третьим способом, используя подстановку subs.
\begin{align*}
    A = \left[\begin{matrix}-3 & 7 & 3 & 4\\3 & 8 & -6 & -3\\c & -7 & -9 & 7\\-6 & -1 & 9 & 7\end{matrix}\right]
\qquad b = \left[\begin{matrix}92\\-10\\-74\\102\end{matrix}\right]
\end{align*}
\newpage
Вариант N 92


Решить СЛАУ c параметром тремя способами (расширенная матрица, список уравнений, матричная форма).

Вначале составить список уравнений и решить вторым способом,
затем список уравнений преобразовать в матричный вид и решить третьим способом.
Затем составить из матрицы левой части и столбца правой расширенную матрицу СЛАУ и решить первым способом.
После этого провести проверку подстановкой.

Затем отдельно рассмотреть значение параметра, при котором решение СЛАУ нельзя найти по общей формуле,
полученной ранее.
Найти решение СЛАУ при этом значении параметра первым или третьим способом, используя подстановку subs.
\begin{align*}
    A = \left[\begin{matrix}-8 & -7 & -4 & 2\\-1 & -5 & -6 & 4\\2 & c & 7 & 8\\-7 & -2 & 2 & -2\end{matrix}\right]
\qquad b = \left[\begin{matrix}38\\26\\-128\\12\end{matrix}\right]
\end{align*}
\newpage
Вариант N 93


Решить СЛАУ c параметром тремя способами (расширенная матрица, список уравнений, матричная форма).

Вначале составить список уравнений и решить вторым способом,
затем список уравнений преобразовать в матричный вид и решить третьим способом.
Затем составить из матрицы левой части и столбца правой расширенную матрицу СЛАУ и решить первым способом.
После этого провести проверку подстановкой.

Затем отдельно рассмотреть значение параметра, при котором решение СЛАУ нельзя найти по общей формуле,
полученной ранее.
Найти решение СЛАУ при этом значении параметра первым или третьим способом, используя подстановку subs.
\begin{align*}
    A = \left[\begin{matrix}5 & 0 & -4 & -7\\-9 & 5 & -5 & -4\\3 & \beta & 4 & 6\\14 & -5 & 1 & -3\end{matrix}\right]
\qquad b = \left[\begin{matrix}96\\-61\\36\\157\end{matrix}\right]
\end{align*}
\newpage
Вариант N 94


Решить СЛАУ c параметром тремя способами (расширенная матрица, список уравнений, матричная форма).

Вначале составить список уравнений и решить вторым способом,
затем список уравнений преобразовать в матричный вид и решить третьим способом.
Затем составить из матрицы левой части и столбца правой расширенную матрицу СЛАУ и решить первым способом.
После этого провести проверку подстановкой.

Затем отдельно рассмотреть значение параметра, при котором решение СЛАУ нельзя найти по общей формуле,
полученной ранее.
Найти решение СЛАУ при этом значении параметра первым или третьим способом, используя подстановку subs.
\begin{align*}
    A = \left[\begin{matrix}-9 & 9 & -4 & -7\\-1 & 8 & 5 & -7\\-2 & t & 1 & -7\\-8 & 1 & -9 & 0\end{matrix}\right]
\qquad b = \left[\begin{matrix}36\\32\\39\\4\end{matrix}\right]
\end{align*}
\newpage
Вариант N 95


Решить СЛАУ c параметром тремя способами (расширенная матрица, список уравнений, матричная форма).

Вначале составить список уравнений и решить вторым способом,
затем список уравнений преобразовать в матричный вид и решить третьим способом.
Затем составить из матрицы левой части и столбца правой расширенную матрицу СЛАУ и решить первым способом.
После этого провести проверку подстановкой.

Затем отдельно рассмотреть значение параметра, при котором решение СЛАУ нельзя найти по общей формуле,
полученной ранее.
Найти решение СЛАУ при этом значении параметра первым или третьим способом, используя подстановку subs.
\begin{align*}
    A = \left[\begin{matrix}-8 & -2 & 1 & -6\\-5 & 9 & -7 & 6\\\gamma & -9 & -9 & 4\\-3 & -11 & 8 & -12\end{matrix}\right]
\qquad b = \left[\begin{matrix}38\\46\\158\\-8\end{matrix}\right]
\end{align*}
\newpage
Вариант N 96


Решить СЛАУ c параметром тремя способами (расширенная матрица, список уравнений, матричная форма).

Вначале составить список уравнений и решить вторым способом,
затем список уравнений преобразовать в матричный вид и решить третьим способом.
Затем составить из матрицы левой части и столбца правой расширенную матрицу СЛАУ и решить первым способом.
После этого провести проверку подстановкой.

Затем отдельно рассмотреть значение параметра, при котором решение СЛАУ нельзя найти по общей формуле,
полученной ранее.
Найти решение СЛАУ при этом значении параметра первым или третьим способом, используя подстановку subs.
\begin{align*}
    A = \left[\begin{matrix}6 & 9 & 3 & -4\\9 & 1 & 8 & 1\\-1 & k & 0 & -9\\-3 & 8 & -5 & -5\end{matrix}\right]
\qquad b = \left[\begin{matrix}-79\\2\\-118\\-81\end{matrix}\right]
\end{align*}
\newpage
Вариант N 97


Решить СЛАУ c параметром тремя способами (расширенная матрица, список уравнений, матричная форма).

Вначале составить список уравнений и решить вторым способом,
затем список уравнений преобразовать в матричный вид и решить третьим способом.
Затем составить из матрицы левой части и столбца правой расширенную матрицу СЛАУ и решить первым способом.
После этого провести проверку подстановкой.

Затем отдельно рассмотреть значение параметра, при котором решение СЛАУ нельзя найти по общей формуле,
полученной ранее.
Найти решение СЛАУ при этом значении параметра первым или третьим способом, используя подстановку subs.
\begin{align*}
    A = \left[\begin{matrix}-7 & 0 & -5 & 5\\-3 & -4 & 6 & 0\\t & 2 & 3 & -3\\-4 & 4 & -11 & 5\end{matrix}\right]
\qquad b = \left[\begin{matrix}-80\\48\\48\\-128\end{matrix}\right]
\end{align*}
\newpage
Вариант N 98


Решить СЛАУ c параметром тремя способами (расширенная матрица, список уравнений, матричная форма).

Вначале составить список уравнений и решить вторым способом,
затем список уравнений преобразовать в матричный вид и решить третьим способом.
Затем составить из матрицы левой части и столбца правой расширенную матрицу СЛАУ и решить первым способом.
После этого провести проверку подстановкой.

Затем отдельно рассмотреть значение параметра, при котором решение СЛАУ нельзя найти по общей формуле,
полученной ранее.
Найти решение СЛАУ при этом значении параметра первым или третьим способом, используя подстановку subs.
\begin{align*}
    A = \left[\begin{matrix}9 & -3 & 6 & 2\\0 & 5 & -6 & -7\\b & 3 & -8 & 1\\9 & -8 & 12 & 9\end{matrix}\right]
\qquad b = \left[\begin{matrix}-48\\-39\\43\\-9\end{matrix}\right]
\end{align*}
\newpage
Вариант N 99


Решить СЛАУ c параметром тремя способами (расширенная матрица, список уравнений, матричная форма).

Вначале составить список уравнений и решить вторым способом,
затем список уравнений преобразовать в матричный вид и решить третьим способом.
Затем составить из матрицы левой части и столбца правой расширенную матрицу СЛАУ и решить первым способом.
После этого провести проверку подстановкой.

Затем отдельно рассмотреть значение параметра, при котором решение СЛАУ нельзя найти по общей формуле,
полученной ранее.
Найти решение СЛАУ при этом значении параметра первым или третьим способом, используя подстановку subs.
\begin{align*}
    A = \left[\begin{matrix}3 & -7 & 8 & 8\\1 & 0 & -3 & -8\\-6 & b & -4 & 8\\2 & -7 & 11 & 16\end{matrix}\right]
\qquad b = \left[\begin{matrix}135\\-10\\-99\\145\end{matrix}\right]
\end{align*}
\newpage
Вариант N 100


Решить СЛАУ c параметром тремя способами (расширенная матрица, список уравнений, матричная форма).

Вначале составить список уравнений и решить вторым способом,
затем список уравнений преобразовать в матричный вид и решить третьим способом.
Затем составить из матрицы левой части и столбца правой расширенную матрицу СЛАУ и решить первым способом.
После этого провести проверку подстановкой.

Затем отдельно рассмотреть значение параметра, при котором решение СЛАУ нельзя найти по общей формуле,
полученной ранее.
Найти решение СЛАУ при этом значении параметра первым или третьим способом, используя подстановку subs.
\begin{align*}
    A = \left[\begin{matrix}8 & 6 & -8 & 9\\9 & 5 & -1 & 9\\-5 & \mu & -9 & 6\\-1 & 1 & -7 & 0\end{matrix}\right]
\qquad b = \left[\begin{matrix}90\\78\\24\\12\end{matrix}\right]
\end{align*}
\newpage
Вариант N 101


Решить СЛАУ c параметром тремя способами (расширенная матрица, список уравнений, матричная форма).

Вначале составить список уравнений и решить вторым способом,
затем список уравнений преобразовать в матричный вид и решить третьим способом.
Затем составить из матрицы левой части и столбца правой расширенную матрицу СЛАУ и решить первым способом.
После этого провести проверку подстановкой.

Затем отдельно рассмотреть значение параметра, при котором решение СЛАУ нельзя найти по общей формуле,
полученной ранее.
Найти решение СЛАУ при этом значении параметра первым или третьим способом, используя подстановку subs.
\begin{align*}
    A = \left[\begin{matrix}-5 & 9 & -4 & 3\\-7 & 2 & -6 & 7\\\delta & 7 & 4 & -9\\2 & 7 & 2 & -4\end{matrix}\right]
\qquad b = \left[\begin{matrix}-55\\-28\\-69\\-27\end{matrix}\right]
\end{align*}
\newpage
Вариант N 102


Решить СЛАУ c параметром тремя способами (расширенная матрица, список уравнений, матричная форма).

Вначале составить список уравнений и решить вторым способом,
затем список уравнений преобразовать в матричный вид и решить третьим способом.
Затем составить из матрицы левой части и столбца правой расширенную матрицу СЛАУ и решить первым способом.
После этого провести проверку подстановкой.

Затем отдельно рассмотреть значение параметра, при котором решение СЛАУ нельзя найти по общей формуле,
полученной ранее.
Найти решение СЛАУ при этом значении параметра первым или третьим способом, используя подстановку subs.
\begin{align*}
    A = \left[\begin{matrix}5 & -3 & 8 & -9\\4 & 4 & -7 & -8\\\beta & -8 & 4 & -2\\1 & -7 & 15 & -1\end{matrix}\right]
\qquad b = \left[\begin{matrix}55\\-85\\68\\140\end{matrix}\right]
\end{align*}
\newpage
Вариант N 103


Решить СЛАУ c параметром тремя способами (расширенная матрица, список уравнений, матричная форма).

Вначале составить список уравнений и решить вторым способом,
затем список уравнений преобразовать в матричный вид и решить третьим способом.
Затем составить из матрицы левой части и столбца правой расширенную матрицу СЛАУ и решить первым способом.
После этого провести проверку подстановкой.

Затем отдельно рассмотреть значение параметра, при котором решение СЛАУ нельзя найти по общей формуле,
полученной ранее.
Найти решение СЛАУ при этом значении параметра первым или третьим способом, используя подстановку subs.
\begin{align*}
    A = \left[\begin{matrix}7 & -9 & -9 & 2\\-5 & -5 & -3 & -6\\9 & \alpha & 4 & -6\\12 & -4 & -6 & 8\end{matrix}\right]
\qquad b = \left[\begin{matrix}-43\\-57\\-161\\14\end{matrix}\right]
\end{align*}
\newpage
Вариант N 104


Решить СЛАУ c параметром тремя способами (расширенная матрица, список уравнений, матричная форма).

Вначале составить список уравнений и решить вторым способом,
затем список уравнений преобразовать в матричный вид и решить третьим способом.
Затем составить из матрицы левой части и столбца правой расширенную матрицу СЛАУ и решить первым способом.
После этого провести проверку подстановкой.

Затем отдельно рассмотреть значение параметра, при котором решение СЛАУ нельзя найти по общей формуле,
полученной ранее.
Найти решение СЛАУ при этом значении параметра первым или третьим способом, используя подстановку subs.
\begin{align*}
    A = \left[\begin{matrix}7 & 7 & 2 & 0\\-6 & 4 & 7 & -8\\\alpha & 0 & -4 & -8\\13 & 3 & -5 & 8\end{matrix}\right]
\qquad b = \left[\begin{matrix}43\\22\\-87\\21\end{matrix}\right]
\end{align*}
\newpage
Вариант N 105


Решить СЛАУ c параметром тремя способами (расширенная матрица, список уравнений, матричная форма).

Вначале составить список уравнений и решить вторым способом,
затем список уравнений преобразовать в матричный вид и решить третьим способом.
Затем составить из матрицы левой части и столбца правой расширенную матрицу СЛАУ и решить первым способом.
После этого провести проверку подстановкой.

Затем отдельно рассмотреть значение параметра, при котором решение СЛАУ нельзя найти по общей формуле,
полученной ранее.
Найти решение СЛАУ при этом значении параметра первым или третьим способом, используя подстановку subs.
\begin{align*}
    A = \left[\begin{matrix}9 & 7 & -7 & -4\\1 & 2 & -9 & -1\\c & -7 & 2 & -3\\8 & 5 & 2 & -3\end{matrix}\right]
\qquad b = \left[\begin{matrix}-112\\-39\\19\\-73\end{matrix}\right]
\end{align*}
\newpage
Вариант N 106


Решить СЛАУ c параметром тремя способами (расширенная матрица, список уравнений, матричная форма).

Вначале составить список уравнений и решить вторым способом,
затем список уравнений преобразовать в матричный вид и решить третьим способом.
Затем составить из матрицы левой части и столбца правой расширенную матрицу СЛАУ и решить первым способом.
После этого провести проверку подстановкой.

Затем отдельно рассмотреть значение параметра, при котором решение СЛАУ нельзя найти по общей формуле,
полученной ранее.
Найти решение СЛАУ при этом значении параметра первым или третьим способом, используя подстановку subs.
\begin{align*}
    A = \left[\begin{matrix}-6 & 2 & -7 & -4\\1 & -9 & -2 & -2\\k & 7 & -5 & 1\\-7 & 11 & -5 & -2\end{matrix}\right]
\qquad b = \left[\begin{matrix}13\\-96\\17\\109\end{matrix}\right]
\end{align*}
\newpage
Вариант N 107


Решить СЛАУ c параметром тремя способами (расширенная матрица, список уравнений, матричная форма).

Вначале составить список уравнений и решить вторым способом,
затем список уравнений преобразовать в матричный вид и решить третьим способом.
Затем составить из матрицы левой части и столбца правой расширенную матрицу СЛАУ и решить первым способом.
После этого провести проверку подстановкой.

Затем отдельно рассмотреть значение параметра, при котором решение СЛАУ нельзя найти по общей формуле,
полученной ранее.
Найти решение СЛАУ при этом значении параметра первым или третьим способом, используя подстановку subs.
\begin{align*}
    A = \left[\begin{matrix}8 & 9 & -9 & 9\\9 & -7 & -9 & -2\\0 & s & -1 & -9\\-1 & 16 & 0 & 11\end{matrix}\right]
\qquad b = \left[\begin{matrix}111\\-62\\-98\\173\end{matrix}\right]
\end{align*}
\newpage
Вариант N 108


Решить СЛАУ c параметром тремя способами (расширенная матрица, список уравнений, матричная форма).

Вначале составить список уравнений и решить вторым способом,
затем список уравнений преобразовать в матричный вид и решить третьим способом.
Затем составить из матрицы левой части и столбца правой расширенную матрицу СЛАУ и решить первым способом.
После этого провести проверку подстановкой.

Затем отдельно рассмотреть значение параметра, при котором решение СЛАУ нельзя найти по общей формуле,
полученной ранее.
Найти решение СЛАУ при этом значении параметра первым или третьим способом, используя подстановку subs.
\begin{align*}
    A = \left[\begin{matrix}-9 & 1 & 2 & -5\\9 & -6 & -6 & -9\\c & -8 & -3 & 0\\-18 & 7 & 8 & 4\end{matrix}\right]
\qquad b = \left[\begin{matrix}81\\-174\\-23\\255\end{matrix}\right]
\end{align*}
\newpage
Вариант N 109


Решить СЛАУ c параметром тремя способами (расширенная матрица, список уравнений, матричная форма).

Вначале составить список уравнений и решить вторым способом,
затем список уравнений преобразовать в матричный вид и решить третьим способом.
Затем составить из матрицы левой части и столбца правой расширенную матрицу СЛАУ и решить первым способом.
После этого провести проверку подстановкой.

Затем отдельно рассмотреть значение параметра, при котором решение СЛАУ нельзя найти по общей формуле,
полученной ранее.
Найти решение СЛАУ при этом значении параметра первым или третьим способом, используя подстановку subs.
\begin{align*}
    A = \left[\begin{matrix}7 & -4 & -9 & -4\\5 & -4 & 9 & -4\\-1 & \beta & 0 & -7\\2 & 0 & -18 & 0\end{matrix}\right]
\qquad b = \left[\begin{matrix}1\\-29\\10\\30\end{matrix}\right]
\end{align*}
\newpage
Вариант N 110


Решить СЛАУ c параметром тремя способами (расширенная матрица, список уравнений, матричная форма).

Вначале составить список уравнений и решить вторым способом,
затем список уравнений преобразовать в матричный вид и решить третьим способом.
Затем составить из матрицы левой части и столбца правой расширенную матрицу СЛАУ и решить первым способом.
После этого провести проверку подстановкой.

Затем отдельно рассмотреть значение параметра, при котором решение СЛАУ нельзя найти по общей формуле,
полученной ранее.
Найти решение СЛАУ при этом значении параметра первым или третьим способом, используя подстановку subs.
\begin{align*}
    A = \left[\begin{matrix}-8 & 6 & 4 & 8\\-6 & 9 & -9 & -1\\\gamma & 4 & 1 & -9\\-2 & -3 & 13 & 9\end{matrix}\right]
\qquad b = \left[\begin{matrix}-118\\-139\\-23\\21\end{matrix}\right]
\end{align*}
\newpage
Вариант N 111


Решить СЛАУ c параметром тремя способами (расширенная матрица, список уравнений, матричная форма).

Вначале составить список уравнений и решить вторым способом,
затем список уравнений преобразовать в матричный вид и решить третьим способом.
Затем составить из матрицы левой части и столбца правой расширенную матрицу СЛАУ и решить первым способом.
После этого провести проверку подстановкой.

Затем отдельно рассмотреть значение параметра, при котором решение СЛАУ нельзя найти по общей формуле,
полученной ранее.
Найти решение СЛАУ при этом значении параметра первым или третьим способом, используя подстановку subs.
\begin{align*}
    A = \left[\begin{matrix}-7 & -5 & -4 & 6\\-9 & 2 & 9 & -6\\\gamma & 4 & 7 & 1\\2 & -7 & -13 & 12\end{matrix}\right]
\qquad b = \left[\begin{matrix}-29\\124\\53\\-153\end{matrix}\right]
\end{align*}
\newpage
Вариант N 112


Решить СЛАУ c параметром тремя способами (расширенная матрица, список уравнений, матричная форма).

Вначале составить список уравнений и решить вторым способом,
затем список уравнений преобразовать в матричный вид и решить третьим способом.
Затем составить из матрицы левой части и столбца правой расширенную матрицу СЛАУ и решить первым способом.
После этого провести проверку подстановкой.

Затем отдельно рассмотреть значение параметра, при котором решение СЛАУ нельзя найти по общей формуле,
полученной ранее.
Найти решение СЛАУ при этом значении параметра первым или третьим способом, используя подстановку subs.
\begin{align*}
    A = \left[\begin{matrix}-3 & 8 & 0 & -1\\-6 & -7 & 4 & 7\\\alpha & 1 & -4 & -1\\3 & 15 & -4 & -8\end{matrix}\right]
\qquad b = \left[\begin{matrix}32\\72\\-87\\-40\end{matrix}\right]
\end{align*}
\newpage
Вариант N 113


Решить СЛАУ c параметром тремя способами (расширенная матрица, список уравнений, матричная форма).

Вначале составить список уравнений и решить вторым способом,
затем список уравнений преобразовать в матричный вид и решить третьим способом.
Затем составить из матрицы левой части и столбца правой расширенную матрицу СЛАУ и решить первым способом.
После этого провести проверку подстановкой.

Затем отдельно рассмотреть значение параметра, при котором решение СЛАУ нельзя найти по общей формуле,
полученной ранее.
Найти решение СЛАУ при этом значении параметра первым или третьим способом, используя подстановку subs.
\begin{align*}
    A = \left[\begin{matrix}9 & -4 & -8 & 3\\-8 & -1 & 7 & -2\\-2 & s & -6 & 2\\17 & -3 & -15 & 5\end{matrix}\right]
\qquad b = \left[\begin{matrix}119\\-93\\17\\212\end{matrix}\right]
\end{align*}
\newpage
Вариант N 114


Решить СЛАУ c параметром тремя способами (расширенная матрица, список уравнений, матричная форма).

Вначале составить список уравнений и решить вторым способом,
затем список уравнений преобразовать в матричный вид и решить третьим способом.
Затем составить из матрицы левой части и столбца правой расширенную матрицу СЛАУ и решить первым способом.
После этого провести проверку подстановкой.

Затем отдельно рассмотреть значение параметра, при котором решение СЛАУ нельзя найти по общей формуле,
полученной ранее.
Найти решение СЛАУ при этом значении параметра первым или третьим способом, используя подстановку subs.
\begin{align*}
    A = \left[\begin{matrix}-7 & 2 & 3 & 4\\7 & 3 & -7 & 2\\\gamma & 9 & -1 & -8\\-14 & -1 & 10 & 2\end{matrix}\right]
\qquad b = \left[\begin{matrix}69\\-130\\42\\199\end{matrix}\right]
\end{align*}
\newpage
Вариант N 115


Решить СЛАУ c параметром тремя способами (расширенная матрица, список уравнений, матричная форма).

Вначале составить список уравнений и решить вторым способом,
затем список уравнений преобразовать в матричный вид и решить третьим способом.
Затем составить из матрицы левой части и столбца правой расширенную матрицу СЛАУ и решить первым способом.
После этого провести проверку подстановкой.

Затем отдельно рассмотреть значение параметра, при котором решение СЛАУ нельзя найти по общей формуле,
полученной ранее.
Найти решение СЛАУ при этом значении параметра первым или третьим способом, используя подстановку subs.
\begin{align*}
    A = \left[\begin{matrix}8 & -5 & 0 & -6\\1 & 7 & 4 & -3\\3 & \alpha & -1 & 0\\7 & -12 & -4 & -3\end{matrix}\right]
\qquad b = \left[\begin{matrix}-67\\-38\\-21\\-29\end{matrix}\right]
\end{align*}
\newpage
Вариант N 116


Решить СЛАУ c параметром тремя способами (расширенная матрица, список уравнений, матричная форма).

Вначале составить список уравнений и решить вторым способом,
затем список уравнений преобразовать в матричный вид и решить третьим способом.
Затем составить из матрицы левой части и столбца правой расширенную матрицу СЛАУ и решить первым способом.
После этого провести проверку подстановкой.

Затем отдельно рассмотреть значение параметра, при котором решение СЛАУ нельзя найти по общей формуле,
полученной ранее.
Найти решение СЛАУ при этом значении параметра первым или третьим способом, используя подстановку subs.
\begin{align*}
    A = \left[\begin{matrix}8 & -2 & -6 & 3\\7 & -6 & 4 & 7\\\gamma & 3 & -9 & 9\\1 & 4 & -10 & -4\end{matrix}\right]
\qquad b = \left[\begin{matrix}-2\\111\\-156\\-113\end{matrix}\right]
\end{align*}
\newpage
Вариант N 117


Решить СЛАУ c параметром тремя способами (расширенная матрица, список уравнений, матричная форма).

Вначале составить список уравнений и решить вторым способом,
затем список уравнений преобразовать в матричный вид и решить третьим способом.
Затем составить из матрицы левой части и столбца правой расширенную матрицу СЛАУ и решить первым способом.
После этого провести проверку подстановкой.

Затем отдельно рассмотреть значение параметра, при котором решение СЛАУ нельзя найти по общей формуле,
полученной ранее.
Найти решение СЛАУ при этом значении параметра первым или третьим способом, используя подстановку subs.
\begin{align*}
    A = \left[\begin{matrix}3 & -5 & -3 & -8\\-7 & -7 & -1 & 3\\c & -8 & 3 & -5\\10 & 2 & -2 & -11\end{matrix}\right]
\qquad b = \left[\begin{matrix}26\\-49\\20\\75\end{matrix}\right]
\end{align*}
\newpage
Вариант N 118


Решить СЛАУ c параметром тремя способами (расширенная матрица, список уравнений, матричная форма).

Вначале составить список уравнений и решить вторым способом,
затем список уравнений преобразовать в матричный вид и решить третьим способом.
Затем составить из матрицы левой части и столбца правой расширенную матрицу СЛАУ и решить первым способом.
После этого провести проверку подстановкой.

Затем отдельно рассмотреть значение параметра, при котором решение СЛАУ нельзя найти по общей формуле,
полученной ранее.
Найти решение СЛАУ при этом значении параметра первым или третьим способом, используя подстановку subs.
\begin{align*}
    A = \left[\begin{matrix}1 & -4 & 3 & 1\\-4 & -2 & -8 & 3\\s & 1 & -8 & -3\\5 & -2 & 11 & -2\end{matrix}\right]
\qquad b = \left[\begin{matrix}38\\-75\\-36\\113\end{matrix}\right]
\end{align*}
\newpage
Вариант N 119


Решить СЛАУ c параметром тремя способами (расширенная матрица, список уравнений, матричная форма).

Вначале составить список уравнений и решить вторым способом,
затем список уравнений преобразовать в матричный вид и решить третьим способом.
Затем составить из матрицы левой части и столбца правой расширенную матрицу СЛАУ и решить первым способом.
После этого провести проверку подстановкой.

Затем отдельно рассмотреть значение параметра, при котором решение СЛАУ нельзя найти по общей формуле,
полученной ранее.
Найти решение СЛАУ при этом значении параметра первым или третьим способом, используя подстановку subs.
\begin{align*}
    A = \left[\begin{matrix}-6 & -6 & 3 & -7\\-5 & -7 & -9 & -8\\s & -8 & 8 & -7\\-1 & 1 & 12 & 1\end{matrix}\right]
\qquad b = \left[\begin{matrix}5\\-48\\96\\53\end{matrix}\right]
\end{align*}
\newpage
Вариант N 120


Решить СЛАУ c параметром тремя способами (расширенная матрица, список уравнений, матричная форма).

Вначале составить список уравнений и решить вторым способом,
затем список уравнений преобразовать в матричный вид и решить третьим способом.
Затем составить из матрицы левой части и столбца правой расширенную матрицу СЛАУ и решить первым способом.
После этого провести проверку подстановкой.

Затем отдельно рассмотреть значение параметра, при котором решение СЛАУ нельзя найти по общей формуле,
полученной ранее.
Найти решение СЛАУ при этом значении параметра первым или третьим способом, используя подстановку subs.
\begin{align*}
    A = \left[\begin{matrix}8 & 2 & -7 & -8\\-4 & 7 & -5 & -6\\\mu & -3 & -4 & 5\\12 & -5 & -2 & -2\end{matrix}\right]
\qquad b = \left[\begin{matrix}53\\123\\-14\\-70\end{matrix}\right]
\end{align*}
\newpage
Вариант N 121


Решить СЛАУ c параметром тремя способами (расширенная матрица, список уравнений, матричная форма).

Вначале составить список уравнений и решить вторым способом,
затем список уравнений преобразовать в матричный вид и решить третьим способом.
Затем составить из матрицы левой части и столбца правой расширенную матрицу СЛАУ и решить первым способом.
После этого провести проверку подстановкой.

Затем отдельно рассмотреть значение параметра, при котором решение СЛАУ нельзя найти по общей формуле,
полученной ранее.
Найти решение СЛАУ при этом значении параметра первым или третьим способом, используя подстановку subs.
\begin{align*}
    A = \left[\begin{matrix}-6 & 5 & -4 & 5\\8 & 4 & -9 & -9\\-3 & c & 2 & 4\\-14 & 1 & 5 & 14\end{matrix}\right]
\qquad b = \left[\begin{matrix}-23\\-123\\45\\100\end{matrix}\right]
\end{align*}
\newpage
Вариант N 122


Решить СЛАУ c параметром тремя способами (расширенная матрица, список уравнений, матричная форма).

Вначале составить список уравнений и решить вторым способом,
затем список уравнений преобразовать в матричный вид и решить третьим способом.
Затем составить из матрицы левой части и столбца правой расширенную матрицу СЛАУ и решить первым способом.
После этого провести проверку подстановкой.

Затем отдельно рассмотреть значение параметра, при котором решение СЛАУ нельзя найти по общей формуле,
полученной ранее.
Найти решение СЛАУ при этом значении параметра первым или третьим способом, используя подстановку subs.
\begin{align*}
    A = \left[\begin{matrix}-4 & -5 & 4 & 2\\5 & 6 & -7 & -8\\\gamma & -1 & 8 & -5\\-9 & -11 & 11 & 10\end{matrix}\right]
\qquad b = \left[\begin{matrix}39\\-52\\47\\91\end{matrix}\right]
\end{align*}
\newpage
Вариант N 123


Решить СЛАУ c параметром тремя способами (расширенная матрица, список уравнений, матричная форма).

Вначале составить список уравнений и решить вторым способом,
затем список уравнений преобразовать в матричный вид и решить третьим способом.
Затем составить из матрицы левой части и столбца правой расширенную матрицу СЛАУ и решить первым способом.
После этого провести проверку подстановкой.

Затем отдельно рассмотреть значение параметра, при котором решение СЛАУ нельзя найти по общей формуле,
полученной ранее.
Найти решение СЛАУ при этом значении параметра первым или третьим способом, используя подстановку subs.
\begin{align*}
    A = \left[\begin{matrix}-9 & 4 & -8 & 3\\0 & -2 & -2 & 2\\\delta & 0 & 6 & -5\\-9 & 6 & -6 & 1\end{matrix}\right]
\qquad b = \left[\begin{matrix}4\\-10\\11\\14\end{matrix}\right]
\end{align*}
\newpage
Вариант N 124


Решить СЛАУ c параметром тремя способами (расширенная матрица, список уравнений, матричная форма).

Вначале составить список уравнений и решить вторым способом,
затем список уравнений преобразовать в матричный вид и решить третьим способом.
Затем составить из матрицы левой части и столбца правой расширенную матрицу СЛАУ и решить первым способом.
После этого провести проверку подстановкой.

Затем отдельно рассмотреть значение параметра, при котором решение СЛАУ нельзя найти по общей формуле,
полученной ранее.
Найти решение СЛАУ при этом значении параметра первым или третьим способом, используя подстановку subs.
\begin{align*}
    A = \left[\begin{matrix}0 & 3 & 8 & 9\\-9 & -6 & -7 & 0\\-2 & c & 9 & 3\\9 & 9 & 15 & 9\end{matrix}\right]
\qquad b = \left[\begin{matrix}49\\10\\127\\39\end{matrix}\right]
\end{align*}
\newpage
Вариант N 125


Решить СЛАУ c параметром тремя способами (расширенная матрица, список уравнений, матричная форма).

Вначале составить список уравнений и решить вторым способом,
затем список уравнений преобразовать в матричный вид и решить третьим способом.
Затем составить из матрицы левой части и столбца правой расширенную матрицу СЛАУ и решить первым способом.
После этого провести проверку подстановкой.

Затем отдельно рассмотреть значение параметра, при котором решение СЛАУ нельзя найти по общей формуле,
полученной ранее.
Найти решение СЛАУ при этом значении параметра первым или третьим способом, используя подстановку subs.
\begin{align*}
    A = \left[\begin{matrix}3 & -8 & 9 & 0\\3 & 1 & 4 & -4\\2 & c & 4 & 2\\0 & -9 & 5 & 4\end{matrix}\right]
\qquad b = \left[\begin{matrix}-112\\-14\\11\\-98\end{matrix}\right]
\end{align*}
\newpage
Вариант N 126


Решить СЛАУ c параметром тремя способами (расширенная матрица, список уравнений, матричная форма).

Вначале составить список уравнений и решить вторым способом,
затем список уравнений преобразовать в матричный вид и решить третьим способом.
Затем составить из матрицы левой части и столбца правой расширенную матрицу СЛАУ и решить первым способом.
После этого провести проверку подстановкой.

Затем отдельно рассмотреть значение параметра, при котором решение СЛАУ нельзя найти по общей формуле,
полученной ранее.
Найти решение СЛАУ при этом значении параметра первым или третьим способом, используя подстановку subs.
\begin{align*}
    A = \left[\begin{matrix}4 & -7 & 5 & 4\\1 & -8 & 0 & 6\\8 & \delta & 3 & -4\\3 & 1 & 5 & -2\end{matrix}\right]
\qquad b = \left[\begin{matrix}-86\\-29\\-69\\-57\end{matrix}\right]
\end{align*}
\newpage
Вариант N 127


Решить СЛАУ c параметром тремя способами (расширенная матрица, список уравнений, матричная форма).

Вначале составить список уравнений и решить вторым способом,
затем список уравнений преобразовать в матричный вид и решить третьим способом.
Затем составить из матрицы левой части и столбца правой расширенную матрицу СЛАУ и решить первым способом.
После этого провести проверку подстановкой.

Затем отдельно рассмотреть значение параметра, при котором решение СЛАУ нельзя найти по общей формуле,
полученной ранее.
Найти решение СЛАУ при этом значении параметра первым или третьим способом, используя подстановку subs.
\begin{align*}
    A = \left[\begin{matrix}-3 & 0 & 0 & 6\\-2 & 5 & 8 & -7\\0 & \alpha & 2 & -8\\-1 & -5 & -8 & 13\end{matrix}\right]
\qquad b = \left[\begin{matrix}-72\\49\\100\\-121\end{matrix}\right]
\end{align*}
\newpage
Вариант N 128


Решить СЛАУ c параметром тремя способами (расширенная матрица, список уравнений, матричная форма).

Вначале составить список уравнений и решить вторым способом,
затем список уравнений преобразовать в матричный вид и решить третьим способом.
Затем составить из матрицы левой части и столбца правой расширенную матрицу СЛАУ и решить первым способом.
После этого провести проверку подстановкой.

Затем отдельно рассмотреть значение параметра, при котором решение СЛАУ нельзя найти по общей формуле,
полученной ранее.
Найти решение СЛАУ при этом значении параметра первым или третьим способом, используя подстановку subs.
\begin{align*}
    A = \left[\begin{matrix}7 & -2 & 6 & 7\\-7 & -7 & -3 & 4\\8 & b & -5 & 8\\14 & 5 & 9 & 3\end{matrix}\right]
\qquad b = \left[\begin{matrix}-35\\-10\\65\\-25\end{matrix}\right]
\end{align*}
\newpage
Вариант N 129


Решить СЛАУ c параметром тремя способами (расширенная матрица, список уравнений, матричная форма).

Вначале составить список уравнений и решить вторым способом,
затем список уравнений преобразовать в матричный вид и решить третьим способом.
Затем составить из матрицы левой части и столбца правой расширенную матрицу СЛАУ и решить первым способом.
После этого провести проверку подстановкой.

Затем отдельно рассмотреть значение параметра, при котором решение СЛАУ нельзя найти по общей формуле,
полученной ранее.
Найти решение СЛАУ при этом значении параметра первым или третьим способом, используя подстановку subs.
\begin{align*}
    A = \left[\begin{matrix}-2 & -3 & 1 & 1\\-6 & 6 & 7 & -7\\8 & \delta & -9 & -5\\4 & -9 & -6 & 8\end{matrix}\right]
\qquad b = \left[\begin{matrix}20\\167\\-90\\-147\end{matrix}\right]
\end{align*}
\newpage
Вариант N 130


Решить СЛАУ c параметром тремя способами (расширенная матрица, список уравнений, матричная форма).

Вначале составить список уравнений и решить вторым способом,
затем список уравнений преобразовать в матричный вид и решить третьим способом.
Затем составить из матрицы левой части и столбца правой расширенную матрицу СЛАУ и решить первым способом.
После этого провести проверку подстановкой.

Затем отдельно рассмотреть значение параметра, при котором решение СЛАУ нельзя найти по общей формуле,
полученной ранее.
Найти решение СЛАУ при этом значении параметра первым или третьим способом, используя подстановку subs.
\begin{align*}
    A = \left[\begin{matrix}4 & 5 & 0 & 5\\4 & -9 & 9 & 3\\s & -1 & 0 & 0\\0 & 14 & -9 & 2\end{matrix}\right]
\qquad b = \left[\begin{matrix}12\\24\\-5\\-12\end{matrix}\right]
\end{align*}
\newpage
Вариант N 131


Решить СЛАУ c параметром тремя способами (расширенная матрица, список уравнений, матричная форма).

Вначале составить список уравнений и решить вторым способом,
затем список уравнений преобразовать в матричный вид и решить третьим способом.
Затем составить из матрицы левой части и столбца правой расширенную матрицу СЛАУ и решить первым способом.
После этого провести проверку подстановкой.

Затем отдельно рассмотреть значение параметра, при котором решение СЛАУ нельзя найти по общей формуле,
полученной ранее.
Найти решение СЛАУ при этом значении параметра первым или третьим способом, используя подстановку subs.
\begin{align*}
    A = \left[\begin{matrix}9 & 6 & 6 & 2\\2 & -4 & -9 & 2\\b & -2 & -2 & 3\\7 & 10 & 15 & 0\end{matrix}\right]
\qquad b = \left[\begin{matrix}-32\\23\\18\\-55\end{matrix}\right]
\end{align*}
\newpage
Вариант N 132


Решить СЛАУ c параметром тремя способами (расширенная матрица, список уравнений, матричная форма).

Вначале составить список уравнений и решить вторым способом,
затем список уравнений преобразовать в матричный вид и решить третьим способом.
Затем составить из матрицы левой части и столбца правой расширенную матрицу СЛАУ и решить первым способом.
После этого провести проверку подстановкой.

Затем отдельно рассмотреть значение параметра, при котором решение СЛАУ нельзя найти по общей формуле,
полученной ранее.
Найти решение СЛАУ при этом значении параметра первым или третьим способом, используя подстановку subs.
\begin{align*}
    A = \left[\begin{matrix}1 & 8 & 7 & -7\\2 & 9 & 7 & -6\\c & 2 & 1 & -7\\-1 & -1 & 0 & -1\end{matrix}\right]
\qquad b = \left[\begin{matrix}-84\\-93\\-78\\9\end{matrix}\right]
\end{align*}
\newpage
Вариант N 133


Решить СЛАУ c параметром тремя способами (расширенная матрица, список уравнений, матричная форма).

Вначале составить список уравнений и решить вторым способом,
затем список уравнений преобразовать в матричный вид и решить третьим способом.
Затем составить из матрицы левой части и столбца правой расширенную матрицу СЛАУ и решить первым способом.
После этого провести проверку подстановкой.

Затем отдельно рассмотреть значение параметра, при котором решение СЛАУ нельзя найти по общей формуле,
полученной ранее.
Найти решение СЛАУ при этом значении параметра первым или третьим способом, используя подстановку subs.
\begin{align*}
    A = \left[\begin{matrix}-9 & -8 & 0 & 8\\6 & 0 & -4 & 4\\-6 & c & 7 & -7\\-15 & -8 & 4 & 4\end{matrix}\right]
\qquad b = \left[\begin{matrix}-30\\20\\-16\\-50\end{matrix}\right]
\end{align*}
\newpage
Вариант N 134


Решить СЛАУ c параметром тремя способами (расширенная матрица, список уравнений, матричная форма).

Вначале составить список уравнений и решить вторым способом,
затем список уравнений преобразовать в матричный вид и решить третьим способом.
Затем составить из матрицы левой части и столбца правой расширенную матрицу СЛАУ и решить первым способом.
После этого провести проверку подстановкой.

Затем отдельно рассмотреть значение параметра, при котором решение СЛАУ нельзя найти по общей формуле,
полученной ранее.
Найти решение СЛАУ при этом значении параметра первым или третьим способом, используя подстановку subs.
\begin{align*}
    A = \left[\begin{matrix}1 & 6 & -1 & 6\\8 & 2 & 3 & -6\\7 & c & 7 & -8\\-7 & 4 & -4 & 12\end{matrix}\right]
\qquad b = \left[\begin{matrix}-100\\37\\140\\-137\end{matrix}\right]
\end{align*}
\newpage
Вариант N 135


Решить СЛАУ c параметром тремя способами (расширенная матрица, список уравнений, матричная форма).

Вначале составить список уравнений и решить вторым способом,
затем список уравнений преобразовать в матричный вид и решить третьим способом.
Затем составить из матрицы левой части и столбца правой расширенную матрицу СЛАУ и решить первым способом.
После этого провести проверку подстановкой.

Затем отдельно рассмотреть значение параметра, при котором решение СЛАУ нельзя найти по общей формуле,
полученной ранее.
Найти решение СЛАУ при этом значении параметра первым или третьим способом, используя подстановку subs.
\begin{align*}
    A = \left[\begin{matrix}0 & 9 & 5 & -1\\9 & 5 & -9 & 0\\2 & \beta & -4 & -7\\-9 & 4 & 14 & -1\end{matrix}\right]
\qquad b = \left[\begin{matrix}-89\\63\\88\\-152\end{matrix}\right]
\end{align*}
\newpage
Вариант N 136


Решить СЛАУ c параметром тремя способами (расширенная матрица, список уравнений, матричная форма).

Вначале составить список уравнений и решить вторым способом,
затем список уравнений преобразовать в матричный вид и решить третьим способом.
Затем составить из матрицы левой части и столбца правой расширенную матрицу СЛАУ и решить первым способом.
После этого провести проверку подстановкой.

Затем отдельно рассмотреть значение параметра, при котором решение СЛАУ нельзя найти по общей формуле,
полученной ранее.
Найти решение СЛАУ при этом значении параметра первым или третьим способом, используя подстановку subs.
\begin{align*}
    A = \left[\begin{matrix}9 & -9 & -5 & -6\\7 & 0 & -7 & -6\\c & 3 & 6 & -6\\2 & -9 & 2 & 0\end{matrix}\right]
\qquad b = \left[\begin{matrix}-7\\75\\59\\-82\end{matrix}\right]
\end{align*}
\newpage
Вариант N 137


Решить СЛАУ c параметром тремя способами (расширенная матрица, список уравнений, матричная форма).

Вначале составить список уравнений и решить вторым способом,
затем список уравнений преобразовать в матричный вид и решить третьим способом.
Затем составить из матрицы левой части и столбца правой расширенную матрицу СЛАУ и решить первым способом.
После этого провести проверку подстановкой.

Затем отдельно рассмотреть значение параметра, при котором решение СЛАУ нельзя найти по общей формуле,
полученной ранее.
Найти решение СЛАУ при этом значении параметра первым или третьим способом, используя подстановку subs.
\begin{align*}
    A = \left[\begin{matrix}5 & -6 & 9 & 2\\4 & 1 & -7 & 7\\\mu & 9 & -6 & 2\\1 & -7 & 16 & -5\end{matrix}\right]
\qquad b = \left[\begin{matrix}-9\\42\\-45\\-51\end{matrix}\right]
\end{align*}
\newpage
Вариант N 138


Решить СЛАУ c параметром тремя способами (расширенная матрица, список уравнений, матричная форма).

Вначале составить список уравнений и решить вторым способом,
затем список уравнений преобразовать в матричный вид и решить третьим способом.
Затем составить из матрицы левой части и столбца правой расширенную матрицу СЛАУ и решить первым способом.
После этого провести проверку подстановкой.

Затем отдельно рассмотреть значение параметра, при котором решение СЛАУ нельзя найти по общей формуле,
полученной ранее.
Найти решение СЛАУ при этом значении параметра первым или третьим способом, используя подстановку subs.
\begin{align*}
    A = \left[\begin{matrix}-5 & 4 & 3 & 0\\-8 & 0 & 7 & -1\\\delta & 1 & -1 & -6\\3 & 4 & -4 & 1\end{matrix}\right]
\qquad b = \left[\begin{matrix}17\\28\\-49\\-11\end{matrix}\right]
\end{align*}
\newpage
Вариант N 139


Решить СЛАУ c параметром тремя способами (расширенная матрица, список уравнений, матричная форма).

Вначале составить список уравнений и решить вторым способом,
затем список уравнений преобразовать в матричный вид и решить третьим способом.
Затем составить из матрицы левой части и столбца правой расширенную матрицу СЛАУ и решить первым способом.
После этого провести проверку подстановкой.

Затем отдельно рассмотреть значение параметра, при котором решение СЛАУ нельзя найти по общей формуле,
полученной ранее.
Найти решение СЛАУ при этом значении параметра первым или третьим способом, используя подстановку subs.
\begin{align*}
    A = \left[\begin{matrix}-4 & 5 & -6 & -7\\-7 & -7 & 3 & -1\\\delta & 0 & 0 & 4\\3 & 12 & -9 & -6\end{matrix}\right]
\qquad b = \left[\begin{matrix}11\\62\\-28\\-51\end{matrix}\right]
\end{align*}
\newpage
Вариант N 140


Решить СЛАУ c параметром тремя способами (расширенная матрица, список уравнений, матричная форма).

Вначале составить список уравнений и решить вторым способом,
затем список уравнений преобразовать в матричный вид и решить третьим способом.
Затем составить из матрицы левой части и столбца правой расширенную матрицу СЛАУ и решить первым способом.
После этого провести проверку подстановкой.

Затем отдельно рассмотреть значение параметра, при котором решение СЛАУ нельзя найти по общей формуле,
полученной ранее.
Найти решение СЛАУ при этом значении параметра первым или третьим способом, используя подстановку subs.
\begin{align*}
    A = \left[\begin{matrix}3 & 6 & -2 & 6\\9 & 1 & -1 & 1\\6 & b & -2 & -4\\-6 & 5 & -1 & 5\end{matrix}\right]
\qquad b = \left[\begin{matrix}-61\\-39\\0\\-22\end{matrix}\right]
\end{align*}
\newpage
Вариант N 141


Решить СЛАУ c параметром тремя способами (расширенная матрица, список уравнений, матричная форма).

Вначале составить список уравнений и решить вторым способом,
затем список уравнений преобразовать в матричный вид и решить третьим способом.
Затем составить из матрицы левой части и столбца правой расширенную матрицу СЛАУ и решить первым способом.
После этого провести проверку подстановкой.

Затем отдельно рассмотреть значение параметра, при котором решение СЛАУ нельзя найти по общей формуле,
полученной ранее.
Найти решение СЛАУ при этом значении параметра первым или третьим способом, используя подстановку subs.
\begin{align*}
    A = \left[\begin{matrix}-2 & 8 & 9 & 2\\7 & 8 & -5 & -8\\k & 7 & 6 & -3\\-9 & 0 & 14 & 10\end{matrix}\right]
\qquad b = \left[\begin{matrix}-38\\-6\\-5\\-32\end{matrix}\right]
\end{align*}
\newpage
Вариант N 142


Решить СЛАУ c параметром тремя способами (расширенная матрица, список уравнений, матричная форма).

Вначале составить список уравнений и решить вторым способом,
затем список уравнений преобразовать в матричный вид и решить третьим способом.
Затем составить из матрицы левой части и столбца правой расширенную матрицу СЛАУ и решить первым способом.
После этого провести проверку подстановкой.

Затем отдельно рассмотреть значение параметра, при котором решение СЛАУ нельзя найти по общей формуле,
полученной ранее.
Найти решение СЛАУ при этом значении параметра первым или третьим способом, используя подстановку subs.
\begin{align*}
    A = \left[\begin{matrix}3 & 8 & -7 & -2\\-3 & -3 & 4 & -6\\\alpha & -2 & -8 & 2\\6 & 11 & -11 & 4\end{matrix}\right]
\qquad b = \left[\begin{matrix}-71\\11\\-10\\-82\end{matrix}\right]
\end{align*}
\newpage
Вариант N 143


Решить СЛАУ c параметром тремя способами (расширенная матрица, список уравнений, матричная форма).

Вначале составить список уравнений и решить вторым способом,
затем список уравнений преобразовать в матричный вид и решить третьим способом.
Затем составить из матрицы левой части и столбца правой расширенную матрицу СЛАУ и решить первым способом.
После этого провести проверку подстановкой.

Затем отдельно рассмотреть значение параметра, при котором решение СЛАУ нельзя найти по общей формуле,
полученной ранее.
Найти решение СЛАУ при этом значении параметра первым или третьим способом, используя подстановку subs.
\begin{align*}
    A = \left[\begin{matrix}-1 & 3 & -7 & -2\\9 & 9 & -7 & 9\\1 & s & -4 & 1\\-10 & -6 & 0 & -11\end{matrix}\right]
\qquad b = \left[\begin{matrix}-1\\18\\6\\-19\end{matrix}\right]
\end{align*}
\newpage
Вариант N 144


Решить СЛАУ c параметром тремя способами (расширенная матрица, список уравнений, матричная форма).

Вначале составить список уравнений и решить вторым способом,
затем список уравнений преобразовать в матричный вид и решить третьим способом.
Затем составить из матрицы левой части и столбца правой расширенную матрицу СЛАУ и решить первым способом.
После этого провести проверку подстановкой.

Затем отдельно рассмотреть значение параметра, при котором решение СЛАУ нельзя найти по общей формуле,
полученной ранее.
Найти решение СЛАУ при этом значении параметра первым или третьим способом, используя подстановку subs.
\begin{align*}
    A = \left[\begin{matrix}-5 & 8 & 4 & 6\\-5 & -7 & -9 & 1\\t & -7 & 7 & 3\\0 & 15 & 13 & 5\end{matrix}\right]
\qquad b = \left[\begin{matrix}-8\\-76\\16\\68\end{matrix}\right]
\end{align*}
\newpage
Вариант N 145


Решить СЛАУ c параметром тремя способами (расширенная матрица, список уравнений, матричная форма).

Вначале составить список уравнений и решить вторым способом,
затем список уравнений преобразовать в матричный вид и решить третьим способом.
Затем составить из матрицы левой части и столбца правой расширенную матрицу СЛАУ и решить первым способом.
После этого провести проверку подстановкой.

Затем отдельно рассмотреть значение параметра, при котором решение СЛАУ нельзя найти по общей формуле,
полученной ранее.
Найти решение СЛАУ при этом значении параметра первым или третьим способом, используя подстановку subs.
\begin{align*}
    A = \left[\begin{matrix}-9 & 2 & -5 & 1\\-6 & 7 & 2 & -4\\\mu & -3 & 1 & 6\\-3 & -5 & -7 & 5\end{matrix}\right]
\qquad b = \left[\begin{matrix}111\\12\\58\\99\end{matrix}\right]
\end{align*}
\newpage
Вариант N 146


Решить СЛАУ c параметром тремя способами (расширенная матрица, список уравнений, матричная форма).

Вначале составить список уравнений и решить вторым способом,
затем список уравнений преобразовать в матричный вид и решить третьим способом.
Затем составить из матрицы левой части и столбца правой расширенную матрицу СЛАУ и решить первым способом.
После этого провести проверку подстановкой.

Затем отдельно рассмотреть значение параметра, при котором решение СЛАУ нельзя найти по общей формуле,
полученной ранее.
Найти решение СЛАУ при этом значении параметра первым или третьим способом, используя подстановку subs.
\begin{align*}
    A = \left[\begin{matrix}3 & 1 & -9 & 0\\6 & 1 & 2 & 2\\-7 & \alpha & 4 & 5\\-3 & 0 & -11 & -2\end{matrix}\right]
\qquad b = \left[\begin{matrix}51\\55\\-74\\-4\end{matrix}\right]
\end{align*}
\newpage
Вариант N 147


Решить СЛАУ c параметром тремя способами (расширенная матрица, список уравнений, матричная форма).

Вначале составить список уравнений и решить вторым способом,
затем список уравнений преобразовать в матричный вид и решить третьим способом.
Затем составить из матрицы левой части и столбца правой расширенную матрицу СЛАУ и решить первым способом.
После этого провести проверку подстановкой.

Затем отдельно рассмотреть значение параметра, при котором решение СЛАУ нельзя найти по общей формуле,
полученной ранее.
Найти решение СЛАУ при этом значении параметра первым или третьим способом, используя подстановку subs.
\begin{align*}
    A = \left[\begin{matrix}-6 & 6 & -1 & -2\\0 & 3 & 7 & 9\\-2 & \beta & -3 & 4\\-6 & 3 & -8 & -11\end{matrix}\right]
\qquad b = \left[\begin{matrix}-21\\48\\59\\-69\end{matrix}\right]
\end{align*}
\newpage
Вариант N 148


Решить СЛАУ c параметром тремя способами (расширенная матрица, список уравнений, матричная форма).

Вначале составить список уравнений и решить вторым способом,
затем список уравнений преобразовать в матричный вид и решить третьим способом.
Затем составить из матрицы левой части и столбца правой расширенную матрицу СЛАУ и решить первым способом.
После этого провести проверку подстановкой.

Затем отдельно рассмотреть значение параметра, при котором решение СЛАУ нельзя найти по общей формуле,
полученной ранее.
Найти решение СЛАУ при этом значении параметра первым или третьим способом, используя подстановку subs.
\begin{align*}
    A = \left[\begin{matrix}3 & 8 & 4 & 0\\-8 & -3 & -7 & 1\\7 & \alpha & 2 & 6\\11 & 11 & 11 & -1\end{matrix}\right]
\qquad b = \left[\begin{matrix}74\\-44\\-38\\118\end{matrix}\right]
\end{align*}
\newpage
Вариант N 149


Решить СЛАУ c параметром тремя способами (расширенная матрица, список уравнений, матричная форма).

Вначале составить список уравнений и решить вторым способом,
затем список уравнений преобразовать в матричный вид и решить третьим способом.
Затем составить из матрицы левой части и столбца правой расширенную матрицу СЛАУ и решить первым способом.
После этого провести проверку подстановкой.

Затем отдельно рассмотреть значение параметра, при котором решение СЛАУ нельзя найти по общей формуле,
полученной ранее.
Найти решение СЛАУ при этом значении параметра первым или третьим способом, используя подстановку subs.
\begin{align*}
    A = \left[\begin{matrix}-5 & -6 & 2 & 4\\6 & -1 & -5 & -6\\-5 & c & -8 & -5\\-11 & -5 & 7 & 10\end{matrix}\right]
\qquad b = \left[\begin{matrix}-21\\79\\110\\-100\end{matrix}\right]
\end{align*}
\newpage
Вариант N 150


Решить СЛАУ c параметром тремя способами (расширенная матрица, список уравнений, матричная форма).

Вначале составить список уравнений и решить вторым способом,
затем список уравнений преобразовать в матричный вид и решить третьим способом.
Затем составить из матрицы левой части и столбца правой расширенную матрицу СЛАУ и решить первым способом.
После этого провести проверку подстановкой.

Затем отдельно рассмотреть значение параметра, при котором решение СЛАУ нельзя найти по общей формуле,
полученной ранее.
Найти решение СЛАУ при этом значении параметра первым или третьим способом, используя подстановку subs.
\begin{align*}
    A = \left[\begin{matrix}6 & 7 & -6 & -4\\-2 & 6 & 8 & -9\\\alpha & -5 & -3 & 3\\8 & 1 & -14 & 5\end{matrix}\right]
\qquad b = \left[\begin{matrix}11\\-92\\24\\103\end{matrix}\right]
\end{align*}
\newpage
Вариант N 151


Решить СЛАУ c параметром тремя способами (расширенная матрица, список уравнений, матричная форма).

Вначале составить список уравнений и решить вторым способом,
затем список уравнений преобразовать в матричный вид и решить третьим способом.
Затем составить из матрицы левой части и столбца правой расширенную матрицу СЛАУ и решить первым способом.
После этого провести проверку подстановкой.

Затем отдельно рассмотреть значение параметра, при котором решение СЛАУ нельзя найти по общей формуле,
полученной ранее.
Найти решение СЛАУ при этом значении параметра первым или третьим способом, используя подстановку subs.
\begin{align*}
    A = \left[\begin{matrix}-4 & -2 & 0 & 1\\6 & -4 & -2 & -2\\8 & \mu & 2 & -5\\-10 & 2 & 2 & 3\end{matrix}\right]
\qquad b = \left[\begin{matrix}-4\\10\\4\\-14\end{matrix}\right]
\end{align*}
\newpage
Вариант N 152


Решить СЛАУ c параметром тремя способами (расширенная матрица, список уравнений, матричная форма).

Вначале составить список уравнений и решить вторым способом,
затем список уравнений преобразовать в матричный вид и решить третьим способом.
Затем составить из матрицы левой части и столбца правой расширенную матрицу СЛАУ и решить первым способом.
После этого провести проверку подстановкой.

Затем отдельно рассмотреть значение параметра, при котором решение СЛАУ нельзя найти по общей формуле,
полученной ранее.
Найти решение СЛАУ при этом значении параметра первым или третьим способом, используя подстановку subs.
\begin{align*}
    A = \left[\begin{matrix}0 & 0 & 2 & -6\\9 & -4 & 0 & 7\\2 & c & -9 & 2\\-9 & 4 & 2 & -13\end{matrix}\right]
\qquad b = \left[\begin{matrix}10\\-38\\-80\\48\end{matrix}\right]
\end{align*}
\newpage
Вариант N 153


Решить СЛАУ c параметром тремя способами (расширенная матрица, список уравнений, матричная форма).

Вначале составить список уравнений и решить вторым способом,
затем список уравнений преобразовать в матричный вид и решить третьим способом.
Затем составить из матрицы левой части и столбца правой расширенную матрицу СЛАУ и решить первым способом.
После этого провести проверку подстановкой.

Затем отдельно рассмотреть значение параметра, при котором решение СЛАУ нельзя найти по общей формуле,
полученной ранее.
Найти решение СЛАУ при этом значении параметра первым или третьим способом, используя подстановку subs.
\begin{align*}
    A = \left[\begin{matrix}4 & -7 & -2 & 6\\-2 & 8 & 5 & 9\\-4 & \mu & -3 & 1\\6 & -15 & -7 & -3\end{matrix}\right]
\qquad b = \left[\begin{matrix}17\\-4\\-18\\21\end{matrix}\right]
\end{align*}
\newpage
Вариант N 154


Решить СЛАУ c параметром тремя способами (расширенная матрица, список уравнений, матричная форма).

Вначале составить список уравнений и решить вторым способом,
затем список уравнений преобразовать в матричный вид и решить третьим способом.
Затем составить из матрицы левой части и столбца правой расширенную матрицу СЛАУ и решить первым способом.
После этого провести проверку подстановкой.

Затем отдельно рассмотреть значение параметра, при котором решение СЛАУ нельзя найти по общей формуле,
полученной ранее.
Найти решение СЛАУ при этом значении параметра первым или третьим способом, используя подстановку subs.
\begin{align*}
    A = \left[\begin{matrix}-7 & 0 & 7 & 3\\7 & -8 & 1 & 1\\s & -9 & 8 & -9\\-14 & 8 & 6 & 2\end{matrix}\right]
\qquad b = \left[\begin{matrix}38\\-58\\-31\\96\end{matrix}\right]
\end{align*}
\newpage
Вариант N 155


Решить СЛАУ c параметром тремя способами (расширенная матрица, список уравнений, матричная форма).

Вначале составить список уравнений и решить вторым способом,
затем список уравнений преобразовать в матричный вид и решить третьим способом.
Затем составить из матрицы левой части и столбца правой расширенную матрицу СЛАУ и решить первым способом.
После этого провести проверку подстановкой.

Затем отдельно рассмотреть значение параметра, при котором решение СЛАУ нельзя найти по общей формуле,
полученной ранее.
Найти решение СЛАУ при этом значении параметра первым или третьим способом, используя подстановку subs.
\begin{align*}
    A = \left[\begin{matrix}-7 & 0 & 5 & -7\\-3 & -4 & -7 & 5\\c & 7 & 7 & -3\\-4 & 4 & 12 & -12\end{matrix}\right]
\qquad b = \left[\begin{matrix}133\\-91\\114\\224\end{matrix}\right]
\end{align*}
\newpage
Вариант N 156


Решить СЛАУ c параметром тремя способами (расширенная матрица, список уравнений, матричная форма).

Вначале составить список уравнений и решить вторым способом,
затем список уравнений преобразовать в матричный вид и решить третьим способом.
Затем составить из матрицы левой части и столбца правой расширенную матрицу СЛАУ и решить первым способом.
После этого провести проверку подстановкой.

Затем отдельно рассмотреть значение параметра, при котором решение СЛАУ нельзя найти по общей формуле,
полученной ранее.
Найти решение СЛАУ при этом значении параметра первым или третьим способом, используя подстановку subs.
\begin{align*}
    A = \left[\begin{matrix}-3 & -5 & 8 & -5\\5 & -2 & -3 & -6\\\alpha & -1 & 0 & -1\\-8 & -3 & 11 & 1\end{matrix}\right]
\qquad b = \left[\begin{matrix}58\\-37\\24\\95\end{matrix}\right]
\end{align*}
\newpage
Вариант N 157


Решить СЛАУ c параметром тремя способами (расширенная матрица, список уравнений, матричная форма).

Вначале составить список уравнений и решить вторым способом,
затем список уравнений преобразовать в матричный вид и решить третьим способом.
Затем составить из матрицы левой части и столбца правой расширенную матрицу СЛАУ и решить первым способом.
После этого провести проверку подстановкой.

Затем отдельно рассмотреть значение параметра, при котором решение СЛАУ нельзя найти по общей формуле,
полученной ранее.
Найти решение СЛАУ при этом значении параметра первым или третьим способом, используя подстановку subs.
\begin{align*}
    A = \left[\begin{matrix}-6 & -6 & -5 & 8\\9 & 0 & 6 & 8\\4 & \beta & 9 & 0\\-15 & -6 & -11 & 0\end{matrix}\right]
\qquad b = \left[\begin{matrix}-42\\153\\2\\-195\end{matrix}\right]
\end{align*}
\newpage
Вариант N 158


Решить СЛАУ c параметром тремя способами (расширенная матрица, список уравнений, матричная форма).

Вначале составить список уравнений и решить вторым способом,
затем список уравнений преобразовать в матричный вид и решить третьим способом.
Затем составить из матрицы левой части и столбца правой расширенную матрицу СЛАУ и решить первым способом.
После этого провести проверку подстановкой.

Затем отдельно рассмотреть значение параметра, при котором решение СЛАУ нельзя найти по общей формуле,
полученной ранее.
Найти решение СЛАУ при этом значении параметра первым или третьим способом, используя подстановку subs.
\begin{align*}
    A = \left[\begin{matrix}1 & -1 & -8 & -2\\-7 & 4 & 0 & 6\\-4 & \beta & 7 & 4\\8 & -5 & -8 & -8\end{matrix}\right]
\qquad b = \left[\begin{matrix}63\\-26\\-97\\89\end{matrix}\right]
\end{align*}
\newpage
Вариант N 159


Решить СЛАУ c параметром тремя способами (расширенная матрица, список уравнений, матричная форма).

Вначале составить список уравнений и решить вторым способом,
затем список уравнений преобразовать в матричный вид и решить третьим способом.
Затем составить из матрицы левой части и столбца правой расширенную матрицу СЛАУ и решить первым способом.
После этого провести проверку подстановкой.

Затем отдельно рассмотреть значение параметра, при котором решение СЛАУ нельзя найти по общей формуле,
полученной ранее.
Найти решение СЛАУ при этом значении параметра первым или третьим способом, используя подстановку subs.
\begin{align*}
    A = \left[\begin{matrix}-5 & -8 & 8 & -6\\1 & 9 & 7 & 1\\0 & \mu & -6 & 9\\-6 & -17 & 1 & -7\end{matrix}\right]
\qquad b = \left[\begin{matrix}-13\\-29\\69\\16\end{matrix}\right]
\end{align*}
\newpage
Вариант N 160


Решить СЛАУ c параметром тремя способами (расширенная матрица, список уравнений, матричная форма).

Вначале составить список уравнений и решить вторым способом,
затем список уравнений преобразовать в матричный вид и решить третьим способом.
Затем составить из матрицы левой части и столбца правой расширенную матрицу СЛАУ и решить первым способом.
После этого провести проверку подстановкой.

Затем отдельно рассмотреть значение параметра, при котором решение СЛАУ нельзя найти по общей формуле,
полученной ранее.
Найти решение СЛАУ при этом значении параметра первым или третьим способом, используя подстановку subs.
\begin{align*}
    A = \left[\begin{matrix}9 & 8 & 2 & -8\\-9 & 0 & 4 & -9\\-6 & t & 9 & -7\\18 & 8 & -2 & 1\end{matrix}\right]
\qquad b = \left[\begin{matrix}-88\\88\\102\\-176\end{matrix}\right]
\end{align*}
\newpage
Вариант N 161


Решить СЛАУ c параметром тремя способами (расширенная матрица, список уравнений, матричная форма).

Вначале составить список уравнений и решить вторым способом,
затем список уравнений преобразовать в матричный вид и решить третьим способом.
Затем составить из матрицы левой части и столбца правой расширенную матрицу СЛАУ и решить первым способом.
После этого провести проверку подстановкой.

Затем отдельно рассмотреть значение параметра, при котором решение СЛАУ нельзя найти по общей формуле,
полученной ранее.
Найти решение СЛАУ при этом значении параметра первым или третьим способом, используя подстановку subs.
\begin{align*}
    A = \left[\begin{matrix}0 & -1 & -9 & 7\\-6 & -7 & 5 & 5\\s & 1 & -2 & -3\\6 & 6 & -14 & 2\end{matrix}\right]
\qquad b = \left[\begin{matrix}92\\-34\\-79\\126\end{matrix}\right]
\end{align*}
\newpage
Вариант N 162


Решить СЛАУ c параметром тремя способами (расширенная матрица, список уравнений, матричная форма).

Вначале составить список уравнений и решить вторым способом,
затем список уравнений преобразовать в матричный вид и решить третьим способом.
Затем составить из матрицы левой части и столбца правой расширенную матрицу СЛАУ и решить первым способом.
После этого провести проверку подстановкой.

Затем отдельно рассмотреть значение параметра, при котором решение СЛАУ нельзя найти по общей формуле,
полученной ранее.
Найти решение СЛАУ при этом значении параметра первым или третьим способом, используя подстановку subs.
\begin{align*}
    A = \left[\begin{matrix}4 & -7 & -7 & 0\\4 & 7 & -5 & -7\\-7 & t & 0 & -9\\0 & -14 & -2 & 7\end{matrix}\right]
\qquad b = \left[\begin{matrix}9\\-28\\-109\\37\end{matrix}\right]
\end{align*}
\newpage
Вариант N 163


Решить СЛАУ c параметром тремя способами (расширенная матрица, список уравнений, матричная форма).

Вначале составить список уравнений и решить вторым способом,
затем список уравнений преобразовать в матричный вид и решить третьим способом.
Затем составить из матрицы левой части и столбца правой расширенную матрицу СЛАУ и решить первым способом.
После этого провести проверку подстановкой.

Затем отдельно рассмотреть значение параметра, при котором решение СЛАУ нельзя найти по общей формуле,
полученной ранее.
Найти решение СЛАУ при этом значении параметра первым или третьим способом, используя подстановку subs.
\begin{align*}
    A = \left[\begin{matrix}8 & -4 & -3 & 4\\8 & -4 & 8 & -3\\c & -4 & 6 & -7\\0 & 0 & -11 & 7\end{matrix}\right]
\qquad b = \left[\begin{matrix}-35\\-33\\29\\-2\end{matrix}\right]
\end{align*}
\newpage
Вариант N 164


Решить СЛАУ c параметром тремя способами (расширенная матрица, список уравнений, матричная форма).

Вначале составить список уравнений и решить вторым способом,
затем список уравнений преобразовать в матричный вид и решить третьим способом.
Затем составить из матрицы левой части и столбца правой расширенную матрицу СЛАУ и решить первым способом.
После этого провести проверку подстановкой.

Затем отдельно рассмотреть значение параметра, при котором решение СЛАУ нельзя найти по общей формуле,
полученной ранее.
Найти решение СЛАУ при этом значении параметра первым или третьим способом, используя подстановку subs.
\begin{align*}
    A = \left[\begin{matrix}7 & -2 & -2 & -6\\5 & 5 & -1 & 6\\-9 & \alpha & 3 & 2\\2 & -7 & -1 & -12\end{matrix}\right]
\qquad b = \left[\begin{matrix}36\\-96\\90\\132\end{matrix}\right]
\end{align*}
\newpage
Вариант N 165


Решить СЛАУ c параметром тремя способами (расширенная матрица, список уравнений, матричная форма).

Вначале составить список уравнений и решить вторым способом,
затем список уравнений преобразовать в матричный вид и решить третьим способом.
Затем составить из матрицы левой части и столбца правой расширенную матрицу СЛАУ и решить первым способом.
После этого провести проверку подстановкой.

Затем отдельно рассмотреть значение параметра, при котором решение СЛАУ нельзя найти по общей формуле,
полученной ранее.
Найти решение СЛАУ при этом значении параметра первым или третьим способом, используя подстановку subs.
\begin{align*}
    A = \left[\begin{matrix}-8 & -5 & 6 & -6\\6 & -6 & 0 & -2\\s & 0 & 4 & 9\\-14 & 1 & 6 & -4\end{matrix}\right]
\qquad b = \left[\begin{matrix}-77\\32\\-139\\-109\end{matrix}\right]
\end{align*}
\newpage
Вариант N 166


Решить СЛАУ c параметром тремя способами (расширенная матрица, список уравнений, матричная форма).

Вначале составить список уравнений и решить вторым способом,
затем список уравнений преобразовать в матричный вид и решить третьим способом.
Затем составить из матрицы левой части и столбца правой расширенную матрицу СЛАУ и решить первым способом.
После этого провести проверку подстановкой.

Затем отдельно рассмотреть значение параметра, при котором решение СЛАУ нельзя найти по общей формуле,
полученной ранее.
Найти решение СЛАУ при этом значении параметра первым или третьим способом, используя подстановку subs.
\begin{align*}
    A = \left[\begin{matrix}9 & -7 & -3 & 0\\-1 & -4 & -5 & -8\\5 & \gamma & 6 & 0\\10 & -3 & 2 & 8\end{matrix}\right]
\qquad b = \left[\begin{matrix}38\\64\\-12\\-26\end{matrix}\right]
\end{align*}
\newpage
Вариант N 167


Решить СЛАУ c параметром тремя способами (расширенная матрица, список уравнений, матричная форма).

Вначале составить список уравнений и решить вторым способом,
затем список уравнений преобразовать в матричный вид и решить третьим способом.
Затем составить из матрицы левой части и столбца правой расширенную матрицу СЛАУ и решить первым способом.
После этого провести проверку подстановкой.

Затем отдельно рассмотреть значение параметра, при котором решение СЛАУ нельзя найти по общей формуле,
полученной ранее.
Найти решение СЛАУ при этом значении параметра первым или третьим способом, используя подстановку subs.
\begin{align*}
    A = \left[\begin{matrix}-9 & -9 & -5 & -3\\0 & 4 & 7 & -4\\s & 2 & 0 & -6\\-9 & -13 & -12 & 1\end{matrix}\right]
\qquad b = \left[\begin{matrix}-147\\29\\-28\\-176\end{matrix}\right]
\end{align*}
\newpage
Вариант N 168


Решить СЛАУ c параметром тремя способами (расширенная матрица, список уравнений, матричная форма).

Вначале составить список уравнений и решить вторым способом,
затем список уравнений преобразовать в матричный вид и решить третьим способом.
Затем составить из матрицы левой части и столбца правой расширенную матрицу СЛАУ и решить первым способом.
После этого провести проверку подстановкой.

Затем отдельно рассмотреть значение параметра, при котором решение СЛАУ нельзя найти по общей формуле,
полученной ранее.
Найти решение СЛАУ при этом значении параметра первым или третьим способом, используя подстановку subs.
\begin{align*}
    A = \left[\begin{matrix}-7 & 8 & 9 & -9\\5 & -4 & 4 & -5\\8 & \mu & -2 & -3\\-12 & 12 & 5 & -4\end{matrix}\right]
\qquad b = \left[\begin{matrix}-123\\40\\54\\-163\end{matrix}\right]
\end{align*}
\newpage
Вариант N 169


Решить СЛАУ c параметром тремя способами (расширенная матрица, список уравнений, матричная форма).

Вначале составить список уравнений и решить вторым способом,
затем список уравнений преобразовать в матричный вид и решить третьим способом.
Затем составить из матрицы левой части и столбца правой расширенную матрицу СЛАУ и решить первым способом.
После этого провести проверку подстановкой.

Затем отдельно рассмотреть значение параметра, при котором решение СЛАУ нельзя найти по общей формуле,
полученной ранее.
Найти решение СЛАУ при этом значении параметра первым или третьим способом, используя подстановку subs.
\begin{align*}
    A = \left[\begin{matrix}-7 & -2 & 1 & 5\\1 & 2 & 5 & 5\\t & -8 & 4 & 9\\-8 & -4 & -4 & 0\end{matrix}\right]
\qquad b = \left[\begin{matrix}-34\\94\\70\\-128\end{matrix}\right]
\end{align*}
\newpage
Вариант N 170


Решить СЛАУ c параметром тремя способами (расширенная матрица, список уравнений, матричная форма).

Вначале составить список уравнений и решить вторым способом,
затем список уравнений преобразовать в матричный вид и решить третьим способом.
Затем составить из матрицы левой части и столбца правой расширенную матрицу СЛАУ и решить первым способом.
После этого провести проверку подстановкой.

Затем отдельно рассмотреть значение параметра, при котором решение СЛАУ нельзя найти по общей формуле,
полученной ранее.
Найти решение СЛАУ при этом значении параметра первым или третьим способом, используя подстановку subs.
\begin{align*}
    A = \left[\begin{matrix}-2 & -7 & 1 & 5\\-4 & -7 & -5 & -6\\c & -8 & 5 & -4\\2 & 0 & 6 & 11\end{matrix}\right]
\qquad b = \left[\begin{matrix}33\\48\\100\\-15\end{matrix}\right]
\end{align*}
\newpage
Вариант N 171


Решить СЛАУ c параметром тремя способами (расширенная матрица, список уравнений, матричная форма).

Вначале составить список уравнений и решить вторым способом,
затем список уравнений преобразовать в матричный вид и решить третьим способом.
Затем составить из матрицы левой части и столбца правой расширенную матрицу СЛАУ и решить первым способом.
После этого провести проверку подстановкой.

Затем отдельно рассмотреть значение параметра, при котором решение СЛАУ нельзя найти по общей формуле,
полученной ранее.
Найти решение СЛАУ при этом значении параметра первым или третьим способом, используя подстановку subs.
\begin{align*}
    A = \left[\begin{matrix}-4 & 7 & -1 & -5\\-2 & -2 & 3 & 6\\\alpha & -1 & 8 & -5\\-2 & 9 & -4 & -11\end{matrix}\right]
\qquad b = \left[\begin{matrix}37\\-48\\64\\85\end{matrix}\right]
\end{align*}
\newpage
Вариант N 172


Решить СЛАУ c параметром тремя способами (расширенная матрица, список уравнений, матричная форма).

Вначале составить список уравнений и решить вторым способом,
затем список уравнений преобразовать в матричный вид и решить третьим способом.
Затем составить из матрицы левой части и столбца правой расширенную матрицу СЛАУ и решить первым способом.
После этого провести проверку подстановкой.

Затем отдельно рассмотреть значение параметра, при котором решение СЛАУ нельзя найти по общей формуле,
полученной ранее.
Найти решение СЛАУ при этом значении параметра первым или третьим способом, используя подстановку subs.
\begin{align*}
    A = \left[\begin{matrix}-7 & -9 & -8 & -3\\5 & 8 & -9 & 0\\t & 9 & -7 & -3\\-12 & -17 & 1 & -3\end{matrix}\right]
\qquad b = \left[\begin{matrix}-73\\-36\\-3\\-37\end{matrix}\right]
\end{align*}
\newpage
Вариант N 173


Решить СЛАУ c параметром тремя способами (расширенная матрица, список уравнений, матричная форма).

Вначале составить список уравнений и решить вторым способом,
затем список уравнений преобразовать в матричный вид и решить третьим способом.
Затем составить из матрицы левой части и столбца правой расширенную матрицу СЛАУ и решить первым способом.
После этого провести проверку подстановкой.

Затем отдельно рассмотреть значение параметра, при котором решение СЛАУ нельзя найти по общей формуле,
полученной ранее.
Найти решение СЛАУ при этом значении параметра первым или третьим способом, используя подстановку subs.
\begin{align*}
    A = \left[\begin{matrix}7 & -5 & 7 & -9\\0 & 7 & -1 & -9\\\beta & -5 & -9 & -9\\7 & -12 & 8 & 0\end{matrix}\right]
\qquad b = \left[\begin{matrix}61\\2\\109\\59\end{matrix}\right]
\end{align*}
\newpage
Вариант N 174


Решить СЛАУ c параметром тремя способами (расширенная матрица, список уравнений, матричная форма).

Вначале составить список уравнений и решить вторым способом,
затем список уравнений преобразовать в матричный вид и решить третьим способом.
Затем составить из матрицы левой части и столбца правой расширенную матрицу СЛАУ и решить первым способом.
После этого провести проверку подстановкой.

Затем отдельно рассмотреть значение параметра, при котором решение СЛАУ нельзя найти по общей формуле,
полученной ранее.
Найти решение СЛАУ при этом значении параметра первым или третьим способом, используя подстановку subs.
\begin{align*}
    A = \left[\begin{matrix}-4 & 1 & 3 & -1\\0 & 1 & -6 & 2\\-5 & \delta & 7 & 5\\-4 & 0 & 9 & -3\end{matrix}\right]
\qquad b = \left[\begin{matrix}-38\\25\\-90\\-63\end{matrix}\right]
\end{align*}
\newpage
Вариант N 175


Решить СЛАУ c параметром тремя способами (расширенная матрица, список уравнений, матричная форма).

Вначале составить список уравнений и решить вторым способом,
затем список уравнений преобразовать в матричный вид и решить третьим способом.
Затем составить из матрицы левой части и столбца правой расширенную матрицу СЛАУ и решить первым способом.
После этого провести проверку подстановкой.

Затем отдельно рассмотреть значение параметра, при котором решение СЛАУ нельзя найти по общей формуле,
полученной ранее.
Найти решение СЛАУ при этом значении параметра первым или третьим способом, используя подстановку subs.
\begin{align*}
    A = \left[\begin{matrix}-7 & 3 & -4 & -2\\-6 & 2 & -8 & 6\\\gamma & 4 & -9 & -2\\-1 & 1 & 4 & -8\end{matrix}\right]
\qquad b = \left[\begin{matrix}87\\54\\128\\33\end{matrix}\right]
\end{align*}
\newpage
Вариант N 176


Решить СЛАУ c параметром тремя способами (расширенная матрица, список уравнений, матричная форма).

Вначале составить список уравнений и решить вторым способом,
затем список уравнений преобразовать в матричный вид и решить третьим способом.
Затем составить из матрицы левой части и столбца правой расширенную матрицу СЛАУ и решить первым способом.
После этого провести проверку подстановкой.

Затем отдельно рассмотреть значение параметра, при котором решение СЛАУ нельзя найти по общей формуле,
полученной ранее.
Найти решение СЛАУ при этом значении параметра первым или третьим способом, используя подстановку subs.
\begin{align*}
    A = \left[\begin{matrix}7 & -7 & 5 & 8\\-6 & -4 & -5 & -3\\k & -4 & -9 & -9\\13 & -3 & 10 & 11\end{matrix}\right]
\qquad b = \left[\begin{matrix}-56\\-19\\-43\\-37\end{matrix}\right]
\end{align*}
\newpage
Вариант N 177


Решить СЛАУ c параметром тремя способами (расширенная матрица, список уравнений, матричная форма).

Вначале составить список уравнений и решить вторым способом,
затем список уравнений преобразовать в матричный вид и решить третьим способом.
Затем составить из матрицы левой части и столбца правой расширенную матрицу СЛАУ и решить первым способом.
После этого провести проверку подстановкой.

Затем отдельно рассмотреть значение параметра, при котором решение СЛАУ нельзя найти по общей формуле,
полученной ранее.
Найти решение СЛАУ при этом значении параметра первым или третьим способом, используя подстановку subs.
\begin{align*}
    A = \left[\begin{matrix}9 & -7 & 9 & -1\\0 & -3 & -3 & -2\\\alpha & 8 & -9 & -7\\9 & -4 & 12 & 1\end{matrix}\right]
\qquad b = \left[\begin{matrix}1\\-49\\16\\50\end{matrix}\right]
\end{align*}
\newpage
Вариант N 178


Решить СЛАУ c параметром тремя способами (расширенная матрица, список уравнений, матричная форма).

Вначале составить список уравнений и решить вторым способом,
затем список уравнений преобразовать в матричный вид и решить третьим способом.
Затем составить из матрицы левой части и столбца правой расширенную матрицу СЛАУ и решить первым способом.
После этого провести проверку подстановкой.

Затем отдельно рассмотреть значение параметра, при котором решение СЛАУ нельзя найти по общей формуле,
полученной ранее.
Найти решение СЛАУ при этом значении параметра первым или третьим способом, используя подстановку subs.
\begin{align*}
    A = \left[\begin{matrix}-6 & -5 & -6 & -1\\7 & -3 & -4 & 0\\-4 & b & 2 & 4\\-13 & -2 & -2 & -1\end{matrix}\right]
\qquad b = \left[\begin{matrix}-90\\19\\2\\-109\end{matrix}\right]
\end{align*}
\newpage
Вариант N 179


Решить СЛАУ c параметром тремя способами (расширенная матрица, список уравнений, матричная форма).

Вначале составить список уравнений и решить вторым способом,
затем список уравнений преобразовать в матричный вид и решить третьим способом.
Затем составить из матрицы левой части и столбца правой расширенную матрицу СЛАУ и решить первым способом.
После этого провести проверку подстановкой.

Затем отдельно рассмотреть значение параметра, при котором решение СЛАУ нельзя найти по общей формуле,
полученной ранее.
Найти решение СЛАУ при этом значении параметра первым или третьим способом, используя подстановку subs.
\begin{align*}
    A = \left[\begin{matrix}3 & -6 & 7 & 8\\-7 & 4 & -5 & 9\\b & -4 & 4 & -6\\10 & -10 & 12 & -1\end{matrix}\right]
\qquad b = \left[\begin{matrix}-77\\-44\\24\\-33\end{matrix}\right]
\end{align*}
\newpage
Вариант N 180


Решить СЛАУ c параметром тремя способами (расширенная матрица, список уравнений, матричная форма).

Вначале составить список уравнений и решить вторым способом,
затем список уравнений преобразовать в матричный вид и решить третьим способом.
Затем составить из матрицы левой части и столбца правой расширенную матрицу СЛАУ и решить первым способом.
После этого провести проверку подстановкой.

Затем отдельно рассмотреть значение параметра, при котором решение СЛАУ нельзя найти по общей формуле,
полученной ранее.
Найти решение СЛАУ при этом значении параметра первым или третьим способом, используя подстановку subs.
\begin{align*}
    A = \left[\begin{matrix}4 & -8 & 6 & 6\\0 & -5 & 0 & -2\\\alpha & 6 & 6 & -7\\4 & -3 & 6 & 8\end{matrix}\right]
\qquad b = \left[\begin{matrix}18\\3\\-7\\15\end{matrix}\right]
\end{align*}
\newpage
Вариант N 181


Решить СЛАУ c параметром тремя способами (расширенная матрица, список уравнений, матричная форма).

Вначале составить список уравнений и решить вторым способом,
затем список уравнений преобразовать в матричный вид и решить третьим способом.
Затем составить из матрицы левой части и столбца правой расширенную матрицу СЛАУ и решить первым способом.
После этого провести проверку подстановкой.

Затем отдельно рассмотреть значение параметра, при котором решение СЛАУ нельзя найти по общей формуле,
полученной ранее.
Найти решение СЛАУ при этом значении параметра первым или третьим способом, используя подстановку subs.
\begin{align*}
    A = \left[\begin{matrix}-3 & -7 & -2 & -2\\5 & 5 & -8 & -6\\1 & t & 0 & 1\\-8 & -12 & 6 & 4\end{matrix}\right]
\qquad b = \left[\begin{matrix}-109\\-23\\4\\-86\end{matrix}\right]
\end{align*}
\newpage
Вариант N 182


Решить СЛАУ c параметром тремя способами (расширенная матрица, список уравнений, матричная форма).

Вначале составить список уравнений и решить вторым способом,
затем список уравнений преобразовать в матричный вид и решить третьим способом.
Затем составить из матрицы левой части и столбца правой расширенную матрицу СЛАУ и решить первым способом.
После этого провести проверку подстановкой.

Затем отдельно рассмотреть значение параметра, при котором решение СЛАУ нельзя найти по общей формуле,
полученной ранее.
Найти решение СЛАУ при этом значении параметра первым или третьим способом, используя подстановку subs.
\begin{align*}
    A = \left[\begin{matrix}8 & 5 & 6 & -6\\8 & -9 & -6 & 2\\-1 & \alpha & -8 & -9\\0 & 14 & 12 & -8\end{matrix}\right]
\qquad b = \left[\begin{matrix}-88\\-20\\-25\\-68\end{matrix}\right]
\end{align*}
\newpage
Вариант N 183


Решить СЛАУ c параметром тремя способами (расширенная матрица, список уравнений, матричная форма).

Вначале составить список уравнений и решить вторым способом,
затем список уравнений преобразовать в матричный вид и решить третьим способом.
Затем составить из матрицы левой части и столбца правой расширенную матрицу СЛАУ и решить первым способом.
После этого провести проверку подстановкой.

Затем отдельно рассмотреть значение параметра, при котором решение СЛАУ нельзя найти по общей формуле,
полученной ранее.
Найти решение СЛАУ при этом значении параметра первым или третьим способом, используя подстановку subs.
\begin{align*}
    A = \left[\begin{matrix}-3 & 7 & -9 & 9\\5 & -5 & 3 & 5\\c & 3 & 7 & -6\\-8 & 12 & -12 & 4\end{matrix}\right]
\qquad b = \left[\begin{matrix}-47\\-91\\64\\44\end{matrix}\right]
\end{align*}
\newpage
Вариант N 184


Решить СЛАУ c параметром тремя способами (расширенная матрица, список уравнений, матричная форма).

Вначале составить список уравнений и решить вторым способом,
затем список уравнений преобразовать в матричный вид и решить третьим способом.
Затем составить из матрицы левой части и столбца правой расширенную матрицу СЛАУ и решить первым способом.
После этого провести проверку подстановкой.

Затем отдельно рассмотреть значение параметра, при котором решение СЛАУ нельзя найти по общей формуле,
полученной ранее.
Найти решение СЛАУ при этом значении параметра первым или третьим способом, используя подстановку subs.
\begin{align*}
    A = \left[\begin{matrix}0 & 5 & 1 & -9\\-6 & 4 & -4 & -4\\k & -6 & -3 & 5\\6 & 1 & 5 & -5\end{matrix}\right]
\qquad b = \left[\begin{matrix}-111\\-98\\110\\-13\end{matrix}\right]
\end{align*}
\newpage
Вариант N 185


Решить СЛАУ c параметром тремя способами (расширенная матрица, список уравнений, матричная форма).

Вначале составить список уравнений и решить вторым способом,
затем список уравнений преобразовать в матричный вид и решить третьим способом.
Затем составить из матрицы левой части и столбца правой расширенную матрицу СЛАУ и решить первым способом.
После этого провести проверку подстановкой.

Затем отдельно рассмотреть значение параметра, при котором решение СЛАУ нельзя найти по общей формуле,
полученной ранее.
Найти решение СЛАУ при этом значении параметра первым или третьим способом, используя подстановку subs.
\begin{align*}
    A = \left[\begin{matrix}0 & -8 & -6 & -7\\2 & -8 & 8 & -9\\4 & c & 2 & -2\\-2 & 0 & -14 & 2\end{matrix}\right]
\qquad b = \left[\begin{matrix}-33\\-157\\-46\\124\end{matrix}\right]
\end{align*}
\newpage
Вариант N 186


Решить СЛАУ c параметром тремя способами (расширенная матрица, список уравнений, матричная форма).

Вначале составить список уравнений и решить вторым способом,
затем список уравнений преобразовать в матричный вид и решить третьим способом.
Затем составить из матрицы левой части и столбца правой расширенную матрицу СЛАУ и решить первым способом.
После этого провести проверку подстановкой.

Затем отдельно рассмотреть значение параметра, при котором решение СЛАУ нельзя найти по общей формуле,
полученной ранее.
Найти решение СЛАУ при этом значении параметра первым или третьим способом, используя подстановку subs.
\begin{align*}
    A = \left[\begin{matrix}-5 & -9 & 9 & 2\\8 & 1 & 9 & 2\\\beta & -9 & 7 & 5\\-13 & -10 & 0 & 0\end{matrix}\right]
\qquad b = \left[\begin{matrix}-77\\-112\\-105\\35\end{matrix}\right]
\end{align*}
\newpage
Вариант N 187


Решить СЛАУ c параметром тремя способами (расширенная матрица, список уравнений, матричная форма).

Вначале составить список уравнений и решить вторым способом,
затем список уравнений преобразовать в матричный вид и решить третьим способом.
Затем составить из матрицы левой части и столбца правой расширенную матрицу СЛАУ и решить первым способом.
После этого провести проверку подстановкой.

Затем отдельно рассмотреть значение параметра, при котором решение СЛАУ нельзя найти по общей формуле,
полученной ранее.
Найти решение СЛАУ при этом значении параметра первым или третьим способом, используя подстановку subs.
\begin{align*}
    A = \left[\begin{matrix}-8 & -6 & -3 & 2\\7 & 3 & 2 & 0\\-5 & \beta & 9 & -3\\-15 & -9 & -5 & 2\end{matrix}\right]
\qquad b = \left[\begin{matrix}107\\-73\\-70\\180\end{matrix}\right]
\end{align*}
\newpage
Вариант N 188


Решить СЛАУ c параметром тремя способами (расширенная матрица, список уравнений, матричная форма).

Вначале составить список уравнений и решить вторым способом,
затем список уравнений преобразовать в матричный вид и решить третьим способом.
Затем составить из матрицы левой части и столбца правой расширенную матрицу СЛАУ и решить первым способом.
После этого провести проверку подстановкой.

Затем отдельно рассмотреть значение параметра, при котором решение СЛАУ нельзя найти по общей формуле,
полученной ранее.
Найти решение СЛАУ при этом значении параметра первым или третьим способом, используя подстановку subs.
\begin{align*}
    A = \left[\begin{matrix}5 & -4 & -8 & 9\\-1 & -6 & 6 & 9\\\mu & -9 & -5 & -5\\6 & 2 & -14 & 0\end{matrix}\right]
\qquad b = \left[\begin{matrix}-15\\-19\\19\\4\end{matrix}\right]
\end{align*}
\newpage
Вариант N 189


Решить СЛАУ c параметром тремя способами (расширенная матрица, список уравнений, матричная форма).

Вначале составить список уравнений и решить вторым способом,
затем список уравнений преобразовать в матричный вид и решить третьим способом.
Затем составить из матрицы левой части и столбца правой расширенную матрицу СЛАУ и решить первым способом.
После этого провести проверку подстановкой.

Затем отдельно рассмотреть значение параметра, при котором решение СЛАУ нельзя найти по общей формуле,
полученной ранее.
Найти решение СЛАУ при этом значении параметра первым или третьим способом, используя подстановку subs.
\begin{align*}
    A = \left[\begin{matrix}-5 & -4 & -8 & 7\\-1 & 4 & 4 & 3\\8 & k & 0 & 2\\-4 & -8 & -12 & 4\end{matrix}\right]
\qquad b = \left[\begin{matrix}-105\\31\\-39\\-136\end{matrix}\right]
\end{align*}
\newpage
Вариант N 190


Решить СЛАУ c параметром тремя способами (расширенная матрица, список уравнений, матричная форма).

Вначале составить список уравнений и решить вторым способом,
затем список уравнений преобразовать в матричный вид и решить третьим способом.
Затем составить из матрицы левой части и столбца правой расширенную матрицу СЛАУ и решить первым способом.
После этого провести проверку подстановкой.

Затем отдельно рассмотреть значение параметра, при котором решение СЛАУ нельзя найти по общей формуле,
полученной ранее.
Найти решение СЛАУ при этом значении параметра первым или третьим способом, используя подстановку subs.
\begin{align*}
    A = \left[\begin{matrix}-3 & -7 & 8 & -5\\-2 & 2 & -8 & -7\\8 & \beta & 3 & 7\\-1 & -9 & 16 & 2\end{matrix}\right]
\qquad b = \left[\begin{matrix}14\\61\\-17\\-47\end{matrix}\right]
\end{align*}
\newpage
Вариант N 191


Решить СЛАУ c параметром тремя способами (расширенная матрица, список уравнений, матричная форма).

Вначале составить список уравнений и решить вторым способом,
затем список уравнений преобразовать в матричный вид и решить третьим способом.
Затем составить из матрицы левой части и столбца правой расширенную матрицу СЛАУ и решить первым способом.
После этого провести проверку подстановкой.

Затем отдельно рассмотреть значение параметра, при котором решение СЛАУ нельзя найти по общей формуле,
полученной ранее.
Найти решение СЛАУ при этом значении параметра первым или третьим способом, используя подстановку subs.
\begin{align*}
    A = \left[\begin{matrix}9 & -1 & -9 & 0\\-6 & -1 & 2 & 5\\\delta & -5 & -1 & 5\\15 & 0 & -11 & -5\end{matrix}\right]
\qquad b = \left[\begin{matrix}-2\\81\\18\\-83\end{matrix}\right]
\end{align*}
\newpage
Вариант N 192


Решить СЛАУ c параметром тремя способами (расширенная матрица, список уравнений, матричная форма).

Вначале составить список уравнений и решить вторым способом,
затем список уравнений преобразовать в матричный вид и решить третьим способом.
Затем составить из матрицы левой части и столбца правой расширенную матрицу СЛАУ и решить первым способом.
После этого провести проверку подстановкой.

Затем отдельно рассмотреть значение параметра, при котором решение СЛАУ нельзя найти по общей формуле,
полученной ранее.
Найти решение СЛАУ при этом значении параметра первым или третьим способом, используя подстановку subs.
\begin{align*}
    A = \left[\begin{matrix}-3 & -4 & -1 & 4\\7 & 3 & 5 & 7\\-9 & t & 8 & -1\\-10 & -7 & -6 & -3\end{matrix}\right]
\qquad b = \left[\begin{matrix}41\\-31\\-52\\72\end{matrix}\right]
\end{align*}
\newpage
Вариант N 193


Решить СЛАУ c параметром тремя способами (расширенная матрица, список уравнений, матричная форма).

Вначале составить список уравнений и решить вторым способом,
затем список уравнений преобразовать в матричный вид и решить третьим способом.
Затем составить из матрицы левой части и столбца правой расширенную матрицу СЛАУ и решить первым способом.
После этого провести проверку подстановкой.

Затем отдельно рассмотреть значение параметра, при котором решение СЛАУ нельзя найти по общей формуле,
полученной ранее.
Найти решение СЛАУ при этом значении параметра первым или третьим способом, используя подстановку subs.
\begin{align*}
    A = \left[\begin{matrix}1 & 6 & -8 & -7\\8 & 9 & 5 & -5\\t & 5 & -1 & -9\\-7 & -3 & -13 & -2\end{matrix}\right]
\qquad b = \left[\begin{matrix}-2\\41\\-27\\-43\end{matrix}\right]
\end{align*}
\newpage
Вариант N 194


Решить СЛАУ c параметром тремя способами (расширенная матрица, список уравнений, матричная форма).

Вначале составить список уравнений и решить вторым способом,
затем список уравнений преобразовать в матричный вид и решить третьим способом.
Затем составить из матрицы левой части и столбца правой расширенную матрицу СЛАУ и решить первым способом.
После этого провести проверку подстановкой.

Затем отдельно рассмотреть значение параметра, при котором решение СЛАУ нельзя найти по общей формуле,
полученной ранее.
Найти решение СЛАУ при этом значении параметра первым или третьим способом, используя подстановку subs.
\begin{align*}
    A = \left[\begin{matrix}-7 & 1 & -4 & -9\\5 & 7 & 4 & 2\\2 & s & 8 & 6\\-12 & -6 & -8 & -11\end{matrix}\right]
\qquad b = \left[\begin{matrix}-38\\22\\66\\-60\end{matrix}\right]
\end{align*}
\newpage
Вариант N 195


Решить СЛАУ c параметром тремя способами (расширенная матрица, список уравнений, матричная форма).

Вначале составить список уравнений и решить вторым способом,
затем список уравнений преобразовать в матричный вид и решить третьим способом.
Затем составить из матрицы левой части и столбца правой расширенную матрицу СЛАУ и решить первым способом.
После этого провести проверку подстановкой.

Затем отдельно рассмотреть значение параметра, при котором решение СЛАУ нельзя найти по общей формуле,
полученной ранее.
Найти решение СЛАУ при этом значении параметра первым или третьим способом, используя подстановку subs.
\begin{align*}
    A = \left[\begin{matrix}-9 & 2 & -8 & 8\\-7 & 4 & 2 & 7\\-9 & t & -8 & 3\\-2 & -2 & -10 & 1\end{matrix}\right]
\qquad b = \left[\begin{matrix}17\\47\\47\\-30\end{matrix}\right]
\end{align*}
\newpage
Вариант N 196


Решить СЛАУ c параметром тремя способами (расширенная матрица, список уравнений, матричная форма).

Вначале составить список уравнений и решить вторым способом,
затем список уравнений преобразовать в матричный вид и решить третьим способом.
Затем составить из матрицы левой части и столбца правой расширенную матрицу СЛАУ и решить первым способом.
После этого провести проверку подстановкой.

Затем отдельно рассмотреть значение параметра, при котором решение СЛАУ нельзя найти по общей формуле,
полученной ранее.
Найти решение СЛАУ при этом значении параметра первым или третьим способом, используя подстановку subs.
\begin{align*}
    A = \left[\begin{matrix}-7 & 4 & 8 & 0\\1 & 9 & 9 & -2\\-9 & \delta & 3 & -1\\-8 & -5 & -1 & 2\end{matrix}\right]
\qquad b = \left[\begin{matrix}-32\\-18\\-18\\-14\end{matrix}\right]
\end{align*}
\newpage
Вариант N 197


Решить СЛАУ c параметром тремя способами (расширенная матрица, список уравнений, матричная форма).

Вначале составить список уравнений и решить вторым способом,
затем список уравнений преобразовать в матричный вид и решить третьим способом.
Затем составить из матрицы левой части и столбца правой расширенную матрицу СЛАУ и решить первым способом.
После этого провести проверку подстановкой.

Затем отдельно рассмотреть значение параметра, при котором решение СЛАУ нельзя найти по общей формуле,
полученной ранее.
Найти решение СЛАУ при этом значении параметра первым или третьим способом, используя подстановку subs.
\begin{align*}
    A = \left[\begin{matrix}-6 & -3 & 7 & -7\\-3 & 3 & 5 & -8\\\gamma & -3 & 7 & 1\\-3 & -6 & 2 & 1\end{matrix}\right]
\qquad b = \left[\begin{matrix}5\\-44\\77\\49\end{matrix}\right]
\end{align*}
\newpage
Вариант N 198


Решить СЛАУ c параметром тремя способами (расширенная матрица, список уравнений, матричная форма).

Вначале составить список уравнений и решить вторым способом,
затем список уравнений преобразовать в матричный вид и решить третьим способом.
Затем составить из матрицы левой части и столбца правой расширенную матрицу СЛАУ и решить первым способом.
После этого провести проверку подстановкой.

Затем отдельно рассмотреть значение параметра, при котором решение СЛАУ нельзя найти по общей формуле,
полученной ранее.
Найти решение СЛАУ при этом значении параметра первым или третьим способом, используя подстановку subs.
\begin{align*}
    A = \left[\begin{matrix}2 & 1 & -3 & 9\\6 & 8 & -1 & 7\\-8 & k & 6 & 5\\-4 & -7 & -2 & 2\end{matrix}\right]
\qquad b = \left[\begin{matrix}-101\\-99\\-21\\-2\end{matrix}\right]
\end{align*}
\newpage
Вариант N 199


Решить СЛАУ c параметром тремя способами (расширенная матрица, список уравнений, матричная форма).

Вначале составить список уравнений и решить вторым способом,
затем список уравнений преобразовать в матричный вид и решить третьим способом.
Затем составить из матрицы левой части и столбца правой расширенную матрицу СЛАУ и решить первым способом.
После этого провести проверку подстановкой.

Затем отдельно рассмотреть значение параметра, при котором решение СЛАУ нельзя найти по общей формуле,
полученной ранее.
Найти решение СЛАУ при этом значении параметра первым или третьим способом, используя подстановку subs.
\begin{align*}
    A = \left[\begin{matrix}6 & -4 & 7 & -4\\-8 & 2 & -1 & -5\\-1 & c & -5 & -3\\14 & -6 & 8 & 1\end{matrix}\right]
\qquad b = \left[\begin{matrix}88\\-31\\1\\119\end{matrix}\right]
\end{align*}
\newpage
Вариант N 200


Решить СЛАУ c параметром тремя способами (расширенная матрица, список уравнений, матричная форма).

Вначале составить список уравнений и решить вторым способом,
затем список уравнений преобразовать в матричный вид и решить третьим способом.
Затем составить из матрицы левой части и столбца правой расширенную матрицу СЛАУ и решить первым способом.
После этого провести проверку подстановкой.

Затем отдельно рассмотреть значение параметра, при котором решение СЛАУ нельзя найти по общей формуле,
полученной ранее.
Найти решение СЛАУ при этом значении параметра первым или третьим способом, используя подстановку subs.
\begin{align*}
    A = \left[\begin{matrix}4 & 9 & -2 & 4\\5 & 9 & 6 & -3\\7 & \mu & 7 & 7\\-1 & 0 & -8 & 7\end{matrix}\right]
\qquad b = \left[\begin{matrix}-20\\-82\\-50\\62\end{matrix}\right]
\end{align*}
\newpage
\end{document}