 \documentclass[11pt]{report}

\usepackage[T2A]{fontenc}

\usepackage[utf8]{inputenc}

\usepackage[russian]{babel}

\usepackage{amsmath,amssymb}

\usepackage{graphicx}

\oddsidemargin=-19mm

\topmargin=-30mm

\textheight 26cm 

\hsize 18cm

\textwidth 20cm

\begin{document}

\pagestyle{empty}

{\bf Индивидуальное задание 4.}

Вариант N 1
Дана СЛАУ $AX = b$,
Проверить совместность по теореме Кронекера-Капелли. Если СЛАУ совместна, проверить единственность решения.
Для соответствующей однородной СЛАУ проверить существование нетривиального решения. В случае, если оно существует,
найти размерность пространства решений и составить ФСР и общее решение однородной  и неоднородной СЛАУ.


\begin{align*}
 A = \left[\begin{matrix}-9 & -5 & 7 & 5 & -9\\-44 & -44 & 20 & -12 & -68\\-212 & -196 & 108 & -28 & -308\\140 & 156 & -52 & 68 & 236\\456 & 488 & -184 & 184 & 744\end{matrix}\right],
\ b = \left[\begin{matrix}-61\\-612\\-2692\\2204\\6856\end{matrix}\right]. 
 \end{align*}

Вариант N 2
Дана СЛАУ $AX = b$,
Проверить совместность по теореме Кронекера-Капелли. Если СЛАУ совместна, проверить единственность решения.
Для соответствующей однородной СЛАУ проверить существование нетривиального решения. В случае, если оно существует,
найти размерность пространства решений и составить ФСР и общее решение однородной  и неоднородной СЛАУ.


\begin{align*}
 A = \left[\begin{matrix}-2 & -4 & 6 & 5 & -9\\9 & 4 & -7 & -2 & -4\\37 & 4 & -11 & 10 & -56\\-53 & -36 & 59 & 30 & -16\\-6 & 6 & 6 & -4 & 8\end{matrix}\right],
\ b = \left[\begin{matrix}116\\-74\\94\\834\\14\end{matrix}\right]. 
 \end{align*}

Вариант N 3
Дана СЛАУ $AX = b$,
Проверить совместность по теореме Кронекера-Капелли. Если СЛАУ совместна, проверить единственность решения.
Для соответствующей однородной СЛАУ проверить существование нетривиального решения. В случае, если оно существует,
найти размерность пространства решений и составить ФСР и общее решение однородной  и неоднородной СЛАУ.


\begin{align*}
 A = \left[\begin{matrix}9 & 5 & 5 & -2 & 2\\-2 & 4 & -3 & -8 & 9\\28 & 36 & 8 & -40 & 44\\44 & 4 & 32 & 24 & -28\\5 & 9 & 0 & -4 & -5\end{matrix}\right],
\ b = \left[\begin{matrix}-141\\-55\\-784\\-344\\-76\end{matrix}\right]. 
 \end{align*}

Вариант N 4
Дана СЛАУ $AX = b$,
Проверить совместность по теореме Кронекера-Капелли. Если СЛАУ совместна, проверить единственность решения.
Для соответствующей однородной СЛАУ проверить существование нетривиального решения. В случае, если оно существует,
найти размерность пространства решений и составить ФСР и общее решение однородной  и неоднородной СЛАУ.


\begin{align*}
 A = \left[\begin{matrix}5 & 3 & 6 & -5 & 3\\9 & 2 & -8 & 9 & 6\\-21 & 1 & 50 & -51 & -15\\-2 & -3 & 4 & -7 & -1\\-7 & 1 & 2 & 7 & -2\end{matrix}\right],
\ b = \left[\begin{matrix}27\\-90\\441\\36\\40\end{matrix}\right]. 
 \end{align*}

Вариант N 5
Дана СЛАУ $AX = b$,
Проверить совместность по теореме Кронекера-Капелли. Если СЛАУ совместна, проверить единственность решения.
Для соответствующей однородной СЛАУ проверить существование нетривиального решения. В случае, если оно существует,
найти размерность пространства решений и составить ФСР и общее решение однородной  и неоднородной СЛАУ.


\begin{align*}
 A = \left[\begin{matrix}3 & -8 & 8 & -1 & -6\\37 & -56 & 20 & 9 & -30\\157 & -248 & 104 & 33 & -138\\-139 & 200 & -56 & -39 & 102\\-426 & 624 & -192 & -114 & 324\end{matrix}\right],
\ b = \left[\begin{matrix}206\\1146\\5202\\-3966\\-12516\end{matrix}\right]. 
 \end{align*}

Вариант N 6
Дана СЛАУ $AX = b$,
Проверить совместность по теореме Кронекера-Капелли. Если СЛАУ совместна, проверить единственность решения.
Для соответствующей однородной СЛАУ проверить существование нетривиального решения. В случае, если оно существует,
найти размерность пространства решений и составить ФСР и общее решение однородной  и неоднородной СЛАУ.


\begin{align*}
 A = \left[\begin{matrix}3 & 4 & -6 & 2 & 2\\8 & -2 & 4 & 1 & 2\\-31 & 22 & -38 & 1 & -4\\-6 & -7 & -5 & 5 & -7\\-7 & -5 & 3 & -6 & -7\end{matrix}\right],
\ b = \left[\begin{matrix}-46\\7\\-173\\-33\\50\end{matrix}\right]. 
 \end{align*}

Вариант N 7
Дана СЛАУ $AX = b$,
Проверить совместность по теореме Кронекера-Капелли. Если СЛАУ совместна, проверить единственность решения.
Для соответствующей однородной СЛАУ проверить существование нетривиального решения. В случае, если оно существует,
найти размерность пространства решений и составить ФСР и общее решение однородной  и неоднородной СЛАУ.


\begin{align*}
 A = \left[\begin{matrix}2 & 4 & 5 & 5 & 8\\-6 & 7 & -5 & -5 & -7\\32 & -12 & 40 & 40 & 60\\-6 & -4 & 4 & 0 & 9\\-3 & -7 & 5 & 6 & -2\end{matrix}\right],
\ b = \left[\begin{matrix}-115\\43\\-632\\-94\\-23\end{matrix}\right]. 
 \end{align*}

Вариант N 8
Дана СЛАУ $AX = b$,
Проверить совместность по теореме Кронекера-Капелли. Если СЛАУ совместна, проверить единственность решения.
Для соответствующей однородной СЛАУ проверить существование нетривиального решения. В случае, если оно существует,
найти размерность пространства решений и составить ФСР и общее решение однородной  и неоднородной СЛАУ.


\begin{align*}
 A = \left[\begin{matrix}-3 & -3 & 6 & -6 & 8\\7 & 9 & 4 & -9 & 2\\-47 & -57 & 4 & 21 & 22\\6 & -9 & 8 & 5 & 0\\1 & 3 & 5 & -2 & 5\end{matrix}\right],
\ b = \left[\begin{matrix}-171\\-15\\-609\\-51\\-66\end{matrix}\right]. 
 \end{align*}

Вариант N 9
Дана СЛАУ $AX = b$,
Проверить совместность по теореме Кронекера-Капелли. Если СЛАУ совместна, проверить единственность решения.
Для соответствующей однородной СЛАУ проверить существование нетривиального решения. В случае, если оно существует,
найти размерность пространства решений и составить ФСР и общее решение однородной  и неоднородной СЛАУ.


\begin{align*}
 A = \left[\begin{matrix}5 & 3 & 9 & 0 & 2\\4 & 9 & -5 & 0 & -4\\4 & -24 & 56 & 0 & 24\\-2 & 1 & 6 & 2 & -5\\5 & 9 & -3 & -7 & -4\end{matrix}\right],
\ b = \left[\begin{matrix}26\\37\\-44\\12\\66\end{matrix}\right]. 
 \end{align*}

Вариант N 10
Дана СЛАУ $AX = b$,
Проверить совместность по теореме Кронекера-Капелли. Если СЛАУ совместна, проверить единственность решения.
Для соответствующей однородной СЛАУ проверить существование нетривиального решения. В случае, если оно существует,
найти размерность пространства решений и составить ФСР и общее решение однородной  и неоднородной СЛАУ.


\begin{align*}
 A = \left[\begin{matrix}0 & -3 & -1 & 5 & -4\\-8 & -3 & -1 & -1 & -8\\-32 & -21 & -7 & 11 & -44\\32 & 3 & 1 & 19 & 20\\-2 & -8 & -3 & 2 & -9\end{matrix}\right],
\ b = \left[\begin{matrix}33\\-53\\-113\\311\\-28\end{matrix}\right]. 
 \end{align*}

Вариант N 11
Дана СЛАУ $AX = b$,
Проверить совместность по теореме Кронекера-Капелли. Если СЛАУ совместна, проверить единственность решения.
Для соответствующей однородной СЛАУ проверить существование нетривиального решения. В случае, если оно существует,
найти размерность пространства решений и составить ФСР и общее решение однородной  и неоднородной СЛАУ.


\begin{align*}
 A = \left[\begin{matrix}9 & 5 & -7 & 8 & -2\\0 & -7 & 5 & -2 & 5\\36 & -8 & -8 & 24 & 12\\36 & 48 & -48 & 40 & -28\\6 & -5 & 1 & -7 & -2\end{matrix}\right],
\ b = \left[\begin{matrix}46\\-6\\160\\208\\24\end{matrix}\right]. 
 \end{align*}

Вариант N 12
Дана СЛАУ $AX = b$,
Проверить совместность по теореме Кронекера-Капелли. Если СЛАУ совместна, проверить единственность решения.
Для соответствующей однородной СЛАУ проверить существование нетривиального решения. В случае, если оно существует,
найти размерность пространства решений и составить ФСР и общее решение однородной  и неоднородной СЛАУ.


\begin{align*}
 A = \left[\begin{matrix}4 & 3 & -2 & -5 & 8\\4 & 7 & 5 & -6 & 4\\-4 & -23 & -33 & 10 & 12\\9 & 4 & -4 & -8 & -6\\2 & 1 & 6 & 7 & -6\end{matrix}\right],
\ b = \left[\begin{matrix}-41\\64\\-484\\97\\143\end{matrix}\right]. 
 \end{align*}

Вариант N 13
Дана СЛАУ $AX = b$,
Проверить совместность по теореме Кронекера-Капелли. Если СЛАУ совместна, проверить единственность решения.
Для соответствующей однородной СЛАУ проверить существование нетривиального решения. В случае, если оно существует,
найти размерность пространства решений и составить ФСР и общее решение однородной  и неоднородной СЛАУ.


\begin{align*}
 A = \left[\begin{matrix}-8 & 3 & -2 & -2 & -6\\-9 & 4 & 1 & -1 & -4\\13 & -8 & -13 & -3 & -4\\-1 & 1 & 2 & -6 & -2\\6 & -4 & 4 & 0 & -2\end{matrix}\right],
\ b = \left[\begin{matrix}80\\59\\25\\29\\-78\end{matrix}\right]. 
 \end{align*}

Вариант N 14
Дана СЛАУ $AX = b$,
Проверить совместность по теореме Кронекера-Капелли. Если СЛАУ совместна, проверить единственность решения.
Для соответствующей однородной СЛАУ проверить существование нетривиального решения. В случае, если оно существует,
найти размерность пространства решений и составить ФСР и общее решение однородной  и неоднородной СЛАУ.


\begin{align*}
 A = \left[\begin{matrix}-3 & 7 & 9 & -2 & -7\\-1 & 53 & 11 & -22 & -29\\-13 & 233 & 71 & -94 & -137\\-5 & -191 & -17 & 82 & 95\\-6 & -594 & -78 & 252 & 306\end{matrix}\right],
\ b = \left[\begin{matrix}-2\\58\\226\\-238\\-708\end{matrix}\right]. 
 \end{align*}

Вариант N 15
Дана СЛАУ $AX = b$,
Проверить совместность по теореме Кронекера-Капелли. Если СЛАУ совместна, проверить единственность решения.
Для соответствующей однородной СЛАУ проверить существование нетривиального решения. В случае, если оно существует,
найти размерность пространства решений и составить ФСР и общее решение однородной  и неоднородной СЛАУ.


\begin{align*}
 A = \left[\begin{matrix}-6 & -4 & -9 & 0 & -6\\-1 & 1 & 7 & 8 & -8\\-22 & -8 & 1 & 32 & -50\\-14 & -16 & -55 & -32 & 14\\-2 & 2 & 8 & -5 & -3\end{matrix}\right],
\ b = \left[\begin{matrix}110\\13\\382\\278\\-104\end{matrix}\right]. 
 \end{align*}

Вариант N 16
Дана СЛАУ $AX = b$,
Проверить совместность по теореме Кронекера-Капелли. Если СЛАУ совместна, проверить единственность решения.
Для соответствующей однородной СЛАУ проверить существование нетривиального решения. В случае, если оно существует,
найти размерность пространства решений и составить ФСР и общее решение однородной  и неоднородной СЛАУ.


\begin{align*}
 A = \left[\begin{matrix}-9 & -8 & 7 & 9 & 8\\-3 & 8 & 6 & 4 & -3\\-39 & 8 & 45 & 43 & 12\\-15 & -56 & -3 & 11 & 36\\-5 & 1 & -7 & -1 & -4\end{matrix}\right],
\ b = \left[\begin{matrix}191\\28\\685\\461\\-4\end{matrix}\right]. 
 \end{align*}

Вариант N 17
Дана СЛАУ $AX = b$,
Проверить совместность по теореме Кронекера-Капелли. Если СЛАУ совместна, проверить единственность решения.
Для соответствующей однородной СЛАУ проверить существование нетривиального решения. В случае, если оно существует,
найти размерность пространства решений и составить ФСР и общее решение однородной  и неоднородной СЛАУ.


\begin{align*}
 A = \left[\begin{matrix}5 & -3 & 3 & 8 & 7\\-2 & 5 & -2 & 8 & -6\\28 & -32 & 20 & 0 & 52\\-3 & 0 & 2 & 1 & 0\\3 & -4 & 8 & 1 & 1\end{matrix}\right],
\ b = \left[\begin{matrix}-64\\-40\\-96\\-23\\58\end{matrix}\right]. 
 \end{align*}

Вариант N 18
Дана СЛАУ $AX = b$,
Проверить совместность по теореме Кронекера-Капелли. Если СЛАУ совместна, проверить единственность решения.
Для соответствующей однородной СЛАУ проверить существование нетривиального решения. В случае, если оно существует,
найти размерность пространства решений и составить ФСР и общее решение однородной  и неоднородной СЛАУ.


\begin{align*}
 A = \left[\begin{matrix}5 & 6 & 4 & 6 & -7\\7 & 0 & 7 & -8 & -1\\-15 & 24 & -19 & 64 & -23\\-4 & 0 & 0 & 2 & 7\\-3 & 2 & 6 & 8 & 5\end{matrix}\right],
\ b = \left[\begin{matrix}-24\\15\\-171\\43\\54\end{matrix}\right]. 
 \end{align*}

Вариант N 19
Дана СЛАУ $AX = b$,
Проверить совместность по теореме Кронекера-Капелли. Если СЛАУ совместна, проверить единственность решения.
Для соответствующей однородной СЛАУ проверить существование нетривиального решения. В случае, если оно существует,
найти размерность пространства решений и составить ФСР и общее решение однородной  и неоднородной СЛАУ.


\begin{align*}
 A = \left[\begin{matrix}4 & -2 & 5 & -6 & 3\\36 & 30 & 7 & -6 & 25\\156 & 114 & 43 & -42 & 109\\-132 & -126 & -13 & 6 & -91\\-408 & -372 & -54 & 36 & -282\end{matrix}\right],
\ b = \left[\begin{matrix}-15\\115\\415\\-505\\-1470\end{matrix}\right]. 
 \end{align*}

Вариант N 20
Дана СЛАУ $AX = b$,
Проверить совместность по теореме Кронекера-Капелли. Если СЛАУ совместна, проверить единственность решения.
Для соответствующей однородной СЛАУ проверить существование нетривиального решения. В случае, если оно существует,
найти размерность пространства решений и составить ФСР и общее решение однородной  и неоднородной СЛАУ.


\begin{align*}
 A = \left[\begin{matrix}0 & 8 & 4 & 5 & 8\\-6 & 9 & 2 & 8 & -5\\-30 & 77 & 26 & 60 & 7\\30 & -13 & 6 & -20 & 57\\7 & 8 & -9 & -3 & 0\end{matrix}\right],
\ b = \left[\begin{matrix}-26\\66\\226\\-434\\-10\end{matrix}\right]. 
 \end{align*}

Вариант N 21
Дана СЛАУ $AX = b$,
Проверить совместность по теореме Кронекера-Капелли. Если СЛАУ совместна, проверить единственность решения.
Для соответствующей однородной СЛАУ проверить существование нетривиального решения. В случае, если оно существует,
найти размерность пространства решений и составить ФСР и общее решение однородной  и неоднородной СЛАУ.


\begin{align*}
 A = \left[\begin{matrix}-5 & -7 & -1 & 6 & 3\\-9 & -9 & 2 & 3 & 9\\-56 & -64 & 4 & 36 & 48\\16 & 8 & -12 & 12 & -24\\-8 & -1 & 6 & -9 & -3\end{matrix}\right],
\ b = \left[\begin{matrix}-20\\-50\\-280\\120\\90\end{matrix}\right]. 
 \end{align*}

Вариант N 22
Дана СЛАУ $AX = b$,
Проверить совместность по теореме Кронекера-Капелли. Если СЛАУ совместна, проверить единственность решения.
Для соответствующей однородной СЛАУ проверить существование нетривиального решения. В случае, если оно существует,
найти размерность пространства решений и составить ФСР и общее решение однородной  и неоднородной СЛАУ.


\begin{align*}
 A = \left[\begin{matrix}7 & 7 & 1 & 9 & -5\\-4 & 46 & -42 & 2 & -15\\1 & 251 & -207 & 37 & -90\\41 & -209 & 213 & 17 & 60\\102 & -648 & 636 & 24 & 195\end{matrix}\right],
\ b = \left[\begin{matrix}28\\439\\2279\\-2111\\-6417\end{matrix}\right]. 
 \end{align*}

Вариант N 23
Дана СЛАУ $AX = b$,
Проверить совместность по теореме Кронекера-Капелли. Если СЛАУ совместна, проверить единственность решения.
Для соответствующей однородной СЛАУ проверить существование нетривиального решения. В случае, если оно существует,
найти размерность пространства решений и составить ФСР и общее решение однородной  и неоднородной СЛАУ.


\begin{align*}
 A = \left[\begin{matrix}5 & 3 & -2 & -3 & -8\\5 & -9 & -4 & -3 & -5\\-5 & 45 & 10 & 3 & -4\\-2 & -3 & -9 & -6 & 1\\-3 & 6 & -6 & -2 & -8\end{matrix}\right],
\ b = \left[\begin{matrix}-3\\-9\\27\\-11\\48\end{matrix}\right]. 
 \end{align*}

Вариант N 24
Дана СЛАУ $AX = b$,
Проверить совместность по теореме Кронекера-Капелли. Если СЛАУ совместна, проверить единственность решения.
Для соответствующей однородной СЛАУ проверить существование нетривиального решения. В случае, если оно существует,
найти размерность пространства решений и составить ФСР и общее решение однородной  и неоднородной СЛАУ.


\begin{align*}
 A = \left[\begin{matrix}0 & -4 & 5 & -3 & -9\\-40 & -12 & 20 & 16 & 8\\-200 & -72 & 115 & 71 & 13\\200 & 48 & -85 & -89 & -67\\600 & 156 & -270 & -258 & -174\end{matrix}\right],
\ b = \left[\begin{matrix}34\\252\\1362\\-1158\\-3576\end{matrix}\right]. 
 \end{align*}

Вариант N 25
Дана СЛАУ $AX = b$,
Проверить совместность по теореме Кронекера-Капелли. Если СЛАУ совместна, проверить единственность решения.
Для соответствующей однородной СЛАУ проверить существование нетривиального решения. В случае, если оно существует,
найти размерность пространства решений и составить ФСР и общее решение однородной  и неоднородной СЛАУ.


\begin{align*}
 A = \left[\begin{matrix}-1 & 9 & 4 & -2 & -2\\-9 & 8 & -6 & 2 & 9\\33 & -5 & 36 & -14 & -42\\-5 & 2 & 5 & 5 & -4\\-2 & 1 & 9 & -7 & -9\end{matrix}\right],
\ b = \left[\begin{matrix}-61\\-215\\677\\-84\\75\end{matrix}\right]. 
 \end{align*}

Вариант N 26
Дана СЛАУ $AX = b$,
Проверить совместность по теореме Кронекера-Капелли. Если СЛАУ совместна, проверить единственность решения.
Для соответствующей однородной СЛАУ проверить существование нетривиального решения. В случае, если оно существует,
найти размерность пространства решений и составить ФСР и общее решение однородной  и неоднородной СЛАУ.


\begin{align*}
 A = \left[\begin{matrix}-3 & -7 & 2 & -9 & -7\\9 & 2 & -3 & 0 & 1\\-57 & -38 & 23 & -36 & -33\\2 & 9 & 2 & -1 & 5\\1 & -2 & -9 & 6 & 9\end{matrix}\right],
\ b = \left[\begin{matrix}-128\\68\\-852\\93\\-24\end{matrix}\right]. 
 \end{align*}

Вариант N 27
Дана СЛАУ $AX = b$,
Проверить совместность по теореме Кронекера-Капелли. Если СЛАУ совместна, проверить единственность решения.
Для соответствующей однородной СЛАУ проверить существование нетривиального решения. В случае, если оно существует,
найти размерность пространства решений и составить ФСР и общее решение однородной  и неоднородной СЛАУ.


\begin{align*}
 A = \left[\begin{matrix}6 & 0 & 3 & -9 & -1\\-6 & -1 & -3 & 2 & 2\\54 & 5 & 27 & -46 & -14\\9 & -7 & 6 & 3 & 5\\-6 & -2 & -2 & 7 & -5\end{matrix}\right],
\ b = \left[\begin{matrix}-13\\-5\\-27\\-12\\-5\end{matrix}\right]. 
 \end{align*}

Вариант N 28
Дана СЛАУ $AX = b$,
Проверить совместность по теореме Кронекера-Капелли. Если СЛАУ совместна, проверить единственность решения.
Для соответствующей однородной СЛАУ проверить существование нетривиального решения. В случае, если оно существует,
найти размерность пространства решений и составить ФСР и общее решение однородной  и неоднородной СЛАУ.


\begin{align*}
 A = \left[\begin{matrix}1 & -8 & -5 & 7 & -3\\-41 & -32 & -20 & -7 & -47\\-201 & -192 & -120 & -7 & -247\\209 & 128 & 80 & 63 & 223\\623 & 416 & 260 & 161 & 681\end{matrix}\right],
\ b = \left[\begin{matrix}114\\286\\1886\\-974\\-3378\end{matrix}\right]. 
 \end{align*}

Вариант N 29
Дана СЛАУ $AX = b$,
Проверить совместность по теореме Кронекера-Капелли. Если СЛАУ совместна, проверить единственность решения.
Для соответствующей однородной СЛАУ проверить существование нетривиального решения. В случае, если оно существует,
найти размерность пространства решений и составить ФСР и общее решение однородной  и неоднородной СЛАУ.


\begin{align*}
 A = \left[\begin{matrix}-1 & -8 & 2 & -6 & 5\\-40 & -32 & -12 & -36 & 52\\-164 & -160 & -40 & -168 & 228\\156 & 96 & 56 & 120 & -188\\472 & 320 & 160 & 384 & -584\end{matrix}\right],
\ b = \left[\begin{matrix}-21\\-492\\-2052\\1884\\5736\end{matrix}\right]. 
 \end{align*}

Вариант N 30
Дана СЛАУ $AX = b$,
Проверить совместность по теореме Кронекера-Капелли. Если СЛАУ совместна, проверить единственность решения.
Для соответствующей однородной СЛАУ проверить существование нетривиального решения. В случае, если оно существует,
найти размерность пространства решений и составить ФСР и общее решение однородной  и неоднородной СЛАУ.


\begin{align*}
 A = \left[\begin{matrix}-4 & 9 & 3 & 8 & 2\\-4 & 5 & -4 & 7 & -4\\0 & 16 & 28 & 4 & 24\\-7 & 5 & 4 & 0 & 8\\1 & -1 & 5 & 2 & -6\end{matrix}\right],
\ b = \left[\begin{matrix}84\\-31\\460\\134\\-41\end{matrix}\right]. 
 \end{align*}

Вариант N 31
Дана СЛАУ $AX = b$,
Проверить совместность по теореме Кронекера-Капелли. Если СЛАУ совместна, проверить единственность решения.
Для соответствующей однородной СЛАУ проверить существование нетривиального решения. В случае, если оно существует,
найти размерность пространства решений и составить ФСР и общее решение однородной  и неоднородной СЛАУ.


\begin{align*}
 A = \left[\begin{matrix}0 & 3 & 0 & 8 & -6\\36 & 44 & 28 & 32 & -48\\144 & 188 & 112 & 160 & -216\\-144 & -164 & -112 & -96 & 168\\-432 & -504 & -336 & -320 & 528\end{matrix}\right],
\ b = \left[\begin{matrix}70\\412\\1928\\-1368\\-4384\end{matrix}\right]. 
 \end{align*}

Вариант N 32
Дана СЛАУ $AX = b$,
Проверить совместность по теореме Кронекера-Капелли. Если СЛАУ совместна, проверить единственность решения.
Для соответствующей однородной СЛАУ проверить существование нетривиального решения. В случае, если оно существует,
найти размерность пространства решений и составить ФСР и общее решение однородной  и неоднородной СЛАУ.


\begin{align*}
 A = \left[\begin{matrix}-3 & 5 & 6 & -3 & -7\\1 & 5 & 8 & 1 & 7\\-13 & -5 & -14 & -13 & -49\\2 & -3 & 3 & -1 & 9\\-5 & -1 & 7 & -5 & 1\end{matrix}\right],
\ b = \left[\begin{matrix}87\\9\\225\\-54\\-15\end{matrix}\right]. 
 \end{align*}

Вариант N 33
Дана СЛАУ $AX = b$,
Проверить совместность по теореме Кронекера-Капелли. Если СЛАУ совместна, проверить единственность решения.
Для соответствующей однородной СЛАУ проверить существование нетривиального решения. В случае, если оно существует,
найти размерность пространства решений и составить ФСР и общее решение однородной  и неоднородной СЛАУ.


\begin{align*}
 A = \left[\begin{matrix}-8 & 8 & -2 & -1 & 0\\-28 & 16 & -10 & -3 & 8\\-136 & 88 & -46 & -15 & 32\\88 & -40 & 34 & 9 & -32\\288 & -144 & 108 & 30 & -96\end{matrix}\right],
\ b = \left[\begin{matrix}-24\\-140\\-632\\488\\1536\end{matrix}\right]. 
 \end{align*}

Вариант N 34
Дана СЛАУ $AX = b$,
Проверить совместность по теореме Кронекера-Капелли. Если СЛАУ совместна, проверить единственность решения.
Для соответствующей однородной СЛАУ проверить существование нетривиального решения. В случае, если оно существует,
найти размерность пространства решений и составить ФСР и общее решение однородной  и неоднородной СЛАУ.


\begin{align*}
 A = \left[\begin{matrix}4 & 0 & -8 & 6 & 7\\28 & 4 & -20 & -2 & 33\\124 & 16 & -104 & 10 & 153\\-100 & -16 & 56 & 26 & -111\\-312 & -48 & 192 & 60 & -354\end{matrix}\right],
\ b = \left[\begin{matrix}-107\\-365\\-1781\\1139\\3738\end{matrix}\right]. 
 \end{align*}

Вариант N 35
Дана СЛАУ $AX = b$,
Проверить совместность по теореме Кронекера-Капелли. Если СЛАУ совместна, проверить единственность решения.
Для соответствующей однородной СЛАУ проверить существование нетривиального решения. В случае, если оно существует,
найти размерность пространства решений и составить ФСР и общее решение однородной  и неоднородной СЛАУ.


\begin{align*}
 A = \left[\begin{matrix}-3 & -9 & 6 & -8 & 3\\-48 & -4 & 8 & -64 & 28\\-204 & -52 & 56 & -288 & 124\\180 & -20 & -8 & 224 & -100\\552 & -24 & -48 & 704 & -312\end{matrix}\right],
\ b = \left[\begin{matrix}59\\0\\236\\236\\472\end{matrix}\right]. 
 \end{align*}

Вариант N 36
Дана СЛАУ $AX = b$,
Проверить совместность по теореме Кронекера-Капелли. Если СЛАУ совместна, проверить единственность решения.
Для соответствующей однородной СЛАУ проверить существование нетривиального решения. В случае, если оно существует,
найти размерность пространства решений и составить ФСР и общее решение однородной  и неоднородной СЛАУ.


\begin{align*}
 A = \left[\begin{matrix}-8 & -9 & -2 & -5 & 5\\-24 & -2 & 9 & -25 & -15\\-144 & -37 & 39 & -140 & -60\\96 & -17 & -51 & 110 & 90\\312 & -24 & -147 & 345 & 255\end{matrix}\right],
\ b = \left[\begin{matrix}37\\-14\\41\\181\\432\end{matrix}\right]. 
 \end{align*}

Вариант N 37
Дана СЛАУ $AX = b$,
Проверить совместность по теореме Кронекера-Капелли. Если СЛАУ совместна, проверить единственность решения.
Для соответствующей однородной СЛАУ проверить существование нетривиального решения. В случае, если оно существует,
найти размерность пространства решений и составить ФСР и общее решение однородной  и неоднородной СЛАУ.


\begin{align*}
 A = \left[\begin{matrix}-9 & -5 & 8 & 1 & -9\\3 & 7 & 3 & -7 & -7\\-24 & 8 & 44 & -24 & -64\\-48 & -48 & 20 & 32 & -8\\-2 & 9 & -8 & -8 & -7\end{matrix}\right],
\ b = \left[\begin{matrix}10\\-37\\-108\\188\\44\end{matrix}\right]. 
 \end{align*}

Вариант N 38
Дана СЛАУ $AX = b$,
Проверить совместность по теореме Кронекера-Капелли. Если СЛАУ совместна, проверить единственность решения.
Для соответствующей однородной СЛАУ проверить существование нетривиального решения. В случае, если оно существует,
найти размерность пространства решений и составить ФСР и общее решение однородной  и неоднородной СЛАУ.


\begin{align*}
 A = \left[\begin{matrix}5 & 3 & 9 & 2 & 2\\43 & 17 & 59 & -6 & -18\\187 & 77 & 263 & -18 & -66\\-157 & -59 & -209 & 30 & 78\\-486 & -186 & -654 & 84 & 228\end{matrix}\right],
\ b = \left[\begin{matrix}-72\\-452\\-2024\\1592\\4992\end{matrix}\right]. 
 \end{align*}

Вариант N 39
Дана СЛАУ $AX = b$,
Проверить совместность по теореме Кронекера-Капелли. Если СЛАУ совместна, проверить единственность решения.
Для соответствующей однородной СЛАУ проверить существование нетривиального решения. В случае, если оно существует,
найти размерность пространства решений и составить ФСР и общее решение однородной  и неоднородной СЛАУ.


\begin{align*}
 A = \left[\begin{matrix}6 & -3 & 7 & -9 & -4\\-3 & -7 & 3 & -2 & -6\\39 & 23 & 13 & -26 & 14\\2 & 6 & 0 & 6 & 8\\9 & -5 & -6 & -5 & -9\end{matrix}\right],
\ b = \left[\begin{matrix}-34\\-35\\39\\20\\-52\end{matrix}\right]. 
 \end{align*}

Вариант N 40
Дана СЛАУ $AX = b$,
Проверить совместность по теореме Кронекера-Капелли. Если СЛАУ совместна, проверить единственность решения.
Для соответствующей однородной СЛАУ проверить существование нетривиального решения. В случае, если оно существует,
найти размерность пространства решений и составить ФСР и общее решение однородной  и неоднородной СЛАУ.


\begin{align*}
 A = \left[\begin{matrix}-6 & 8 & 6 & -2 & 4\\9 & 8 & -5 & -9 & -9\\18 & 56 & -2 & -42 & -24\\-54 & -8 & 38 & 30 & 48\\-2 & -2 & 0 & 0 & 7\end{matrix}\right],
\ b = \left[\begin{matrix}26\\-211\\-766\\922\\58\end{matrix}\right]. 
 \end{align*}

Вариант N 41
Дана СЛАУ $AX = b$,
Проверить совместность по теореме Кронекера-Капелли. Если СЛАУ совместна, проверить единственность решения.
Для соответствующей однородной СЛАУ проверить существование нетривиального решения. В случае, если оно существует,
найти размерность пространства решений и составить ФСР и общее решение однородной  и неоднородной СЛАУ.


\begin{align*}
 A = \left[\begin{matrix}6 & 6 & -1 & -5 & -9\\-2 & 4 & 0 & 3 & -3\\32 & 8 & -4 & -32 & -24\\5 & 5 & 2 & 2 & 5\\-6 & -1 & -8 & 2 & 8\end{matrix}\right],
\ b = \left[\begin{matrix}4\\51\\-188\\-7\\-22\end{matrix}\right]. 
 \end{align*}

Вариант N 42
Дана СЛАУ $AX = b$,
Проверить совместность по теореме Кронекера-Капелли. Если СЛАУ совместна, проверить единственность решения.
Для соответствующей однородной СЛАУ проверить существование нетривиального решения. В случае, если оно существует,
найти размерность пространства решений и составить ФСР и общее решение однородной  и неоднородной СЛАУ.


\begin{align*}
 A = \left[\begin{matrix}6 & -7 & -9 & 7 & -9\\-3 & 2 & 0 & -3 & -2\\30 & -29 & -27 & 33 & -19\\3 & 5 & 1 & 2 & -6\\-7 & -7 & 4 & 4 & 5\end{matrix}\right],
\ b = \left[\begin{matrix}-2\\0\\-6\\-5\\113\end{matrix}\right]. 
 \end{align*}

Вариант N 43
Дана СЛАУ $AX = b$,
Проверить совместность по теореме Кронекера-Капелли. Если СЛАУ совместна, проверить единственность решения.
Для соответствующей однородной СЛАУ проверить существование нетривиального решения. В случае, если оно существует,
найти размерность пространства решений и составить ФСР и общее решение однородной  и неоднородной СЛАУ.


\begin{align*}
 A = \left[\begin{matrix}-2 & -3 & 1 & 8 & 5\\9 & 2 & -1 & 5 & 8\\-44 & -20 & 8 & 12 & -12\\-6 & 0 & -8 & 2 & 8\\9 & 3 & 4 & -3 & -9\end{matrix}\right],
\ b = \left[\begin{matrix}37\\65\\-112\\-26\\8\end{matrix}\right]. 
 \end{align*}

Вариант N 44
Дана СЛАУ $AX = b$,
Проверить совместность по теореме Кронекера-Капелли. Если СЛАУ совместна, проверить единственность решения.
Для соответствующей однородной СЛАУ проверить существование нетривиального решения. В случае, если оно существует,
найти размерность пространства решений и составить ФСР и общее решение однородной  и неоднородной СЛАУ.


\begin{align*}
 A = \left[\begin{matrix}7 & -7 & -4 & 3 & -1\\-6 & 5 & 1 & -5 & 8\\58 & -53 & -21 & 37 & -44\\-4 & -1 & -8 & -4 & -2\\-4 & -6 & -8 & -7 & 5\end{matrix}\right],
\ b = \left[\begin{matrix}-28\\-16\\-32\\41\\37\end{matrix}\right]. 
 \end{align*}

Вариант N 45
Дана СЛАУ $AX = b$,
Проверить совместность по теореме Кронекера-Капелли. Если СЛАУ совместна, проверить единственность решения.
Для соответствующей однородной СЛАУ проверить существование нетривиального решения. В случае, если оно существует,
найти размерность пространства решений и составить ФСР и общее решение однородной  и неоднородной СЛАУ.


\begin{align*}
 A = \left[\begin{matrix}-8 & -4 & -6 & -2 & -4\\7 & 9 & -1 & -2 & -1\\-60 & -52 & -20 & 0 & -12\\8 & 8 & 2 & 4 & 3\\0 & 3 & -1 & 2 & -7\end{matrix}\right],
\ b = \left[\begin{matrix}-54\\-55\\4\\11\\21\end{matrix}\right]. 
 \end{align*}

Вариант N 46
Дана СЛАУ $AX = b$,
Проверить совместность по теореме Кронекера-Капелли. Если СЛАУ совместна, проверить единственность решения.
Для соответствующей однородной СЛАУ проверить существование нетривиального решения. В случае, если оно существует,
найти размерность пространства решений и составить ФСР и общее решение однородной  и неоднородной СЛАУ.


\begin{align*}
 A = \left[\begin{matrix}-4 & 4 & 5 & 7 & -4\\0 & 6 & 9 & 4 & -5\\-12 & 36 & 51 & 37 & -32\\-12 & -12 & -21 & 5 & 8\\-7 & -7 & 7 & -6 & 9\end{matrix}\right],
\ b = \left[\begin{matrix}-3\\-39\\-165\\147\\45\end{matrix}\right]. 
 \end{align*}

Вариант N 47
Дана СЛАУ $AX = b$,
Проверить совместность по теореме Кронекера-Капелли. Если СЛАУ совместна, проверить единственность решения.
Для соответствующей однородной СЛАУ проверить существование нетривиального решения. В случае, если оно существует,
найти размерность пространства решений и составить ФСР и общее решение однородной  и неоднородной СЛАУ.


\begin{align*}
 A = \left[\begin{matrix}-6 & -9 & -3 & 1 & 9\\-69 & -76 & 18 & -11 & 6\\-369 & -416 & 78 & -51 & 66\\321 & 344 & -102 & 59 & 6\\987 & 1068 & -294 & 173 & -18\end{matrix}\right],
\ b = \left[\begin{matrix}6\\79\\419\\-371\\-1137\end{matrix}\right]. 
 \end{align*}

Вариант N 48
Дана СЛАУ $AX = b$,
Проверить совместность по теореме Кронекера-Капелли. Если СЛАУ совместна, проверить единственность решения.
Для соответствующей однородной СЛАУ проверить существование нетривиального решения. В случае, если оно существует,
найти размерность пространства решений и составить ФСР и общее решение однородной  и неоднородной СЛАУ.


\begin{align*}
 A = \left[\begin{matrix}-4 & -5 & 9 & 3 & -1\\-1 & -5 & 0 & 8 & 2\\-7 & 10 & 27 & -31 & -13\\0 & 2 & -2 & -8 & 3\\-5 & -3 & 3 & 6 & 8\end{matrix}\right],
\ b = \left[\begin{matrix}-20\\78\\-450\\-51\\27\end{matrix}\right]. 
 \end{align*}

Вариант N 49
Дана СЛАУ $AX = b$,
Проверить совместность по теореме Кронекера-Капелли. Если СЛАУ совместна, проверить единственность решения.
Для соответствующей однородной СЛАУ проверить существование нетривиального решения. В случае, если оно существует,
найти размерность пространства решений и составить ФСР и общее решение однородной  и неоднородной СЛАУ.


\begin{align*}
 A = \left[\begin{matrix}-2 & 8 & 8 & -7 & -1\\-41 & 19 & -16 & -11 & -43\\-211 & 119 & -56 & -76 & -218\\199 & -71 & 104 & 34 & 212\\603 & -237 & 288 & 123 & 639\end{matrix}\right],
\ b = \left[\begin{matrix}45\\-155\\-640\\910\\2595\end{matrix}\right]. 
 \end{align*}

Вариант N 50
Дана СЛАУ $AX = b$,
Проверить совместность по теореме Кронекера-Капелли. Если СЛАУ совместна, проверить единственность решения.
Для соответствующей однородной СЛАУ проверить существование нетривиального решения. В случае, если оно существует,
найти размерность пространства решений и составить ФСР и общее решение однородной  и неоднородной СЛАУ.


\begin{align*}
 A = \left[\begin{matrix}-7 & -8 & -5 & 0 & 9\\-43 & 3 & -30 & -30 & 11\\-243 & -17 & -170 & -150 & 91\\187 & -47 & 130 & 150 & -19\\589 & -109 & 410 & 450 & -93\end{matrix}\right],
\ b = \left[\begin{matrix}37\\33\\313\\-17\\-199\end{matrix}\right]. 
 \end{align*}

Вариант N 51
Дана СЛАУ $AX = b$,
Проверить совместность по теореме Кронекера-Капелли. Если СЛАУ совместна, проверить единственность решения.
Для соответствующей однородной СЛАУ проверить существование нетривиального решения. В случае, если оно существует,
найти размерность пространства решений и составить ФСР и общее решение однородной  и неоднородной СЛАУ.


\begin{align*}
 A = \left[\begin{matrix}-6 & 1 & 8 & -7 & -9\\-6 & -4 & 6 & -4 & 1\\-48 & -17 & 54 & -41 & -22\\12 & 23 & -6 & -1 & -32\\6 & -1 & 2 & 7 & -9\end{matrix}\right],
\ b = \left[\begin{matrix}11\\-3\\18\\48\\-115\end{matrix}\right]. 
 \end{align*}

Вариант N 52
Дана СЛАУ $AX = b$,
Проверить совместность по теореме Кронекера-Капелли. Если СЛАУ совместна, проверить единственность решения.
Для соответствующей однородной СЛАУ проверить существование нетривиального решения. В случае, если оно существует,
найти размерность пространства решений и составить ФСР и общее решение однородной  и неоднородной СЛАУ.


\begin{align*}
 A = \left[\begin{matrix}-3 & 2 & -1 & -5 & 7\\23 & -10 & -15 & 21 & 37\\83 & -34 & -63 & 69 & 169\\-101 & 46 & 57 & -99 & -127\\-294 & 132 & 174 & -282 & -402\end{matrix}\right],
\ b = \left[\begin{matrix}42\\-138\\-426\\678\\1908\end{matrix}\right]. 
 \end{align*}

Вариант N 53
Дана СЛАУ $AX = b$,
Проверить совместность по теореме Кронекера-Капелли. Если СЛАУ совместна, проверить единственность решения.
Для соответствующей однородной СЛАУ проверить существование нетривиального решения. В случае, если оно существует,
найти размерность пространства решений и составить ФСР и общее решение однородной  и неоднородной СЛАУ.


\begin{align*}
 A = \left[\begin{matrix}7 & 9 & 4 & 8 & 9\\9 & 4 & -3 & -1 & -1\\73 & 56 & 1 & 27 & 31\\-17 & 16 & 31 & 37 & 41\\-4 & -2 & 9 & 9 & -6\end{matrix}\right],
\ b = \left[\begin{matrix}-30\\-63\\-435\\195\\28\end{matrix}\right]. 
 \end{align*}

Вариант N 54
Дана СЛАУ $AX = b$,
Проверить совместность по теореме Кронекера-Капелли. Если СЛАУ совместна, проверить единственность решения.
Для соответствующей однородной СЛАУ проверить существование нетривиального решения. В случае, если оно существует,
найти размерность пространства решений и составить ФСР и общее решение однородной  и неоднородной СЛАУ.


\begin{align*}
 A = \left[\begin{matrix}1 & -1 & 6 & 0 & 7\\33 & 12 & 58 & -20 & 51\\168 & 57 & 308 & -100 & 276\\-162 & -63 & -272 & 100 & -234\\-489 & -186 & -834 & 300 & -723\end{matrix}\right],
\ b = \left[\begin{matrix}-50\\-470\\-2500\\2200\\6750\end{matrix}\right]. 
 \end{align*}

Вариант N 55
Дана СЛАУ $AX = b$,
Проверить совместность по теореме Кронекера-Капелли. Если СЛАУ совместна, проверить единственность решения.
Для соответствующей однородной СЛАУ проверить существование нетривиального решения. В случае, если оно существует,
найти размерность пространства решений и составить ФСР и общее решение однородной  и неоднородной СЛАУ.


\begin{align*}
 A = \left[\begin{matrix}-2 & -5 & 3 & -4 & -2\\2 & 3 & 7 & 4 & -4\\-16 & -30 & -26 & -32 & 14\\5 & 2 & 7 & -2 & -5\\-6 & 0 & 8 & 2 & 0\end{matrix}\right],
\ b = \left[\begin{matrix}-53\\81\\-564\\54\\34\end{matrix}\right]. 
 \end{align*}

Вариант N 56
Дана СЛАУ $AX = b$,
Проверить совместность по теореме Кронекера-Капелли. Если СЛАУ совместна, проверить единственность решения.
Для соответствующей однородной СЛАУ проверить существование нетривиального решения. В случае, если оно существует,
найти размерность пространства решений и составить ФСР и общее решение однородной  и неоднородной СЛАУ.


\begin{align*}
 A = \left[\begin{matrix}-4 & -2 & -6 & -1 & 5\\5 & -7 & 4 & -2 & -8\\-32 & 22 & -34 & 5 & 47\\7 & 7 & -4 & 9 & -7\\-9 & 8 & 9 & -8 & 5\end{matrix}\right],
\ b = \left[\begin{matrix}72\\-2\\224\\-176\\35\end{matrix}\right]. 
 \end{align*}

Вариант N 57
Дана СЛАУ $AX = b$,
Проверить совместность по теореме Кронекера-Капелли. Если СЛАУ совместна, проверить единственность решения.
Для соответствующей однородной СЛАУ проверить существование нетривиального решения. В случае, если оно существует,
найти размерность пространства решений и составить ФСР и общее решение однородной  и неоднородной СЛАУ.


\begin{align*}
 A = \left[\begin{matrix}-4 & -6 & 1 & -3 & 3\\8 & -9 & 9 & 2 & -5\\-44 & 18 & -33 & -17 & 29\\4 & 5 & 4 & 0 & 2\\-6 & -1 & -4 & 9 & 2\end{matrix}\right],
\ b = \left[\begin{matrix}83\\33\\117\\2\\-68\end{matrix}\right]. 
 \end{align*}

Вариант N 58
Дана СЛАУ $AX = b$,
Проверить совместность по теореме Кронекера-Капелли. Если СЛАУ совместна, проверить единственность решения.
Для соответствующей однородной СЛАУ проверить существование нетривиального решения. В случае, если оно существует,
найти размерность пространства решений и составить ФСР и общее решение однородной  и неоднородной СЛАУ.


\begin{align*}
 A = \left[\begin{matrix}9 & 2 & 1 & -5 & -5\\7 & 9 & 0 & 0 & 6\\64 & 44 & 4 & -20 & 4\\8 & -28 & 4 & -20 & -44\\6 & 8 & -4 & 6 & 1\end{matrix}\right],
\ b = \left[\begin{matrix}-27\\-94\\-484\\268\\-55\end{matrix}\right]. 
 \end{align*}

Вариант N 59
Дана СЛАУ $AX = b$,
Проверить совместность по теореме Кронекера-Капелли. Если СЛАУ совместна, проверить единственность решения.
Для соответствующей однородной СЛАУ проверить существование нетривиального решения. В случае, если оно существует,
найти размерность пространства решений и составить ФСР и общее решение однородной  и неоднородной СЛАУ.


\begin{align*}
 A = \left[\begin{matrix}6 & 8 & 8 & -9 & -5\\-6 & 6 & 3 & -5 & -1\\-6 & 48 & 36 & -47 & -19\\42 & 0 & 12 & -7 & -11\\5 & 7 & 8 & -2 & -5\end{matrix}\right],
\ b = \left[\begin{matrix}-47\\54\\75\\-357\\-99\end{matrix}\right]. 
 \end{align*}

Вариант N 60
Дана СЛАУ $AX = b$,
Проверить совместность по теореме Кронекера-Капелли. Если СЛАУ совместна, проверить единственность решения.
Для соответствующей однородной СЛАУ проверить существование нетривиального решения. В случае, если оно существует,
найти размерность пространства решений и составить ФСР и общее решение однородной  и неоднородной СЛАУ.


\begin{align*}
 A = \left[\begin{matrix}-3 & -7 & 4 & -6 & 2\\-21 & -5 & 44 & -54 & 38\\-93 & -41 & 188 & -234 & 158\\75 & -1 & -164 & 198 & -146\\234 & 18 & -504 & 612 & -444\end{matrix}\right],
\ b = \left[\begin{matrix}-38\\110\\326\\-554\\-1548\end{matrix}\right]. 
 \end{align*}

Вариант N 61
Дана СЛАУ $AX = b$,
Проверить совместность по теореме Кронекера-Капелли. Если СЛАУ совместна, проверить единственность решения.
Для соответствующей однородной СЛАУ проверить существование нетривиального решения. В случае, если оно существует,
найти размерность пространства решений и составить ФСР и общее решение однородной  и неоднородной СЛАУ.


\begin{align*}
 A = \left[\begin{matrix}2 & 0 & -8 & 2 & 8\\-7 & -7 & 5 & 1 & -3\\43 & 35 & -57 & 3 & 47\\-8 & -7 & 9 & 9 & 8\\-5 & 4 & -6 & 1 & -6\end{matrix}\right],
\ b = \left[\begin{matrix}62\\-80\\648\\-27\\-35\end{matrix}\right]. 
 \end{align*}

Вариант N 62
Дана СЛАУ $AX = b$,
Проверить совместность по теореме Кронекера-Капелли. Если СЛАУ совместна, проверить единственность решения.
Для соответствующей однородной СЛАУ проверить существование нетривиального решения. В случае, если оно существует,
найти размерность пространства решений и составить ФСР и общее решение однородной  и неоднородной СЛАУ.


\begin{align*}
 A = \left[\begin{matrix}-3 & 3 & 1 & 1 & 0\\36 & -26 & -42 & -17 & -25\\171 & -121 & -207 & -82 & -125\\-189 & 139 & 213 & 88 & 125\\-558 & 408 & 636 & 261 & 375\end{matrix}\right],
\ b = \left[\begin{matrix}-33\\461\\2206\\-2404\\-7113\end{matrix}\right]. 
 \end{align*}

Вариант N 63
Дана СЛАУ $AX = b$,
Проверить совместность по теореме Кронекера-Капелли. Если СЛАУ совместна, проверить единственность решения.
Для соответствующей однородной СЛАУ проверить существование нетривиального решения. В случае, если оно существует,
найти размерность пространства решений и составить ФСР и общее решение однородной  и неоднородной СЛАУ.


\begin{align*}
 A = \left[\begin{matrix}5 & 8 & 0 & -7 & 6\\55 & 42 & -30 & 17 & 39\\295 & 242 & -150 & 57 & 219\\-255 & -178 & 150 & -113 & -171\\-785 & -566 & 450 & -311 & -537\end{matrix}\right],
\ b = \left[\begin{matrix}80\\380\\2220\\-1580\\-5060\end{matrix}\right]. 
 \end{align*}

Вариант N 64
Дана СЛАУ $AX = b$,
Проверить совместность по теореме Кронекера-Капелли. Если СЛАУ совместна, проверить единственность решения.
Для соответствующей однородной СЛАУ проверить существование нетривиального решения. В случае, если оно существует,
найти размерность пространства решений и составить ФСР и общее решение однородной  и неоднородной СЛАУ.


\begin{align*}
 A = \left[\begin{matrix}2 & 8 & -4 & -2 & -7\\26 & 54 & 3 & -11 & -26\\136 & 294 & 3 & -61 & -151\\-124 & -246 & -27 & 49 & 109\\-378 & -762 & -69 & 153 & 348\end{matrix}\right],
\ b = \left[\begin{matrix}-37\\-36\\-291\\69\\318\end{matrix}\right]. 
 \end{align*}

Вариант N 65
Дана СЛАУ $AX = b$,
Проверить совместность по теореме Кронекера-Капелли. Если СЛАУ совместна, проверить единственность решения.
Для соответствующей однородной СЛАУ проверить существование нетривиального решения. В случае, если оно существует,
найти размерность пространства решений и составить ФСР и общее решение однородной  и неоднородной СЛАУ.


\begin{align*}
 A = \left[\begin{matrix}-6 & 3 & -7 & 2 & -3\\-14 & 45 & -21 & 38 & -25\\-74 & 189 & -105 & 158 & -109\\38 & -171 & 63 & -146 & 91\\132 & -522 & 210 & -444 & 282\end{matrix}\right],
\ b = \left[\begin{matrix}-49\\-175\\-847\\553\\1806\end{matrix}\right]. 
 \end{align*}

Вариант N 66
Дана СЛАУ $AX = b$,
Проверить совместность по теореме Кронекера-Капелли. Если СЛАУ совместна, проверить единственность решения.
Для соответствующей однородной СЛАУ проверить существование нетривиального решения. В случае, если оно существует,
найти размерность пространства решений и составить ФСР и общее решение однородной  и неоднородной СЛАУ.


\begin{align*}
 A = \left[\begin{matrix}-6 & 7 & 3 & 3 & 5\\-7 & -9 & 4 & -6 & 4\\-52 & -8 & 28 & -12 & 36\\4 & 64 & -4 & 36 & 4\\-8 & 5 & -1 & -3 & 7\end{matrix}\right],
\ b = \left[\begin{matrix}64\\-46\\72\\440\\22\end{matrix}\right]. 
 \end{align*}

Вариант N 67
Дана СЛАУ $AX = b$,
Проверить совместность по теореме Кронекера-Капелли. Если СЛАУ совместна, проверить единственность решения.
Для соответствующей однородной СЛАУ проверить существование нетривиального решения. В случае, если оно существует,
найти размерность пространства решений и составить ФСР и общее решение однородной  и неоднородной СЛАУ.


\begin{align*}
 A = \left[\begin{matrix}-2 & 0 & -5 & -5 & -3\\-1 & -4 & -5 & 5 & -7\\-13 & -20 & -45 & 5 & -47\\-3 & 20 & 5 & -45 & 23\\1 & 1 & -3 & 2 & -6\end{matrix}\right],
\ b = \left[\begin{matrix}-12\\69\\297\\-393\\22\end{matrix}\right]. 
 \end{align*}

Вариант N 68
Дана СЛАУ $AX = b$,
Проверить совместность по теореме Кронекера-Капелли. Если СЛАУ совместна, проверить единственность решения.
Для соответствующей однородной СЛАУ проверить существование нетривиального решения. В случае, если оно существует,
найти размерность пространства решений и составить ФСР и общее решение однородной  и неоднородной СЛАУ.


\begin{align*}
 A = \left[\begin{matrix}-3 & -5 & 8 & 8 & 5\\-9 & -25 & 4 & 54 & -5\\-54 & -140 & 44 & 294 & -10\\36 & 110 & 4 & -246 & 40\\117 & 345 & -12 & -762 & 105\end{matrix}\right],
\ b = \left[\begin{matrix}3\\109\\554\\-536\\-1617\end{matrix}\right]. 
 \end{align*}

Вариант N 69
Дана СЛАУ $AX = b$,
Проверить совместность по теореме Кронекера-Капелли. Если СЛАУ совместна, проверить единственность решения.
Для соответствующей однородной СЛАУ проверить существование нетривиального решения. В случае, если оно существует,
найти размерность пространства решений и составить ФСР и общее решение однородной  и неоднородной СЛАУ.


\begin{align*}
 A = \left[\begin{matrix}-8 & 3 & 8 & -5 & -8\\2 & 2 & -3 & 3 & -8\\-14 & 19 & 9 & 0 & -64\\-34 & -1 & 39 & -30 & 16\\8 & 0 & 3 & -4 & -8\end{matrix}\right],
\ b = \left[\begin{matrix}39\\54\\387\\-153\\99\end{matrix}\right]. 
 \end{align*}

Вариант N 70
Дана СЛАУ $AX = b$,
Проверить совместность по теореме Кронекера-Капелли. Если СЛАУ совместна, проверить единственность решения.
Для соответствующей однородной СЛАУ проверить существование нетривиального решения. В случае, если оно существует,
найти размерность пространства решений и составить ФСР и общее решение однородной  и неоднородной СЛАУ.


\begin{align*}
 A = \left[\begin{matrix}2 & -5 & 7 & 6 & 7\\-4 & -8 & 1 & -9 & -3\\-8 & -52 & 32 & -12 & 16\\24 & 12 & 24 & 60 & 40\\5 & -4 & 5 & -2 & 6\end{matrix}\right],
\ b = \left[\begin{matrix}100\\112\\848\\-48\\56\end{matrix}\right]. 
 \end{align*}

Вариант N 71
Дана СЛАУ $AX = b$,
Проверить совместность по теореме Кронекера-Капелли. Если СЛАУ совместна, проверить единственность решения.
Для соответствующей однородной СЛАУ проверить существование нетривиального решения. В случае, если оно существует,
найти размерность пространства решений и составить ФСР и общее решение однородной  и неоднородной СЛАУ.


\begin{align*}
 A = \left[\begin{matrix}4 & -6 & 1 & 8 & 2\\-6 & 1 & -4 & -4 & 5\\46 & -29 & 24 & 52 & -17\\0 & 1 & -1 & -8 & 5\\3 & -3 & 6 & -2 & -1\end{matrix}\right],
\ b = \left[\begin{matrix}-74\\82\\-706\\82\\15\end{matrix}\right]. 
 \end{align*}

Вариант N 72
Дана СЛАУ $AX = b$,
Проверить совместность по теореме Кронекера-Капелли. Если СЛАУ совместна, проверить единственность решения.
Для соответствующей однородной СЛАУ проверить существование нетривиального решения. В случае, если оно существует,
найти размерность пространства решений и составить ФСР и общее решение однородной  и неоднородной СЛАУ.


\begin{align*}
 A = \left[\begin{matrix}3 & 4 & -4 & -8 & 9\\-1 & 2 & -9 & 2 & -7\\8 & 24 & -52 & -24 & 8\\16 & 8 & 20 & -40 & 64\\-2 & -4 & -5 & 8 & 9\end{matrix}\right],
\ b = \left[\begin{matrix}25\\-106\\-324\\524\\-32\end{matrix}\right]. 
 \end{align*}

Вариант N 73
Дана СЛАУ $AX = b$,
Проверить совместность по теореме Кронекера-Капелли. Если СЛАУ совместна, проверить единственность решения.
Для соответствующей однородной СЛАУ проверить существование нетривиального решения. В случае, если оно существует,
найти размерность пространства решений и составить ФСР и общее решение однородной  и неоднородной СЛАУ.


\begin{align*}
 A = \left[\begin{matrix}6 & 4 & -4 & -5 & 2\\63 & 7 & 18 & 30 & 46\\333 & 47 & 78 & 135 & 236\\-297 & -23 & -102 & -165 & -224\\-909 & -81 & -294 & -480 & -678\end{matrix}\right],
\ b = \left[\begin{matrix}-22\\-196\\-1046\\914\\2808\end{matrix}\right]. 
 \end{align*}

Вариант N 74
Дана СЛАУ $AX = b$,
Проверить совместность по теореме Кронекера-Капелли. Если СЛАУ совместна, проверить единственность решения.
Для соответствующей однородной СЛАУ проверить существование нетривиального решения. В случае, если оно существует,
найти размерность пространства решений и составить ФСР и общее решение однородной  и неоднородной СЛАУ.


\begin{align*}
 A = \left[\begin{matrix}-8 & 3 & 3 & -7 & -5\\11 & 29 & 49 & 4 & -10\\31 & 154 & 254 & -1 & -65\\-79 & -136 & -236 & -41 & 35\\-213 & -417 & -717 & -102 & 120\end{matrix}\right],
\ b = \left[\begin{matrix}-37\\-346\\-1841\\1619\\4968\end{matrix}\right]. 
 \end{align*}

Вариант N 75
Дана СЛАУ $AX = b$,
Проверить совместность по теореме Кронекера-Капелли. Если СЛАУ совместна, проверить единственность решения.
Для соответствующей однородной СЛАУ проверить существование нетривиального решения. В случае, если оно существует,
найти размерность пространства решений и составить ФСР и общее решение однородной  и неоднородной СЛАУ.


\begin{align*}
 A = \left[\begin{matrix}7 & 0 & -5 & -4 & -9\\8 & 15 & -30 & -61 & -46\\68 & 75 & -170 & -321 & -266\\-12 & -75 & 130 & 289 & 194\\-64 & -225 & 410 & 883 & 618\end{matrix}\right],
\ b = \left[\begin{matrix}-153\\-947\\-5347\\4123\\12981\end{matrix}\right]. 
 \end{align*}

Вариант N 76
Дана СЛАУ $AX = b$,
Проверить совместность по теореме Кронекера-Капелли. Если СЛАУ совместна, проверить единственность решения.
Для соответствующей однородной СЛАУ проверить существование нетривиального решения. В случае, если оно существует,
найти размерность пространства решений и составить ФСР и общее решение однородной  и неоднородной СЛАУ.


\begin{align*}
 A = \left[\begin{matrix}7 & 1 & 8 & -2 & -7\\-8 & 2 & -5 & 3 & -4\\-12 & 14 & 7 & 7 & -48\\68 & -6 & 57 & -23 & -8\\-1 & -2 & 1 & 2 & -5\end{matrix}\right],
\ b = \left[\begin{matrix}-74\\117\\289\\-881\\17\end{matrix}\right]. 
 \end{align*}

Вариант N 77
Дана СЛАУ $AX = b$,
Проверить совместность по теореме Кронекера-Капелли. Если СЛАУ совместна, проверить единственность решения.
Для соответствующей однородной СЛАУ проверить существование нетривиального решения. В случае, если оно существует,
найти размерность пространства решений и составить ФСР и общее решение однородной  и неоднородной СЛАУ.


\begin{align*}
 A = \left[\begin{matrix}-4 & 3 & 6 & 2 & 3\\5 & 4 & 3 & 4 & 3\\9 & 32 & 39 & 28 & 27\\-41 & -8 & 9 & -12 & -3\\-8 & 3 & -3 & -4 & 7\end{matrix}\right],
\ b = \left[\begin{matrix}19\\3\\91\\61\\0\end{matrix}\right]. 
 \end{align*}

Вариант N 78
Дана СЛАУ $AX = b$,
Проверить совместность по теореме Кронекера-Капелли. Если СЛАУ совместна, проверить единственность решения.
Для соответствующей однородной СЛАУ проверить существование нетривиального решения. В случае, если оно существует,
найти размерность пространства решений и составить ФСР и общее решение однородной  и неоднородной СЛАУ.


\begin{align*}
 A = \left[\begin{matrix}-4 & -6 & 5 & 7 & -8\\6 & -3 & 2 & 9 & 2\\12 & -30 & 23 & 57 & -16\\-36 & -6 & 7 & -15 & -32\\9 & -4 & 8 & -5 & -2\end{matrix}\right],
\ b = \left[\begin{matrix}-39\\-50\\-317\\83\\-46\end{matrix}\right]. 
 \end{align*}

Вариант N 79
Дана СЛАУ $AX = b$,
Проверить совместность по теореме Кронекера-Капелли. Если СЛАУ совместна, проверить единственность решения.
Для соответствующей однородной СЛАУ проверить существование нетривиального решения. В случае, если оно существует,
найти размерность пространства решений и составить ФСР и общее решение однородной  и неоднородной СЛАУ.


\begin{align*}
 A = \left[\begin{matrix}-6 & -9 & -9 & -5 & -9\\8 & 2 & 3 & -7 & -7\\22 & -17 & -12 & -50 & -62\\-58 & -37 & -42 & 20 & 8\\-3 & 8 & 7 & -4 & -7\end{matrix}\right],
\ b = \left[\begin{matrix}74\\106\\752\\-308\\-12\end{matrix}\right]. 
 \end{align*}

Вариант N 80
Дана СЛАУ $AX = b$,
Проверить совместность по теореме Кронекера-Капелли. Если СЛАУ совместна, проверить единственность решения.
Для соответствующей однородной СЛАУ проверить существование нетривиального решения. В случае, если оно существует,
найти размерность пространства решений и составить ФСР и общее решение однородной  и неоднородной СЛАУ.


\begin{align*}
 A = \left[\begin{matrix}1 & -9 & -9 & 8 & 8\\-28 & -32 & -44 & 36 & 56\\-108 & -164 & -212 & 176 & 256\\116 & 92 & 140 & -112 & -192\\344 & 312 & 456 & -368 & -608\end{matrix}\right],
\ b = \left[\begin{matrix}68\\-96\\-112\\656\\1696\end{matrix}\right]. 
 \end{align*}

Вариант N 81
Дана СЛАУ $AX = b$,
Проверить совместность по теореме Кронекера-Капелли. Если СЛАУ совместна, проверить единственность решения.
Для соответствующей однородной СЛАУ проверить существование нетривиального решения. В случае, если оно существует,
найти размерность пространства решений и составить ФСР и общее решение однородной  и неоднородной СЛАУ.


\begin{align*}
 A = \left[\begin{matrix}-6 & 7 & 9 & -1 & 1\\0 & 2 & 3 & 2 & -8\\-18 & 13 & 15 & -11 & 35\\3 & -9 & -6 & 9 & -4\\1 & 0 & -9 & -7 & 9\end{matrix}\right],
\ b = \left[\begin{matrix}25\\35\\-65\\-12\\-53\end{matrix}\right]. 
 \end{align*}

Вариант N 82
Дана СЛАУ $AX = b$,
Проверить совместность по теореме Кронекера-Капелли. Если СЛАУ совместна, проверить единственность решения.
Для соответствующей однородной СЛАУ проверить существование нетривиального решения. В случае, если оно существует,
найти размерность пространства решений и составить ФСР и общее решение однородной  и неоднородной СЛАУ.


\begin{align*}
 A = \left[\begin{matrix}2 & -5 & -9 & 2 & 4\\-8 & -2 & -2 & -8 & 3\\40 & -12 & -28 & 40 & 4\\2 & 2 & 1 & -3 & 0\\-5 & -8 & 2 & -6 & 0\end{matrix}\right],
\ b = \left[\begin{matrix}-116\\53\\-676\\26\\40\end{matrix}\right]. 
 \end{align*}

Вариант N 83
Дана СЛАУ $AX = b$,
Проверить совместность по теореме Кронекера-Капелли. Если СЛАУ совместна, проверить единственность решения.
Для соответствующей однородной СЛАУ проверить существование нетривиального решения. В случае, если оно существует,
найти размерность пространства решений и составить ФСР и общее решение однородной  и неоднородной СЛАУ.


\begin{align*}
 A = \left[\begin{matrix}-6 & -1 & 4 & 1 & 9\\9 & 8 & 2 & 6 & 4\\12 & 28 & 24 & 28 & 52\\-60 & -36 & 8 & -20 & 20\\-8 & 6 & -3 & 6 & 8\end{matrix}\right],
\ b = \left[\begin{matrix}5\\41\\184\\-144\\-24\end{matrix}\right]. 
 \end{align*}

Вариант N 84
Дана СЛАУ $AX = b$,
Проверить совместность по теореме Кронекера-Капелли. Если СЛАУ совместна, проверить единственность решения.
Для соответствующей однородной СЛАУ проверить существование нетривиального решения. В случае, если оно существует,
найти размерность пространства решений и составить ФСР и общее решение однородной  и неоднородной СЛАУ.


\begin{align*}
 A = \left[\begin{matrix}3 & 6 & 2 & -7 & 4\\7 & 1 & 6 & 4 & 6\\-23 & 19 & -22 & -48 & -14\\-9 & 9 & 1 & 1 & -5\\9 & 1 & -9 & 2 & 2\end{matrix}\right],
\ b = \left[\begin{matrix}35\\62\\-170\\81\\-88\end{matrix}\right]. 
 \end{align*}

Вариант N 85
Дана СЛАУ $AX = b$,
Проверить совместность по теореме Кронекера-Капелли. Если СЛАУ совместна, проверить единственность решения.
Для соответствующей однородной СЛАУ проверить существование нетривиального решения. В случае, если оно существует,
найти размерность пространства решений и составить ФСР и общее решение однородной  и неоднородной СЛАУ.


\begin{align*}
 A = \left[\begin{matrix}6 & 3 & 7 & -6 & 7\\-9 & -3 & 3 & -2 & -7\\-21 & -3 & 43 & -34 & -7\\69 & 27 & 13 & -14 & 63\\-3 & -3 & -5 & -2 & 8\end{matrix}\right],
\ b = \left[\begin{matrix}-82\\32\\-168\\-488\\82\end{matrix}\right]. 
 \end{align*}

Вариант N 86
Дана СЛАУ $AX = b$,
Проверить совместность по теореме Кронекера-Капелли. Если СЛАУ совместна, проверить единственность решения.
Для соответствующей однородной СЛАУ проверить существование нетривиального решения. В случае, если оно существует,
найти размерность пространства решений и составить ФСР и общее решение однородной  и неоднородной СЛАУ.


\begin{align*}
 A = \left[\begin{matrix}7 & -2 & 6 & 2 & 9\\3 & 1 & 7 & -3 & -6\\40 & -4 & 52 & -4 & 12\\16 & -12 & -4 & 20 & 60\\-3 & 9 & 6 & 1 & 6\end{matrix}\right],
\ b = \left[\begin{matrix}47\\-42\\20\\356\\-63\end{matrix}\right]. 
 \end{align*}

Вариант N 87
Дана СЛАУ $AX = b$,
Проверить совместность по теореме Кронекера-Капелли. Если СЛАУ совместна, проверить единственность решения.
Для соответствующей однородной СЛАУ проверить существование нетривиального решения. В случае, если оно существует,
найти размерность пространства решений и составить ФСР и общее решение однородной  и неоднородной СЛАУ.


\begin{align*}
 A = \left[\begin{matrix}-7 & 2 & -7 & -5 & -1\\-9 & 2 & -7 & 7 & 4\\8 & 0 & 0 & -48 & -20\\-1 & -1 & 6 & 4 & -8\\9 & -6 & -7 & 1 & -3\end{matrix}\right],
\ b = \left[\begin{matrix}-16\\-21\\20\\-13\\-44\end{matrix}\right]. 
 \end{align*}

Вариант N 88
Дана СЛАУ $AX = b$,
Проверить совместность по теореме Кронекера-Капелли. Если СЛАУ совместна, проверить единственность решения.
Для соответствующей однородной СЛАУ проверить существование нетривиального решения. В случае, если оно существует,
найти размерность пространства решений и составить ФСР и общее решение однородной  и неоднородной СЛАУ.


\begin{align*}
 A = \left[\begin{matrix}2 & 8 & 5 & 9 & 1\\-6 & -5 & -8 & 3 & 0\\-24 & -1 & -25 & 42 & 3\\36 & 49 & 55 & 12 & 3\\-3 & 9 & -3 & -3 & -2\end{matrix}\right],
\ b = \left[\begin{matrix}-3\\32\\151\\-169\\75\end{matrix}\right]. 
 \end{align*}

Вариант N 89
Дана СЛАУ $AX = b$,
Проверить совместность по теореме Кронекера-Капелли. Если СЛАУ совместна, проверить единственность решения.
Для соответствующей однородной СЛАУ проверить существование нетривиального решения. В случае, если оно существует,
найти размерность пространства решений и составить ФСР и общее решение однородной  и неоднородной СЛАУ.


\begin{align*}
 A = \left[\begin{matrix}-1 & 3 & 9 & 8 & -1\\-1 & -4 & 7 & -9 & 7\\-7 & -7 & 55 & -12 & 25\\1 & 25 & -1 & 60 & -31\\-1 & 8 & 6 & -5 & 1\end{matrix}\right],
\ b = \left[\begin{matrix}-105\\-96\\-699\\69\\-83\end{matrix}\right]. 
 \end{align*}

Вариант N 90
Дана СЛАУ $AX = b$,
Проверить совместность по теореме Кронекера-Капелли. Если СЛАУ совместна, проверить единственность решения.
Для соответствующей однородной СЛАУ проверить существование нетривиального решения. В случае, если оно существует,
найти размерность пространства решений и составить ФСР и общее решение однородной  и неоднородной СЛАУ.


\begin{align*}
 A = \left[\begin{matrix}-1 & 5 & 2 & -1 & 2\\-8 & -5 & -2 & 8 & 3\\-43 & -10 & -4 & 37 & 21\\37 & 40 & 16 & -43 & -9\\7 & -5 & -4 & 8 & -6\end{matrix}\right],
\ b = \left[\begin{matrix}14\\-54\\-228\\312\\41\end{matrix}\right]. 
 \end{align*}

Вариант N 91
Дана СЛАУ $AX = b$,
Проверить совместность по теореме Кронекера-Капелли. Если СЛАУ совместна, проверить единственность решения.
Для соответствующей однородной СЛАУ проверить существование нетривиального решения. В случае, если оно существует,
найти размерность пространства решений и составить ФСР и общее решение однородной  и неоднородной СЛАУ.


\begin{align*}
 A = \left[\begin{matrix}-1 & 4 & 2 & -3 & -1\\1 & -1 & 1 & -3 & -4\\0 & 12 & 12 & -24 & -20\\-8 & 20 & 4 & 0 & 12\\-5 & 5 & 2 & 1 & 1\end{matrix}\right],
\ b = \left[\begin{matrix}-38\\-12\\-200\\-104\\-66\end{matrix}\right]. 
 \end{align*}

Вариант N 92
Дана СЛАУ $AX = b$,
Проверить совместность по теореме Кронекера-Капелли. Если СЛАУ совместна, проверить единственность решения.
Для соответствующей однородной СЛАУ проверить существование нетривиального решения. В случае, если оно существует,
найти размерность пространства решений и составить ФСР и общее решение однородной  и неоднородной СЛАУ.


\begin{align*}
 A = \left[\begin{matrix}-4 & -1 & 6 & -8 & -1\\12 & -40 & 52 & -8 & -16\\32 & -164 & 232 & -64 & -68\\-64 & 156 & -184 & 0 & 60\\-176 & 472 & -576 & 32 & 184\end{matrix}\right],
\ b = \left[\begin{matrix}-89\\-356\\-1780\\1068\\3560\end{matrix}\right]. 
 \end{align*}

Вариант N 93
Дана СЛАУ $AX = b$,
Проверить совместность по теореме Кронекера-Капелли. Если СЛАУ совместна, проверить единственность решения.
Для соответствующей однородной СЛАУ проверить существование нетривиального решения. В случае, если оно существует,
найти размерность пространства решений и составить ФСР и общее решение однородной  и неоднородной СЛАУ.


\begin{align*}
 A = \left[\begin{matrix}-8 & -4 & 2 & -4 & -5\\1 & -2 & -8 & -7 & 8\\-27 & -26 & -32 & -51 & 20\\-37 & -6 & 48 & 19 & -60\\2 & 3 & 4 & -6 & -1\end{matrix}\right],
\ b = \left[\begin{matrix}56\\65\\549\\-101\\-15\end{matrix}\right]. 
 \end{align*}

Вариант N 94
Дана СЛАУ $AX = b$,
Проверить совместность по теореме Кронекера-Капелли. Если СЛАУ совместна, проверить единственность решения.
Для соответствующей однородной СЛАУ проверить существование нетривиального решения. В случае, если оно существует,
найти размерность пространства решений и составить ФСР и общее решение однородной  и неоднородной СЛАУ.


\begin{align*}
 A = \left[\begin{matrix}7 & 1 & -4 & -7 & -9\\5 & 4 & 1 & 4 & -8\\41 & 19 & -8 & -5 & -59\\1 & -13 & -16 & -37 & 5\\0 & 4 & 2 & 8 & -1\end{matrix}\right],
\ b = \left[\begin{matrix}11\\80\\353\\-287\\63\end{matrix}\right]. 
 \end{align*}

Вариант N 95
Дана СЛАУ $AX = b$,
Проверить совместность по теореме Кронекера-Капелли. Если СЛАУ совместна, проверить единственность решения.
Для соответствующей однородной СЛАУ проверить существование нетривиального решения. В случае, если оно существует,
найти размерность пространства решений и составить ФСР и общее решение однородной  и неоднородной СЛАУ.


\begin{align*}
 A = \left[\begin{matrix}-8 & 5 & 7 & 4 & -6\\9 & 7 & 8 & 2 & -9\\-68 & -8 & -4 & 8 & 12\\2 & -7 & 6 & -8 & 6\\-3 & 7 & -8 & -5 & 6\end{matrix}\right],
\ b = \left[\begin{matrix}76\\54\\88\\-99\\-46\end{matrix}\right]. 
 \end{align*}

Вариант N 96
Дана СЛАУ $AX = b$,
Проверить совместность по теореме Кронекера-Капелли. Если СЛАУ совместна, проверить единственность решения.
Для соответствующей однородной СЛАУ проверить существование нетривиального решения. В случае, если оно существует,
найти размерность пространства решений и составить ФСР и общее решение однородной  и неоднородной СЛАУ.


\begin{align*}
 A = \left[\begin{matrix}-8 & -3 & 0 & 0 & -8\\-4 & -1 & 5 & 4 & 1\\-52 & -17 & 25 & 20 & -27\\-12 & -7 & -25 & -20 & -37\\-9 & 7 & 6 & 6 & -4\end{matrix}\right],
\ b = \left[\begin{matrix}-71\\-63\\-599\\31\\-160\end{matrix}\right]. 
 \end{align*}

Вариант N 97
Дана СЛАУ $AX = b$,
Проверить совместность по теореме Кронекера-Капелли. Если СЛАУ совместна, проверить единственность решения.
Для соответствующей однородной СЛАУ проверить существование нетривиального решения. В случае, если оно существует,
найти размерность пространства решений и составить ФСР и общее решение однородной  и неоднородной СЛАУ.


\begin{align*}
 A = \left[\begin{matrix}4 & 3 & 4 & 0 & 2\\-5 & 0 & -7 & 5 & 9\\32 & 9 & 40 & -20 & -30\\-2 & -2 & 2 & 3 & -2\\4 & 7 & 4 & 2 & 4\end{matrix}\right],
\ b = \left[\begin{matrix}-64\\-35\\-52\\32\\-100\end{matrix}\right]. 
 \end{align*}

Вариант N 98
Дана СЛАУ $AX = b$,
Проверить совместность по теореме Кронекера-Капелли. Если СЛАУ совместна, проверить единственность решения.
Для соответствующей однородной СЛАУ проверить существование нетривиального решения. В случае, если оно существует,
найти размерность пространства решений и составить ФСР и общее решение однородной  и неоднородной СЛАУ.


\begin{align*}
 A = \left[\begin{matrix}2 & -6 & -4 & -3 & 1\\31 & -8 & -47 & -29 & -27\\161 & -58 & -247 & -154 & -132\\-149 & 22 & 223 & 136 & 138\\-453 & 84 & 681 & 417 & 411\end{matrix}\right],
\ b = \left[\begin{matrix}35\\-205\\-920\\1130\\3285\end{matrix}\right]. 
 \end{align*}

Вариант N 99
Дана СЛАУ $AX = b$,
Проверить совместность по теореме Кронекера-Капелли. Если СЛАУ совместна, проверить единственность решения.
Для соответствующей однородной СЛАУ проверить существование нетривиального решения. В случае, если оно существует,
найти размерность пространства решений и составить ФСР и общее решение однородной  и неоднородной СЛАУ.


\begin{align*}
 A = \left[\begin{matrix}-5 & -2 & 6 & -4 & -9\\5 & 7 & 8 & 4 & 5\\10 & 29 & 58 & 8 & -2\\-40 & -41 & -22 & -32 & -52\\6 & -5 & 6 & -4 & 2\end{matrix}\right],
\ b = \left[\begin{matrix}-45\\28\\5\\-275\\59\end{matrix}\right]. 
 \end{align*}

Вариант N 100
Дана СЛАУ $AX = b$,
Проверить совместность по теореме Кронекера-Капелли. Если СЛАУ совместна, проверить единственность решения.
Для соответствующей однородной СЛАУ проверить существование нетривиального решения. В случае, если оно существует,
найти размерность пространства решений и составить ФСР и общее решение однородной  и неоднородной СЛАУ.


\begin{align*}
 A = \left[\begin{matrix}-1 & 2 & -8 & 9 & 1\\-4 & -7 & 1 & -7 & -6\\13 & 34 & -28 & 55 & 27\\-5 & -3 & 0 & 5 & 4\\3 & -2 & 0 & 5 & 8\end{matrix}\right],
\ b = \left[\begin{matrix}-35\\-13\\-53\\-33\\20\end{matrix}\right]. 
 \end{align*}

Вариант N 101
Дана СЛАУ $AX = b$,
Проверить совместность по теореме Кронекера-Капелли. Если СЛАУ совместна, проверить единственность решения.
Для соответствующей однородной СЛАУ проверить существование нетривиального решения. В случае, если оно существует,
найти размерность пространства решений и составить ФСР и общее решение однородной  и неоднородной СЛАУ.


\begin{align*}
 A = \left[\begin{matrix}2 & 8 & -9 & 1 & 0\\1 & 0 & -8 & -8 & -5\\10 & 24 & -59 & -29 & -20\\2 & 24 & 5 & 35 & 20\\3 & 8 & -9 & -4 & -3\end{matrix}\right],
\ b = \left[\begin{matrix}110\\108\\762\\-102\\143\end{matrix}\right]. 
 \end{align*}

Вариант N 102
Дана СЛАУ $AX = b$,
Проверить совместность по теореме Кронекера-Капелли. Если СЛАУ совместна, проверить единственность решения.
Для соответствующей однородной СЛАУ проверить существование нетривиального решения. В случае, если оно существует,
найти размерность пространства решений и составить ФСР и общее решение однородной  и неоднородной СЛАУ.


\begin{align*}
 A = \left[\begin{matrix}3 & 3 & 9 & -8 & 1\\8 & -8 & 2 & 6 & -3\\-23 & 41 & 19 & -48 & 15\\8 & 1 & 7 & -5 & 8\\0 & -8 & 5 & -2 & 0\end{matrix}\right],
\ b = \left[\begin{matrix}-22\\36\\-210\\2\\60\end{matrix}\right]. 
 \end{align*}

Вариант N 103
Дана СЛАУ $AX = b$,
Проверить совместность по теореме Кронекера-Капелли. Если СЛАУ совместна, проверить единственность решения.
Для соответствующей однородной СЛАУ проверить существование нетривиального решения. В случае, если оно существует,
найти размерность пространства решений и составить ФСР и общее решение однородной  и неоднородной СЛАУ.


\begin{align*}
 A = \left[\begin{matrix}8 & 1 & -7 & -1 & -1\\0 & -5 & 6 & -6 & 4\\24 & -17 & 3 & -27 & 13\\24 & 23 & -45 & 21 & -19\\-1 & -6 & 6 & -3 & 0\end{matrix}\right],
\ b = \left[\begin{matrix}56\\16\\232\\104\\-18\end{matrix}\right]. 
 \end{align*}

Вариант N 104
Дана СЛАУ $AX = b$,
Проверить совместность по теореме Кронекера-Капелли. Если СЛАУ совместна, проверить единственность решения.
Для соответствующей однородной СЛАУ проверить существование нетривиального решения. В случае, если оно существует,
найти размерность пространства решений и составить ФСР и общее решение однородной  и неоднородной СЛАУ.


\begin{align*}
 A = \left[\begin{matrix}3 & -1 & 9 & -7 & -1\\39 & 27 & 27 & 19 & 2\\204 & 132 & 162 & 74 & 7\\-186 & -138 & -108 & -116 & -13\\-567 & -411 & -351 & -327 & -36\end{matrix}\right],
\ b = \left[\begin{matrix}-49\\-47\\-382\\88\\411\end{matrix}\right]. 
 \end{align*}

Вариант N 105
Дана СЛАУ $AX = b$,
Проверить совместность по теореме Кронекера-Капелли. Если СЛАУ совместна, проверить единственность решения.
Для соответствующей однородной СЛАУ проверить существование нетривиального решения. В случае, если оно существует,
найти размерность пространства решений и составить ФСР и общее решение однородной  и неоднородной СЛАУ.


\begin{align*}
 A = \left[\begin{matrix}-9 & 1 & -9 & -1 & 0\\2 & -9 & 0 & -1 & -2\\-19 & -33 & -27 & -7 & -8\\-35 & 39 & -27 & 1 & 8\\3 & 3 & 7 & 9 & -8\end{matrix}\right],
\ b = \left[\begin{matrix}50\\-21\\66\\234\\36\end{matrix}\right]. 
 \end{align*}

Вариант N 106
Дана СЛАУ $AX = b$,
Проверить совместность по теореме Кронекера-Капелли. Если СЛАУ совместна, проверить единственность решения.
Для соответствующей однородной СЛАУ проверить существование нетривиального решения. В случае, если оно существует,
найти размерность пространства решений и составить ФСР и общее решение однородной  и неоднородной СЛАУ.


\begin{align*}
 A = \left[\begin{matrix}9 & 4 & 1 & 2 & -5\\-1 & -2 & 2 & -4 & 5\\23 & 4 & 11 & -10 & 5\\31 & 20 & -5 & 22 & -35\\-3 & 1 & -4 & 4 & 6\end{matrix}\right],
\ b = \left[\begin{matrix}-31\\-50\\-293\\107\\6\end{matrix}\right]. 
 \end{align*}

Вариант N 107
Дана СЛАУ $AX = b$,
Проверить совместность по теореме Кронекера-Капелли. Если СЛАУ совместна, проверить единственность решения.
Для соответствующей однородной СЛАУ проверить существование нетривиального решения. В случае, если оно существует,
найти размерность пространства решений и составить ФСР и общее решение однородной  и неоднородной СЛАУ.


\begin{align*}
 A = \left[\begin{matrix}8 & 0 & 3 & 0 & 9\\-5 & 5 & 9 & 4 & 1\\7 & 25 & 57 & 20 & 41\\57 & -25 & -33 & -20 & 31\\8 & -8 & 0 & 8 & 9\end{matrix}\right],
\ b = \left[\begin{matrix}-120\\23\\-365\\-595\\-183\end{matrix}\right]. 
 \end{align*}

Вариант N 108
Дана СЛАУ $AX = b$,
Проверить совместность по теореме Кронекера-Капелли. Если СЛАУ совместна, проверить единственность решения.
Для соответствующей однородной СЛАУ проверить существование нетривиального решения. В случае, если оно существует,
найти размерность пространства решений и составить ФСР и общее решение однородной  и неоднородной СЛАУ.


\begin{align*}
 A = \left[\begin{matrix}8 & -7 & 3 & 0 & -9\\-2 & -2 & 5 & -2 & 4\\22 & -38 & 37 & -10 & -16\\42 & -18 & -13 & 10 & -56\\2 & 1 & -1 & -1 & -1\end{matrix}\right],
\ b = \left[\begin{matrix}33\\-58\\-158\\422\\20\end{matrix}\right]. 
 \end{align*}

Вариант N 109
Дана СЛАУ $AX = b$,
Проверить совместность по теореме Кронекера-Капелли. Если СЛАУ совместна, проверить единственность решения.
Для соответствующей однородной СЛАУ проверить существование нетривиального решения. В случае, если оно существует,
найти размерность пространства решений и составить ФСР и общее решение однородной  и неоднородной СЛАУ.


\begin{align*}
 A = \left[\begin{matrix}6 & -5 & -7 & 9 & 2\\50 & 1 & -29 & 15 & 22\\218 & -11 & -137 & 87 & 94\\-182 & -19 & 95 & -33 & -82\\-564 & -42 & 306 & -126 & -252\end{matrix}\right],
\ b = \left[\begin{matrix}-76\\-216\\-1092\\636\\2136\end{matrix}\right]. 
 \end{align*}

Вариант N 110
Дана СЛАУ $AX = b$,
Проверить совместность по теореме Кронекера-Капелли. Если СЛАУ совместна, проверить единственность решения.
Для соответствующей однородной СЛАУ проверить существование нетривиального решения. В случае, если оно существует,
найти размерность пространства решений и составить ФСР и общее решение однородной  и неоднородной СЛАУ.


\begin{align*}
 A = \left[\begin{matrix}-4 & 9 & 1 & 9 & -2\\8 & 1 & 9 & -4 & -4\\-48 & 32 & -32 & 52 & 8\\3 & 3 & -4 & 5 & 8\\-9 & 2 & 3 & -7 & 7\end{matrix}\right],
\ b = \left[\begin{matrix}26\\164\\-552\\-21\\-26\end{matrix}\right]. 
 \end{align*}

Вариант N 111
Дана СЛАУ $AX = b$,
Проверить совместность по теореме Кронекера-Капелли. Если СЛАУ совместна, проверить единственность решения.
Для соответствующей однородной СЛАУ проверить существование нетривиального решения. В случае, если оно существует,
найти размерность пространства решений и составить ФСР и общее решение однородной  и неоднородной СЛАУ.


\begin{align*}
 A = \left[\begin{matrix}5 & -9 & 8 & -2 & -2\\5 & 6 & 2 & 6 & 4\\35 & -3 & 32 & 18 & 10\\-5 & -51 & 16 & -30 & -22\\-2 & -1 & -5 & -3 & -8\end{matrix}\right],
\ b = \left[\begin{matrix}28\\13\\136\\32\\-35\end{matrix}\right]. 
 \end{align*}

Вариант N 112
Дана СЛАУ $AX = b$,
Проверить совместность по теореме Кронекера-Капелли. Если СЛАУ совместна, проверить единственность решения.
Для соответствующей однородной СЛАУ проверить существование нетривиального решения. В случае, если оно существует,
найти размерность пространства решений и составить ФСР и общее решение однородной  и неоднородной СЛАУ.


\begin{align*}
 A = \left[\begin{matrix}-5 & 6 & -6 & -7 & 1\\-45 & 58 & 12 & -26 & 23\\-240 & 308 & 42 & -151 & 118\\210 & -272 & -78 & 109 & -112\\645 & -834 & -216 & 348 & -339\end{matrix}\right],
\ b = \left[\begin{matrix}-35\\-225\\-1230\\1020\\3165\end{matrix}\right]. 
 \end{align*}

Вариант N 113
Дана СЛАУ $AX = b$,
Проверить совместность по теореме Кронекера-Капелли. Если СЛАУ совместна, проверить единственность решения.
Для соответствующей однородной СЛАУ проверить существование нетривиального решения. В случае, если оно существует,
найти размерность пространства решений и составить ФСР и общее решение однородной  и неоднородной СЛАУ.


\begin{align*}
 A = \left[\begin{matrix}-5 & -7 & -7 & 9 & 0\\-7 & 0 & 8 & -3 & 0\\8 & -28 & -60 & 48 & 0\\2 & -4 & -6 & -7 & 5\\-5 & -2 & 1 & -9 & 9\end{matrix}\right],
\ b = \left[\begin{matrix}82\\-79\\644\\-62\\-149\end{matrix}\right]. 
 \end{align*}

Вариант N 114
Дана СЛАУ $AX = b$,
Проверить совместность по теореме Кронекера-Капелли. Если СЛАУ совместна, проверить единственность решения.
Для соответствующей однородной СЛАУ проверить существование нетривиального решения. В случае, если оно существует,
найти размерность пространства решений и составить ФСР и общее решение однородной  и неоднородной СЛАУ.


\begin{align*}
 A = \left[\begin{matrix}1 & -9 & -4 & -3 & 3\\9 & -3 & -9 & -2 & 9\\-33 & -15 & 24 & -1 & -27\\7 & 0 & -9 & 3 & -5\\6 & 3 & -1 & -3 & 7\end{matrix}\right],
\ b = \left[\begin{matrix}-39\\51\\-321\\46\\51\end{matrix}\right]. 
 \end{align*}

Вариант N 115
Дана СЛАУ $AX = b$,
Проверить совместность по теореме Кронекера-Капелли. Если СЛАУ совместна, проверить единственность решения.
Для соответствующей однородной СЛАУ проверить существование нетривиального решения. В случае, если оно существует,
найти размерность пространства решений и составить ФСР и общее решение однородной  и неоднородной СЛАУ.


\begin{align*}
 A = \left[\begin{matrix}7 & -5 & 7 & 4 & -6\\-2 & -9 & -5 & -6 & -5\\36 & 16 & 48 & 40 & -4\\6 & -6 & -1 & -6 & 9\\8 & -8 & 8 & -7 & 5\end{matrix}\right],
\ b = \left[\begin{matrix}94\\104\\-40\\-5\\70\end{matrix}\right]. 
 \end{align*}

Вариант N 116
Дана СЛАУ $AX = b$,
Проверить совместность по теореме Кронекера-Капелли. Если СЛАУ совместна, проверить единственность решения.
Для соответствующей однородной СЛАУ проверить существование нетривиального решения. В случае, если оно существует,
найти размерность пространства решений и составить ФСР и общее решение однородной  и неоднородной СЛАУ.


\begin{align*}
 A = \left[\begin{matrix}2 & 7 & -6 & 2 & 9\\-9 & 4 & -5 & -6 & 7\\42 & 5 & 2 & 30 & -1\\9 & 9 & 6 & 0 & -3\\2 & 0 & 7 & -8 & -6\end{matrix}\right],
\ b = \left[\begin{matrix}-75\\84\\-561\\-48\\91\end{matrix}\right]. 
 \end{align*}

Вариант N 117
Дана СЛАУ $AX = b$,
Проверить совместность по теореме Кронекера-Капелли. Если СЛАУ совместна, проверить единственность решения.
Для соответствующей однородной СЛАУ проверить существование нетривиального решения. В случае, если оно существует,
найти размерность пространства решений и составить ФСР и общее решение однородной  и неоднородной СЛАУ.


\begin{align*}
 A = \left[\begin{matrix}-2 & -7 & -5 & 3 & 8\\2 & -7 & 9 & -1 & 1\\-18 & 7 & -65 & 17 & 27\\4 & 3 & -9 & -2 & 4\\-6 & 1 & -5 & -1 & -5\end{matrix}\right],
\ b = \left[\begin{matrix}-95\\-41\\-175\\-29\\23\end{matrix}\right]. 
 \end{align*}

Вариант N 118
Дана СЛАУ $AX = b$,
Проверить совместность по теореме Кронекера-Капелли. Если СЛАУ совместна, проверить единственность решения.
Для соответствующей однородной СЛАУ проверить существование нетривиального решения. В случае, если оно существует,
найти размерность пространства решений и составить ФСР и общее решение однородной  и неоднородной СЛАУ.


\begin{align*}
 A = \left[\begin{matrix}4 & -4 & -6 & 7 & 5\\-7 & 4 & 8 & 5 & 6\\44 & -32 & -56 & 8 & -4\\-6 & -4 & 1 & -4 & -3\\-3 & -9 & -3 & -6 & 0\end{matrix}\right],
\ b = \left[\begin{matrix}1\\4\\-12\\-69\\-117\end{matrix}\right]. 
 \end{align*}

Вариант N 119
Дана СЛАУ $AX = b$,
Проверить совместность по теореме Кронекера-Капелли. Если СЛАУ совместна, проверить единственность решения.
Для соответствующей однородной СЛАУ проверить существование нетривиального решения. В случае, если оно существует,
найти размерность пространства решений и составить ФСР и общее решение однородной  и неоднородной СЛАУ.


\begin{align*}
 A = \left[\begin{matrix}-6 & 2 & -3 & -6 & -1\\6 & 7 & -6 & 0 & 4\\6 & 34 & -33 & -18 & 13\\-42 & -22 & 15 & -18 & -19\\-4 & 9 & -6 & -7 & -9\end{matrix}\right],
\ b = \left[\begin{matrix}-77\\125\\269\\-731\\-17\end{matrix}\right]. 
 \end{align*}

Вариант N 120
Дана СЛАУ $AX = b$,
Проверить совместность по теореме Кронекера-Капелли. Если СЛАУ совместна, проверить единственность решения.
Для соответствующей однородной СЛАУ проверить существование нетривиального решения. В случае, если оно существует,
найти размерность пространства решений и составить ФСР и общее решение однородной  и неоднородной СЛАУ.


\begin{align*}
 A = \left[\begin{matrix}-3 & 2 & -7 & 1 & 8\\21 & 51 & -21 & -37 & -21\\96 & 261 & -126 & -182 & -81\\-114 & -249 & 84 & 188 & 129\\-333 & -753 & 273 & 561 & 363\end{matrix}\right],
\ b = \left[\begin{matrix}-30\\-285\\-1515\\1335\\4095\end{matrix}\right]. 
 \end{align*}

Вариант N 121
Дана СЛАУ $AX = b$,
Проверить совместность по теореме Кронекера-Капелли. Если СЛАУ совместна, проверить единственность решения.
Для соответствующей однородной СЛАУ проверить существование нетривиального решения. В случае, если оно существует,
найти размерность пространства решений и составить ФСР и общее решение однородной  и неоднородной СЛАУ.


\begin{align*}
 A = \left[\begin{matrix}-2 & -1 & -5 & -3 & 4\\6 & -8 & 5 & 6 & -6\\-36 & 37 & -40 & -39 & 42\\9 & 2 & -2 & 8 & 1\\8 & -5 & 5 & 2 & 1\end{matrix}\right],
\ b = \left[\begin{matrix}56\\-80\\568\\-34\\-16\end{matrix}\right]. 
 \end{align*}

Вариант N 122
Дана СЛАУ $AX = b$,
Проверить совместность по теореме Кронекера-Капелли. Если СЛАУ совместна, проверить единственность решения.
Для соответствующей однородной СЛАУ проверить существование нетривиального решения. В случае, если оно существует,
найти размерность пространства решений и составить ФСР и общее решение однородной  и неоднородной СЛАУ.


\begin{align*}
 A = \left[\begin{matrix}-3 & 7 & -5 & -6 & -2\\-4 & 4 & -16 & -48 & -28\\-28 & 44 & -84 & -216 & -120\\4 & 12 & 44 & 168 & 104\\24 & 8 & 152 & 528 & 320\end{matrix}\right],
\ b = \left[\begin{matrix}55\\252\\1228\\-788\\-2584\end{matrix}\right]. 
 \end{align*}

Вариант N 123
Дана СЛАУ $AX = b$,
Проверить совместность по теореме Кронекера-Капелли. Если СЛАУ совместна, проверить единственность решения.
Для соответствующей однородной СЛАУ проверить существование нетривиального решения. В случае, если оно существует,
найти размерность пространства решений и составить ФСР и общее решение однородной  и неоднородной СЛАУ.


\begin{align*}
 A = \left[\begin{matrix}-4 & 9 & -3 & -1 & 2\\-5 & -8 & 6 & 1 & -5\\-41 & -4 & 18 & 1 & -17\\9 & 76 & -42 & -9 & 33\\-5 & -8 & 7 & 2 & 5\end{matrix}\right],
\ b = \left[\begin{matrix}125\\-87\\65\\935\\-30\end{matrix}\right]. 
 \end{align*}

Вариант N 124
Дана СЛАУ $AX = b$,
Проверить совместность по теореме Кронекера-Капелли. Если СЛАУ совместна, проверить единственность решения.
Для соответствующей однородной СЛАУ проверить существование нетривиального решения. В случае, если оно существует,
найти размерность пространства решений и составить ФСР и общее решение однородной  и неоднородной СЛАУ.


\begin{align*}
 A = \left[\begin{matrix}-8 & 2 & -4 & 9 & -6\\-2 & -6 & -2 & 7 & -5\\-34 & -24 & -22 & 62 & -43\\-14 & 36 & -2 & -8 & 7\\-8 & -8 & 8 & 9 & -9\end{matrix}\right],
\ b = \left[\begin{matrix}16\\28\\188\\-92\\-70\end{matrix}\right]. 
 \end{align*}

Вариант N 125
Дана СЛАУ $AX = b$,
Проверить совместность по теореме Кронекера-Капелли. Если СЛАУ совместна, проверить единственность решения.
Для соответствующей однородной СЛАУ проверить существование нетривиального решения. В случае, если оно существует,
найти размерность пространства решений и составить ФСР и общее решение однородной  и неоднородной СЛАУ.


\begin{align*}
 A = \left[\begin{matrix}0 & -5 & 9 & 1 & 5\\-3 & -7 & 9 & -2 & -3\\-15 & -50 & 72 & -7 & 0\\15 & 20 & -18 & 13 & 30\\-9 & -5 & 7 & -6 & 5\end{matrix}\right],
\ b = \left[\begin{matrix}65\\18\\285\\105\\64\end{matrix}\right]. 
 \end{align*}

Вариант N 126
Дана СЛАУ $AX = b$,
Проверить совместность по теореме Кронекера-Капелли. Если СЛАУ совместна, проверить единственность решения.
Для соответствующей однородной СЛАУ проверить существование нетривиального решения. В случае, если оно существует,
найти размерность пространства решений и составить ФСР и общее решение однородной  и неоднородной СЛАУ.


\begin{align*}
 A = \left[\begin{matrix}-2 & -7 & -2 & -7 & -3\\-5 & -4 & 6 & -6 & -4\\17 & -8 & -38 & 2 & 8\\7 & -1 & -8 & -3 & 8\\6 & -1 & 6 & 3 & 6\end{matrix}\right],
\ b = \left[\begin{matrix}-85\\-115\\235\\86\\119\end{matrix}\right]. 
 \end{align*}

Вариант N 127
Дана СЛАУ $AX = b$,
Проверить совместность по теореме Кронекера-Капелли. Если СЛАУ совместна, проверить единственность решения.
Для соответствующей однородной СЛАУ проверить существование нетривиального решения. В случае, если оно существует,
найти размерность пространства решений и составить ФСР и общее решение однородной  и неоднородной СЛАУ.


\begin{align*}
 A = \left[\begin{matrix}-4 & 6 & -3 & -1 & -9\\-36 & 34 & -2 & -44 & 9\\-196 & 194 & -22 & -224 & 9\\164 & -146 & -2 & 216 & -81\\508 & -462 & 6 & 652 & -207\end{matrix}\right],
\ b = \left[\begin{matrix}-6\\56\\256\\-304\\-888\end{matrix}\right]. 
 \end{align*}

Вариант N 128
Дана СЛАУ $AX = b$,
Проверить совместность по теореме Кронекера-Капелли. Если СЛАУ совместна, проверить единственность решения.
Для соответствующей однородной СЛАУ проверить существование нетривиального решения. В случае, если оно существует,
найти размерность пространства решений и составить ФСР и общее решение однородной  и неоднородной СЛАУ.


\begin{align*}
 A = \left[\begin{matrix}-1 & -8 & 6 & -9 & 2\\-28 & -48 & 28 & -8 & -8\\-116 & -224 & 136 & -68 & -24\\108 & 160 & -88 & -4 & 40\\328 & 512 & -288 & 24 & 112\end{matrix}\right],
\ b = \left[\begin{matrix}-150\\-356\\-2024\\824\\3072\end{matrix}\right]. 
 \end{align*}

Вариант N 129
Дана СЛАУ $AX = b$,
Проверить совместность по теореме Кронекера-Капелли. Если СЛАУ совместна, проверить единственность решения.
Для соответствующей однородной СЛАУ проверить существование нетривиального решения. В случае, если оно существует,
найти размерность пространства решений и составить ФСР и общее решение однородной  и неоднородной СЛАУ.


\begin{align*}
 A = \left[\begin{matrix}-5 & 5 & -9 & -3 & 6\\30 & -20 & -57 & -14 & 33\\135 & -85 & -312 & -79 & 183\\-165 & 115 & 258 & 61 & -147\\-480 & 330 & 801 & 192 & -459\end{matrix}\right],
\ b = \left[\begin{matrix}-7\\-681\\-3426\\3384\\10173\end{matrix}\right]. 
 \end{align*}

Вариант N 130
Дана СЛАУ $AX = b$,
Проверить совместность по теореме Кронекера-Капелли. Если СЛАУ совместна, проверить единственность решения.
Для соответствующей однородной СЛАУ проверить существование нетривиального решения. В случае, если оно существует,
найти размерность пространства решений и составить ФСР и общее решение однородной  и неоднородной СЛАУ.


\begin{align*}
 A = \left[\begin{matrix}4 & -6 & -6 & 5 & 7\\-7 & -4 & -6 & -6 & -6\\-16 & -34 & -42 & -9 & -3\\40 & -2 & 6 & 39 & 45\\-3 & 4 & -8 & -5 & -2\end{matrix}\right],
\ b = \left[\begin{matrix}-65\\-61\\-439\\49\\-11\end{matrix}\right]. 
 \end{align*}

Вариант N 131
Дана СЛАУ $AX = b$,
Проверить совместность по теореме Кронекера-Капелли. Если СЛАУ совместна, проверить единственность решения.
Для соответствующей однородной СЛАУ проверить существование нетривиального решения. В случае, если оно существует,
найти размерность пространства решений и составить ФСР и общее решение однородной  и неоднородной СЛАУ.


\begin{align*}
 A = \left[\begin{matrix}6 & -2 & -8 & 6 & 4\\63 & -46 & -29 & -22 & 22\\333 & -236 & -169 & -92 & 122\\-297 & 224 & 121 & 128 & -98\\-909 & 678 & 387 & 366 & -306\end{matrix}\right],
\ b = \left[\begin{matrix}-60\\-545\\-2905\\2545\\7815\end{matrix}\right]. 
 \end{align*}

Вариант N 132
Дана СЛАУ $AX = b$,
Проверить совместность по теореме Кронекера-Капелли. Если СЛАУ совместна, проверить единственность решения.
Для соответствующей однородной СЛАУ проверить существование нетривиального решения. В случае, если оно существует,
найти размерность пространства решений и составить ФСР и общее решение однородной  и неоднородной СЛАУ.


\begin{align*}
 A = \left[\begin{matrix}-1 & 0 & -2 & -3 & 5\\2 & 7 & 9 & 8 & 5\\6 & 35 & 37 & 28 & 45\\-14 & -35 & -53 & -52 & -5\\4 & 8 & -2 & -8 & 2\end{matrix}\right],
\ b = \left[\begin{matrix}62\\-69\\-97\\593\\8\end{matrix}\right]. 
 \end{align*}

Вариант N 133
Дана СЛАУ $AX = b$,
Проверить совместность по теореме Кронекера-Капелли. Если СЛАУ совместна, проверить единственность решения.
Для соответствующей однородной СЛАУ проверить существование нетривиального решения. В случае, если оно существует,
найти размерность пространства решений и составить ФСР и общее решение однородной  и неоднородной СЛАУ.


\begin{align*}
 A = \left[\begin{matrix}2 & -4 & -5 & -7 & -7\\-4 & -32 & 12 & -32 & 8\\-8 & -144 & 28 & -156 & 4\\24 & 112 & -68 & 100 & -60\\64 & 352 & -184 & 328 & -152\end{matrix}\right],
\ b = \left[\begin{matrix}-116\\-512\\-2512\\1584\\5216\end{matrix}\right]. 
 \end{align*}

Вариант N 134
Дана СЛАУ $AX = b$,
Проверить совместность по теореме Кронекера-Капелли. Если СЛАУ совместна, проверить единственность решения.
Для соответствующей однородной СЛАУ проверить существование нетривиального решения. В случае, если оно существует,
найти размерность пространства решений и составить ФСР и общее решение однородной  и неоднородной СЛАУ.


\begin{align*}
 A = \left[\begin{matrix}4 & 7 & -4 & 1 & 2\\-4 & 5 & 0 & 0 & 6\\36 & 3 & -16 & 4 & -22\\7 & 8 & 0 & 8 & 0\\-8 & 8 & 6 & 7 & -8\end{matrix}\right],
\ b = \left[\begin{matrix}32\\-40\\328\\79\\-74\end{matrix}\right]. 
 \end{align*}

Вариант N 135
Дана СЛАУ $AX = b$,
Проверить совместность по теореме Кронекера-Капелли. Если СЛАУ совместна, проверить единственность решения.
Для соответствующей однородной СЛАУ проверить существование нетривиального решения. В случае, если оно существует,
найти размерность пространства решений и составить ФСР и общее решение однородной  и неоднородной СЛАУ.


\begin{align*}
 A = \left[\begin{matrix}-5 & 9 & 7 & 4 & 4\\-9 & 9 & 6 & -6 & -6\\-51 & 63 & 45 & -12 & -12\\21 & -9 & -3 & 36 & 36\\-1 & 3 & 7 & 4 & -3\end{matrix}\right],
\ b = \left[\begin{matrix}63\\-27\\81\\297\\13\end{matrix}\right]. 
 \end{align*}

Вариант N 136
Дана СЛАУ $AX = b$,
Проверить совместность по теореме Кронекера-Капелли. Если СЛАУ совместна, проверить единственность решения.
Для соответствующей однородной СЛАУ проверить существование нетривиального решения. В случае, если оно существует,
найти размерность пространства решений и составить ФСР и общее решение однородной  и неоднородной СЛАУ.


\begin{align*}
 A = \left[\begin{matrix}-7 & -5 & 2 & -9 & -7\\9 & 6 & -1 & -1 & -3\\17 & 10 & 3 & -41 & -43\\-73 & -50 & 13 & -31 & -13\\-8 & -1 & -5 & -4 & -8\end{matrix}\right],
\ b = \left[\begin{matrix}-26\\49\\141\\-349\\-79\end{matrix}\right]. 
 \end{align*}

Вариант N 137
Дана СЛАУ $AX = b$,
Проверить совместность по теореме Кронекера-Капелли. Если СЛАУ совместна, проверить единственность решения.
Для соответствующей однородной СЛАУ проверить существование нетривиального решения. В случае, если оно существует,
найти размерность пространства решений и составить ФСР и общее решение однородной  и неоднородной СЛАУ.


\begin{align*}
 A = \left[\begin{matrix}2 & -8 & -8 & -3 & 0\\4 & 5 & -9 & -7 & -6\\28 & -7 & -77 & -47 & -30\\-12 & -57 & 13 & 23 & 30\\7 & -2 & 7 & -4 & 0\end{matrix}\right],
\ b = \left[\begin{matrix}19\\-90\\-374\\526\\31\end{matrix}\right]. 
 \end{align*}

Вариант N 138
Дана СЛАУ $AX = b$,
Проверить совместность по теореме Кронекера-Капелли. Если СЛАУ совместна, проверить единственность решения.
Для соответствующей однородной СЛАУ проверить существование нетривиального решения. В случае, если оно существует,
найти размерность пространства решений и составить ФСР и общее решение однородной  и неоднородной СЛАУ.


\begin{align*}
 A = \left[\begin{matrix}-3 & -8 & 5 & -6 & 7\\-7 & -6 & 4 & -8 & 1\\-47 & -62 & 40 & -64 & 33\\23 & -2 & 0 & 16 & 23\\1 & -4 & 1 & -4 & -7\end{matrix}\right],
\ b = \left[\begin{matrix}11\\98\\534\\-446\\55\end{matrix}\right]. 
 \end{align*}

Вариант N 139
Дана СЛАУ $AX = b$,
Проверить совместность по теореме Кронекера-Капелли. Если СЛАУ совместна, проверить единственность решения.
Для соответствующей однородной СЛАУ проверить существование нетривиального решения. В случае, если оно существует,
найти размерность пространства решений и составить ФСР и общее решение однородной  и неоднородной СЛАУ.


\begin{align*}
 A = \left[\begin{matrix}-5 & 1 & -5 & 6 & -1\\4 & -1 & -7 & -7 & 4\\-36 & 8 & 8 & 52 & -20\\-3 & 5 & -2 & 0 & 7\\7 & -1 & -6 & 8 & 2\end{matrix}\right],
\ b = \left[\begin{matrix}-57\\103\\-640\\5\\2\end{matrix}\right]. 
 \end{align*}

Вариант N 140
Дана СЛАУ $AX = b$,
Проверить совместность по теореме Кронекера-Капелли. Если СЛАУ совместна, проверить единственность решения.
Для соответствующей однородной СЛАУ проверить существование нетривиального решения. В случае, если оно существует,
найти размерность пространства решений и составить ФСР и общее решение однородной  и неоднородной СЛАУ.


\begin{align*}
 A = \left[\begin{matrix}6 & -1 & 7 & 8 & 3\\-4 & -16 & 4 & 28 & 20\\8 & -68 & 44 & 144 & 92\\40 & 60 & 12 & -80 & -68\\96 & 184 & 8 & -272 & -216\end{matrix}\right],
\ b = \left[\begin{matrix}-19\\208\\756\\-908\\-2648\end{matrix}\right]. 
 \end{align*}

Вариант N 141
Дана СЛАУ $AX = b$,
Проверить совместность по теореме Кронекера-Капелли. Если СЛАУ совместна, проверить единственность решения.
Для соответствующей однородной СЛАУ проверить существование нетривиального решения. В случае, если оно существует,
найти размерность пространства решений и составить ФСР и общее решение однородной  и неоднородной СЛАУ.


\begin{align*}
 A = \left[\begin{matrix}-4 & 3 & -5 & 0 & 0\\-44 & -11 & 21 & 24 & -20\\-188 & -35 & 69 & 96 & -80\\164 & 53 & -99 & -96 & 80\\504 & 150 & -282 & -288 & 240\end{matrix}\right],
\ b = \left[\begin{matrix}-12\\-256\\-1060\\988\\3000\end{matrix}\right]. 
 \end{align*}

Вариант N 142
Дана СЛАУ $AX = b$,
Проверить совместность по теореме Кронекера-Капелли. Если СЛАУ совместна, проверить единственность решения.
Для соответствующей однородной СЛАУ проверить существование нетривиального решения. В случае, если оно существует,
найти размерность пространства решений и составить ФСР и общее решение однородной  и неоднородной СЛАУ.


\begin{align*}
 A = \left[\begin{matrix}3 & -5 & -9 & -9 & -7\\32 & -55 & -51 & -56 & 17\\172 & -295 & -291 & -316 & 57\\-148 & 255 & 219 & 244 & -113\\-456 & 785 & 693 & 768 & -311\end{matrix}\right],
\ b = \left[\begin{matrix}166\\999\\5659\\-4331\\-13657\end{matrix}\right]. 
 \end{align*}

Вариант N 143
Дана СЛАУ $AX = b$,
Проверить совместность по теореме Кронекера-Капелли. Если СЛАУ совместна, проверить единственность решения.
Для соответствующей однородной СЛАУ проверить существование нетривиального решения. В случае, если оно существует,
найти размерность пространства решений и составить ФСР и общее решение однородной  и неоднородной СЛАУ.


\begin{align*}
 A = \left[\begin{matrix}4 & -4 & 6 & 1 & 8\\-4 & 0 & 3 & 6 & -9\\32 & -16 & 12 & -20 & 68\\-6 & 2 & -5 & 7 & 4\\7 & 1 & 3 & -5 & -3\end{matrix}\right],
\ b = \left[\begin{matrix}58\\-22\\320\\35\\-26\end{matrix}\right]. 
 \end{align*}

Вариант N 144
Дана СЛАУ $AX = b$,
Проверить совместность по теореме Кронекера-Капелли. Если СЛАУ совместна, проверить единственность решения.
Для соответствующей однородной СЛАУ проверить существование нетривиального решения. В случае, если оно существует,
найти размерность пространства решений и составить ФСР и общее решение однородной  и неоднородной СЛАУ.


\begin{align*}
 A = \left[\begin{matrix}3 & 7 & -8 & -4 & 4\\-6 & 9 & -5 & -2 & -3\\42 & -17 & -7 & -6 & 31\\3 & 8 & -5 & -3 & 8\\5 & -5 & -2 & 4 & -6\end{matrix}\right],
\ b = \left[\begin{matrix}-8\\53\\-297\\-18\\26\end{matrix}\right]. 
 \end{align*}

Вариант N 145
Дана СЛАУ $AX = b$,
Проверить совместность по теореме Кронекера-Капелли. Если СЛАУ совместна, проверить единственность решения.
Для соответствующей однородной СЛАУ проверить существование нетривиального решения. В случае, если оно существует,
найти размерность пространства решений и составить ФСР и общее решение однородной  и неоднородной СЛАУ.


\begin{align*}
 A = \left[\begin{matrix}-6 & 4 & 1 & -1 & -4\\-10 & 28 & -29 & 13 & -20\\-58 & 124 & -113 & 49 & -92\\22 & -100 & 119 & -55 & 68\\84 & -312 & 354 & -162 & 216\end{matrix}\right],
\ b = \left[\begin{matrix}-4\\36\\132\\-156\\-456\end{matrix}\right]. 
 \end{align*}

Вариант N 146
Дана СЛАУ $AX = b$,
Проверить совместность по теореме Кронекера-Капелли. Если СЛАУ совместна, проверить единственность решения.
Для соответствующей однородной СЛАУ проверить существование нетривиального решения. В случае, если оно существует,
найти размерность пространства решений и составить ФСР и общее решение однородной  и неоднородной СЛАУ.


\begin{align*}
 A = \left[\begin{matrix}7 & -7 & -8 & -2 & 4\\-7 & 0 & -7 & -4 & -7\\56 & -28 & -4 & 8 & 44\\-4 & -3 & 9 & -6 & -3\\-2 & 2 & 7 & 4 & 9\end{matrix}\right],
\ b = \left[\begin{matrix}73\\-34\\428\\-45\\19\end{matrix}\right]. 
 \end{align*}

Вариант N 147
Дана СЛАУ $AX = b$,
Проверить совместность по теореме Кронекера-Капелли. Если СЛАУ совместна, проверить единственность решения.
Для соответствующей однородной СЛАУ проверить существование нетривиального решения. В случае, если оно существует,
найти размерность пространства решений и составить ФСР и общее решение однородной  и неоднородной СЛАУ.


\begin{align*}
 A = \left[\begin{matrix}-2 & 9 & 6 & -8 & -1\\-2 & 9 & -5 & 9 & -7\\-16 & 72 & 4 & 4 & -32\\0 & 0 & 44 & -68 & 24\\3 & 0 & -4 & -4 & -3\end{matrix}\right],
\ b = \left[\begin{matrix}-100\\46\\-216\\-584\\-18\end{matrix}\right]. 
 \end{align*}

Вариант N 148
Дана СЛАУ $AX = b$,
Проверить совместность по теореме Кронекера-Капелли. Если СЛАУ совместна, проверить единственность решения.
Для соответствующей однородной СЛАУ проверить существование нетривиального решения. В случае, если оно существует,
найти размерность пространства решений и составить ФСР и общее решение однородной  и неоднородной СЛАУ.


\begin{align*}
 A = \left[\begin{matrix}-5 & 4 & -9 & -5 & 7\\0 & 2 & -8 & 4 & -8\\-15 & 4 & 5 & -31 & 53\\4 & 2 & -3 & -9 & 7\\-4 & 6 & 4 & 1 & -1\end{matrix}\right],
\ b = \left[\begin{matrix}71\\80\\-107\\13\\-23\end{matrix}\right]. 
 \end{align*}

Вариант N 149
Дана СЛАУ $AX = b$,
Проверить совместность по теореме Кронекера-Капелли. Если СЛАУ совместна, проверить единственность решения.
Для соответствующей однородной СЛАУ проверить существование нетривиального решения. В случае, если оно существует,
найти размерность пространства решений и составить ФСР и общее решение однородной  и неоднородной СЛАУ.


\begin{align*}
 A = \left[\begin{matrix}-5 & 0 & -3 & -5 & -1\\-5 & -3 & 9 & -6 & -4\\5 & 12 & -45 & 9 & 13\\-1 & 9 & 7 & 0 & 6\\8 & 6 & 2 & 9 & -3\end{matrix}\right],
\ b = \left[\begin{matrix}33\\101\\-305\\-51\\-99\end{matrix}\right]. 
 \end{align*}

Вариант N 150
Дана СЛАУ $AX = b$,
Проверить совместность по теореме Кронекера-Капелли. Если СЛАУ совместна, проверить единственность решения.
Для соответствующей однородной СЛАУ проверить существование нетривиального решения. В случае, если оно существует,
найти размерность пространства решений и составить ФСР и общее решение однородной  и неоднородной СЛАУ.


\begin{align*}
 A = \left[\begin{matrix}-1 & 3 & -8 & 0 & -1\\-3 & 34 & -14 & -15 & 37\\-18 & 179 & -94 & -75 & 182\\12 & -161 & 46 & 75 & -188\\39 & -492 & 162 & 225 & -561\end{matrix}\right],
\ b = \left[\begin{matrix}29\\82\\497\\-323\\-1056\end{matrix}\right]. 
 \end{align*}

Вариант N 151
Дана СЛАУ $AX = b$,
Проверить совместность по теореме Кронекера-Капелли. Если СЛАУ совместна, проверить единственность решения.
Для соответствующей однородной СЛАУ проверить существование нетривиального решения. В случае, если оно существует,
найти размерность пространства решений и составить ФСР и общее решение однородной  и неоднородной СЛАУ.


\begin{align*}
 A = \left[\begin{matrix}-2 & -6 & 0 & 0 & -2\\-8 & 1 & 2 & -6 & 2\\-46 & -13 & 10 & -30 & 4\\34 & -23 & -10 & 30 & -16\\-3 & -7 & -3 & -1 & 8\end{matrix}\right],
\ b = \left[\begin{matrix}-52\\5\\-131\\-181\\-4\end{matrix}\right]. 
 \end{align*}

Вариант N 152
Дана СЛАУ $AX = b$,
Проверить совместность по теореме Кронекера-Капелли. Если СЛАУ совместна, проверить единственность решения.
Для соответствующей однородной СЛАУ проверить существование нетривиального решения. В случае, если оно существует,
найти размерность пространства решений и составить ФСР и общее решение однородной  и неоднородной СЛАУ.


\begin{align*}
 A = \left[\begin{matrix}8 & -8 & -8 & 5 & -7\\-9 & 6 & -3 & 5 & -2\\-21 & 6 & -39 & 40 & -31\\69 & -54 & -9 & -10 & -11\\4 & -6 & -6 & 5 & 7\end{matrix}\right],
\ b = \left[\begin{matrix}-117\\27\\-216\\-486\\29\end{matrix}\right]. 
 \end{align*}

Вариант N 153
Дана СЛАУ $AX = b$,
Проверить совместность по теореме Кронекера-Капелли. Если СЛАУ совместна, проверить единственность решения.
Для соответствующей однородной СЛАУ проверить существование нетривиального решения. В случае, если оно существует,
найти размерность пространства решений и составить ФСР и общее решение однородной  и неоднородной СЛАУ.


\begin{align*}
 A = \left[\begin{matrix}-4 & 3 & -6 & 5 & 9\\-6 & -7 & 2 & 0 & 0\\8 & 40 & -32 & 20 & 36\\3 & 5 & -1 & -8 & 3\\2 & -2 & -6 & 6 & 1\end{matrix}\right],
\ b = \left[\begin{matrix}-87\\-82\\-20\\22\\-43\end{matrix}\right]. 
 \end{align*}

Вариант N 154
Дана СЛАУ $AX = b$,
Проверить совместность по теореме Кронекера-Капелли. Если СЛАУ совместна, проверить единственность решения.
Для соответствующей однородной СЛАУ проверить существование нетривиального решения. В случае, если оно существует,
найти размерность пространства решений и составить ФСР и общее решение однородной  и неоднородной СЛАУ.


\begin{align*}
 A = \left[\begin{matrix}2 & 3 & 0 & -5 & 7\\-22 & -27 & -28 & -43 & 5\\-82 & -99 & -112 & -187 & 41\\94 & 117 & 112 & 157 & 1\\276 & 342 & 336 & 486 & -18\end{matrix}\right],
\ b = \left[\begin{matrix}5\\-221\\-869\\899\\2682\end{matrix}\right]. 
 \end{align*}

Вариант N 155
Дана СЛАУ $AX = b$,
Проверить совместность по теореме Кронекера-Капелли. Если СЛАУ совместна, проверить единственность решения.
Для соответствующей однородной СЛАУ проверить существование нетривиального решения. В случае, если оно существует,
найти размерность пространства решений и составить ФСР и общее решение однородной  и неоднородной СЛАУ.


\begin{align*}
 A = \left[\begin{matrix}3 & -7 & 2 & 1 & -4\\8 & -32 & -20 & -20 & -32\\44 & -156 & -72 & -76 & -144\\-20 & 100 & 88 & 84 & 112\\-72 & 328 & 256 & 248 & 352\end{matrix}\right],
\ b = \left[\begin{matrix}-8\\92\\336\\-400\\-1168\end{matrix}\right]. 
 \end{align*}

Вариант N 156
Дана СЛАУ $AX = b$,
Проверить совместность по теореме Кронекера-Капелли. Если СЛАУ совместна, проверить единственность решения.
Для соответствующей однородной СЛАУ проверить существование нетривиального решения. В случае, если оно существует,
найти размерность пространства решений и составить ФСР и общее решение однородной  и неоднородной СЛАУ.


\begin{align*}
 A = \left[\begin{matrix}5 & -2 & 2 & 4 & 8\\-10 & -11 & -34 & -13 & 49\\-35 & -61 & -164 & -53 & 269\\65 & 49 & 176 & 77 & -221\\180 & 153 & 522 & 219 & -687\end{matrix}\right],
\ b = \left[\begin{matrix}9\\-768\\-3813\\3867\\11574\end{matrix}\right]. 
 \end{align*}

Вариант N 157
Дана СЛАУ $AX = b$,
Проверить совместность по теореме Кронекера-Капелли. Если СЛАУ совместна, проверить единственность решения.
Для соответствующей однородной СЛАУ проверить существование нетривиального решения. В случае, если оно существует,
найти размерность пространства решений и составить ФСР и общее решение однородной  и неоднородной СЛАУ.


\begin{align*}
 A = \left[\begin{matrix}4 & 0 & 9 & -7 & 7\\2 & -2 & -8 & -9 & -7\\20 & -8 & -5 & -57 & -7\\4 & 8 & 59 & 15 & 49\\2 & -8 & -1 & -4 & 4\end{matrix}\right],
\ b = \left[\begin{matrix}15\\52\\253\\-163\\-9\end{matrix}\right]. 
 \end{align*}

Вариант N 158
Дана СЛАУ $AX = b$,
Проверить совместность по теореме Кронекера-Капелли. Если СЛАУ совместна, проверить единственность решения.
Для соответствующей однородной СЛАУ проверить существование нетривиального решения. В случае, если оно существует,
найти размерность пространства решений и составить ФСР и общее решение однородной  и неоднородной СЛАУ.


\begin{align*}
 A = \left[\begin{matrix}2 & 3 & -7 & -4 & 4\\-9 & -8 & -2 & 0 & -7\\-39 & -31 & -31 & -12 & -23\\51 & 49 & -11 & -12 & 47\\0 & -9 & -9 & 6 & 8\end{matrix}\right],
\ b = \left[\begin{matrix}-34\\-19\\-197\\-7\\-100\end{matrix}\right]. 
 \end{align*}

Вариант N 159
Дана СЛАУ $AX = b$,
Проверить совместность по теореме Кронекера-Капелли. Если СЛАУ совместна, проверить единственность решения.
Для соответствующей однородной СЛАУ проверить существование нетривиального решения. В случае, если оно существует,
найти размерность пространства решений и составить ФСР и общее решение однородной  и неоднородной СЛАУ.


\begin{align*}
 A = \left[\begin{matrix}0 & 8 & 1 & -5 & 2\\1 & -9 & -4 & 3 & 5\\-5 & 77 & 24 & -35 & -17\\6 & -9 & 5 & 7 & 8\\-1 & 9 & -1 & 6 & 5\end{matrix}\right],
\ b = \left[\begin{matrix}23\\46\\-138\\39\\147\end{matrix}\right]. 
 \end{align*}

Вариант N 160
Дана СЛАУ $AX = b$,
Проверить совместность по теореме Кронекера-Капелли. Если СЛАУ совместна, проверить единственность решения.
Для соответствующей однородной СЛАУ проверить существование нетривиального решения. В случае, если оно существует,
найти размерность пространства решений и составить ФСР и общее решение однородной  и неоднородной СЛАУ.


\begin{align*}
 A = \left[\begin{matrix}-6 & 8 & 6 & -2 & 8\\6 & 3 & 4 & 3 & 9\\-54 & 17 & 4 & -23 & -13\\2 & -3 & 1 & 1 & 5\\7 & 8 & -2 & -6 & 2\end{matrix}\right],
\ b = \left[\begin{matrix}-20\\-23\\35\\-6\\-60\end{matrix}\right]. 
 \end{align*}

Вариант N 161
Дана СЛАУ $AX = b$,
Проверить совместность по теореме Кронекера-Капелли. Если СЛАУ совместна, проверить единственность решения.
Для соответствующей однородной СЛАУ проверить существование нетривиального решения. В случае, если оно существует,
найти размерность пространства решений и составить ФСР и общее решение однородной  и неоднородной СЛАУ.


\begin{align*}
 A = \left[\begin{matrix}6 & 2 & -7 & -6 & 0\\6 & 1 & 4 & -7 & -7\\48 & 11 & -1 & -53 & -35\\-12 & 1 & -41 & 17 & 35\\2 & 8 & 5 & 1 & -8\end{matrix}\right],
\ b = \left[\begin{matrix}-19\\90\\393\\-507\\138\end{matrix}\right]. 
 \end{align*}

Вариант N 162
Дана СЛАУ $AX = b$,
Проверить совместность по теореме Кронекера-Капелли. Если СЛАУ совместна, проверить единственность решения.
Для соответствующей однородной СЛАУ проверить существование нетривиального решения. В случае, если оно существует,
найти размерность пространства решений и составить ФСР и общее решение однородной  и неоднородной СЛАУ.


\begin{align*}
 A = \left[\begin{matrix}9 & 0 & 4 & -9 & 9\\42 & 10 & -23 & -22 & -3\\237 & 50 & -103 & -137 & 12\\-183 & -50 & 127 & 83 & 42\\-576 & -150 & 369 & 276 & 99\end{matrix}\right],
\ b = \left[\begin{matrix}-36\\607\\2927\\-3143\\-9321\end{matrix}\right]. 
 \end{align*}

Вариант N 163
Дана СЛАУ $AX = b$,
Проверить совместность по теореме Кронекера-Капелли. Если СЛАУ совместна, проверить единственность решения.
Для соответствующей однородной СЛАУ проверить существование нетривиального решения. В случае, если оно существует,
найти размерность пространства решений и составить ФСР и общее решение однородной  и неоднородной СЛАУ.


\begin{align*}
 A = \left[\begin{matrix}8 & -9 & -6 & 8 & 3\\7 & -9 & 1 & -1 & 0\\59 & -72 & -13 & 19 & 9\\-11 & 18 & -23 & 29 & 9\\-9 & -4 & -2 & 9 & -4\end{matrix}\right],
\ b = \left[\begin{matrix}-114\\-98\\-832\\148\\22\end{matrix}\right]. 
 \end{align*}

Вариант N 164
Дана СЛАУ $AX = b$,
Проверить совместность по теореме Кронекера-Капелли. Если СЛАУ совместна, проверить единственность решения.
Для соответствующей однородной СЛАУ проверить существование нетривиального решения. В случае, если оно существует,
найти размерность пространства решений и составить ФСР и общее решение однородной  и неоднородной СЛАУ.


\begin{align*}
 A = \left[\begin{matrix}-7 & 5 & 8 & 7 & -1\\7 & -7 & 1 & 6 & 1\\0 & -8 & 36 & 52 & 0\\-56 & 48 & 28 & 4 & -8\\6 & 0 & -5 & -7 & 6\end{matrix}\right],
\ b = \left[\begin{matrix}32\\8\\160\\96\\-60\end{matrix}\right]. 
 \end{align*}

Вариант N 165
Дана СЛАУ $AX = b$,
Проверить совместность по теореме Кронекера-Капелли. Если СЛАУ совместна, проверить единственность решения.
Для соответствующей однородной СЛАУ проверить существование нетривиального решения. В случае, если оно существует,
найти размерность пространства решений и составить ФСР и общее решение однородной  и неоднородной СЛАУ.


\begin{align*}
 A = \left[\begin{matrix}-4 & -6 & -8 & 2 & -5\\4 & 0 & 2 & -6 & -5\\8 & -18 & -14 & -24 & -40\\-32 & -18 & -34 & 36 & 10\\-9 & 3 & 6 & 1 & -6\end{matrix}\right],
\ b = \left[\begin{matrix}-10\\48\\210\\-270\\14\end{matrix}\right]. 
 \end{align*}

Вариант N 166
Дана СЛАУ $AX = b$,
Проверить совместность по теореме Кронекера-Капелли. Если СЛАУ совместна, проверить единственность решения.
Для соответствующей однородной СЛАУ проверить существование нетривиального решения. В случае, если оно существует,
найти размерность пространства решений и составить ФСР и общее решение однородной  и неоднородной СЛАУ.


\begin{align*}
 A = \left[\begin{matrix}4 & 0 & 0 & 9 & 3\\-5 & 3 & -3 & 0 & 2\\36 & -12 & 12 & 36 & 4\\-9 & 5 & 9 & -1 & -7\\-1 & 4 & -8 & -2 & -9\end{matrix}\right],
\ b = \left[\begin{matrix}-25\\32\\-228\\-89\\39\end{matrix}\right]. 
 \end{align*}

Вариант N 167
Дана СЛАУ $AX = b$,
Проверить совместность по теореме Кронекера-Капелли. Если СЛАУ совместна, проверить единственность решения.
Для соответствующей однородной СЛАУ проверить существование нетривиального решения. В случае, если оно существует,
найти размерность пространства решений и составить ФСР и общее решение однородной  и неоднородной СЛАУ.


\begin{align*}
 A = \left[\begin{matrix}-8 & 3 & -5 & -5 & -9\\8 & 2 & 3 & -4 & -8\\8 & 22 & -5 & -40 & -76\\-72 & 2 & -35 & 0 & 4\\8 & 4 & -7 & -7 & -9\end{matrix}\right],
\ b = \left[\begin{matrix}41\\32\\324\\4\\34\end{matrix}\right]. 
 \end{align*}

Вариант N 168
Дана СЛАУ $AX = b$,
Проверить совместность по теореме Кронекера-Капелли. Если СЛАУ совместна, проверить единственность решения.
Для соответствующей однородной СЛАУ проверить существование нетривиального решения. В случае, если оно существует,
найти размерность пространства решений и составить ФСР и общее решение однородной  и неоднородной СЛАУ.


\begin{align*}
 A = \left[\begin{matrix}-8 & 0 & -7 & 4 & -4\\5 & -6 & -3 & 4 & -4\\-7 & -30 & -43 & 36 & -36\\-57 & 30 & -13 & -4 & 4\\-9 & -9 & 6 & -5 & 3\end{matrix}\right],
\ b = \left[\begin{matrix}-30\\21\\-15\\-225\\-69\end{matrix}\right]. 
 \end{align*}

Вариант N 169
Дана СЛАУ $AX = b$,
Проверить совместность по теореме Кронекера-Капелли. Если СЛАУ совместна, проверить единственность решения.
Для соответствующей однородной СЛАУ проверить существование нетривиального решения. В случае, если оно существует,
найти размерность пространства решений и составить ФСР и общее решение однородной  и неоднородной СЛАУ.


\begin{align*}
 A = \left[\begin{matrix}-1 & -7 & -7 & 3 & 1\\4 & -4 & 2 & 6 & 1\\-19 & -5 & -29 & -15 & -1\\8 & -7 & 8 & 6 & 2\\4 & 4 & 1 & -3 & 1\end{matrix}\right],
\ b = \left[\begin{matrix}36\\-33\\240\\-76\\-47\end{matrix}\right]. 
 \end{align*}

Вариант N 170
Дана СЛАУ $AX = b$,
Проверить совместность по теореме Кронекера-Капелли. Если СЛАУ совместна, проверить единственность решения.
Для соответствующей однородной СЛАУ проверить существование нетривиального решения. В случае, если оно существует,
найти размерность пространства решений и составить ФСР и общее решение однородной  и неоднородной СЛАУ.


\begin{align*}
 A = \left[\begin{matrix}9 & 0 & -9 & 2 & 8\\44 & 0 & -16 & 40 & 56\\212 & 0 & -100 & 168 & 256\\-140 & 0 & 28 & -152 & -192\\-456 & 0 & 120 & -464 & -608\end{matrix}\right],
\ b = \left[\begin{matrix}-18\\92\\296\\-440\\-1248\end{matrix}\right]. 
 \end{align*}

Вариант N 171
Дана СЛАУ $AX = b$,
Проверить совместность по теореме Кронекера-Капелли. Если СЛАУ совместна, проверить единственность решения.
Для соответствующей однородной СЛАУ проверить существование нетривиального решения. В случае, если оно существует,
найти размерность пространства решений и составить ФСР и общее решение однородной  и неоднородной СЛАУ.


\begin{align*}
 A = \left[\begin{matrix}6 & -2 & 1 & 9 & -9\\63 & 4 & 3 & 22 & 18\\333 & 14 & 18 & 137 & 63\\-297 & -26 & -12 & -83 & -117\\-909 & -72 & -39 & -276 & -324\end{matrix}\right],
\ b = \left[\begin{matrix}-11\\27\\102\\-168\\-471\end{matrix}\right]. 
 \end{align*}

Вариант N 172
Дана СЛАУ $AX = b$,
Проверить совместность по теореме Кронекера-Капелли. Если СЛАУ совместна, проверить единственность решения.
Для соответствующей однородной СЛАУ проверить существование нетривиального решения. В случае, если оно существует,
найти размерность пространства решений и составить ФСР и общее решение однородной  и неоднородной СЛАУ.


\begin{align*}
 A = \left[\begin{matrix}5 & -6 & -9 & 5 & -5\\-9 & -4 & -5 & 6 & -3\\-25 & -44 & -61 & 50 & -35\\65 & -4 & -11 & -10 & -5\\0 & 4 & 5 & -5 & 7\end{matrix}\right],
\ b = \left[\begin{matrix}66\\65\\589\\-61\\-38\end{matrix}\right]. 
 \end{align*}

Вариант N 173
Дана СЛАУ $AX = b$,
Проверить совместность по теореме Кронекера-Капелли. Если СЛАУ совместна, проверить единственность решения.
Для соответствующей однородной СЛАУ проверить существование нетривиального решения. В случае, если оно существует,
найти размерность пространства решений и составить ФСР и общее решение однородной  и неоднородной СЛАУ.


\begin{align*}
 A = \left[\begin{matrix}8 & -9 & -6 & 5 & 1\\2 & -9 & 7 & -4 & 4\\22 & 9 & -59 & 40 & -16\\-1 & 8 & -8 & 9 & -3\\5 & 3 & 6 & -3 & 5\end{matrix}\right],
\ b = \left[\begin{matrix}39\\39\\-39\\-57\\42\end{matrix}\right]. 
 \end{align*}

Вариант N 174
Дана СЛАУ $AX = b$,
Проверить совместность по теореме Кронекера-Капелли. Если СЛАУ совместна, проверить единственность решения.
Для соответствующей однородной СЛАУ проверить существование нетривиального решения. В случае, если оно существует,
найти размерность пространства решений и составить ФСР и общее решение однородной  и неоднородной СЛАУ.


\begin{align*}
 A = \left[\begin{matrix}-4 & -3 & 5 & 1 & -5\\3 & 0 & -7 & -9 & 1\\0 & -9 & -13 & -33 & -11\\-24 & -9 & 43 & 39 & -19\\-3 & 2 & -7 & -3 & -8\end{matrix}\right],
\ b = \left[\begin{matrix}18\\38\\206\\-98\\-18\end{matrix}\right]. 
 \end{align*}

Вариант N 175
Дана СЛАУ $AX = b$,
Проверить совместность по теореме Кронекера-Капелли. Если СЛАУ совместна, проверить единственность решения.
Для соответствующей однородной СЛАУ проверить существование нетривиального решения. В случае, если оно существует,
найти размерность пространства решений и составить ФСР и общее решение однородной  и неоднородной СЛАУ.


\begin{align*}
 A = \left[\begin{matrix}0 & 6 & 3 & -6 & 0\\20 & 34 & 37 & -34 & -36\\80 & 154 & 157 & -154 & -144\\-80 & -118 & -139 & 118 & 144\\-240 & -372 & -426 & 372 & 432\end{matrix}\right],
\ b = \left[\begin{matrix}24\\68\\344\\-200\\-672\end{matrix}\right]. 
 \end{align*}

Вариант N 176
Дана СЛАУ $AX = b$,
Проверить совместность по теореме Кронекера-Капелли. Если СЛАУ совместна, проверить единственность решения.
Для соответствующей однородной СЛАУ проверить существование нетривиального решения. В случае, если оно существует,
найти размерность пространства решений и составить ФСР и общее решение однородной  и неоднородной СЛАУ.


\begin{align*}
 A = \left[\begin{matrix}3 & -4 & 7 & 9 & 5\\7 & -3 & -6 & 9 & 9\\-26 & 3 & 51 & -18 & -30\\-2 & 8 & 5 & -3 & 1\\2 & -3 & -6 & 9 & -2\end{matrix}\right],
\ b = \left[\begin{matrix}12\\-27\\171\\64\\-52\end{matrix}\right]. 
 \end{align*}

Вариант N 177
Дана СЛАУ $AX = b$,
Проверить совместность по теореме Кронекера-Капелли. Если СЛАУ совместна, проверить единственность решения.
Для соответствующей однородной СЛАУ проверить существование нетривиального решения. В случае, если оно существует,
найти размерность пространства решений и составить ФСР и общее решение однородной  и неоднородной СЛАУ.


\begin{align*}
 A = \left[\begin{matrix}-8 & 8 & 0 & 6 & -5\\-2 & 2 & -1 & -5 & -4\\-34 & 34 & -5 & -7 & -35\\-14 & 14 & 5 & 43 & 5\\-8 & 4 & -8 & -9 & 9\end{matrix}\right],
\ b = \left[\begin{matrix}-71\\-2\\-223\\-203\\3\end{matrix}\right]. 
 \end{align*}

Вариант N 178
Дана СЛАУ $AX = b$,
Проверить совместность по теореме Кронекера-Капелли. Если СЛАУ совместна, проверить единственность решения.
Для соответствующей однородной СЛАУ проверить существование нетривиального решения. В случае, если оно существует,
найти размерность пространства решений и составить ФСР и общее решение однородной  и неоднородной СЛАУ.


\begin{align*}
 A = \left[\begin{matrix}0 & 1 & 5 & 7 & -9\\7 & 7 & 8 & -9 & -3\\-35 & -31 & -20 & 73 & -21\\2 & -2 & -7 & -6 & -5\\7 & 0 & -7 & 3 & -9\end{matrix}\right],
\ b = \left[\begin{matrix}36\\97\\-341\\90\\101\end{matrix}\right]. 
 \end{align*}

Вариант N 179
Дана СЛАУ $AX = b$,
Проверить совместность по теореме Кронекера-Капелли. Если СЛАУ совместна, проверить единственность решения.
Для соответствующей однородной СЛАУ проверить существование нетривиального решения. В случае, если оно существует,
найти размерность пространства решений и составить ФСР и общее решение однородной  и неоднородной СЛАУ.


\begin{align*}
 A = \left[\begin{matrix}-8 & -1 & -7 & 9 & -1\\-9 & -6 & -1 & -2 & -9\\13 & 26 & -23 & 46 & 41\\9 & -9 & -4 & -1 & 7\\-5 & -9 & -2 & -2 & 7\end{matrix}\right],
\ b = \left[\begin{matrix}61\\62\\-66\\-56\\1\end{matrix}\right]. 
 \end{align*}

Вариант N 180
Дана СЛАУ $AX = b$,
Проверить совместность по теореме Кронекера-Капелли. Если СЛАУ совместна, проверить единственность решения.
Для соответствующей однородной СЛАУ проверить существование нетривиального решения. В случае, если оно существует,
найти размерность пространства решений и составить ФСР и общее решение однородной  и неоднородной СЛАУ.


\begin{align*}
 A = \left[\begin{matrix}7 & 4 & -1 & -6 & -3\\66 & -23 & 17 & 2 & 21\\351 & -103 & 82 & -8 & 96\\-309 & 127 & -88 & -28 & -114\\-948 & 369 & -261 & -66 & -333\end{matrix}\right],
\ b = \left[\begin{matrix}-11\\-838\\-4223\\4157\\12504\end{matrix}\right]. 
 \end{align*}

Вариант N 181
Дана СЛАУ $AX = b$,
Проверить совместность по теореме Кронекера-Капелли. Если СЛАУ совместна, проверить единственность решения.
Для соответствующей однородной СЛАУ проверить существование нетривиального решения. В случае, если оно существует,
найти размерность пространства решений и составить ФСР и общее решение однородной  и неоднородной СЛАУ.


\begin{align*}
 A = \left[\begin{matrix}2 & -9 & -2 & -7 & 6\\6 & 4 & 8 & 5 & 0\\-24 & -47 & -46 & -46 & 18\\-3 & -2 & 0 & -2 & 2\\3 & 7 & -9 & 4 & 0\end{matrix}\right],
\ b = \left[\begin{matrix}-47\\98\\-631\\-39\\81\end{matrix}\right]. 
 \end{align*}

Вариант N 182
Дана СЛАУ $AX = b$,
Проверить совместность по теореме Кронекера-Капелли. Если СЛАУ совместна, проверить единственность решения.
Для соответствующей однородной СЛАУ проверить существование нетривиального решения. В случае, если оно существует,
найти размерность пространства решений и составить ФСР и общее решение однородной  и неоднородной СЛАУ.


\begin{align*}
 A = \left[\begin{matrix}-9 & 3 & 8 & -3 & 9\\-27 & 5 & 32 & 23 & 15\\-135 & 29 & 152 & 83 & 87\\81 & -11 & -104 & -101 & -33\\270 & -42 & -336 & -294 & -126\end{matrix}\right],
\ b = \left[\begin{matrix}-54\\-170\\-842\\518\\1716\end{matrix}\right]. 
 \end{align*}

Вариант N 183
Дана СЛАУ $AX = b$,
Проверить совместность по теореме Кронекера-Капелли. Если СЛАУ совместна, проверить единственность решения.
Для соответствующей однородной СЛАУ проверить существование нетривиального решения. В случае, если оно существует,
найти размерность пространства решений и составить ФСР и общее решение однородной  и неоднородной СЛАУ.


\begin{align*}
 A = \left[\begin{matrix}-3 & -3 & -4 & 0 & 6\\-6 & 3 & 3 & -5 & -4\\21 & -24 & -27 & 25 & 38\\-6 & 8 & 3 & -3 & -2\\-4 & -8 & -1 & -8 & -7\end{matrix}\right],
\ b = \left[\begin{matrix}4\\43\\-203\\-18\\169\end{matrix}\right]. 
 \end{align*}

Вариант N 184
Дана СЛАУ $AX = b$,
Проверить совместность по теореме Кронекера-Капелли. Если СЛАУ совместна, проверить единственность решения.
Для соответствующей однородной СЛАУ проверить существование нетривиального решения. В случае, если оно существует,
найти размерность пространства решений и составить ФСР и общее решение однородной  и неоднородной СЛАУ.


\begin{align*}
 A = \left[\begin{matrix}5 & 0 & 6 & -7 & 7\\15 & 45 & 29 & -68 & 58\\95 & 225 & 169 & -368 & 318\\-55 & -225 & -121 & 312 & -262\\-185 & -675 & -387 & 964 & -814\end{matrix}\right],
\ b = \left[\begin{matrix}-34\\4\\-116\\-156\\-332\end{matrix}\right]. 
 \end{align*}

Вариант N 185
Дана СЛАУ $AX = b$,
Проверить совместность по теореме Кронекера-Капелли. Если СЛАУ совместна, проверить единственность решения.
Для соответствующей однородной СЛАУ проверить существование нетривиального решения. В случае, если оно существует,
найти размерность пространства решений и составить ФСР и общее решение однородной  и неоднородной СЛАУ.


\begin{align*}
 A = \left[\begin{matrix}-5 & 2 & -8 & -1 & -8\\-15 & 26 & 6 & 7 & -29\\-90 & 136 & 6 & 32 & -169\\60 & -124 & -54 & -38 & 121\\195 & -378 & -138 & -111 & 387\end{matrix}\right],
\ b = \left[\begin{matrix}4\\132\\672\\-648\\-1956\end{matrix}\right]. 
 \end{align*}

Вариант N 186
Дана СЛАУ $AX = b$,
Проверить совместность по теореме Кронекера-Капелли. Если СЛАУ совместна, проверить единственность решения.
Для соответствующей однородной СЛАУ проверить существование нетривиального решения. В случае, если оно существует,
найти размерность пространства решений и составить ФСР и общее решение однородной  и неоднородной СЛАУ.


\begin{align*}
 A = \left[\begin{matrix}-4 & 6 & 9 & 0 & 2\\9 & 7 & -4 & -6 & 2\\29 & 59 & 16 & -30 & 18\\-61 & -11 & 56 & 30 & -2\\-2 & -1 & 6 & -8 & 4\end{matrix}\right],
\ b = \left[\begin{matrix}-30\\-108\\-660\\420\\-7\end{matrix}\right]. 
 \end{align*}

Вариант N 187
Дана СЛАУ $AX = b$,
Проверить совместность по теореме Кронекера-Капелли. Если СЛАУ совместна, проверить единственность решения.
Для соответствующей однородной СЛАУ проверить существование нетривиального решения. В случае, если оно существует,
найти размерность пространства решений и составить ФСР и общее решение однородной  и неоднородной СЛАУ.


\begin{align*}
 A = \left[\begin{matrix}7 & -7 & -2 & 5 & -4\\1 & -9 & 9 & -2 & -2\\16 & 24 & -51 & 25 & -2\\-2 & 1 & -3 & -5 & -7\\-5 & 1 & -4 & -9 & 5\end{matrix}\right],
\ b = \left[\begin{matrix}11\\95\\-442\\6\\28\end{matrix}\right]. 
 \end{align*}

Вариант N 188
Дана СЛАУ $AX = b$,
Проверить совместность по теореме Кронекера-Капелли. Если СЛАУ совместна, проверить единственность решения.
Для соответствующей однородной СЛАУ проверить существование нетривиального решения. В случае, если оно существует,
найти размерность пространства решений и составить ФСР и общее решение однородной  и неоднородной СЛАУ.


\begin{align*}
 A = \left[\begin{matrix}4 & 4 & 9 & -6 & -6\\5 & 5 & 1 & -3 & -1\\-4 & -4 & 32 & -12 & -20\\-1 & 4 & -8 & -3 & 1\\2 & -7 & -7 & -7 & 3\end{matrix}\right],
\ b = \left[\begin{matrix}29\\-28\\228\\-5\\52\end{matrix}\right]. 
 \end{align*}

Вариант N 189
Дана СЛАУ $AX = b$,
Проверить совместность по теореме Кронекера-Капелли. Если СЛАУ совместна, проверить единственность решения.
Для соответствующей однородной СЛАУ проверить существование нетривиального решения. В случае, если оно существует,
найти размерность пространства решений и составить ФСР и общее решение однородной  и неоднородной СЛАУ.


\begin{align*}
 A = \left[\begin{matrix}2 & 4 & -5 & -4 & -3\\-39 & 7 & 15 & -42 & -44\\-189 & 47 & 60 & -222 & -229\\201 & -23 & -90 & 198 & 211\\597 & -81 & -255 & 606 & 642\end{matrix}\right],
\ b = \left[\begin{matrix}47\\731\\3796\\-3514\\-10683\end{matrix}\right]. 
 \end{align*}

Вариант N 190
Дана СЛАУ $AX = b$,
Проверить совместность по теореме Кронекера-Капелли. Если СЛАУ совместна, проверить единственность решения.
Для соответствующей однородной СЛАУ проверить существование нетривиального решения. В случае, если оно существует,
найти размерность пространства решений и составить ФСР и общее решение однородной  и неоднородной СЛАУ.


\begin{align*}
 A = \left[\begin{matrix}-5 & -4 & -8 & 0 & -1\\0 & -7 & 0 & 8 & 1\\-20 & -51 & -32 & 40 & 1\\-20 & 19 & -32 & -40 & -9\\-1 & 2 & -7 & 6 & 3\end{matrix}\right],
\ b = \left[\begin{matrix}94\\79\\771\\-19\\66\end{matrix}\right]. 
 \end{align*}

Вариант N 191
Дана СЛАУ $AX = b$,
Проверить совместность по теореме Кронекера-Капелли. Если СЛАУ совместна, проверить единственность решения.
Для соответствующей однородной СЛАУ проверить существование нетривиального решения. В случае, если оно существует,
найти размерность пространства решений и составить ФСР и общее решение однородной  и неоднородной СЛАУ.


\begin{align*}
 A = \left[\begin{matrix}-9 & -5 & -4 & -7 & -1\\0 & 9 & 3 & 0 & 7\\-27 & -51 & -24 & -21 & -31\\7 & 5 & -7 & -8 & 9\\2 & -4 & 1 & 5 & 8\end{matrix}\right],
\ b = \left[\begin{matrix}16\\24\\-48\\67\\79\end{matrix}\right]. 
 \end{align*}

Вариант N 192
Дана СЛАУ $AX = b$,
Проверить совместность по теореме Кронекера-Капелли. Если СЛАУ совместна, проверить единственность решения.
Для соответствующей однородной СЛАУ проверить существование нетривиального решения. В случае, если оно существует,
найти размерность пространства решений и составить ФСР и общее решение однородной  и неоднородной СЛАУ.


\begin{align*}
 A = \left[\begin{matrix}-7 & -6 & 8 & -3 & -7\\9 & 6 & 3 & -6 & -1\\-66 & -48 & 9 & 21 & -16\\-2 & 1 & 4 & -3 & 1\\4 & -8 & -2 & 9 & -5\end{matrix}\right],
\ b = \left[\begin{matrix}42\\-23\\241\\12\\-69\end{matrix}\right]. 
 \end{align*}

Вариант N 193
Дана СЛАУ $AX = b$,
Проверить совместность по теореме Кронекера-Капелли. Если СЛАУ совместна, проверить единственность решения.
Для соответствующей однородной СЛАУ проверить существование нетривиального решения. В случае, если оно существует,
найти размерность пространства решений и составить ФСР и общее решение однородной  и неоднородной СЛАУ.


\begin{align*}
 A = \left[\begin{matrix}-4 & -1 & -1 & 3 & 4\\7 & -9 & -2 & 8 & -9\\-51 & 41 & 6 & -28 & 61\\7 & -6 & 5 & 7 & 5\\1 & -4 & 5 & 2 & 0\end{matrix}\right],
\ b = \left[\begin{matrix}-44\\-117\\409\\-153\\-14\end{matrix}\right]. 
 \end{align*}

Вариант N 194
Дана СЛАУ $AX = b$,
Проверить совместность по теореме Кронекера-Капелли. Если СЛАУ совместна, проверить единственность решения.
Для соответствующей однородной СЛАУ проверить существование нетривиального решения. В случае, если оно существует,
найти размерность пространства решений и составить ФСР и общее решение однородной  и неоднородной СЛАУ.


\begin{align*}
 A = \left[\begin{matrix}-4 & 9 & -6 & 4 & -7\\7 & -1 & -1 & 4 & 0\\16 & 23 & -22 & 28 & -21\\-40 & 31 & -14 & -4 & -21\\-5 & -5 & 0 & 8 & -3\end{matrix}\right],
\ b = \left[\begin{matrix}-86\\4\\-242\\-274\\-163\end{matrix}\right]. 
 \end{align*}

Вариант N 195
Дана СЛАУ $AX = b$,
Проверить совместность по теореме Кронекера-Капелли. Если СЛАУ совместна, проверить единственность решения.
Для соответствующей однородной СЛАУ проверить существование нетривиального решения. В случае, если оно существует,
найти размерность пространства решений и составить ФСР и общее решение однородной  и неоднородной СЛАУ.


\begin{align*}
 A = \left[\begin{matrix}-8 & -8 & 7 & 8 & 9\\1 & -9 & -9 & 6 & -9\\-27 & -77 & -17 & 62 & -9\\-37 & 13 & 73 & 2 & 81\\8 & 7 & -5 & -3 & 3\end{matrix}\right],
\ b = \left[\begin{matrix}-112\\13\\-383\\-513\\42\end{matrix}\right]. 
 \end{align*}

Вариант N 196
Дана СЛАУ $AX = b$,
Проверить совместность по теореме Кронекера-Капелли. Если СЛАУ совместна, проверить единственность решения.
Для соответствующей однородной СЛАУ проверить существование нетривиального решения. В случае, если оно существует,
найти размерность пространства решений и составить ФСР и общее решение однородной  и неоднородной СЛАУ.


\begin{align*}
 A = \left[\begin{matrix}4 & 4 & 3 & -9 & 5\\36 & 26 & 32 & -16 & 0\\196 & 146 & 172 & -116 & 20\\-164 & -114 & -148 & 44 & 20\\-508 & -358 & -456 & 168 & 40\end{matrix}\right],
\ b = \left[\begin{matrix}-78\\-262\\-1622\\998\\3306\end{matrix}\right]. 
 \end{align*}

Вариант N 197
Дана СЛАУ $AX = b$,
Проверить совместность по теореме Кронекера-Капелли. Если СЛАУ совместна, проверить единственность решения.
Для соответствующей однородной СЛАУ проверить существование нетривиального решения. В случае, если оно существует,
найти размерность пространства решений и составить ФСР и общее решение однородной  и неоднородной СЛАУ.


\begin{align*}
 A = \left[\begin{matrix}-6 & 0 & -9 & -4 & 2\\-1 & 9 & -6 & -1 & -2\\-28 & 36 & -60 & -20 & 0\\-20 & -36 & -12 & -12 & 16\\-8 & -2 & -1 & -3 & -2\end{matrix}\right],
\ b = \left[\begin{matrix}33\\-30\\12\\252\\26\end{matrix}\right]. 
 \end{align*}

Вариант N 198
Дана СЛАУ $AX = b$,
Проверить совместность по теореме Кронекера-Капелли. Если СЛАУ совместна, проверить единственность решения.
Для соответствующей однородной СЛАУ проверить существование нетривиального решения. В случае, если оно существует,
найти размерность пространства решений и составить ФСР и общее решение однородной  и неоднородной СЛАУ.


\begin{align*}
 A = \left[\begin{matrix}0 & -5 & -8 & 9 & 0\\8 & 4 & 0 & 4 & -9\\32 & 1 & -24 & 43 & -36\\-32 & -31 & -24 & 11 & 36\\7 & 9 & -7 & -9 & -3\end{matrix}\right],
\ b = \left[\begin{matrix}127\\20\\461\\301\\7\end{matrix}\right]. 
 \end{align*}

Вариант N 199
Дана СЛАУ $AX = b$,
Проверить совместность по теореме Кронекера-Капелли. Если СЛАУ совместна, проверить единственность решения.
Для соответствующей однородной СЛАУ проверить существование нетривиального решения. В случае, если оно существует,
найти размерность пространства решений и составить ФСР и общее решение однородной  и неоднородной СЛАУ.


\begin{align*}
 A = \left[\begin{matrix}-4 & 4 & 7 & -8 & -7\\9 & 8 & -1 & -2 & 5\\20 & 48 & 24 & -40 & -8\\-52 & -16 & 32 & -24 & -48\\9 & -6 & -8 & 7 & -4\end{matrix}\right],
\ b = \left[\begin{matrix}76\\77\\612\\-4\\-14\end{matrix}\right]. 
 \end{align*}

Вариант N 200
Дана СЛАУ $AX = b$,
Проверить совместность по теореме Кронекера-Капелли. Если СЛАУ совместна, проверить единственность решения.
Для соответствующей однородной СЛАУ проверить существование нетривиального решения. В случае, если оно существует,
найти размерность пространства решений и составить ФСР и общее решение однородной  и неоднородной СЛАУ.


\begin{align*}
 A = \left[\begin{matrix}-1 & -6 & -2 & 4 & -6\\-7 & 6 & -4 & 8 & 7\\-39 & 6 & -28 & 56 & 11\\31 & -54 & 12 & -24 & -59\\-1 & 5 & 1 & 7 & -3\end{matrix}\right],
\ b = \left[\begin{matrix}-47\\-24\\-308\\-68\\-59\end{matrix}\right]. 
 \end{align*}

\end{document}